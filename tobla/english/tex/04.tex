
\documentclass[10pt,fleqn]{article}
%\documentclass[a4paper,10pt]{article}
%\documentclass[letterpaper,10pt]{article}

\usepackage[dvips]{geometry}
\geometry{papersize={162.0mm,234.0mm}}
\geometry{totalwidth=141.0mm,totalheight=198.0mm}

\usepackage[english]{babel}
\usepackage[latin1]{inputenc}
\usepackage{amsfonts}
\usepackage{amsmath,bm}

\usepackage{hyperref}
\hypersetup{colorlinks=true,linkcolor=black,bookmarksopen=true}
\hypersetup{bookmarksnumbered=true,pdfstartview=FitH,pdfpagemode=UseNone}
\hypersetup{pdftitle={A Reformulation of Special Relativity}}
\hypersetup{pdfauthor={Agust�n A. Tobla}}

\setlength{\arraycolsep}{1.74pt}

\newcommand{\tauxi}{{\smash{t}\hspace{-0.39em}\rule[2.8pt]{3.6pt}{0.3pt}}}
\newcommand{\sauxi}{{\smash{r}\hspace{-0.39em}\rule[5.7pt]{3.6pt}{0.3pt}}}

\begin{document}

\addcontentsline{toc}{section}{A Reformulation of Special Relativity}

\begin{center}

{\LARGE A Reformulation of Special Relativity}

\bigskip \medskip

{\large Agust�n A. Tobla}

\bigskip \medskip

\small

Creative Commons Attribution 3.0 License

\smallskip

(2024) Buenos Aires

\medskip

Argentina

\smallskip

\bigskip \medskip

\parbox{109.50mm}{This paper presents a reformulation of special relativity, whose kinematic and dynamic magnitudes are invariant under transformations between inertial and non-inertial reference frames, which can be applied in massive and non-massive particles, and where the relationship between net force and special acceleration is as in Newton's second law. Additionally, new universal forces are proposed.}

\end{center}

\normalsize

\vspace{-1.20em}

\par \bigskip {\centering\subsection*{Introduction}}\addcontentsline{toc}{subsection}{1. Introduction}

\bigskip \smallskip

\noindent The reformulation of special relativity presented in this paper is obtained starting from an auxiliary massive particle ( called auxiliary-point ) that is used to obtain kinematic magnitudes ( such as relational time,\, relational position,\, relational velocity,\, etc. \hspace{-0.30em}) that are invariant under transformations between inertial and non-inertial reference frames.
\par \bigskip \smallskip
\noindent The relational time $( \, t \, )$, the relational position $( \, \mathbf{r} \, )$, the relational velocity $( \, \mathbf{v} \, )$ and the relational acceleration $( \, \mathbf{a} \, )$ of a particle ( massive or non-massive ) \hbox {relative} to an inertial reference frame S, are given by:
\par \vspace{+0.21em}
\begin{eqnarray}
t \hspace{+0.12em}~\doteq~ \tauxi ~=~ \hspace{+0.00em} \gamma \left ( \mathtt{t} - \frac{\vec{r} \cdot \vec{u}}{c^2} \right )
\end{eqnarray}
\vspace{-0.45em}
\begin{eqnarray}
\mathbf{r} ~\doteq~ (\hspace{+0.090em}\sauxi\hspace{+0.090em}) ~=~ \hspace{-0.12em} \left [ \; \vec{r} + \frac{\gamma^2}{\gamma + 1} \frac{( \vec{r} \cdot \vec{u} \hspace{+0.09em}) \, \vec{u}}{c^2} - \gamma \, \vec{u} \; \mathtt{t} \; \right ]
\end{eqnarray}
\vspace{-0.30em}
\begin{eqnarray}
\mathbf{v} \hspace{-0.15em}~\doteq~ d\hspace{+0.090em}(\hspace{+0.090em}\sauxi\hspace{+0.090em})\hspace{+0.045em}/d\hspace{+0.03em}\tauxi ~=~ \hspace{-0.12em} \left [ \; \vec{v} + \frac{\gamma^2}{\gamma + 1} \frac{( \vec{v} \cdot \vec{u} \hspace{+0.09em}) \, \vec{u}}{c^2} - \hspace{+0.114em} \gamma \, \vec{u} \hspace{+0.114em} \; \right ] \frac{1}{\gamma \, ( 1 - \frac{\vec{v} \cdot \vec{u}}{c^2} )}
\end{eqnarray}
\vspace{-0.30em}
\begin{eqnarray}
\mathbf{a} \hspace{-0.09em}~\doteq~ d^{\hspace{+0.060em}2}(\hspace{+0.090em}\sauxi\hspace{+0.090em})\hspace{+0.045em}/d\hspace{+0.03em}\tauxi^{\hspace{+0.090em}2} ~=~ \hspace{-0.12em} \left [ \; \vec{a} - \frac{\gamma}{\gamma + 1} \frac{( \vec{a} \cdot \vec{u} \hspace{+0.09em}) \, \vec{u}}{c^2} + \frac{( \vec{a} \times \vec{v} \hspace{+0.09em}) \times \vec{u}}{c^2} \; \right ] \frac{1}{\gamma^2 \, ( 1 - \frac{\vec{v} \cdot \vec{u}}{c^2} )^3}
\end{eqnarray}
\par \vspace{+1.20em}
\noindent where $( \, {\tauxi}\hspace{+0.060em}, \, \sauxi \, )$ are the time and the position of the particle relative to the auxiliary frame, $( \, \mathtt{t}, \, \vec{r}, \, \vec{v}, \, \vec{a} \, )$ are the time, the position, the velocity and the acceleration of the particle relative to the frame S, $( \, \vec{u} \, )$ is the velocity of the auxiliary-point relative to the frame S, $( \, c \, )$ is the speed of light in vacuum, and {\small $\gamma \doteq (\hspace{+0.03em}{1 - \vec{u} \cdot \vec{u}/c^2}\hspace{+0.06em})^{-1/2}$} \hyperlink{p1a1}{(\hspace{+0.120em}see {\small A}nnex {\small I}\hspace{+0.120em})}
\par \bigskip \smallskip
\noindent The auxiliary-point is an arbitrary massive particle free of external forces ( or that the net force acting on it is equal to zero ) The auxiliary frame is an inertial reference frame whose origin always coincides with the auxiliary-point (\hspace{+0.180em}$\vec{u} = 0$ and $\gamma = 1$\hspace{+0.180em})

\newpage

\noindent On the other hand, the relational frequency $( \, \nu \, )$ of a non-massive particle relative to an inertial reference frame S, is given by:
\par \vspace{-0.60em}
\begin{eqnarray}
\nu ~\doteq~ \mathtt{v} \;\, \dfrac{\left ( 1 - \dfrac{\vec{c} \cdot \vec{u}}{c^2} \right )}{\sqrt{1 - \dfrac{\vec{u} \cdot \vec{u}}{c^2}}}
\end{eqnarray}
\par \vspace{+0.45em}
\noindent where $( \, \mathtt{v} \, )$ is the ( ordinary ) frequency of the non-massive particle relative to the frame S, $( \, \vec{c} \, )$ is the velocity of the non-massive particle relative to the frame S, $( \, \vec{u} \, )$ is the velocity of the auxiliary-point relative to the frame S, and $( \, c \, )$ is the speed of light in \hbox {vacuum \hyperlink{p1a2}{(\hspace{+0.120em}see {\small A}nnex {\small II}\hspace{+0.120em})}}

\vspace{+0.90em}

\noindent Note : In arbitrary inertial reference frames ( $t_{\alpha} \ne \tau_{\alpha}$ or \hspace{+0.06em}$\mathbf{r}_{\alpha} \ne 0$ ) \hbox {( $\alpha$ = auxiliary-point )} \hbox {a constant} must be add in the equation of relational time since the relational time and the proper time of the auxiliary-point must be equal \hbox {( $t_{\alpha} = \tau_{\alpha}$ )} and another constant must be add in the equation of relational position since the relational position of the auxiliary-point must be zero \hbox {( $\mathbf{r}_{\alpha} = 0$ )}

\vspace{-0.90em}

\par \bigskip {\centering\subsection*{Intrinsic Mass \& Relativistic Factor}}\addcontentsline{toc}{subsection}{2. Intrinsic Mass \& Relativistic Factor}

\bigskip \smallskip

\noindent The intrinsic mass $( \, m \, )$ and the relativistic factor $( \, f \, )$ of a massive particle, \hbox {are given by}:
\par \vspace{-0.60em}
\begin{eqnarray}
m ~\doteq~ m_o
\end{eqnarray}
\vspace{-0.90em}
\begin{eqnarray}
f ~\doteq~ \Big ( 1 - \dfrac{\mathbf{v} \cdot \mathbf{v}}{c^2} \hspace{+0.15em} \Big )^{\hspace{-0.24em}-\hspace{+0.03em}1/2}
\end{eqnarray}
\par \vspace{+0.60em}
\noindent where $( \, m_o \, )$ is the rest mass of the massive particle, $( \, \mathbf{v} \, )$ is the relational velocity of the massive particle, and $( \, c \, )$ is the speed of light in vacuum.
\par \vspace{+0.60em}
\noindent The intrinsic mass $( \, m \, )$ and the relativistic factor $( \, f \, )$ of a non-massive particle, \hbox {are given by}:
\par \vspace{-0.60em}
\begin{eqnarray}
m ~\doteq~ \dfrac{h \, \kappa}{c^2}
\end{eqnarray}
\vspace{-0.60em}
\begin{eqnarray}
f ~\doteq~ \dfrac{\nu}{\kappa}
\end{eqnarray}
\par \vspace{+0.60em}
\noindent where $( \hspace{+0.33em} h \hspace{+0.33em} )$ is the Planck constant, \hspace{+0.06em}$( \hspace{+0.30em} \nu \hspace{+0.30em} )$ is the relational frequency of the \hbox {non-massive} particle, $( \, \kappa \, )$ is a positive universal constant with dimension of frequency\hspace{-0.03em}, and $( \, c \, )$ is the speed of light in vacuum.
\par \vspace{+0.60em}
\noindent According to this paper, a massive particle is a particle with non-zero rest mass ( or a particle whose speed in vacuum is less than $c$ ) and a non-massive particle is a particle with zero rest mass ( or a particle whose speed in vacuum is $c$ )
\par \vspace{+0.60em}
\noindent Note : The rest mass $( \, m_o \, )$ and the intrinsic mass $( \, m \, )$ are in general not additive, and the relativistic mass $( \, {\mathrm{m}} \, )$ of a particle ( massive or non-massive ) is given by : \hbox {\hspace{-0.09em}( ${\mathrm{m}} ~\doteq~ m \, f$ )}

\newpage

\par \bigskip {\centering\subsection*{The Special Kinematics}}\addcontentsline{toc}{subsection}{3. The Special Kinematics}

\bigskip \smallskip

\noindent The special position $( \, \bar{\mathbf{r}} \, )$, the special velocity $( \, \bar{\mathbf{v}} \, )$ and the special acceleration $( \, \bar{\mathbf{a}} \, )$ of a particle \hbox {( massive or non-massive )} are given by:
\par \vspace{-0.30em}
\begin{eqnarray}
\bar{\mathbf{r}} \hspace{+0.12em}~\doteq~ \int f \, \mathbf{v} \; d\hspace{+0.012em}t
\end{eqnarray}
\vspace{-0.45em}
\begin{eqnarray}
\bar{\mathbf{v}} \hspace{-0.03em}~\doteq~ \dfrac{d\hspace{+0.036em}\bar{\mathbf{r}}}{d\hspace{+0.012em}t} ~=~ f \, \mathbf{v}
\end{eqnarray}
\vspace{-0.30em}
\begin{eqnarray}
\bar{\mathbf{a}} \hspace{+0.03em}~\doteq~ \dfrac{d\hspace{+0.021em}\bar{\mathbf{v}}}{d\hspace{+0.012em}t} ~=~ f \, \dfrac{d\hspace{+0.021em}\mathbf{v}}{d\hspace{+0.012em}t} + \dfrac{d\hspace{-0.12em}f}{d\hspace{+0.012em}t} \, \mathbf{v}
\end{eqnarray}
\par \vspace{+1.20em}
\noindent where $( \, f \, )$ is the relativistic factor of the particle, $( \, \mathbf{v} \, )$ is the relational velocity of the particle, and $( \, t \, )$ is the relational time of the particle.

\vspace{+0.60em}

\par \bigskip {\centering\subsection*{The Special Dynamics}}\addcontentsline{toc}{subsection}{4. The Special Dynamics}

\bigskip \smallskip

\noindent If we consider a particle ( massive or non-massive ) with intrinsic mass $( \, m \, )$ then the linear momentum $( \, \mathbf{P} \, )$ of the particle, the angular momentum $( \, \mathbf{L} \, )$ of the particle, the net force $( \, \mathbf{F} \, )$ acting on the particle, the work $( \, \mathrm{W} \, )$ done by the net force acting on the particle, and the kinetic energy $( \, \mathrm{K} \, )$ of the particle, are given by:
\par \vspace{-0.30em}
\begin{eqnarray}
\mathbf{P} \hspace{-0.06em}~\doteq~ m \, \bar{\mathbf{v}} ~=~ m \, f \, \mathbf{v}
\end{eqnarray}
\vspace{-0.30em}
\begin{eqnarray}
\mathbf{L} \hspace{+0.03em}~\doteq~ \mathbf{r} \times \mathbf{P} ~=~ m \; \mathbf{r} \times \bar{\mathbf{v}} ~=~ m \, f \: \mathbf{r} \times \mathbf{v}
\end{eqnarray}
\vspace{-0.30em}
\hypertarget{p1e15}{}
\begin{eqnarray}
\mathbf{F} ~=~ \dfrac{d\hspace{+0.045em}\mathbf{P}}{d\hspace{+0.012em}t} ~=~ m \, \bar{\mathbf{a}} ~=~ m \, \bigg [ \, f \, \dfrac{d\hspace{+0.021em}\mathbf{v}}{d\hspace{+0.012em}t} + \dfrac{d\hspace{-0.12em}f}{d\hspace{+0.012em}t} \, \mathbf{v} \, \bigg ]
\end{eqnarray}
\vspace{-0.15em}
\begin{eqnarray}
\mathrm{W} \hspace{-0.312em}~\doteq~ \hspace{-0.36em} \int_{\scriptscriptstyle 1}^{\hspace{+0.09em}{\scriptscriptstyle 2}} \mathbf{F} \cdot d\hspace{+0.036em}\mathbf{r} ~=~ \hspace{-0.36em} \int_{\scriptscriptstyle 1}^{\hspace{+0.09em}{\scriptscriptstyle 2}} \dfrac{d\hspace{+0.045em}\mathbf{P}}{d\hspace{+0.012em}t} \cdot d\hspace{+0.036em}\mathbf{r} ~=~ \Delta \, \mathrm{K}
\end{eqnarray}
\vspace{-0.30em}
\begin{eqnarray}
\mathrm{K} \hspace{-0.06em}~\doteq~ m \, f \, c^2
\end{eqnarray}
\par \vspace{+0.90em}
\noindent where $( \: f, \: \mathbf{r}, \: \mathbf{v}, \: t, \: \bar{\mathbf{v}}, \: \bar{\mathbf{a}} \: )$ are the relativistic factor, the relational position, the relational velocity, the relational time, the special velocity and the special acceleration of the particle, and $( \, c \, )$ is the speed of light in vacuum. The kinetic energy $( \, \mathrm{K}_o \, )$ of a massive particle at relational rest is $( \, m_o \, c^2 \, )$ since in this paper the relativistic total energy $( \, \mathrm{E} \,\doteq\, \mathrm{T} + m_o \, c^2 \, )$ and the kinetic energy $( \, \mathrm{K} \,\doteq\, m \, f \, c^2 \, )$ are the same $( \, \mathrm{E} \,=\, \mathrm{K} \, )$
\par \vspace{+0.60em}
\noindent {\small Note :} {\footnotesize $\mathrm{E}^2 - \mathbf{P}^2 c^2 = m^2 \, f^2 \, c^4 \, (\hspace{+0.12em} 1 - \mathbf{v}^2/c^2 \hspace{+0.15em})$} {\small [ in massive particle :} {\footnotesize $f^2 \, (\hspace{+0.12em} 1 - \mathbf{v}^2/c^2 \hspace{+0.15em}) = 1$ \hspace{+0.27em}$\rightarrow$\hspace{+0.27em} \hbox {{$\mathrm{E}^2 - \mathbf{P}^2 c^2 = {m_o}^2 c^4$}} \,and\, $m \ne 0$} {\small ]} \,{\footnotesize\&}\, {\small [ in non-massive particle :} {\footnotesize $\mathbf{v}^2 = c^2$ \hspace{+0.27em}$\rightarrow$\hspace{+0.27em} $(\hspace{+0.12em} 1 - \mathbf{v}^2/c^2 \hspace{+0.15em}) = 0$ \hspace{+0.27em}$\rightarrow$\hspace{+0.27em} \hbox {{$\mathrm{E}^2 - \mathbf{P}^2 c^2 = 0$ \:and\: $m \ne 0$} {\small ]}}}

\newpage

\par \bigskip {\centering\subsection*{General Observations}}\addcontentsline{toc}{subsection}{5. General Observations}

\bigskip \smallskip

\noindent According to this paper, in the auxiliary reference frame relational magnitudes and ordinary magnitudes are always the same.
\par \bigskip \smallskip
\noindent The special acceleration $\bar{\mathbf{a}}$ of a particle (\hspace{+0.12em}massive or non-massive\hspace{+0.12em}) is always in the direction of the net force $\mathbf{F}$ acting on the particle (\hspace{+0.12em}as in Newton's second law\hspace{+0.12em})
\par \bigskip \smallskip
\noindent Finally,
\par \bigskip \smallskip
\noindent The intrinsic mass magnitude {\small $( \: m \: )$} is invariant under transformations between inertial and non-inertial reference frames.
\par \bigskip \smallskip
\noindent The relational magnitudes {\small $( \: \nu, t, \mathbf{r}, \mathbf{v}, \mathbf{a} \: )$} are invariant under transformations between inertial and non-inertial reference frames (\hspace{+0.12em}since these relational magnitudes are the proper (\hspace{+0.06em}own\hspace{+0.06em}) ordinary magnitudes of the auxiliary reference frame\hspace{+0.12em})
\par \bigskip \smallskip
\noindent Therefore, the kinematic and dynamic magnitudes {\small $( \: f, \bar{\mathbf{r}}, \bar{\mathbf{v}}, \bar{\mathbf{a}}, \mathbf{P}, \mathbf{L}, \mathbf{F}, \mathrm{W}, \mathrm{K} \: )$} are also invariant under transformations between inertial and non-inertial reference frames.
\par \bigskip \smallskip
\noindent The special dynamics can be applied in any inertial or non-inertial reference frame (\hspace{+0.12em}and non-inertial observers must not introduce fictitious forces into $\mathbf{F}$\hspace{+0.12em})
\par \bigskip \smallskip
\noindent Therefore, the reformulation of special relativity presented in this paper is in accordance with the general principle of relativity.
\par \bigskip \smallskip
\noindent However,
\par \bigskip \smallskip
\noindent In this paper, starting from an auxiliary massive particle, we only obtained the form that the relational magnitudes {\small $( \: \nu, t, \mathbf{r}, \mathbf{v}, \mathbf{a} \: )$} have in any inertial reference frame.
\par \bigskip \smallskip
\noindent Consequently,
\par \bigskip \smallskip
\noindent Starting from an auxiliary massive particle, we must obtain the form that the relational magnitudes {\small $( \: \nu, t, \mathbf{r}, \mathbf{v}, \mathbf{a} \: )$} have in any non-rotating reference frame (\hspace{+0.12em}non-rotating reference frame relative to an inertial reference frame\hspace{+0.12em})
\par \bigskip \smallskip
\noindent Later, starting from an auxiliary system of massive particles, we must obtain the form that the relational magnitudes {\small $( \: \nu, t, \mathbf{r}, \mathbf{v}, \mathbf{a} \: )$} have in any rotating reference frame (\hspace{+0.12em}rotating reference frame relative to an inertial reference frame\hspace{+0.12em})

\vspace{+0.60em}

\par \bigskip {\centering\subsection*{Bibliography}}\addcontentsline{toc}{subsection}{6. Bibliography}

\bigskip \smallskip

\par \noindent \textbf{W\hspace{-0.18em}. Pauli}, Theory of Relativity.
\bigskip \smallskip
\par \noindent \textbf{C. M{\o}ller}, The Theory of Relativity.
\bigskip \smallskip
\par \noindent \textbf{A. Blato}, Vector Lorentz Transformations.

\newpage

\par \bigskip {\centering\subsection*{Annex I}}\hypertarget{p1a1}{}

\par \bigskip {\centering\subsection*{Vector Lorentz Transformations}}\addcontentsline{toc}{subsection}{Annex I : Vector Lorentz Transformations}

\bigskip \smallskip

\noindent If we consider two inertial reference frames ( S and S' ) whose origins coincide at time zero \hbox {( in both frames )} then the time $( \, \mathtt{t}\hspace{+0.03em}' \, )$, the position $( \, \vec{r}\hspace{+0.18em}' \, )$, the velocity $( \, \vec{v}\hspace{+0.18em}' \, )$ and the acceleration $( \, \vec{a}\hspace{+0.18em}' \, )$ of a particle ( massive or non-massive ) relative to the inertial reference frame S', are given by:
\par \vspace{+0.21em}
\begin{eqnarray}
\mathtt{t}\hspace{+0.03em}' \hspace{-0.03em}~=~ \hspace{+0.09em} \gamma \left ( \mathtt{t} - \frac{\vec{r} \cdot \vec{u}}{c^2} \right )
\end{eqnarray}
\vspace{-0.45em}
\begin{eqnarray}
\vec{r}\hspace{+0.21em}' \hspace{-0.18em}~=~ \hspace{-0.12em} \left [ \; \vec{r} + \frac{\gamma^2}{\gamma + 1} \frac{( \vec{r} \cdot \vec{u} \hspace{+0.09em}) \, \vec{u}}{c^2} - \gamma \, \vec{u} \; \mathtt{t} \; \right ]
\end{eqnarray}
\vspace{-0.30em}
\begin{eqnarray}
\vec{v}\hspace{+0.21em}' \hspace{-0.21em}~=~ \hspace{-0.12em} \left [ \; \vec{v} + \frac{\gamma^2}{\gamma + 1} \frac{( \vec{v} \cdot \vec{u} \hspace{+0.09em}) \, \vec{u}}{c^2} - \hspace{+0.114em} \gamma \, \vec{u} \hspace{+0.114em} \; \right ] \frac{1}{\gamma \, ( 1 - \frac{\vec{v} \cdot \vec{u}}{c^2} )}
\end{eqnarray}
\vspace{-0.30em}
\begin{eqnarray}
\vec{a}\hspace{+0.21em}' \hspace{-0.21em}~=~ \hspace{-0.12em} \left [ \; \vec{a} - \frac{\gamma}{\gamma + 1} \frac{( \vec{a} \cdot \vec{u} \hspace{+0.09em}) \, \vec{u}}{c^2} + \frac{( \vec{a} \times \vec{v} \hspace{+0.09em}) \times \vec{u}}{c^2} \; \right ] \frac{1}{\gamma^2 \, ( 1 - \frac{\vec{v} \cdot \vec{u}}{c^2} )^3}
\end{eqnarray}
\par \vspace{+1.20em}
\noindent where $( \, \mathtt{t}, \, \vec{r}, \, \vec{v}, \, \vec{a} \, )$ are the time, the position, the velocity and the acceleration of the particle relative to the frame S, $( \, \vec{u} \, )$ is the velocity of the frame S' relative to the frame S, $( \, c \, )$ is the speed of light in vacuum, and {\small $\gamma \doteq (\hspace{+0.03em}{1 - \vec{u} \hspace{-0.114em}\cdot\hspace{-0.114em} \vec{u}/c^2}\hspace{+0.06em})^{-1/2}$} \hfill \hbox {{\small Note : $\gamma^2 / (\gamma + 1) \, c^2 = (\gamma - 1) /\hspace{+0.03em} \vec{u}^{\hspace{+0.09em}2}$ \,$\leftarrow$\, $\vec{u}\ne 0$}}

\vspace{+0.60em}

\par \bigskip {\centering\subsection*{Annex II}}\hypertarget{p1a2}{}

\par \bigskip {\centering\subsection*{Transformation of Frequency}}\addcontentsline{toc}{subsection}{Annex II : Transformation of Frequency}

\bigskip \smallskip

\noindent On the other hand, the frequency $(\hspace{+0.24em}\mathtt{v}\hspace{+0.15em}'\hspace{+0.12em})$ of a non-massive particle relative to an inertial reference frame S', is given by:
\par \vspace{-0.60em}
\begin{eqnarray}
\mathtt{v}\hspace{+0.15em}' ~=~ \: \mathtt{v} \;\, \dfrac{\left ( 1 - \dfrac{\vec{c} \cdot \vec{u}}{c^2} \right )}{\sqrt{1 - \dfrac{\vec{u} \cdot \vec{u}}{c^2}}}
\end{eqnarray}
\par \vspace{+0.45em}
\noindent where $( \: \mathtt{v} \: )$ is the frequency of the non-massive particle relative to another inertial reference frame S, $( \: \vec{c} \: )$ is the velocity of the non-massive particle relative to the frame S, $( \, \vec{u} \, )$ is the velocity of the frame S' relative to the frame S, and $( \, c \, )$ is the speed of light in vacuum.

\newpage

\par \bigskip {\centering\subsection*{Annex III}}

\par \bigskip {\centering\subsection*{The Kinetic Forces}}\addcontentsline{toc}{subsection}{Annex III : The Kinetic Forces}

\bigskip \smallskip

\noindent The kinetic force \hbox {$\mathbf{K}^{a}_{\hspace{+0.012em}ij}$ exerted} on a particle $i$ with intrinsic mass $m_i$ by another particle $j$ with intrinsic mass $m_j$, \hbox {is given by}:
\par \vspace{-0.54em}
\begin{eqnarray}
\mathbf{K}^{a}_{\hspace{+0.012em}ij} \,=\, - \; \Bigg [ \; \dfrac{m_i \, m_j}{\mathbb{M}} \, ( \, \bar{\mathbf{a}}_{\hspace{+0.045em}i} \hspace{+0.045em}-\, \bar{\mathbf{a}}_{j} \, ) \; \Bigg ]
\end{eqnarray}
\par \vspace{+0.60em}
\noindent where $\bar{\mathbf{a}}_{\hspace{+0.045em}i}$ is the special acceleration of particle $i$, $\bar{\mathbf{a}}_{j}$ is the special acceleration of particle $j$ and $\mathbb{M}$ {\small ( $ = \sum_z^{\scriptscriptstyle{\mathit{All}}} m_z$ )} is the sum of the intrinsic masses of all the particles of the Universe.
\par \vspace{+0.60em}
\noindent On the other hand, the kinetic force $\mathbf{K}^{u}_{\hspace{+0.030em}i}$ exerted on a particle $i$ with intrinsic mass $m_i$ by the Universe, is given by:
\par \vspace{-0.45em}
\begin{eqnarray}
\mathbf{K}^{u}_{\hspace{+0.030em}i} \,=\, - \; m_i \; \dfrac{\sum_z^{\scriptscriptstyle{\mathit{All}}} m_z \, \bar{\mathbf{a}}_{\hspace{+0.045em}z}}{\sum_z^{\scriptscriptstyle{\mathit{All}}} m_z}
\end{eqnarray}
\par \vspace{+0.60em}
\noindent where $m_z$ and $\bar{\mathbf{a}}_{\hspace{+0.045em}z}$ are the intrinsic mass and the special acceleration of the \textit{z}-th particle of the Universe.
\par \vspace{+0.60em}
\noindent From the above equations it follows that the net kinetic force $\mathbf{K}_i$ {\small ( $ = \sum_j^{\scriptscriptstyle{\mathit{All}}} \, \mathbf{K}^{a}_{\hspace{+0.012em}ij}$} {\small $+ \; \mathbf{K}^{u}_{\hspace{+0.030em}i}$ )} acting on a particle $i$ with intrinsic mass $m_i$, is given by:
\par \vspace{-0.60em}
\begin{eqnarray}
\mathbf{K}_i \,=\, - \, m_i \, \bar{\mathbf{a}}_{\hspace{+0.045em}i}
\end{eqnarray}
\par \vspace{+0.60em}
\noindent where $\bar{\mathbf{a}}_{\hspace{+0.045em}i}$ is the special acceleration of particle $i$.
\par \vspace{+0.60em}
\noindent Now, from the special dynamics \hyperlink{p1e15}{(15)}, we have:
\par \vspace{-0.60em}
\begin{eqnarray}
\mathbf{F}_i \,=\, m_i \, \bar{\mathbf{a}}_{\hspace{+0.045em}i}
\end{eqnarray}
\par \vspace{+0.60em}
\noindent Since {\small (\hspace{+0.240em}$\mathbf{K}_i \,=\, - \, m_i \, \bar{\mathbf{a}}_{\hspace{+0.045em}i}$\hspace{+0.240em})} we obtain:
\par \vspace{-0.81em}
\begin{eqnarray}
\mathbf{F}_i \,=\, - \, \mathbf{K}_i
\end{eqnarray}
\par \vspace{+0.30em}
\noindent that is:
\par \vspace{-0.81em}
\begin{eqnarray}
\mathbf{K}_i \, + \, \mathbf{F}_i \,=\, 0
\end{eqnarray}
\par \vspace{+0.60em}
\noindent If {\small (\hspace{+0.240em}$\mathbf{T}_i \,\doteq\, \mathbf{K}_i \, + \, \mathbf{F}_i$\hspace{+0.240em})} then:
\par \vspace{-0.81em}
\begin{eqnarray}
\mathbf{T}_i \,=\, 0
\end{eqnarray}
\par \vspace{+0.60em}
\noindent Therefore, if the net kinetic force $\mathbf{K}_i$ is added in the special dynamics then the total \hbox {force $\mathbf{T}_i$} acting on a ( massive or non-massive ) particle $i$ is always zero.
\par \vspace{+0.21em}
\noindent Note : According to this paper, the kinetic forces ${\stackrel{\scriptstyle{au}}{\smash{\mathbf{K}}\rule{0pt}{+0.63em}}}$ are directly related to kinetic energy $\mathrm{K}$.

\newpage

\enlargethispage{+0.00em}

\setlength{\unitlength}{1pt}

\par \bigskip {\centering\subsection*{Annex IV}}

\vspace{-0.03em}

\par \bigskip {\centering\subsection*{Special Relativity}}\addcontentsline{toc}{subsection}{Annex IV : Special Relativity}

\vspace{+1.50em}

\newcommand{\cafa}{\textsf{A Reformulation with\vspace{+0.30em}\\Relational Magnitudes}}
\newcommand{\cafb}{\textsf{A Reformulation with\vspace{+0.30em}\\Ordinary Magnitudes}}
\newcommand{\cbfa}{\textsf{\vspace{-0.48em}\\$\mathit{\nu}$ , $\mathit{t}$ , $\mathbf{r}$ , $\mathbf{v}$ , $\mathbf{a}$\vspace{+0.27em}\\( invariant )}}
\newcommand{\cbfb}{\textsf{$\mathtt{v}$ , $\mathtt{t}$ , \hspace{-0.06em}$\vec{\hspace{+0.06em}\mathit{r}}$ , \hspace{-0.06em}$\vec{\hspace{+0.06em}\mathit{v}}$ , \hspace{-0.06em}$\vec{\hspace{+0.06em}\mathit{a}}$\vspace{+0.27em}\\( not invariant )}}
\newcommand{\ccfa}{\textsf{$\mathit{m_o}$ , $\mathit{m}$\vspace{+0.30em}\\( invariant )\vspace{+0.30em}\\$\mathit{f}$ , $\mathrm{m}$\vspace{+0.30em}\\( invariant )}}
\newcommand{\ccfb}{\textsf{$\mathit{m_o}$ , $\mathit{m}$\vspace{+0.30em}\\( invariant )\vspace{+0.30em}\\$\mathrm{f}$ , $\mathtt{m}$\vspace{+0.30em}\\( not invariant )}}
\newcommand{\cdfa}{\textsf{\vspace{-0.33em}\\$\bar{\mathbf{r}}$ , $\bar{\mathbf{v}}$ , $\bar{\mathbf{a}}$\vspace{+0.27em}\\( invariant )}}
\newcommand{\cdfb}{\textsf{$\vec{\hspace{+0.09em}\bar{\hspace{-0.09em}{\mathit{r}}}}$ , $\vec{\hspace{+0.09em}\bar{\hspace{-0.09em}{\mathit{v}}}}$ , $\vec{\hspace{+0.09em}\bar{\hspace{-0.09em}{\mathit{a}}}}$\vspace{+0.27em}\\( not invariant )}}
\newcommand{\cefa}{\textsf{\vspace{-0.24em}\\$\mathbf{P}$ , $\mathbf{L}$ , $\mathbf{F}$ , $\mathrm{W}$ , $\mathrm{K}$ , ($\mathbf{K}$)\vspace{+0.27em}\\( invariant )}}
\newcommand{\cefb}{\textsf{\hspace{-0.21em}$\vec{\hspace{+0.21em}\mathrm{P}}$ , \hspace{-0.30em}$\vec{\hspace{+0.30em}\mathrm{L}}$ , \hspace{-0.15em}$\vec{\hspace{+0.15em}\mathrm{F}}$ , $\mathit{W}$ , $\mathit{K}$ , (\hspace{-0.15em}$\vec{\hspace{+0.15em}\mathrm{K}}$)\vspace{+0.27em}\\( not invariant )}}
\newcommand{\cffa}{\textsf{The (\hspace{+0.12em}relational\hspace{+0.12em}) special dynamics can be applied in any inertial or non-inertial reference frame (\hspace{+0.12em}and non-inertial observers must {\footnotesize \textbf{NOT}} introduce fictitious forces \hbox {into $\mathbf{F}$\hspace{+0.12em})}}}
\newcommand{\cffb}{\textsf{The (\hspace{+0.12em}ordinary\hspace{+0.12em}) special dynamics can be applied in any inertial or non-inertial reference frame (\hspace{+0.12em}and non-inertial observers must introduce fictitious forces \hbox {into \hspace{-0.15em}$\vec{\hspace{+0.15em}\mathrm{F}}$\hspace{+0.12em})}}}
\newcommand{\cgfa}{\textsf{\textbf{invariant} = The magnitudes are invariant under transformations between inertial and non-inertial reference frames.}}
\newcommand{\cgfb}{\textsf{\textbf{not invariant} = The magnitudes are not invariant under transformations between inertial reference frames.}}
\newcommand{\czid}{\textsf{{\small $=$}\vspace{+1.80em}\\{\small $\ne$}}}

\begin{center}
\begin{tabular}{ccc}
{\framebox(154,45){\parbox{154\unitlength}{\centering \cafa }}} &
{\makebox(10,45){\textbf{{\small $\ne$}}}} &
{\framebox(154,45){\parbox{154\unitlength}{\centering \cafb }}} \\
{\makebox(154,15){$\downarrow$}} & & {\makebox(154,15){$\downarrow$}} \\
{\framebox(154,36){\parbox{154\unitlength}{\centering \cbfa }}} &
{\makebox(10,36){\textbf{{\small $\ne$}}}} &
{\framebox(154,36){\parbox{154\unitlength}{\centering \cbfb }}} \\
{\makebox(154,15){$\downarrow$}} & & {\makebox(154,15){$\downarrow$}} \\
{\framebox(154,63){\parbox{154\unitlength}{\centering \ccfa }}} &
{\makebox(10,63){\parbox{10\unitlength}{\centering \czid }}} &
{\framebox(154,63){\parbox{154\unitlength}{\centering \ccfb }}} \\
{\makebox(154,15){$\downarrow$}} & & {\makebox(154,15){$\downarrow$}} \\
{\framebox(154,36){\parbox{154\unitlength}{\centering \cdfa }}} &
{\makebox(10,36){\textbf{{\small $\ne$}}}} &
{\framebox(154,36){\parbox{154\unitlength}{\centering \cdfb }}} \\
{\makebox(154,15){$\downarrow$}} & & {\makebox(154,15){$\downarrow$}} \\
{\framebox(154,42){\parbox{154\unitlength}{\centering \cefa }}} &
{\makebox(10,42){\textbf{{\small $\ne$}}}} &
{\framebox(154,42){\parbox{154\unitlength}{\centering \cefb }}} \\
{\makebox(154,15){$\downarrow$}} & & {\makebox(154,15){$\downarrow$}} \\
{\framebox(154,78){\parbox{144\unitlength}{\centering \cffa }}} &
{\makebox(10,78){\textbf{{\small $\ne$}}}} &
{\framebox(154,78){\parbox{144\unitlength}{\centering \cffb }}} \\
{\makebox(154,15){$\downarrow$}} & & {\makebox(154,15){$\downarrow$}}
\end{tabular}
{\framebox(342,66){\parbox{306\unitlength}{ \cgfa \vspace{+0.45em} \\ \cgfb }}}
\end{center}

\end{document}

