
\documentclass[10pt]{article}
%\documentclass[a4paper,10pt]{article}
%\documentclass[letterpaper,10pt]{article}

\usepackage[dvips]{geometry}
\geometry{papersize={162.0mm,234.0mm}}
\geometry{totalwidth=141.0mm,totalheight=198.0mm}

\usepackage[english]{babel}
\usepackage[latin1]{inputenc}
\usepackage{amsfonts}
\usepackage{amsmath,bm}

\usepackage{hyperref}
\hypersetup{colorlinks=true,linkcolor=black,urlcolor=blue,bookmarksopen=true}
\hypersetup{bookmarksnumbered=true,pdfstartview=FitH,pdfpagemode=UseNone}
\hypersetup{pdftitle={On Relational Mechanics}}
\hypersetup{pdfauthor={Agust�n A. Tobla}}

\setlength{\arraycolsep}{1.74pt}

\newcommand{\med}{\raise.5ex\hbox{$\scriptstyle 1$}\kern-.15em/\kern-.09em\lower.25ex\hbox{$\scriptstyle 2$}}
\newcommand{\met}{\raise.5ex\hbox{$\scriptstyle 1$}\kern-.15em/\kern-.09em\lower.25ex\hbox{$\scriptstyle 2$}}
\newcommand{\dix}{\rule[1mm]{4.2mm}{0.12mm} Differential Equations \rule[1mm]{4.2mm}{0.12mm}}
\newcommand{\diy}{$\nabla \cdot {\mathbf{E}} = 2 \: {\vec{\omega}}^{\hspace{+0.15em}2}$ \hspace{+1.78em},\hspace{+0.17em} $\nabla \cdot {\mathbf{B}} = 0$}
\newcommand{\diz}{$\nabla \hspace{-0.24em}\times\hspace{-0.24em} {\mathbf{E}} = - \partial{\mathbf{B}}/\partial{\hspace{+0.03em}t}$ \hspace{+0.09em},\hspace{+0.15em} $\nabla \hspace{-0.24em}\times\hspace{-0.24em} {\mathbf{B}} = 0$}

\begin{document}

\enlargethispage{+0.00em}

\begin{center}

{\Large ON RELATIONAL MECHANICS}

\bigskip \medskip

{\large Agust�n A. Tobla}

\bigskip \medskip

\small

Creative Commons Attribution 3.0 License

\smallskip

(2024) Buenos Aires

\medskip

Argentina

\smallskip

\bigskip

\parbox{107.40mm}{In classical mechanics, a new reformulation is presented, which is totally in accordance with the general principle of relativity, which is invariant under transformations between inertial and non-inertial reference frames, which can be applied in any reference frame without introducing fictitious forces, which is observationally equivalent to Newtonian mechanics, and which establishes the existence of a new universal force of interaction, called kinetic force \hbox {( which is related} to the force of inertia $-m{\mathbf{a}}$ , and also to \hbox {Mach's principle )}\vspace{+0.60em}\\{\footnotesize Kws : Relational Mechanics � Mach's Principle � Classical Mechanics � Kinetic Force}}

\end{center}

\normalsize

\vspace{-1.20em}

\par \medskip {\centering\subsection*{Introduction}}\addcontentsline{toc}{subsection}{1. Introduction}

\par \medskip \noindent The new reformulation in classical mechanics presented in this paper is obtained starting from an auxiliary system of particles (called Universe) that is used to obtain kinematic magnitudes (such as universal position, universal velocity, etc.) that are invariant under transformations between inertial and non-inertial reference frames.

\par \medskip \noindent The universal position ${\mathbf{r}}_i$, the universal velocity ${\mathbf{v}}_i$ and the universal acceleration ${\mathbf{a}}_{\hspace{+0.045em}i}$ of a \hbox {particle $i$} relative to a reference frame S (\hspace{+0.120em}inertial or non-inertial\hspace{+0.120em}) are given by:

\par \medskip\vspace{+0.06em} ${\mathbf{r}}_i \,\:\doteq\; (\hspace{+0.090em}{\stackrel{\scriptscriptstyle\sim}{\smash{r}\rule{0pt}{+0.30em}}}_{\hspace{-0.12em}i}\hspace{+0.090em}) \;=\; ({\vec{\mathit{r}}}_i - {\vec{\mathit{R}}})$

\par \medskip\vspace{+0.36em} ${\mathbf{v}}_i \;\doteq\; d\hspace{+0.090em}(\hspace{+0.090em}{\stackrel{\scriptscriptstyle\sim}{\smash{r}\rule{0pt}{+0.30em}}}_{\hspace{-0.12em}i}\hspace{+0.090em})\hspace{+0.045em}/dt \;=\; ({\vec{\mathit{v}}}_i - \hspace{-0.120em}{\vec{\mathit{V}}}) - {\vec{\omega}} \times ({\vec{\mathit{r}}}_i - {\vec{\mathit{R}}})$

\par \medskip\vspace{+0.36em} ${\mathbf{a}}_{\hspace{+0.045em}i} \;\doteq\; d^2\hspace{+0.030em}(\hspace{+0.090em}{\stackrel{\scriptscriptstyle\sim}{\smash{r}\rule{0pt}{+0.30em}}}_{\hspace{-0.12em}i}\hspace{+0.090em})\hspace{+0.045em}/dt^2 \:=\; ({\vec{\mathit{a}}}_i - {\vec{\mathit{A}}}) - 2 \; {\vec{\omega}} \times ({\vec{\mathit{v}}}_i - \hspace{-0.120em}{\vec{\mathit{V}}}) + {\vec{\omega}} \times [ \, {\vec{\omega}} \times ({\vec{\mathit{r}}}_i - {\vec{\mathit{R}}}) \, ] - {\vec{\alpha}} \times ({\vec{\mathit{r}}}_i - {\vec{\mathit{R}}})$

\par \medskip\vspace{+0.36em} \noindent where ${\stackrel{\scriptscriptstyle\sim}{\smash{r}\rule{0pt}{+0.30em}}}_{\hspace{-0.12em}i}$ is the position vector of particle $i$ relative to the universal frame [\hspace{+0.120em}${\vec{\mathit{r}}}_i$ is the position vector of particle $i$, ${\vec{\mathit{R}}}$ is the position vector of the center of mass of the Universe, and ${\vec{\omega}}$ is the angular velocity vector of the Universe\hspace{+0.120em}] [\hspace{+0.120em}relative to the frame S\hspace{+0.120em}] \hyperlink{p5a1}{(\hspace{+0.120em}see {\small A}nnex {\small I}\hspace{+0.120em})}

\par \medskip\vspace{+0.00em} \noindent The universal frame is a reference frame fixed to the Universe (\hspace{+0.120em}${\vec{\omega}} = 0$\hspace{+0.120em}) whose origin always coincides with the center of mass of the Universe (\hspace{+0.033em}{\small ${\vec{\mathit{R}}} = {\vec{\mathit{V}}} = {\vec{\mathit{A}}} =$} $0$\hspace{+0.120em})

\par \medskip \noindent Any reference frame S is an inertial frame when the angular velocity ${\vec{\omega}}$ of the Universe and the acceleration ${\vec{\mathit{A}}}$ of the center of mass of the Universe are equal to zero \hbox {relative to S.}

\par \medskip \noindent Note\hspace{+0.300em}:\hspace{+0.240em}\hbox{( $\forall \;\; {\mathbf{m}} \, \in$\hspace{+0.12em} Universal Magnitudes \hspace{+0.27em}:\hspace{+0.27em} If \hspace{+0.42em}${\mathbf{m}} \,=\, {\vec{\mathit{n}}}$ \hspace{+0.54em}$\longrightarrow$\hspace{+0.54em} $d(\hspace{+0.090em}{\mathbf{m}}\hspace{+0.090em})/dt \,=\, d(\hspace{+0.090em}{\vec{\mathit{n}}}\hspace{+0.090em})/dt \,-\, {\vec{\omega}} \times {\vec{\mathit{n}}}$ )}

\vspace{+0.90em}

\par {\centering\subsection*{The New Dynamics}}\addcontentsline{toc}{subsection}{2. The New Dynamics}

\par \medskip \noindent $[\,1\,]$ A force is always caused by the interaction between two or more particles.

\par \medskip \noindent $[\,2\,]$ The total force ${\mathbf{T}}_i$ acting on a particle $i$ is always equal to zero : $[ \; {\mathbf{T}}_i \,=\, 0 \; ]$

\par \medskip \noindent $[\,3\,]$ In this paper, we assume that all dynamic forces ( that is, all non-kinetic forces ) always obey Newton's third law in its weak form and in its strong form.

\newpage

\par \bigskip {\centering\subsection*{The Kinetic Force}}\addcontentsline{toc}{subsection}{3. The Kinetic Force}

\par \bigskip \noindent The kinetic force ${\mathbf{K}}_{\hspace{+0.060em}ij}$ exerted on a particle $i$ of mass $m_i$ by another particle $j$ of mass $m_j$, caused by the interaction between particle $i$ and particle $j$, is given by:

\par \bigskip ${\mathbf{K}}_{\hspace{+0.060em}ij} ~=~ - \; \dfrac{m_i \, m_j}{\mathit{M}} \, (\hspace{+0.045em}{\mathbf{a}}_{\hspace{+0.045em}i} - {\mathbf{a}}_{j})$

\par \bigskip \noindent where ${\mathbf{a}}_{\hspace{+0.045em}i}$ is the universal acceleration of particle $i$, ${\mathbf{a}}_{j}$ is the universal acceleration of \hbox {particle $j$}, and ${\mathit{M}}$ {\small \hspace{-0.120em}(\hspace{+0.120em}$ = \sum_i^{\scriptscriptstyle{\mathit{All}}} \, m_i$\hspace{+0.120em})} is the mass of the Universe.

\par \bigskip \noindent From the above equation it follows that the net kinetic force ${\mathbf{K}}_{i}$ {\small (\hspace{+0.120em}$ = \sum_{j}^{\scriptscriptstyle{\mathit{All}}} \, {\mathbf{K}}_{\hspace{+0.060em}ij}$\hspace{+0.120em})} acting on a particle $i$ of mass $m_i$ is given by:

\par \bigskip ${\mathbf{K}}_{i} ~=~ - \; m_i \, (\hspace{+0.045em}{\mathbf{a}}_{\hspace{+0.045em}i} - {\mathbf{A}})$

\par \bigskip \noindent where ${\mathbf{a}}_{\hspace{+0.045em}i}$ is the universal acceleration of particle $i$ and ${\mathbf{A}}$ {\small \hspace{-0.120em}(\hspace{+0.120em}$ = {\mathit{M}}^{\scriptscriptstyle -1} \sum_i^{\scriptscriptstyle{\mathit{All}}} \, m_i \, {\mathbf{a}}_{\hspace{+0.045em}i}$\hspace{+0.120em})} is the universal acceleration of the center of mass of the Universe.

\par \bigskip \noindent Since the universal acceleration of the center of mass of the Universe ${\mathbf{A}}$ is always zero, then the net kinetic force ${\mathbf{K}}_{i}$ acting on a particle $i$ of mass $m_i$ is certainly given by:

\par \bigskip ${\mathbf{K}}_{i} ~=~ - \; m_i \, {\mathbf{a}}_{\hspace{+0.045em}i}$

\par \bigskip \noindent where ${\mathbf{a}}_{\hspace{+0.045em}i}$ is the universal acceleration of particle $i$.

\par \bigskip \noindent The net kinetic force ${\mathbf{K}}_{i}$ is related to the force of inertia $-m{\mathbf{a}}$ ( vis insita ) and the kinetic force ${\mathbf{K}}_{\hspace{+0.060em}ij}$ ( as the origin of $-m{\mathbf{a}}$ ) is related to Mach's principle.

\par \bigskip \noindent The force ${\mathbf{K}}_{i}$ is the force that balances the net dynamic force in each particle of the Universe and the force ${\mathbf{K}}_{\hspace{+0.060em}ij}$ always obey Newton's third law in its strong form or in its weak form.

\par \bigskip \noindent On fields and potentials of the forces ${\mathbf{K}}_{i}$ and ${\mathbf{K}}_{\hspace{+0.060em}ij}$ see : \hyperlink{p5aA}{{\small A}nnex {\small A}} and \hyperlink{p5aB}{{\small A}nnex {\small B}}. The force ${\mathbf{K}}_{\hspace{+0.060em}ij}$ \hbox {is obtained} starting from Newtonian mechanics in : {\small \url{https://doi.org/10.5281/zenodo.1215207}}

\vspace{+1.20em}

\par {\centering\subsection*{The [$\,$2$\,$] Principle}}\addcontentsline{toc}{subsection}{4. The Second Principle}

\par \bigskip \noindent The second principle of the new dynamics establishes that the total force ${\mathbf{T}}_i$ acting on a \hbox {particle $i$} is always equal to zero.

\par \bigskip ${\mathbf{T}}_i ~=~ 0$

\par \bigskip \noindent If the total force ${\mathbf{T}}_i$ is divided into the following two parts: the net kinetic force ${\mathbf{K}}_{i}$ and the net dynamic force ${\mathbf{F}}_{i}$ (\hspace{+0.180em}$\sum$ of gravitational forces, electrostatic forces, etc.\hspace{+0.180em}) then we \hbox{have\hspace{+0.060em}:}

\par \bigskip ${\mathbf{K}}_{i} + {\mathbf{F}}_{i} ~=~ 0$

\par \bigskip \noindent Now, substituting ${\mathbf{K}}_{i}$ {\normalsize \hspace{-0.120em}(\hspace{+0.120em}$ = - \; m_i \, {\mathbf{a}}_{\hspace{+0.045em}i}$\hspace{+0.120em})} and rearranging, we finally obtain:

\par \bigskip ${\mathbf{F}}_{i} ~=~ m_i \, {\mathbf{a}}_{\hspace{+0.045em}i}$

\par \bigskip \noindent This equation (\hspace{+0.180em}similar to Newton's second law\hspace{+0.180em}) will be used throughout this paper.

\newpage

\par \bigskip {\centering\subsection*{The Equation of Motion}}\addcontentsline{toc}{subsection}{5. The Equation of Motion}

\par \bigskip \noindent The net dynamic force ${\mathbf{F}}_i$ acting on a particle $i$ of mass $m_i$ is related to the universal acceleration ${\mathbf{a}}_{\hspace{+0.045em}i}$ of particle $i$ according to the following equation:

\par \bigskip ${\mathbf{F}}_i \,=\, m_i \, {\mathbf{a}}_{\hspace{+0.045em}i}$

\par \bigskip \noindent From the above equation it follows that the (\hspace{+0.120em}ordinary\hspace{+0.120em}) acceleration ${\vec{\mathit{a}}}_{i}$ of particle $i$ relative to a reference frame S (\hspace{+0.120em}inertial or non-inertial\hspace{+0.120em}) is given by:

\par \bigskip ${\vec{\mathit{a}}}_{i} \;=\: {\mathbf{F}}_{\hspace{-0.09em}i}/m_i + {\vec{\mathit{A}}} \hspace{+0.15em} + 2 \; {\vec{\omega}} \times ({\vec{\mathit{v}}}_{i} - {\vec{\mathit{V}}}) - {\vec{\omega}} \times [ \, {\vec{\omega}} \times ({\vec{\mathit{r}}}_{i} - {\vec{\mathit{R}}}) \, ] + {\vec{\alpha}} \times ({\vec{\mathit{r}}}_{i} - {\vec{\mathit{R}}})$

\par \bigskip \noindent where ${\vec{\mathit{r}}}_i$ is the position vector of particle $i$, ${\vec{\mathit{R}}}$ is the position vector of the center of mass of the Universe, and ${\vec{\omega}}$ is the angular velocity vector of the Universe relative to S \hbox {\hyperlink{p5a1}{(\hspace{+0.120em}see {\small A}nnex {\small I}\hspace{+0.120em})}}

\par \bigskip \noindent From the above equation it follows that particle $i$ can have a non-zero \hbox {acceleration (\hspace{+0.120em}${\vec{\mathit{a}}}_{i} \ne 0$\hspace{+0.120em})} even if there is no dynamic force acting on particle $i$, and also that particle $i$ can have zero acceleration (\hspace{+0.120em}${\vec{\mathit{a}}}_{i} = 0$\hspace{+0.120em}) \hbox {(\hspace{+0.120em}state of} rest or of uniform linear motion\hspace{+0.120em}) even if there is an unbalanced net dynamic force acting \hbox {on particle $i$.}

\par \bigskip \noindent However, from the above equation it also follows that Newton's first and second laws are valid in any inertial reference frame, since the angular velocity ${\vec{\omega}}$ of the Universe and the acceleration ${\vec{\mathit{A}}}$ of the center of mass of the Universe are equal to zero relative to any inertial reference frame.

\par \bigskip \noindent In this paper, any reference frame S is an inertial frame when the angular velocity ${\vec{\omega}}$ of the Universe and the acceleration ${\vec{\mathit{A}}}$ of the center of mass of the Universe are equal to zero relative to the frame S (\hspace{+0.120em}a Machian definition of inertial frame\hspace{+0.120em})

\par \bigskip \noindent On the other hand, the new reformulation of classical mechanics presented in this paper is observationally equivalent to Newtonian mechanics.

\par \bigskip \noindent However, non-inertial observers can use Newtonian mechanics only if they introduce fictitious forces into ${\mathbf{F}}_{\hspace{-0.09em}i}$ (\hspace{+0.120em}such as the centrifugal force, the Coriolis force, etc.\hspace{+0.120em}) From the above equation it follows that \,:\, {\small ${\mathbf{F}}_{fictitious} \;=\; m_i \; \{ \; + \hspace{+0.240em} {\vec{\mathit{A}}} \, + \, 2 \; {\vec{\omega}} \times ({\vec{\mathit{v}}}_{i} - {\vec{\mathit{V}}}) \, - \, {\vec{\omega}} \times [ \, {\vec{\omega}} \times ({\vec{\mathit{r}}}_{i} - {\vec{\mathit{R}}}) \, ] \, + \, {\vec{\alpha}} \times ({\vec{\mathit{r}}}_{i} - {\vec{\mathit{R}}}) \; \}$}

\par \bigskip \noindent Additionally, the new reformulation of classical mechanics presented in this paper is also a relational reformulation of classical mechanics since it is obtained starting from relative magnitudes (\hspace{+0.120em}position, velocity and acceleration\hspace{+0.120em}) between particles.

\par \bigskip \noindent However, as already stated above, the new reformulation of classical mechanics presented in this paper is observationally equivalent to Newtonian mechanics.

\par \bigskip \noindent Finally, the equation $[ \: {\mathbf{F}}_i = m_i \, {\mathbf{a}}_{\hspace{+0.045em}i} \: ]$ is valid in all reference frames (\hspace{+0.180em}inertial \hbox {or non-inertial\hspace{+0.180em})} only if all dynamic forces always obey Newton's third law in its weak form and in its strong form ( that is, \dots\,only if the equation $[ \; {\mathbf{F}}_i = m_i \, {\vec{\mathit{a}}}_i \; ]$ is always valid in the universal frame )

\vspace{+0.09em}

\noindent \rule[0mm]{141mm}{0.15mm}

\par \medskip \noindent {\footnotesize \textbf{A. Tobla}, A Reformulation of Classical Mechanics ( I \& II )\hspace{+1.12em} : \url{https://doi.org/10.5281/zenodo.11207437}}

\par \medskip \noindent {\footnotesize \textbf{A. Tobla}, A Reformulation of Classical Mechanics ( III \& IV ) : \url{https://doi.org/10.5281/zenodo.11207459}}

\newpage

\par \bigskip {\centering\subsection*{Annex I}}

\par \medskip {\centering\subsubsection*{The Universe}}\addcontentsline{toc}{subsection}{Annex I : The Universe}\hypertarget{p5a1}{}

\par \medskip \noindent The Universe is a system that contains all particles, that is always free of external forces, and that all internal dynamic forces always obey Newton's third law in its weak form and in its strong form.

\par \bigskip \noindent The position ${\vec{\mathit{R}}}$, the velocity ${\vec{\mathit{V}}}$ and the acceleration ${\vec{\mathit{A}}}$ of the center of mass of the Universe relative to a reference frame S (and the angular velocity ${\vec{\omega}}$ and the angular acceleration ${\vec{\alpha}}$ \hbox {of the Universe} relative to the reference frame S) are given by:

\par \bigskip\smallskip \hspace{-2.40em} \begin{tabular}{l}
${\mathrm{M}} ~\doteq~ \sum_i^{\scriptscriptstyle{\mathit{All}}} \, m_i$ \vspace{+1.20em} \\
${\vec{\mathit{R}}} ~\doteq~ {\mathrm{M}}^{\scriptscriptstyle -1} \, \sum_i^{\scriptscriptstyle{\mathit{All}}} \, m_i \, {\vec{\mathit{r}}_{i}}$ \vspace{+1.20em} \\
${\vec{\mathit{V}}} ~\doteq~ {\mathrm{M}}^{\scriptscriptstyle -1} \, \sum_i^{\scriptscriptstyle{\mathit{All}}} \, m_i \, {\vec{\mathit{v}}_{i}}$ \vspace{+1.20em} \\
${\vec{\mathit{A}}} ~\doteq~ {\mathrm{M}}^{\scriptscriptstyle -1} \, \sum_i^{\scriptscriptstyle{\mathit{All}}} \, m_i \, {\vec{\mathit{a}}_{i}}$ \vspace{+1.20em} \\
${\vec{\omega}} ~\doteq~ {\mathit{I}}^{\scriptscriptstyle -1}{\vphantom{\sum_1^2}}^{\hspace{-1.500em}\leftrightarrow}\hspace{+0.600em} \cdot {\vec{\mathit{L}}}$ \vspace{+1.20em} \\
${\vec{\alpha}} ~\doteq~ d({\vec{\omega}})/dt$ \vspace{+1.20em} \\
${\mathit{I}}{\vphantom{\sum_1^2}}^{\hspace{-0.555em}\leftrightarrow}\hspace{-0.210em} ~\doteq~ \sum_i^{\scriptscriptstyle{\mathit{All}}} \, m_i \, [ \, |\hspace{+0.090em}{\vec{\mathit{r}}_{i}} - {\vec{\mathit{R}}}\,|^2 \hspace{+0.309em} {\mathrm{1}}{\vphantom{\sum_1^2}}^{\hspace{-0.639em}\leftrightarrow}\hspace{-0.129em} - ({\vec{\mathit{r}}_{i}} - {\vec{\mathit{R}}}) \otimes ({\vec{\mathit{r}}_{i}} - {\vec{\mathit{R}}}) \, ]$ \vspace{+1.20em} \\
${\vec{\mathit{L}}} ~\doteq~ \sum_i^{\scriptscriptstyle{\mathit{All}}} \, m_i \, ({\vec{\mathit{r}}_{i}} - {\vec{\mathit{R}}}) \times ({\vec{\mathit{v}}_{i}} - \hspace{-0.120em}{\vec{\mathit{V}}})$
\end{tabular}

\par \bigskip \noindent where ${\mathrm{M}}$ is the mass of the Universe, ${\mathit{I}}{\vphantom{\sum_1^2}}^{\hspace{-0.555em}\leftrightarrow}\hspace{-0.300em}$ is the inertia tensor of the Universe (relative \hbox {to ${\vec{\mathit{R}}}$)} and ${\vec{\mathit{L}}}$ is the angular momentum of the Universe relative to the reference frame S.

\vspace{+1.50em}

\par {\centering\subsubsection*{The Transformations}}\addcontentsline{toc}{subsection}{Annex I : The Transformations}

\par \bigskip \noindent The transformations of position, velocity and acceleration of a particle $i$ between a reference frame S and another reference frame S', are given by:

\par \bigskip\medskip \hspace{-1.80em} $({\vec{\mathit{r}}}_i - {\vec{\mathit{R}}}) ~=~ {\mathbf{r}}_i ~=~ {\mathbf{r}}_i\hspace{-0.300em}'$

\par \bigskip \hspace{-1.80em} $({\vec{\mathit{r}}}_i\hspace{-0.150em}' - {\vec{\mathit{R}}}\hspace{+0.015em}') ~=~ {\mathbf{r}}_i\hspace{-0.300em}' ~=~ {\mathbf{r}}_i$

\par \bigskip \hspace{-1.80em} $({\vec{\mathit{v}}}_i - \hspace{-0.120em}{\vec{\mathit{V}}}) - {\vec{\omega}} \times ({\vec{\mathit{r}}}_i - {\vec{\mathit{R}}}) ~=~ {\mathbf{v}}_i ~=~ {\mathbf{v}}_i\hspace{-0.300em}'$

\par \bigskip \hspace{-1.80em} $({\vec{\mathit{v}}}_i\hspace{-0.150em}' - \hspace{-0.120em}{\vec{\mathit{V}}}\hspace{-0.045em}') - {\vec{\omega}}\hspace{+0.060em}' \times ({\vec{\mathit{r}}}_i\hspace{-0.150em}' - {\vec{\mathit{R}}}\hspace{+0.015em}') ~=~ {\mathbf{v}}_i\hspace{-0.300em}' ~=~ {\mathbf{v}}_i$

\par \bigskip \hspace{-1.80em} $({\vec{\mathit{a}}}_i - {\vec{\mathit{A}}}) - 2 \; {\vec{\omega}} \times ({\vec{\mathit{v}}}_i - \hspace{-0.120em}{\vec{\mathit{V}}}) + {\vec{\omega}} \times [ \, {\vec{\omega}} \times ({\vec{\mathit{r}}}_i - {\vec{\mathit{R}}}) \, ] - {\vec{\alpha}} \times ({\vec{\mathit{r}}}_i - {\vec{\mathit{R}}}) ~=~ {\mathbf{a}}_{\hspace{+0.045em}i} ~=~ {\mathbf{a}}_{\hspace{+0.045em}i}\hspace{-0.360em}'$

\par \bigskip \hspace{-1.80em} $({\vec{\mathit{a}}}_i\hspace{-0.150em}' - {\vec{\mathit{A}}}\hspace{-0.045em}') - 2 \; {\vec{\omega}}\hspace{+0.060em}' \times ({\vec{\mathit{v}}}_i\hspace{-0.150em}' - \hspace{-0.120em}{\vec{\mathit{V}}}\hspace{-0.045em}') + {\vec{\omega}}\hspace{+0.060em}' \times [ \, {\vec{\omega}}\hspace{+0.060em}' \times ({\vec{\mathit{r}}}_i\hspace{-0.150em}' - {\vec{\mathit{R}}}\hspace{+0.015em}') \, ] - {\vec{\alpha}}\hspace{+0.060em}' \times ({\vec{\mathit{r}}}_i\hspace{-0.150em}' - {\vec{\mathit{R}}}\hspace{+0.015em}') ~=~ {\mathbf{a}}_{\hspace{+0.045em}i}\hspace{-0.360em}' ~=~ {\mathbf{a}}_{\hspace{+0.045em}i}$

\newpage

\par \bigskip {\centering\subsection*{Annex A}}

\par \medskip {\centering\subsection*{Fields and Potentials I}}\addcontentsline{toc}{subsection}{Annex A : Fields and Potentials I}\hypertarget{p5aA}{}

\par \medskip \noindent The net kinetic force ${\mathbf{K}}_{\hspace{+0.009em}i}$ acting on a particle $i$ of mass $m_i$ can also be expressed as follows:

\par \bigskip ${\mathbf{K}}_{\hspace{+0.009em}i} \;=\: + \; m_i \, \Big[ \: {\mathbf{E}} \, + ({\vec{\mathit{v}}}_{i} - {\vec{\mathit{V}}}) \times {\mathbf{B}} \, \Big]$

\par \bigskip ${\mathbf{K}}_{\hspace{+0.009em}i} \;=\: + \; m_i \, \Big[ - \nabla \phi \, - \partial{\mathbf{A}}/\partial{\hspace{+0.03em}t}\, + ({\vec{\mathit{v}}}_{i} - {\vec{\mathit{V}}}) \times (\nabla \times {\mathbf{A}}) \, \Big]$

\par \bigskip ${\mathbf{K}}_{\hspace{+0.009em}i} \;=\: + \; m_i \, \Big[ - ({\vec{\mathit{a}}}_{i} - {\vec{\mathit{A}}}\hspace{+0.06em}) + 2 \; {\vec{\omega}} \times ({\vec{\mathit{v}}}_{i} - {\vec{\mathit{V}}}) - {\vec{\omega}} \times [ \, {\vec{\omega}} \times ({\vec{\mathit{r}}}_{i} - {\vec{\mathit{R}}}) \, ] + {\vec{\alpha}} \times ({\vec{\mathit{r}}}_{i} - {\vec{\mathit{R}}}) \, \Big]$

\par \bigskip \noindent where:

\vspace{+0.42em}

\par \bigskip ${\mathbf{E}} \;=\: - \; \nabla \phi \, - \partial{\mathbf{A}}/\partial{\hspace{+0.03em}t}$

\par \bigskip ${\mathbf{B}} \;=\: \nabla \times {\mathbf{A}}$

\par \bigskip $\phi \;=\: - \; \med \: [ \, {\vec{\omega}} \times ({\vec{\mathit{r}}}_{i} - {\vec{\mathit{R}}}) \, ]^{\hspace{+0.006em}2} \, + \, \med \, ( {\vec{\mathit{v}}}_{i} - {\vec{\mathit{V}}})^{\hspace{+0.03em}2}$

\par \bigskip ${\mathbf{A}} \;=\: - \; [ \, {\vec{\omega}} \times ({\vec{\mathit{r}}}_{i} - {\vec{\mathit{R}}}) \, ] \, + ( {\vec{\mathit{v}}}_{i} - {\vec{\mathit{V}}})$

\par \bigskip $\partial{\mathbf{A}}/\partial{\hspace{+0.03em}t} \;=\: - \; {\vec{\alpha}} \times ({\vec{\mathit{r}}}_{i} - {\vec{\mathit{R}}}) + ( {\vec{\mathit{a}}}_{i} - {\vec{\mathit{A}}}\hspace{+0.06em})$ * \hfill {\dix}

\par \bigskip $\nabla \phi \;=\: {\vec{\omega}} \times [ \, {\vec{\omega}} \times ({\vec{\mathit{r}}}_{i} - {\vec{\mathit{R}}}) \, ] \hspace{+0.06em}$ \hfill {\diy}

\par \bigskip $\nabla \times {\mathbf{A}} \;=\: - \; 2 \; {\vec{\omega}}$ \hfill {\diz}

\vspace{+0.60em}

\par \bigskip \noindent The net kinetic force ${\mathbf{K}}_{\hspace{+0.009em}i}$ acting on a particle $i$ of mass $m_i$ can also be obtained starting from the following kinetic energy:

\par \bigskip ${\mathit{K}}_{\hspace{+0.009em}i} \;=\: - \; m_i \, \big[ \, \phi \, - ({\vec{\mathit{v}}}_{i} - {\vec{\mathit{V}}}) \cdot {\mathbf{A}} \, \big]$

\par \bigskip ${\mathit{K}}_{\hspace{+0.009em}i} \;=\: \med \; m_i \, \big[ \, ({\vec{\mathit{v}}}_i - \hspace{-0.120em}{\vec{\mathit{V}}}) - {\vec{\omega}} \times ({\vec{\mathit{r}}}_i - {\vec{\mathit{R}}}) \, \big]^2$

\par \bigskip ${\mathit{K}}_{\hspace{+0.009em}i} \;=\: \med \; m_i \, \big[ \, {\mathbf{v}}_i \, \big]^2$

\vspace{+0.03em}

\par \bigskip \noindent Since the kinetic energy ${\mathit{K}}_{\hspace{+0.009em}i}$ must be positive, then applying the following Euler-Lagrange equation, we obtain:

\vspace{+0.06em}

\par \bigskip ${\mathbf{K}}_{\hspace{+0.009em}i} \;=\: - \; \dfrac{d}{dt} \, \Bigg[ \, \dfrac{\partial{\, \med \; m_i \, \big[ \, {\mathbf{v}}_i \, \big]^2}}{\partial{\,\mathbf{v}}_i} \, \Bigg] + \dfrac{\partial{\, \med \; m_i \, \big[ \, {\mathbf{v}}_i \, \big]^2}}{\partial{\,\mathbf{r}}_i} \;=\: - \; m_i \, {\mathbf{a}}_{\hspace{+0.045em}i}$

\vspace{+0.09em}

\par \bigskip \noindent where ${\mathbf{r}}_i, \hspace{+0.180em} {\mathbf{v}}_i$ and ${\mathbf{a}}_{\hspace{+0.045em}i}$ are the universal position, the universal velocity and the universal \hbox {acceleration} of particle $i$.

\vspace{+0.09em}

\noindent \rule[0mm]{21em}{0.15mm}

\vspace{-0.30em}

\par \medskip \noindent {\small * In the temporal partial derivative, the spatial coordinates must be treated as constants [ or replace this in the first equation:} ${\scriptstyle + \; {\small \met} \; ({\vec{\mathit{v}}}_{i} - {\vec{\mathit{V}}}) \times {\mathbf{B}}}$ {\small , and this in the second equation:} ${\scriptstyle + \; {\small \met} \; ({\vec{\mathit{v}}}_{i} - {\vec{\mathit{V}}}) \times (\nabla \times {\mathbf{A}})}$ ]

\newpage

\par \bigskip {\centering\subsection*{Annex B}}

\par \medskip {\centering\subsection*{Fields and Potentials II}}\addcontentsline{toc}{subsection}{Annex B : Fields and Potentials II}\hypertarget{p5aB}{}

\par \medskip \noindent The kinetic force ${\mathbf{K}}_{\hspace{+0.009em}ij}$ exerted on a particle $i$ of mass $m_i$ by another particle $j$ of mass $m_j$ can also be expressed as follows:

\par \bigskip ${\mathbf{K}}_{\hspace{+0.009em}ij} \;=\: + \; m_i \: m_j \, {\mathit{M}}^{\scriptscriptstyle -1} \, \Big[ \: {\mathbf{E}} \, + ({\vec{\mathit{v}}}_{i} - {\vec{\mathit{v}}}_{j}) \times {\mathbf{B}} \, \Big]$

\par \bigskip ${\mathbf{K}}_{\hspace{+0.009em}ij} \;=\: + \; m_i \: m_j \, {\mathit{M}}^{\scriptscriptstyle -1} \, \Big[ - \nabla \phi \, - \partial{\mathbf{A}}/\partial{\hspace{+0.03em}t}\, + ({\vec{\mathit{v}}}_{i} - {\vec{\mathit{v}}}_{j}) \times (\nabla \times {\mathbf{A}}) \, \Big]$

\par \bigskip ${\mathbf{K}}_{\hspace{+0.009em}ij} \;=\: + \; m_i \: m_j \, {\mathit{M}}^{\scriptscriptstyle -1} \, \Big[ - ({\vec{\mathit{a}}}_{i} - {\vec{\mathit{a}}}_{j}\hspace{+0.06em}) + 2 \; {\vec{\omega}} \times ({\vec{\mathit{v}}}_{i} - {\vec{\mathit{v}}}_{j}) - {\vec{\omega}} \times [ \, {\vec{\omega}} \times ({\vec{\mathit{r}}}_{i} - {\vec{\mathit{r}}}_{j}) \, ] + {\vec{\alpha}} \times ({\vec{\mathit{r}}}_{i} - {\vec{\mathit{r}}}_{j}) \, \Big]$

\par \bigskip \noindent where:

\vspace{+0.42em}

\par \bigskip ${\mathbf{E}} \;=\: - \; \nabla \phi \, - \partial{\mathbf{A}}/\partial{\hspace{+0.03em}t}$

\par \bigskip ${\mathbf{B}} \;=\: \nabla \times {\mathbf{A}}$

\par \bigskip $\phi \;=\: - \; \med \: [ \, {\vec{\omega}} \times ({\vec{\mathit{r}}}_{i} - {\vec{\mathit{r}}}_{j}) \, ]^{\hspace{+0.006em}2} \, + \, \med \, ( {\vec{\mathit{v}}}_{i} - {\vec{\mathit{v}}}_{j})^{\hspace{+0.03em}2}$

\par \bigskip ${\mathbf{A}} \;=\: - \; [ \, {\vec{\omega}} \times ({\vec{\mathit{r}}}_{i} - {\vec{\mathit{r}}}_{j}) \, ] \, + ( {\vec{\mathit{v}}}_{i} - {\vec{\mathit{v}}}_{j})$

\par \bigskip $\partial{\mathbf{A}}/\partial{\hspace{+0.03em}t} \;=\: - \; {\vec{\alpha}} \times ({\vec{\mathit{r}}}_{i} - {\vec{\mathit{r}}}_{j}) + ( {\vec{\mathit{a}}}_{i} - {\vec{\mathit{a}}}_{j}\hspace{+0.06em})$ * \hfill {\dix}

\par \bigskip $\nabla \phi \;=\: {\vec{\omega}} \times [ \, {\vec{\omega}} \times ({\vec{\mathit{r}}}_{i} - {\vec{\mathit{r}}}_{j}) \, ] \hspace{+0.06em}$ \hfill {\diy}

\par \bigskip $\nabla \times {\mathbf{A}} \;=\: - \; 2 \; {\vec{\omega}}$ \hfill {\diz}

\vspace{+0.60em}

\par \bigskip \noindent The kinetic force ${\mathbf{K}}_{\hspace{+0.009em}ij}$ exerted on a particle $i$ of mass $m_i$ by another particle $j$ of mass $m_j$ can also be obtained starting from the following kinetic energy:

\par \bigskip ${\mathit{K}}_{\hspace{+0.03em}ij} \;=\: - \; m_i \: m_j \, {\mathit{M}}^{\scriptscriptstyle -1} \, \big[ \, \phi \, - ({\vec{\mathit{v}}}_{i} - {\vec{\mathit{v}}}_{j}) \cdot {\mathbf{A}} \, \big]$

\par \bigskip ${\mathit{K}}_{\hspace{+0.03em}ij} \;=\: \med \; m_i \: m_j \, {\mathit{M}}^{\scriptscriptstyle -1} \, \big[ \, ({\vec{\mathit{v}}}_i - \hspace{-0.120em}{\vec{\mathit{v}}}_{j}) - {\vec{\omega}} \times ({\vec{\mathit{r}}}_i - {\vec{\mathit{r}}}_{j}) \, \big]^2$

\par \bigskip ${\mathit{K}}_{\hspace{+0.03em}ij} \;=\: \med \; m_i \: m_j \, {\mathit{M}}^{\scriptscriptstyle -1} \, \big[ \, {\mathbf{v}}_i - {\mathbf{v}}_{\hspace{-0.06em}j} \, \big]^2$

\vspace{+0.03em}

\par \bigskip \noindent Since the kinetic energy ${\mathit{K}}_{\hspace{+0.03em}ij}$ must be positive, then applying the following Euler-Lagrange equation, we obtain:

\vspace{+0.06em}

\par \bigskip ${\mathbf{K}}_{\hspace{+0.009em}ij} \;=\: - \; \dfrac{d}{dt} \, \Bigg[ \, \dfrac{\partial \, \med \, \frac{m_i \, m_j}{\mathit{M}} \big[ \, {\mathbf{v}}_i - {\mathbf{v}}_{\hspace{-0.06em}j} \, \big]^2}{\partial \, [ \, {\mathbf{v}}_i - {\mathbf{v}}_{\hspace{-0.06em}j} \, ]} \, \Bigg] + \dfrac{\partial \, \med \, \frac{m_i \, m_j}{\mathit{M}} \big[ \, {\mathbf{v}}_i - {\mathbf{v}}_{\hspace{-0.06em}j} \, \big]^2}{\partial \, [ \, {\mathbf{r}}_i - {\mathbf{r}}_{\hspace{-0.015em}j} \, ]} \;=\: - \; \dfrac{m_i \, m_j}{\mathit{M}} \, \big[ \, {\mathbf{a}}_{\hspace{+0.045em}i} - {\mathbf{a}}_{\hspace{-0.015em}j} \, \big]$

\vspace{+0.09em}

\par \bigskip \noindent where ${\mathbf{r}}_i, {\mathbf{v}}_i, {\mathbf{a}}_{\hspace{+0.045em}i}, {\mathbf{r}}_{\hspace{-0.015em}j}, {\mathbf{v}}_{\hspace{-0.06em}j}$ and ${\mathbf{a}}_{\hspace{-0.015em}j}$ are the universal positions, the universal velocities and the universal accelerations of particles $i$ and $j$.

\vspace{+0.09em}

\noindent \rule[0mm]{21em}{0.15mm}

\vspace{-0.30em}

\par \medskip \noindent {\small * In the temporal partial derivative, the spatial coordinates must be treated as constants [ or replace this in the first equation:} ${\scriptstyle + \; {\small \met} \; ({\vec{\mathit{v}}}_{i} - {\vec{\mathit{v}}}_{j}) \times {\mathbf{B}}}$ {\small , and this in the second equation:} ${\scriptstyle + \; {\small \met} \; ({\vec{\mathit{v}}}_{i} - {\vec{\mathit{v}}}_{j}) \times (\nabla \times {\mathbf{A}})}$ ]

\end{document}

