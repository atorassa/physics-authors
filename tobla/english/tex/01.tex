
\documentclass[10pt]{article}
%\documentclass[a4paper,10pt]{article}
%\documentclass[letterpaper,10pt]{article}

\usepackage[dvips]{geometry}
\geometry{papersize={163.05mm,240.0mm}}
\geometry{totalwidth=142.05mm,totalheight=204.0mm}

\usepackage[spanish]{babel}
\usepackage[latin1]{inputenc}
\usepackage{amsfonts}
\usepackage{amsmath,bm}

\frenchspacing

\usepackage{hyperref}
\hypersetup{colorlinks=true,linkcolor=black}
\hypersetup{bookmarksnumbered=true,pdfstartview=FitH,pdfpagemode=UseNone}
\hypersetup{pdftitle={Linear, Radial \& Scalar Magnitudes}}
\hypersetup{pdfauthor={Agust�n A. Tobla}}

\setlength{\arraycolsep}{1.74pt}

\newcommand{\med}{\raise.5ex\hbox{$\scriptstyle 1$}\kern-.15em/\kern-.09em\lower.25ex\hbox{$\scriptstyle 2$}}

\begin{document}

\begin{center}

{\LARGE Linear, Radial \& Scalar Magnitudes}

\bigskip \medskip

{\large Agust�n A. Tobla}

\bigskip \medskip

\small

Creative Commons Attribution 3.0 License

\smallskip

(2015) Buenos Aires

\smallskip

Argentina

\smallskip

\bigskip \medskip

\parbox{104.70mm}{In classical mechanics, this paper presents the definitions and the relations of the linear, radial and scalar magnitudes of a pair of particles $ij$.}

\end{center}

\normalsize

\vspace{-1.50em}
 
\par \bigskip {\centering\subsection*{Introduction}}

\par \bigskip \noindent i) The definitions of the linear, radial and scalar magnitudes of a pair of particles $ij$, \hbox {where ${\vec{\mathit{r}}}_{i}$} and ${\vec{\mathit{r}}}_{j}$ are the positions of the particles $i$ and $j$, are as follows:

\par \bigskip \noindent $\S$ The linear position ${\vec{\mathit{r}}}_{\hspace{+0.060em}ij}$, the linear velocity ${\vec{\mathit{v}}}_{\hspace{+0.060em}ij}$ and the linear acceleration ${\vec{\mathit{a}}}_{\hspace{+0.060em}ij}$, are given by:

\par \bigskip \noindent ${\vec{\mathit{r}}}_{\hspace{+0.060em}ij} \doteq ({\vec{\mathit{r}}}_{i} - {\vec{\mathit{r}}}_{j})$

\par \bigskip \noindent ${\vec{\mathit{v}}}_{\hspace{+0.060em}ij} \doteq d(\hspace{+0.045em}{\vec{\mathit{r}}}_{\hspace{+0.060em}ij})/dt = ({\vec{\mathit{v}}}_{i} - {\vec{\mathit{v}}}_{j})$

\par \bigskip \noindent ${\vec{\mathit{a}}}_{\hspace{+0.060em}ij} \doteq d^2(\hspace{+0.045em}{\vec{\mathit{r}}}_{\hspace{+0.060em}ij})/dt^2 = ({\vec{\mathit{a}}}_{i} - {\vec{\mathit{a}}}_{j})$

\par \bigskip \noindent $\S$ The radial position ${\mathit{r}}_{\hspace{+0.060em}ij}$, the radial velocity ${\dot{\mathit{r}}}_{\hspace{+0.060em}ij}$ and the radial acceleration ${\ddot{\mathit{r}}}_{\hspace{+0.060em}ij}$, are given by:

\par \bigskip \noindent ${\mathit{r}}_{\hspace{+0.060em}ij} \doteq | \hspace{+0.090em} {\vec{\mathit{r}}}_{i} - {\vec{\mathit{r}}}_{j} \hspace{+0.090em} |$

\par \bigskip \noindent ${\dot{\mathit{r}}}_{\hspace{+0.060em}ij} \doteq d(\hspace{+0.045em}{\mathit{r}}_{\hspace{+0.060em}ij})/dt = [ \, ({\vec{\mathit{v}}}_i - {\vec{\mathit{v}}}_j) \cdot ({\vec{\mathit{r}}}_i - {\vec{\mathit{r}}}_j) \, ] \,/\, | \, {\vec{\mathit{r}}}_i - {\vec{\mathit{r}}}_j \, |$

\par \bigskip \noindent ${\ddot{\mathit{r}}}_{\hspace{+0.060em}ij} \doteq d^2(\hspace{+0.045em}{\mathit{r}}_{\hspace{+0.060em}ij})/dt^2 = [ \, ({\vec{\mathit{a}}}_i - {\vec{\mathit{a}}}_j) \cdot ({\vec{\mathit{r}}}_i - {\vec{\mathit{r}}}_j) + ({\vec{\mathit{v}}}_i - {\vec{\mathit{v}}}_j) \cdot ({\vec{\mathit{v}}}_i - {\vec{\mathit{v}}}_j) - [ ({\vec{\mathit{v}}}_i - {\vec{\mathit{v}}}_j) \cdot ({\vec{\mathit{r}}}_i - {\vec{\mathit{r}}}_j) ]^2 \hspace{-0.06em} / ({\vec{\mathit{r}}}_i - {\vec{\mathit{r}}}_j)^2 \, ] \,/\, | \, {\vec{\mathit{r}}}_i - {\vec{\mathit{r}}}_j \, |$

\par \bigskip \noindent $\S$ The scalar position ${\tau}_{\hspace{+0.060em}ij}$, the scalar velocity ${\dot{\tau}}_{\hspace{+0.060em}ij}$ and the scalar acceleration ${\ddot{\tau}}_{\hspace{+0.060em}ij}$, are given by:

\par \bigskip \noindent ${\tau}_{\hspace{+0.060em}ij} \doteq \med \; ({\vec{\mathit{r}}}_{i} - {\vec{\mathit{r}}}_{j}) \cdot ({\vec{\mathit{r}}}_{i} - {\vec{\mathit{r}}}_{j})$

\par \bigskip \noindent ${\dot{\tau}}_{\hspace{+0.060em}ij} \doteq d(\hspace{+0.045em}{\tau}_{\hspace{+0.060em}ij})/dt = ({\vec{\mathit{v}}}_{i} - {\vec{\mathit{v}}}_{j}) \cdot ({\vec{\mathit{r}}}_{i} - {\vec{\mathit{r}}}_{j})$

\par \bigskip \noindent ${\ddot{\tau}}_{\hspace{+0.060em}ij} \doteq d^2(\hspace{+0.045em}{\tau}_{\hspace{+0.060em}ij})/dt^2 = ({\vec{\mathit{a}}}_{i} - {\vec{\mathit{a}}}_{j}) \cdot ({\vec{\mathit{r}}}_{i} - {\vec{\mathit{r}}}_{j}) \hspace{+0.060em}+\hspace{+0.060em} ({\vec{\mathit{v}}}_{i} - {\vec{\mathit{v}}}_{j}) \cdot ({\vec{\mathit{v}}}_{i} - {\vec{\mathit{v}}}_{j})$

\par \bigskip \noindent ii) The relations between the linear, radial and scalar magnitudes of a pair of particles $ij$, they can be obtained from the above definitions, are as follows:

\par \bigskip \noindent ${\tau}_{\hspace{+0.060em}ij} = \med \: {\mathit{r}}_{\hspace{+0.060em}ij} \, {\mathit{r}}_{\hspace{+0.060em}ij} = \med \: {\vec{\mathit{r}}}_{\hspace{+0.060em}ij} \cdot {\vec{\mathit{r}}}_{\hspace{+0.060em}ij}$

\par \bigskip \noindent ${\dot{\tau}}_{\hspace{+0.060em}ij} = {\dot{\mathit{r}}}_{\hspace{+0.060em}ij} \, {\mathit{r}}_{\hspace{+0.060em}ij} = {\vec{\mathit{v}}}_{\hspace{+0.060em}ij} \cdot {\vec{\mathit{r}}}_{\hspace{+0.060em}ij}$

\par \bigskip \noindent ${\ddot{\tau}}_{\hspace{+0.060em}ij} = {\ddot{\mathit{r}}}_{\hspace{+0.060em}ij} \, {\mathit{r}}_{\hspace{+0.060em}ij} + {\dot{\mathit{r}}}_{\hspace{+0.060em}ij} \, {\dot{\mathit{r}}}_{\hspace{+0.060em}ij} = {\vec{\mathit{a}}}_{\hspace{+0.060em}ij} \cdot {\vec{\mathit{r}}}_{\hspace{+0.060em}ij} + {\vec{\mathit{v}}}_{\hspace{+0.060em}ij} \cdot {\vec{\mathit{v}}}_{\hspace{+0.060em}ij}$

\par \bigskip \noindent iii) The magnitudes $[ \: {\vec{\mathit{r}}}_{\hspace{+0.060em}ij}, {\mathit{r}}_{\hspace{+0.060em}ij}, {\dot{\mathit{r}}}_{\hspace{+0.060em}ij}, {\ddot{\mathit{r}}}_{\hspace{+0.060em}ij}, {\tau}_{\hspace{+0.060em}ij}, {\dot{\tau}}_{\hspace{+0.060em}ij}$ and ${\ddot{\tau}}_{\hspace{+0.060em}ij} \: ]$ are invariant under transformations between inertial and non-inertial reference frames.

\end{document}

