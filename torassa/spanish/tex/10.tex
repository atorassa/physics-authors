
\documentclass[12pt]{article}
%\documentclass[a4paper,12pt]{article}
%\documentclass[letterpaper,12pt]{article}

\usepackage[dvips]{geometry}
\geometry{papersize={127mm,165.1mm}}
\geometry{totalwidth=106mm,totalheight=129.1mm}

\usepackage[spanish]{babel}
\usepackage[latin1]{inputenc}
\usepackage[T1]{fontenc}
\usepackage{mathptmx}

\frenchspacing

\usepackage{hyperref}
\hypersetup{colorlinks=true,linkcolor=black}
\hypersetup{bookmarksnumbered=true,pdfstartview=FitH,pdfpagemode=UseNone}
\hypersetup{pdftitle={Mec�nica Cl�sica (Diagrama)}}
\hypersetup{pdfsubject={Diagrama de fuerzas resultantes sobre sistemas de referencia.}}
\hypersetup{pdfauthor={Alejandro A. Torassa}}
\hypersetup{pdfkeywords={mec�nica cl�sica, diagrama, fuerza, resultante, sistema, referencia}}

\setlength{\unitlength}{1pt}

\begin{document}

\pagestyle{empty}

\enlargethispage{+3em}

\begin{center}

{\Large Mec�nica Cl�sica}

\bigskip \medskip

{\large Diagrama [A]}

\bigskip \medskip

{\normalsize Diagrama de fuerzas resultantes sobre un sistema de referencia \\ no rotante no acelerado respecto a un sistema inercial}

\bigskip \medskip

{\normalsize ( 9 puntos )}

\bigskip \bigskip

\begin{picture}(240,240)
\put(120,120){\vector(0,+1){120}}
\put(120,120){\vector(+1,0){120}}
\put(120,120){\vector(0,-1){120}}
\put(120,120){\vector(-1,0){120}}
\put(120,120){\circle*{6}}
\put(150,150){\circle*{6}}
\put(195,195){\circle*{6}}
\put(45,195){\circle*{6}}
\put(90,150){\circle*{6}}
\put(150,90){\circle*{6}}
\put(195,45){\circle*{6}}
\put(45,45){\circle*{6}}
\put(90,90){\circle*{6}}
\put(124,231){{$y$}}
\put(231,124){{$x$}}
\end{picture}

\end{center}

\newpage

\pagestyle{empty}

\enlargethispage{+3em}

\begin{center}

{\Large Mec�nica Cl�sica}

\bigskip \medskip

{\large Diagrama [B]}

\bigskip \medskip

{\normalsize Diagrama de fuerzas resultantes sobre un sistema de referencia \\ no rotante acelerado respecto a un sistema inercial}

\bigskip \medskip

{\normalsize ( 9 puntos )}

\bigskip \bigskip

\begin{picture}(240,240)
\put(120,120){\vector(0,+1){120}}
\put(120,120){\vector(+1,0){120}}
\put(120,120){\vector(0,-1){120}}
\put(120,120){\vector(-1,0){120}}
\put(120,120){\vector(1,1){21}}
\put(150,150){\vector(1,1){21}}
\put(195,195){\vector(1,1){21}}
\put(45,195){\vector(1,1){21}}
\put(90,150){\vector(1,1){21}}
\put(150,90){\vector(1,1){21}}
\put(195,45){\vector(1,1){21}}
\put(45,45){\vector(1,1){21}}
\put(90,90){\vector(1,1){21}}
\put(120,120){\circle*{6}}
\put(150,150){\circle*{6}}
\put(195,195){\circle*{6}}
\put(45,195){\circle*{6}}
\put(90,150){\circle*{6}}
\put(150,90){\circle*{6}}
\put(195,45){\circle*{6}}
\put(45,45){\circle*{6}}
\put(90,90){\circle*{6}}
\put(124,231){{$y$}}
\put(231,124){{$x$}}
\end{picture}

\end{center}

\newpage

\pagestyle{empty}

\enlargethispage{+3em}

\begin{center}

{\Large Mec�nica Cl�sica}

\bigskip \medskip

{\large Diagrama [C]}

\bigskip \medskip

{\normalsize Diagrama de fuerzas resultantes sobre un sistema de referencia \\ rotante no acelerado respecto a un sistema inercial}

\bigskip \medskip

{\normalsize ( 9 puntos )}

\bigskip \bigskip

\begin{picture}(240,240)
\put(120,120){\vector(0,+1){120}}
\put(120,120){\vector(+1,0){120}}
\put(120,120){\vector(0,-1){120}}
\put(120,120){\vector(-1,0){120}}
\put(150,150){\vector(-1,-1){21}}
\put(195,195){\vector(-1,-1){36}}
\put(45,195){\vector(1,-1){36}}
\put(90,150){\vector(1,-1){21}}
\put(150,90){\vector(-1,1){21}}
\put(195,45){\vector(-1,1){36}}
\put(45,45){\vector(1,1){36}}
\put(90,90){\vector(1,1){21}}
\put(120,120){\circle*{6}}
\put(150,150){\circle*{6}}
\put(195,195){\circle*{6}}
\put(45,195){\circle*{6}}
\put(90,150){\circle*{6}}
\put(150,90){\circle*{6}}
\put(195,45){\circle*{6}}
\put(45,45){\circle*{6}}
\put(90,90){\circle*{6}}
\put(124,231){{$y$}}
\put(231,124){{$x$}}
\end{picture}

\end{center}

\end{document}

