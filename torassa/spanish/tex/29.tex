
\documentclass[10pt]{article}
%\documentclass[a4paper,10pt]{article}
%\documentclass[letterpaper,10pt]{article}

\usepackage[dvips]{geometry}
\geometry{papersize={128.0mm,150.0mm}}
\geometry{totalwidth=106.6mm,totalheight=114.0mm}

\usepackage[spanish]{babel}
\usepackage[latin1]{inputenc}
\usepackage[T1]{fontenc}
\usepackage{mathptmx}

\frenchspacing

\usepackage{hyperref}
\hypersetup{colorlinks=true,linkcolor=black}
\hypersetup{bookmarksnumbered=true,pdfstartview=FitH,pdfpagemode=UseNone}
\hypersetup{pdftitle={Energ�a Mec�nica Angular}}
\hypersetup{pdfauthor={Alejandro A. Torassa}}

\setlength{\arraycolsep}{1.74pt}

\newcommand{\mM}{m}
\newcommand{\mE}{E}
\newcommand{\ra}{_a}
\newcommand{\vR}{\mathbf{r}}
\newcommand{\vV}{\mathbf{v}}
\newcommand{\vA}{\mathbf{a}}
\newcommand{\med}{\raise.5ex\hbox{$\scriptstyle 1$}\kern-.15em/\kern-.15em\lower.25ex\hbox{$\scriptstyle 2$}}

\begin{document}

\begin{center}

{\LARGE Energ�a Mec�nica Angular}

\bigskip \medskip

Alejandro A. Torassa

\bigskip \medskip

\footnotesize

Licencia Creative Commons Atribuci�n 3.0

(2014) Buenos Aires, Argentina

atorassa@gmail.com

\bigskip \smallskip

\small

{\bf Resumen}

\bigskip \smallskip

\parbox{75mm}{Este trabajo presenta el principio de conservaci�n de la energ�a mec�nica angular para una part�cula que se mueve en un campo de fuerzas uniforme.}

\end{center}

\normalsize

\vspace{-0.30em}

{\centering\subsubsection*{Energ�a Mec�nica Angular}}

\vspace{+0.75em}

\par La energ�a mec�nica angular $\mE\ra$ de una part�cula A de masa $\mM\ra$ que se mueve en un campo de fuerzas uniforme, est� dada por:
\begin{eqnarray*}
\mE\ra = \med \hspace{+0.24em} \mM\ra \hspace{+0.06em} (\hspace{+0.03em} \vR \times \vV\ra)^{\hspace{+0.03em} 2} - \hspace{+0.09em} \mM\ra \hspace{+0.06em} (\hspace{+0.03em} \vR \times \vA\ra) \cdot (\hspace{+0.03em} \vR \times \vR\ra)
\end{eqnarray*}
\noindent donde $\vR$ es un vector posici�n que es constante en magnitud y \hbox {direcci�n, y} $\vV\ra$, $\vA\ra$ y $\vR\ra$ son la velocidad, la aceleraci�n constante y la posici�n de la part�cula A.
\medskip
\par El principio de conservaci�n de la energ�a mec�nica angular establece que si una part�cula A se mueve en un campo de fuerzas uniforme entonces la energ�a mec�nica angular de la part�cula A permanece constante.

\end{document}

