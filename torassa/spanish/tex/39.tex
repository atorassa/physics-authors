
\documentclass[10pt]{article}
%\documentclass[a4paper,10pt]{article}
%\documentclass[letterpaper,10pt]{article}

\usepackage[dvips]{geometry}
\geometry{papersize={168.9mm,190.8mm}}
\geometry{totalwidth=147.9mm,totalheight=148.8mm}

\usepackage[spanish]{babel}
\usepackage[latin1]{inputenc}
\usepackage[T1]{fontenc}
\usepackage{mathptmx}

\frenchspacing

\usepackage{hyperref}
\hypersetup{colorlinks=true,linkcolor=black}
\hypersetup{bookmarksnumbered=true,pdfstartview=FitH,pdfpagemode=UseNone}
\hypersetup{pdftitle={Mec�nica Cl�sica General}}
\hypersetup{pdfauthor={Alejandro A. Torassa}}

\setlength{\arraycolsep}{1.74pt}

\newcommand{\mT}{t}
\newcommand{\mM}{m}
\newcommand{\ri}{_i}
\newcommand{\bre}{\breve}
\newcommand{\vR}{\mathbf{r}}
\newcommand{\vV}{\mathbf{v}}
\newcommand{\vA}{\mathbf{a}}
\newcommand{\vF}{\mathbf{F}}
\newcommand{\sen}{\operatorname{\sen}}
\newcommand{\rj}{_{\hspace{-0.081em}j}}
\newcommand{\rij}{_{i\hspace{-0.081em}j}}
\newcommand{\med}{\raise.5ex\hbox{$\scriptstyle 1$}\kern-.15em/\kern-.15em\lower.25ex\hbox{$\scriptstyle 2$}\:}

\begin{document}

\begin{center}

{\huge Mec�nica Cl�sica General}

\bigskip \bigskip

{\large Alejandro A. Torassa}

\bigskip \bigskip

\small

Licencia Creative Commons Atribuci�n 3.0

(2014) Buenos Aires, Argentina

atorassa@gmail.com

\bigskip \medskip

{\bf Resumen}

\bigskip

\parbox{96mm}{Este trabajo presenta una mec�nica cl�sica general que es invariante bajo transformaciones entre sistemas de referencia y que puede ser aplicada en cualquier sistema de referencia (rotante o no rotante) (inercial o no inercial) sin necesidad de introducir fuerzas ficticias.}

\end{center}

\normalsize

\vspace{-0.30em}

{\centering\subsection*{Introducci�n}}

\vspace{+1.20em}

\par La posici�n $\vR\ri$, la velocidad $\vV\ri$ y la aceleraci�n $\vA\ri$ de una part�cula i de masa $\mM\ri$, est�n dadas por:
\bigskip
\begin{center}
\begin{tabular}{l}
\hspace{-3.09em} $\vR\ri = (\vR\ri)$ \vspace{+1.20em} \\
\hspace{-3.09em} $\vV\ri = d(\vR\ri)/d\mT$ \vspace{+1.20em} \\
\hspace{-3.09em} $\vA\ri = d^2(\vR\ri)/d\mT^2$
\end{tabular}
\end{center}
\bigskip
\noindent donde $\vR\ri$ es el vector posici�n de la part�cula i.
\medskip
\par Y la posici�n din�mica $\bre\vR\ri$, la velocidad din�mica $\bre\vV\ri$ y la aceleraci�n din�mica $\bre\vA\ri$, est�n dadas por:
\medskip
\begin{center}
\begin{tabular}{l}
$\bre\vR\ri = \int\int \hspace{+0.12em} (\vF\ri/\mM\ri) \; d\mT \; d\mT$ \vspace{+1.20em} \\
$\bre\vV\ri = \int \hspace{+0.12em} (\vF\ri/\mM\ri) \; d\mT$ \vspace{+1.20em} \\
$\bre\vA\ri = (\vF\ri/\mM\ri)$
\end{tabular}
\end{center}
\medskip
\noindent donde $\vF\ri$ es la fuerza resultante que act�a sobre la part�cula i.

\newpage

{\centering\subsection*{Ecuaciones de Movimiento}}

\vspace{+1.20em}

\par Dadas dos part�culas i y j entonces para un sistema de referencia S las ecuaciones de movimiento son:
\vspace{+0.00em}
\par $\med \hspace{+0.15em} \mM\ri\hspace{+0.09em}\mM\rj \hspace{+0.12em} \big[ \hspace{+0.12em} (\vR \hspace{+0.18em}{\S}\hspace{+0.18em} \vR\rij) \cdot (\vR \hspace{+0.18em}{\S}\hspace{+0.18em} \vR\rij) - (\vR \hspace{+0.18em}{\S}\hspace{+0.18em} \bre\vR\rij) \cdot (\vR \hspace{+0.18em}{\S}\hspace{+0.18em} \bre\vR\rij) \hspace{+0.12em} \big] = \hspace{+0.12em} 0$ \\
\vspace{+0.30em}
\par $\med \hspace{+0.15em} \mM\ri\hspace{+0.09em}\mM\rj \hspace{+0.12em} \big[ \hspace{+0.12em} (\vR \hspace{+0.18em}{\S}\hspace{+0.18em} \vR\rij) \cdot (\vR \hspace{+0.18em}{\S}\hspace{+0.18em} \vV\rij) - (\vR \hspace{+0.18em}{\S}\hspace{+0.18em} \vR\rij) \cdot \int (\vR \hspace{+0.18em}{\S}\hspace{+0.18em} \bre\vA\rij) \hspace{+0.18em} dt \hspace{+0.12em} \big] = \hspace{+0.12em} 0$ \\
\vspace{+0.30em}
\par $\med \hspace{+0.15em} \mM\ri\hspace{+0.09em}\mM\rj \hspace{+0.12em} \big[ \hspace{+0.12em} (\vR \hspace{+0.18em}{\S}\hspace{+0.18em} \vV\rij) \cdot (\vR \hspace{+0.18em}{\S}\hspace{+0.18em} \vV\rij) + (\vR \hspace{+0.18em}{\S}\hspace{+0.18em} \vA\rij) \cdot (\vR \hspace{+0.18em}{\S}\hspace{+0.18em} \vR\rij) - 2 \int (\vR \hspace{+0.18em}{\S}\hspace{+0.18em} \bre\vA\rij) \cdot d(\vR \hspace{+0.18em}{\S}\hspace{+0.18em} \vR\rij) - (\vR \hspace{+0.18em}{\S}\hspace{+0.18em} \bre\vA\rij) \cdot (\vR \hspace{+0.18em}{\S}\hspace{+0.18em} \vR\rij) \hspace{+0.12em} \big] = \hspace{+0.12em} 0$
\vspace{+1.20em}
\par \noindent donde $\vR\rij = \vR\ri - \vR\rj$, $\vV\rij = \vV\ri - \vV\rj$, $\vA\rij = \vA\ri - \vA\rj$, $\bre\vR\rij = \bre\vR\ri - \bre\vR\rj$, $\bre\vA\rij = \bre\vA\ri - \bre\vA\rj$, $\mM\ri$ y $\mM\rj$ son las masas de las part�culas i y j, $\vR\ri$, $\vR\rj$, $\vV\ri$, $\vV\rj$, $\vA\ri$ y $\vA\rj$ son las posiciones, las velocidades y las aceleraciones de las part�culas i y j, y $\bre\vR\ri$, $\bre\vR\rj$, $\bre\vA\ri$ y $\bre\vA\rj$ son las posiciones din�micas y las aceleraciones din�micas de las part�culas i y j.
\vspace{+1.20em}
\par \noindent $\vR$ es un vector posici�n definido por dos puntos fijos 1 y 2 del sistema de referencia S ($\vR = \vR_1 - \vR_2$) en el que la aceleraci�n din�mica del punto 1 es igual a la aceleraci�n din�mica del punto 2 ($\bre\vA_1 = \bre\vA_2$)
\vspace{+1.20em}
\par \noindent $\vR$ es invariante bajo transformaciones entre sistemas de referencia.
\vspace{+1.20em}
\par \noindent ${\S}$ puede cambiarse por los siguientes operadores:
\vspace{+1.20em}
\par\hspace{-1.20em}\begin{tabular}{cl}
\vspace{+1.20em}
\par $\ast$ & Producto primero excluido: $\vec{A} \ast \vec{B} = \vec{B}$ \\
\vspace{+1.20em}
\par $\cdot$ & Producto punto: $\vec{A} \cdot \vec{B} = |\vec{A}| |\vec{B}| \cos \theta$ \\
\vspace{+1.20em}
\par $\times$ & Producto cruz: $\vec{A} \times \vec{B} = |\vec{A}| |\vec{B}| \sen \theta \hspace{+0.12em} \hat{n}$ \hspace{+0.12em} (direcci�n: regla mano derecha) \\
\vspace{+1.20em}
\par $:$ & Producto punto vectorial: $\vec{A} : \vec{B} = |\vec{A}| |\vec{B}| \cos \theta \hspace{+0.12em} \hat{n}$ \hspace{+0.12em} (direcci�n: igual que $\vec{A}$) \\
\vspace{+1.20em}
\par $\dot{\times}$ & Producto cruz escalar: $\vec{A} \hspace{+0.12em}\dot{\times}\hspace{+0.12em} \vec{B} = |\vec{A}| |\vec{B}| \sen \theta$ \hspace{+0.12em} (signo: regla mano derecha)
\end{tabular}
\vspace{+0.15em}
\par \noindent \textbf{Nota}: En este trabajo la siguiente regla es v�lida:
\vspace{+0.90em}
\par \noindent (escalar)\hspace{+0.09em}$\cdot$\hspace{+0.09em}(escalar)\hspace{+0.27em}=\hspace{+0.27em}(escalar)\hspace{+0.27em}(escalar)

\newpage

{\centering\subsection*{Sistemas de Referencia}}

\vspace{+1.20em}

\par Las magnitudes $\bre\vR\ri$, $\bre\vV\ri$ y $\bre\vA\ri$ son invariantes bajo transformaciones entre sistemas de referencia.
\bigskip
\par En cualquier sistema de referencia $\vR\rij = \bre\vR\rij$. Por lo tanto, $\vR\rij$ es invariante bajo transformaciones entre sistemas de referencia.
\bigskip
\par En cualquier sistema de referencia no rotante $\vV\rij = \bre\vV\rij$ y $\vA\rij = \bre\vA\rij$. Por lo tanto, $\vV\rij$ y $\vA\rij$ son invariantes bajo transformaciones entre sistemas de referencia no rotantes.
\bigskip
\par En cualquier sistema de referencia inercial $\vA\ri = \bre\vA\ri$. Por lo tanto, $\vA\ri$ es invariante bajo transformaciones entre sistemas de referencia inerciales. Cualquier sistema de referencia inercial es un sistema de referencia no rotante.
\bigskip
\par En el sistema de referencia universal $\vR\ri = \bre\vR\ri$, $\vV\ri = \bre\vV\ri$ y $\vA\ri = \bre\vA\ri$. Por lo tanto, el sistema de referencia universal es un sistema de referencia inercial.
\bigskip
\par El sistema de referencia universal es un sistema de referencia fijo al centro de masa del universo (si la fuerza resultante que act�a sobre el centro de masa del universo es siempre cero)

\vspace{+1.50em}

{\centering\subsection*{Observaciones}}

\vspace{+1.20em}

\par Las ecuaciones de movimiento son tambi�n ecuaciones de conservaci�n.
\bigskip
\par Las ecuaciones de movimiento son invariantes bajo transformaciones entre sistemas de referencia.
\bigskip
\par Las ecuaciones de movimiento pueden ser aplicadas en cualquier sistema de referencia (rotante o no rotante) (inercial o no inercial) sin necesidad de introducir fuerzas ficticias.
\bigskip
\par Las ecuaciones de movimiento ser�an v�lidas incluso si la tercera ley de movimiento de Newton fuera falsa en un sistema de referencia inercial.
\bigskip
\par Las ecuaciones de movimiento ser�an v�lidas incluso si las tres leyes de movimiento de Newton fueran falsas en un sistema de referencia no inercial.
\bigskip
\par Las ecuaciones de movimiento son ecuaciones de transformaci�n entre sistemas de referencia y pueden obtenerse desde la ecuaci�n general de movimiento ( \hspace{-0.45em} \textbf{A. Torassa}, Ecuaci�n General de Movimiento \hspace{-0.45em} )

\newpage

{\centering\subsection*{Anexo}}

\vspace{+1.20em}

{\centering\subsection*{\rm Trabajo, K y U}}

\vspace{+1.50em}

\par $W\rij = \med \hspace{+0.15em} \mM\ri\hspace{+0.09em}\mM\rj \hspace{+0.12em} \big[ \hspace{+0.12em} 2 \int_{\scriptscriptstyle \hspace{-0.03em} 1}^{\scriptscriptstyle 2} (\vR \hspace{+0.18em}{\S}\hspace{+0.18em} \bre\vA\rij) \cdot d(\vR \hspace{+0.18em}{\S}\hspace{+0.18em} \vR\rij) + \Delta \hspace{+0.18em} (\vR \hspace{+0.18em}{\S}\hspace{+0.18em} \bre\vA\rij) \cdot (\vR \hspace{+0.18em}{\S}\hspace{+0.18em} \vR\rij) \hspace{+0.12em} \big]$
\vspace{+1.20em}
\par $W\rij = \Delta \hspace{+0.24em} K\rij$
\vspace{+1.20em}
\par $\Delta \hspace{+0.24em} K\rij = \hspace{+0.09em} \Delta \hspace{+0.24em} \med \hspace{+0.15em} \mM\ri\hspace{+0.09em}\mM\rj \hspace{+0.12em} \big[ \hspace{+0.12em} (\vR \hspace{+0.18em}{\S}\hspace{+0.18em} \vV\rij) \cdot (\vR \hspace{+0.18em}{\S}\hspace{+0.18em} \vV\rij) + (\vR \hspace{+0.18em}{\S}\hspace{+0.18em} \vA\rij) \cdot (\vR \hspace{+0.18em}{\S}\hspace{+0.18em} \vR\rij) \hspace{+0.12em} \big]$
\vspace{+1.20em}
\par $\Delta \hspace{+0.24em} U\rij = - \hspace{+0.18em} \med \hspace{+0.15em} \mM\ri\hspace{+0.09em}\mM\rj \hspace{+0.12em} \big[ \hspace{+0.12em} 2 \int_{\scriptscriptstyle \hspace{-0.03em} 1}^{\scriptscriptstyle 2} (\vR \hspace{+0.18em}{\S}\hspace{+0.18em} \bre\vA\rij) \cdot d(\vR \hspace{+0.18em}{\S}\hspace{+0.18em} \vR\rij) + \Delta \hspace{+0.18em} (\vR \hspace{+0.18em}{\S}\hspace{+0.18em} \bre\vA\rij) \cdot (\vR \hspace{+0.18em}{\S}\hspace{+0.18em} \vR\rij) \hspace{+0.12em} \big]$

\vspace{+2.10em}

{\centering\subsection*{\rm Principio de M�nima Acci�n}}

\vspace{+1.50em}

\par $\delta \int_{\scriptscriptstyle \hspace{-0.03em} 1}^{\scriptscriptstyle 2} \hspace{+0.18em} L\rij \hspace{+0.24em} dt = 0$
\vspace{+1.20em}
\par $\delta \int_{\scriptscriptstyle \hspace{-0.03em} 1}^{\scriptscriptstyle 2} \hspace{+0.18em} (T\rij - V\rij) \hspace{+0.24em} dt = 0$
\vspace{+1.20em}
\par $T\rij = + \hspace{+0.18em} \med \hspace{+0.15em} \mM\ri\hspace{+0.09em}\mM\rj \hspace{+0.12em} \big[ \hspace{+0.12em} (\vR \hspace{+0.18em}{\S}\hspace{+0.18em} \vV\rij) \cdot (\vR \hspace{+0.18em}{\S}\hspace{+0.18em} \vV\rij) + (\vR \hspace{+0.18em}{\S}\hspace{+0.18em} \vA\rij) \cdot (\vR \hspace{+0.18em}{\S}\hspace{+0.18em} \vR\rij) \hspace{+0.12em} \big]$
\vspace{+1.20em}
\par $V\rij = - \hspace{+0.18em} \med \hspace{+0.15em} \mM\ri\hspace{+0.09em}\mM\rj \hspace{+0.12em} \big[ \hspace{+0.12em} 2 \int (\vR \hspace{+0.18em}{\S}\hspace{+0.18em} \bre\vA\rij) \cdot d(\vR \hspace{+0.18em}{\S}\hspace{+0.18em} \vR\rij) + (\vR \hspace{+0.18em}{\S}\hspace{+0.18em} \bre\vA\rij) \cdot (\vR \hspace{+0.18em}{\S}\hspace{+0.18em} \vR\rij) \hspace{+0.12em} \big]$

\vspace{+2.10em}

{\centering\subsection*{\rm Grupo de Ecuaciones Invariantes}}

\vspace{+1.50em}

\par $\med \hspace{+0.15em} \mM\ri\hspace{+0.09em}\mM\rj \hspace{+0.12em} \big[ \hspace{+0.12em} (\vR \hspace{+0.18em}{\S}\hspace{+0.18em} \vR\rij) \cdot (\vR \hspace{+0.18em}{\S}\hspace{+0.18em} \vR\rij) \hspace{+0.12em} \big] = \hspace{+0.12em}$ Invariante desde S a S'
\vspace{+1.20em}
\par $\med \hspace{+0.15em} \mM\ri\hspace{+0.09em}\mM\rj \hspace{+0.12em} \big[ \hspace{+0.12em} (\vR \hspace{+0.18em}{\S}\hspace{+0.18em} \vR\rij) \cdot (\vR \hspace{+0.18em}{\S}\hspace{+0.18em} \vV\rij) \hspace{+0.12em} \big] = \hspace{+0.12em}$ Invariante desde S a S'
\vspace{+1.20em}
\par $\med \hspace{+0.15em} \mM\ri\hspace{+0.09em}\mM\rj \hspace{+0.12em} \big[ \hspace{+0.12em} (\vR \hspace{+0.18em}{\S}\hspace{+0.18em} \vV\rij) \cdot (\vR \hspace{+0.18em}{\S}\hspace{+0.18em} \vV\rij) + (\vR \hspace{+0.18em}{\S}\hspace{+0.18em} \vA\rij) \cdot (\vR \hspace{+0.18em}{\S}\hspace{+0.18em} \vR\rij) \hspace{+0.12em} \big] = \hspace{+0.12em}$ Invariante desde S a S'

\end{document}

