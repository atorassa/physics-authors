
\documentclass[11pt]{article}
%\documentclass[a4paper,11pt]{article}
%\documentclass[letterpaper,11pt]{article}
\usepackage[totalwidth=114mm,totalheight=174mm]{geometry}

\usepackage{graphicx}

\usepackage[spanish]{babel}
\usepackage[latin1]{inputenc}
\usepackage[T1]{fontenc}
\usepackage{mathptmx}

\frenchspacing

\parindent=3mm

\usepackage{hyperref}
\hypersetup{colorlinks=true,linkcolor=black}
\hypersetup{bookmarksnumbered=true,pdfstartview=FitH,pdfpagemode=UseNone}
\hypersetup{pdftitle={Sobre los Sistemas de Referencia}}
\hypersetup{pdfauthor={Author: A. Torassa - Editor: W. Babin}}

\setlength{\unitlength}{0.57pt}
\setlength{\arraycolsep}{1.74pt}

\newcommand{\vV}{\mathbf{v}}
\newcommand{\vA}{\mathbf{a}}
\newcommand{\mX}{x}
\newcommand{\mY}{y}
\newcommand{\mZ}{z}
\newcommand{\mT}{t}
\newcommand{\rt}{'}
\newcommand{\rot}{_{o'}}
\newcommand{\tf}{Figura}

\begin{document}

\begin{figure}
\includegraphics{logo6.jpg}
%\includegraphics{logo6.eps}
\end{figure}

\enlargethispage{0em}

\begin{center}

\ \vspace{-1.5em}

{\LARGE Sobre los Sistemas de Referencia}

\vspace{+1.5em}

\normalsize

Alejandro A. Torassa

\smallskip

\footnotesize

Buenos Aires, Argentina, E-mail: atorassa@gmail.com

\smallskip

Licencia Creative Commons Atribuci�n 3.0

\smallskip

(Copyright 2009)

\vspace{+1.2em}

\end{center}

\begin{abstract}

\noindent En este trabajo se establece, por un lado, que cualquier sistema de referencia debe estar fijo a un cuerpo material y, por otro lado, que es posible convenir que cualquier sistema de referencia fijo a un cuerpo material debe ser no rotante.

\medskip

\noindent Keywords: mec�nica cl�sica, observador, cuerpo material, centro de masa, sistema de referencia rotante, sistema de referencia no rotante.

\vspace{+0.3em}

\end{abstract}

\normalsize

{\centering\subsection*{Primera Parte}}

\par Es sabido que a partir de las observaciones de un sistema de \hbox {referencia S} y utilizando las leyes de transformaci�n adecuadas es posible conocer las observaciones de otro sistema de referencia S'.
\smallskip
\par Sin embargo, las observaciones del sistema de referencia S' obtenidas a trav�s del m�todo anterior son observaciones hipot�ticas, puesto que las observaciones reales del sistema de referencia S' son las observaciones hechas por el propio sistema de referencia S'.
\smallskip
\par Seg�n este trabajo, para conocer las observaciones reales de un sistema de referencia es necesario que el sistema de referencia exista. Y para que un sistema de referencia exista es necesario que el sistema de referencia est� fijo a un cuerpo material.
\smallskip
\par En consecuencia, cualquier sistema de referencia debe estar fijo a un cuerpo material.

\newpage \enlargethispage{0em}

{\centering\subsection*{Segunda Parte}}

\par �C�mo debe estar fijo un sistema de referencia a un cuerpo material?
\smallskip
\par Seg�n este trabajo, si se considera que cualquier cuerpo material es un cuerpo formado por una part�cula o por un sistema de part�culas, entonces el origen de coordenadas de cualquier sistema de referencia debe estar fijo al centro de masa de un cuerpo material.
\smallskip
\par Como el centro de masa de cualquier cuerpo material es un punto en el espacio sin rotaci�n, entonces es posible que el origen de coordenadas de un sistema de referencia no rotante est� siempre fijo al centro de masa de cualquier cuerpo material. Sin embargo, no es posible que el origen de coordenadas de un sistema de referencia inercial est� siempre fijo al centro de masa de cualquier cuerpo material.
\smallskip
\par En consecuencia, si el origen de coordenadas de cualquier sistema de referencia debe estar fijo al centro de masa de un cuerpo material, entonces es posible convenir que cualquier sistema de referencia fijo a un cuerpo material debe ser no rotante.

\bigskip

{\centering\subsection*{Conclusiones}}

\par Seg�n este trabajo, cada sistema de referencia no rotante est� definido por un origen de coordenadas fijo al centro de masa de un cuerpo material, y por tres ejes de coordenadas perpendiculares entre s�, donde cada eje es paralelo al correspondiente eje de un sistema de referencia universal definido por cuatro estrellas lejanas muy distantes entre s�.
\smallskip
\par Por otro lado, varias leyes de la f�sica adoptar�an una forma m�s simple si ning�n sistema de referencia fuese un sistema de referencia rotante. Sin embargo, la rotaci�n de un cuerpo material ser�a absoluta; por ejemplo, la rotaci�n de la Tierra ser�a absoluta. Pero, en la teor�a de relatividad, la velocidad de la luz es tambi�n absoluta.
\smallskip
\par En adici�n, seg�n este trabajo, cada cuerpo material es un sistema de referencia no rotante posible. Por lo tanto, cada cuerpo material es tambi�n un observador posible.
\smallskip
\par Por �ltimo, las leyes de la f�sica deben ser las mismas para todos los observadores. Por lo tanto, seg�n este trabajo, las leyes de la f�sica deben tener la misma forma en todos los sistemas de referencia no rotantes.

\newpage \enlargethispage{+0.3em}

{\centering\subsection*{Ap�ndice}}

{\centering\subsection*{Transformaciones de la Mec�nica Cl�sica}}

\par Si cualquier sistema de referencia es un sistema de referencia no rotante, entonces los ejes de un sistema de referencia S y otro S' permanecer�n siempre fijos entre s�. Por lo tanto, se puede convenir, para facilitar los c�lculos, que los ejes de los sistemas de referencia S y S' tengan la misma orientaci�n entre s�, seg�n como muestra la Figura 1.

\vspace{+1.2em}

\begin{center}
\begin{picture}(228,198)
\multiput(75,75)(45,18){2}{\vector(1,0){90}}
\multiput(75,75)(45,18){2}{\vector(0,1){90}}
\multiput(75,75)(45,18){2}{\vector(-1,-1){60}}
\put(72,171){$\mZ$}\put(117,189){$\mZ\rt$}
\put(171,72){$\mX$}\put(216,90){$\mX\rt$}
\put(3,3){$\mY$}\put(45,18){$\mY\rt$}
\put(78,78){$O$}\put(123,96){$O\rt$}
\put(24,96){S}\put(162,141){S'}
\end{picture}
\\* \tf \ 1
\end{center}

\smallskip

\par Se puede pasar de las coordenadas $\mX$, $\mY$, $\mZ$, $\mT$ del sistema de \hbox {referencia S} a las coordenadas $\mX\rt$, $\mY\rt$, $\mZ\rt$, $\mT\rt$ del sistema de referencia S' cuyo origen de coordenadas $O\rt$ se encuentra en la posici�n $\mX\rot$, $\mY\rot$, $\mZ\rot$ con respecto al sistema de referencia S, aplicando las siguientes ecuaciones:
\begin{eqnarray*}
\mX\rt & = & \mX - \mX\rot \\*
\mY\rt & = & \mY - \mY\rot \\*
\mZ\rt & = & \mZ - \mZ\rot \\*
\mT\rt & = & \mT
\end{eqnarray*}
\par De estas ecuaciones, se deduce como se transforman las velocidades y las aceleraciones del sistema de referencia S al sistema de referencia S', que en forma vectorial pueden ser expresadas como sigue:
\begin{eqnarray*}
\vV\rt & = & \vV - \vV\rot \\*
\vA\rt & = & \vA - \vA\rot
\end{eqnarray*}
\noindent donde $\vV\rot$ y $\vA\rot$ es la velocidad y la aceleraci�n respectivamente del sistema de referencia S' con respecto al sistema de referencia S.

\end{document}

