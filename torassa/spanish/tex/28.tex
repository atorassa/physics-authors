
\documentclass[10pt]{article}
%\documentclass[a4paper,10pt]{article}
%\documentclass[letterpaper,10pt]{article}

\usepackage[dvips]{geometry}
\geometry{papersize={123.0mm,150.0mm}}
\geometry{totalwidth=102.0mm,totalheight=114.0mm}

\usepackage[spanish]{babel}
\usepackage[latin1]{inputenc}
\usepackage[T1]{fontenc}
\usepackage{mathptmx}

\frenchspacing

\usepackage{hyperref}
\hypersetup{colorlinks=true,linkcolor=black}
\hypersetup{bookmarksnumbered=true,pdfstartview=FitH,pdfpagemode=UseNone}
\hypersetup{pdftitle={Energ�a Potencial Angular}}
\hypersetup{pdfauthor={Alejandro A. Torassa}}

\setlength{\arraycolsep}{1.74pt}

\newcommand{\mM}{m}
\newcommand{\mU}{U}
\newcommand{\ra}{_a}
\newcommand{\vR}{\mathbf{r}}
\newcommand{\vA}{\mathbf{a}}
\newcommand{\vF}{\mathbf{F}}

\begin{document}

\begin{center}

{\LARGE Energ�a Potencial Angular}

\bigskip \medskip

Alejandro A. Torassa

\bigskip \medskip

\footnotesize

Licencia Creative Commons Atribuci�n 3.0

(2014) Buenos Aires, Argentina

atorassa@gmail.com

\bigskip \smallskip

\small

{\bf Resumen}

\bigskip \smallskip

\parbox{87mm}{Este trabajo presenta una ecuaci�n para calcular la energ�a potencial angular de una part�cula.}

\end{center}

\normalsize

\vspace{-0.30em}

{\centering\subsubsection*{Energ�a Potencial Angular}}

\vspace{+0.75em}

\par La energ�a potencial angular $\mU\ra$ de una part�cula A sobre la cual act�a una fuerza resultante $\vF\ra$, est� dada por:
\begin{eqnarray*}
\mU\ra = - \int (\hspace{+0.03em} \vR \times \vF\ra) \cdot d(\hspace{+0.03em} \vR \times \vR\ra)
\end{eqnarray*}
\noindent donde $\vR$ es un vector posici�n que es constante en magnitud y direcci�n y $\vR\ra$ es la posici�n de la part�cula A.
\medskip
\par Si $\vF\ra$ es constante y como $\vF\ra = \mM\ra \hspace{+0.09em} \vA\ra$, entonces se deduce:
\begin{eqnarray*}
\mU\ra = - \hspace{+0.18em} \mM\ra \hspace{+0.06em} (\hspace{+0.03em} \vR \times \vA\ra) \cdot (\hspace{+0.03em} \vR \times \vR\ra)
\end{eqnarray*}
\noindent donde $\mM\ra$ es la masa de la part�cula A y $\vA\ra$ es la aceleraci�n constante de la part�cula A.

\end{document}

