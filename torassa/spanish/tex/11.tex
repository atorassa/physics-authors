
\documentclass[10pt]{article}
%\documentclass[a4paper,10pt]{article}
%\documentclass[letterpaper,10pt]{article}

\usepackage[dvips]{geometry}
\geometry{papersize={127.0mm,165.1mm}}
\geometry{totalwidth=106.0mm,totalheight=132.9mm}

\usepackage[spanish]{babel}
\usepackage[latin1]{inputenc}
\usepackage[T1]{fontenc}
\usepackage{mathptmx}

\frenchspacing

\usepackage{hyperref}
\hypersetup{colorlinks=true,linkcolor=black}
\hypersetup{bookmarksnumbered=true,pdfstartview=FitH,pdfpagemode=UseNone}
\hypersetup{pdftitle={Sistema de Referencia Central}}
\hypersetup{pdfauthor={Alejandro A. Torassa}}

\setlength{\arraycolsep}{2pt}

\newcommand{\mm}{m}
\newcommand{\mW}{W}
\newcommand{\vR}{\mathbf{r}}
\newcommand{\vV}{\mathbf{v}}
\newcommand{\vA}{\mathbf{a}}
\newcommand{\vF}{\mathbf{F}}
\newcommand{\ri}{_{\scriptstyle \mathit i}}
\newcommand{\ra}{_{\scriptscriptstyle \mathrm A}}
\newcommand{\rc}{^{\scriptscriptstyle \mathrm (c)}}
\newcommand{\rs}{_{\scriptscriptstyle \mathrm {CM}}}
\newcommand{\med}{\raise.5ex\hbox{$\scriptstyle 1$}\kern-.15em/\kern-.15em\lower.25ex\hbox{$\scriptstyle 2$}}

\begin{document}

\ \vspace{+0.3em}

\begin{center}

{\LARGE Sistema de Referencia Central}

\bigskip \medskip

{\large Alejandro A. Torassa}

\bigskip \medskip

\footnotesize

Licencia Creative Commons Atribuci�n 3.0

(2011) Buenos Aires, Argentina

atorassa@gmail.com

\bigskip \smallskip

\small

{\bf Resumen}

\bigskip

\parbox{89mm}{En este trabajo se presenta un sistema de referencia, que puede ser utilizado por cualquier observador (rotante o no rotante) (inercial o no inercial) para describir el comportamiento din�mico (movimiento) de un sistema de part�culas sin necesidad de introducir fuerzas ficticias.}

\vspace{+0.3em}

\end{center}

\normalsize

{\centering\subsection*{Sistema de Referencia Central}}

\par Un sistema de referencia central S$\rc$ es un sistema de referencia no rotante fijo al centro de masa de un sistema de part�culas.

\vspace{+0.6em}

{\centering\subsection*{Ecuaci�n de Movimiento}}

\par En un sistema de part�culas, la aceleraci�n $\vA\ra\rc$ de una part�cula A con respecto al sistema de referencia central est� dada por:
\begin{eqnarray*}
\vA\ra\rc = \frac{\vF\ra}{\mm\ra} - \frac{\vF\rs}{\mm\rs}
\end{eqnarray*}
\noindent donde $\vF\ra$ es la fuerza resultante que act�a sobre la part�cula A, $\mm\ra$ es la masa de la part�cula A, $\vF\rs$ es la fuerza resultante que act�a sobre el centro de masa y $\mm\rs$ es la masa del centro de masa.

\newpage

{\centering\subsection*{Trabajo y Energ�a Cin�tica}}

\par En un sistema de part�culas, el trabajo total $\mW\rc$ realizado por las fuerzas que act�an sobre el sistema de part�culas con respecto al sistema de referencia central est� dado por:
\begin{eqnarray*}
\mW\rc = \sum \: \int{\vF\ri^{\vphantom{\scriptscriptstyle \mathrm (c)}} \cdot d\vR\ri\rc} = \Delta \, \left( \sum \: \med \, {\mm\ri^{\vphantom{\scriptscriptstyle \mathrm (c)}}}{\vV\ri\rc}^2 \right)
\end{eqnarray*}
\noindent donde ${\vF\ri^{\vphantom{\scriptscriptstyle \mathrm (c)}}}$ es la fuerza resultante que act�a sobre la \textit{i}-�sima part�cula, ${\mm\ri^{\vphantom{\scriptscriptstyle \mathrm (c)}}}$ es la masa de la \textit{i}-�sima part�cula, $\vR\ri\rc$ y $\vV\ri\rc$ son la posici�n y la velocidad de la \textit{i}-�sima part�cula con respecto al sistema de referencia central.

\vspace{+0.6em}

{\centering\subsection*{Conservaci�n de Energ�a Cin�tica}}

\par En un sistema de part�culas, si las fuerzas que act�an sobre el sistema de part�culas no realizan trabajo con respecto al sistema de referencia central entonces la energ�a cin�tica total del sistema de part�culas permanece constante con respecto al sistema de referencia central.

\vspace{+0.6em}

{\centering\subsection*{Ap�ndice}}

\par Las transformaciones entre un sistema de referencia central S$\rc$ y otro sistema de referencia S (rotante o no rotante) (inercial o no inercial) son:

\vspace{-0.6em}

\begin{eqnarray*}
\vR\ra\rc & = & (\vR\ra - \vR\rs) \\ \\
\vV\ra\rc & = & (\vV\ra - \vV\rs) \; + \; {\mathbf{\omega}} \times (\vR\ra - \vR\rs) \\ \\
\vA\ra\rc & = & \frac{\vF\ra}{\mm\ra} - \frac{\vF\rs}{\mm\rs}
\end{eqnarray*}
\smallskip
\par \noindent donde ${\mathbf{\omega}}$ es la velocidad angular de rotaci�n del sistema de referencia S con respecto al sistema de referencia central S$\rc$.

\end{document}

