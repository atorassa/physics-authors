
\documentclass[11pt]{article}
%\documentclass[a4paper,11pt]{article}
%\documentclass[letterpaper,11pt]{article}

\usepackage[dvips]{geometry}
\geometry{papersize={127.0mm,165.1mm}}
\geometry{totalwidth=106.0mm,totalheight=129.0mm}

\usepackage[spanish]{babel}
\usepackage[latin1]{inputenc}
\usepackage[T1]{fontenc}
\usepackage{mathptmx}

\frenchspacing

\usepackage{hyperref}
\hypersetup{colorlinks=true,linkcolor=black}
\hypersetup{bookmarksnumbered=true,pdfstartview=FitH,pdfpagemode=UseNone}
\hypersetup{pdftitle={Momento de Inercia}}
\hypersetup{pdfauthor={Alejandro A. Torassa}}

\newcommand{\mM}{m}
\newcommand{\EM}{M}
\newcommand{\EI}{I}
\newcommand{\ri}{_i}
\newcommand{\dos}{^{\:2}}
\newcommand{\ccc}{^{\:cm}}
\newcommand{\vR}{\mathbf{r}}
\newcommand{\rj}{_{\hspace{-0.081em}j}}
\newcommand{\rij}{_{i\hspace{-0.081em}j}}

\begin{document}

\begin{center}

{\fontsize{24}{24}\selectfont Momento de Inercia}

\bigskip \medskip

{\fontsize{12}{12}\selectfont Alejandro A. Torassa}

\bigskip \medskip

\footnotesize

Licencia Creative Commons Atribuci�n 3.0

(2011) Buenos Aires, Argentina

atorassa@gmail.com

\bigskip \medskip

\small

{\bf Resumen}

\bigskip

\parbox{90mm}{En este trabajo una ecuaci�n para calcular el momento de inercia de un sistema de bipart�culas es presentada.}

\end{center}

\normalsize

\bigskip \medskip

\noindent \hspace{+0.6em} El momento de inercia de un sistema de part�culas, est� dado por:
\begin{eqnarray*}
\EI\ri^{\vphantom{\dos}} = \sum_{\scriptscriptstyle i} \; \mM\ri^{\vphantom{\dos}} \, \vR\ri\dos
\end{eqnarray*}

\noindent \hspace{+0.6em} El momento de inercia de un sistema de bipart�culas, est� dado por:
\begin{eqnarray*}
\EI\rij^{\vphantom{\dos}} = \sum_{\scriptscriptstyle i} \, \sum_{\scriptscriptstyle j>i} \; \mM\ri^{\vphantom{\dos}} \, \mM\rj^{\vphantom{\dos}} \left( \vR\ri - \vR\rj \right)^2
\end{eqnarray*}

\noindent \hspace{+0.6em} Un sistema de part�culas forma un sistema de bipart�culas, y desde las ecuaciones anteriores se puede obtener la siguiente relaci�n:
\begin{eqnarray*}
\EI\rij^{\vphantom{\ccc}} = \EM\ri^{\vphantom{\ccc}} \: \EI\ri\ccc
\end{eqnarray*}

\noindent donde $\EI\rij^{\vphantom{\ccc}}$ es el momento de inercia del sistema de bipart�culas, $\EM\ri^{\vphantom{\ccc}}$ es la masa del sistema de part�culas y $\EI\ri\ccc$ es el momento de inercia del sistema de part�culas con respecto al centro de masa.

\end{document}

