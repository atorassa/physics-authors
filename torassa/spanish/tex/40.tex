
\documentclass[10pt,fleqn]{article}
%\documentclass[a4paper,10pt]{article}
%\documentclass[letterpaper,10pt]{article}

\usepackage[dvips]{geometry}
\geometry{papersize={159.0mm,204.0mm}}
\geometry{totalwidth=138.0mm,totalheight=168.0mm}

\usepackage[spanish]{babel}
\usepackage[latin1]{inputenc}
\usepackage[T1]{fontenc}
\usepackage{mathptmx}

\frenchspacing

\usepackage{hyperref}
\hypersetup{colorlinks=true,linkcolor=black}
\hypersetup{bookmarksnumbered=true,pdfstartview=FitH,pdfpagemode=UseNone}
\hypersetup{pdftitle={Un Nuevo Principio de M�nima Acci�n}}
\hypersetup{pdfauthor={Alejandro A. Torassa}}

\setlength{\arraycolsep}{1.74pt}

\newcommand{\mM}{m}
\newcommand{\mL}{L}
\newcommand{\mT}{T}
\newcommand{\mV}{V}
\newcommand{\ri}{_i}
\newcommand{\vR}{\mathbf{r}}
\newcommand{\vV}{\mathbf{v}}
\newcommand{\vA}{\mathbf{a}}
\newcommand{\vF}{\mathbf{F}}
\newcommand{\Hs}{\hspace{+1.50em}}
\newcommand{\rj}{_{\hspace{-0.081em}j}}
\newcommand{\rij}{_{i\hspace{-0.081em}j}}
\newcommand{\med}{\raise.5ex\hbox{$\scriptstyle 1$}\kern-.15em/\kern-.15em\lower.25ex\hbox{$\scriptstyle 2$}}

\begin{document}

\begin{center}

{\LARGE Un Nuevo Principio de M�nima Acci�n}

\bigskip \medskip

Alejandro A. Torassa

\bigskip \medskip

\footnotesize

Licencia Creative Commons Atribuci�n 3.0

(2014) Buenos Aires, Argentina

atorassa@gmail.com

\bigskip \smallskip

\small

{\bf Resumen}

\bigskip

\parbox{108mm}{En mec�nica cl�sica, este trabajo presenta un nuevo principio de m�nima acci�n que es invariante bajo transformaciones entre sistemas de referencia y que puede ser aplicado en cualquier sistema de referencia (rotante o no rotante) (inercial o no inercial) sin necesidad de introducir fuerzas ficticias.}

\end{center}

\normalsize

\vspace{-0.30em}

{\centering\subsubsection*{El Nuevo Principio de M�nima Acci�n}}

\vspace{+0.90em}

\par Si consideramos dos part�culas i y j entonces el nuevo principio de m�nima acci�n es:
\vspace{+0.90em}
\begin{eqnarray*}
\Hs \delta \int_{t_1}^{t_2} \mL\rij \hspace{+0.24em} dt = 0
\end{eqnarray*}
\vspace{+0.30em}
\begin{eqnarray*}
\Hs \delta \int_{t_1}^{t_2} (\mT\rij - \mV\rij) \hspace{+0.24em} dt = 0
\end{eqnarray*}
\vspace{+0.30em}
\begin{eqnarray*}
\Hs \mT\rij = + \hspace{+0.18em} \med \hspace{+0.15em} \mM\ri\hspace{+0.09em}\mM\rj \hspace{+0.12em} \left[ \hspace{+0.12em} (\vV\ri - \vV\rj) \cdot (\vV\ri - \vV\rj) + (\vA\ri - \vA\rj) \cdot (\vR\ri - \vR\rj) \hspace{+0.12em} \right]
\end{eqnarray*}
\vspace{+0.30em}
\begin{eqnarray*}
\Hs \mV\rij = - \hspace{+0.18em} \med \hspace{+0.15em} \mM\ri\hspace{+0.09em}\mM\rj \hspace{+0.03em} \left[ \hspace{+0.12em} 2 \hspace{-0.12em} \int \hspace{-0.12em} \left(\frac{\vF\ri}{\mM\ri} - \frac{\vF\rj}{\mM\rj}\right) \cdot d(\vR\ri - \vR\rj) + \left(\frac{\vF\ri}{\mM\ri} - \frac{\vF\rj}{\mM\rj}\right) \cdot (\vR\ri - \vR\rj) \hspace{+0.12em} \right]
\end{eqnarray*}
\vspace{+0.60em}
\par \noindent donde $\mM\ri$ y $\mM\rj$ son las masas de las part�culas i y j, $\vR\ri$, $\vR\rj$, $\vV\ri$, $\vV\rj$, $\vA\ri$ y $\vA\rj$ son las posiciones, las velocidades y las aceleraciones de las part�culas i y j y $\vF\ri$ y $\vF\rj$ son las fuerzas (conservativas) netas que act�an sobre las part�culas i y j.
\bigskip
\par El Lagrangiano $\mL\rij$ es invariante bajo transformaciones entre sistemas de referencia.
\bigskip
\par El Lagrangiano $\mL\rij$ puede ser aplicado en cualquier sistema de referencia (rotante o no rotante) (inercial o no inercial) sin necesidad de introducir fuerzas ficticias.

\end{document}

