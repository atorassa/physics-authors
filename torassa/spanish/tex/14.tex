
\documentclass[12pt]{article}
%\documentclass[a4paper,12pt]{article}
%\documentclass[letterpaper,12pt]{article}

\usepackage[dvips]{geometry}
\geometry{papersize={127mm,165.1mm}}
\geometry{totalwidth=106mm,totalheight=129.1mm}

\usepackage[spanish]{babel}
\usepackage[latin1]{inputenc}
\usepackage[T1]{fontenc}
\usepackage{mathptmx}

\frenchspacing

\usepackage{hyperref}
\hypersetup{colorlinks=true,linkcolor=black}
\hypersetup{bookmarksnumbered=true,pdfstartview=FitH,pdfpagemode=UseNone}
\hypersetup{pdftitle={Mec�nica Cl�sica (Principio de Energ�a)}}
\hypersetup{pdfsubject={Sobre el principio de energ�a en mec�nica cl�sica.}}
\hypersetup{pdfauthor={Alejandro A. Torassa}}
\hypersetup{pdfkeywords={mec�nica cl�sica, principio, energ�a, part�cula, bipart�cula}}

\setlength{\arraycolsep}{5.4pt}

\newcommand{\mM}{m}
\newcommand{\dos}{^{\,2}}
\newcommand{\vR}{\mathbf{r}}
\newcommand{\vV}{\mathbf{v}}
\newcommand{\vA}{\mathbf{a}}
\newcommand{\ra}{_{\scriptscriptstyle \mathrm A}}
\newcommand{\rab}{_{\scriptscriptstyle \mathrm {AB}}}

\begin{document}

\pagestyle{empty}

\begin{center}

\ \vspace{-0.3em}

{\fontsize{27}{27}\selectfont Mec�nica Cl�sica}

\bigskip \bigskip

{\fontsize{15}{15}\selectfont Principio de Energ�a}

\bigskip \bigskip

{\fontsize{13}{13}\selectfont ( Campo Uniforme )}

\bigskip \bigskip

{\fontsize{14}{14}\selectfont Part�cula}

\end{center}

\vspace{-0.6em}

\begin{eqnarray*}
\hspace{-5.19em}\vA\ra & = & \vA\ra \\ \\
\hspace{-5.19em}\int \vA\ra \cdot d\vR\ra & = & \int \vA\ra \cdot d\vR\ra \\ \\
\hspace{-5.19em}\int \vA\ra^{\vphantom{\dos}} \cdot d\vR\ra^{\vphantom{\dos}} & = & \Delta \; {\textstyle \frac{1}{2}}\vV\ra\dos \\ \\
\hspace{-5.19em}\int \vA\ra \cdot d\vR\ra & = & \Delta \; \vA\ra \cdot \vR\ra \\ \\
\hspace{-5.19em}\Delta \; {\textstyle \frac{1}{2}}\vV\ra\dos & = & \Delta \; \vA\ra^{\vphantom{\dos}} \cdot \vR\ra^{\vphantom{\dos}} \\ \\
\hspace{-5.19em}\Delta \; {\textstyle \frac{1}{2}}\vV\ra\dos - \Delta \; \vA\ra^{\vphantom{\dos}} \cdot \vR\ra^{\vphantom{\dos}} & = & 0 \\ \\
\hspace{-5.19em}\mM\ra^{\vphantom{\dos}} \left( \Delta \; {\textstyle \frac{1}{2}}\vV\ra\dos - \Delta \; \vA\ra^{\vphantom{\dos}} \cdot \vR\ra^{\vphantom{\dos}} \right) & = & 0
\end{eqnarray*}

\newpage

\pagestyle{empty}

\begin{center}

\ \vspace{-0.3em}

{\fontsize{27}{27}\selectfont Mec�nica Cl�sica}

\bigskip \bigskip

{\fontsize{15}{15}\selectfont Principio de Energ�a}

\bigskip \bigskip

{\fontsize{13}{13}\selectfont ( Campo Uniforme )}

\bigskip \bigskip

{\fontsize{14}{14}\selectfont Bipart�cula}

\end{center}

\vspace{-0.6em}

\begin{eqnarray*}
\hspace{-6.00em}\vA\rab & = & \vA\rab \\ \\
\hspace{-6.00em}\int \vA\rab \cdot d\vR\rab & = & \int \vA\rab \cdot d\vR\rab \\ \\
\hspace{-6.00em}\int \vA\rab^{\vphantom{\dos}} \cdot d\vR\rab^{\vphantom{\dos}} & = & \Delta \; {\textstyle \frac{1}{2}}\vV\rab\dos \\ \\
\hspace{-6.00em}\int \vA\rab \cdot d\vR\rab & = & \Delta \; \vA\rab \cdot \vR\rab \\ \\
\hspace{-6.00em}\Delta \; {\textstyle \frac{1}{2}}\vV\rab\dos & = & \Delta \; \vA\rab^{\vphantom{\dos}} \cdot \vR\rab^{\vphantom{\dos}} \\ \\
\hspace{-6.00em}\Delta \; {\textstyle \frac{1}{2}}\vV\rab\dos - \Delta \; \vA\rab^{\vphantom{\dos}} \cdot \vR\rab^{\vphantom{\dos}} & = & 0 \\ \\
\hspace{-6.00em}\mM\rab^{\vphantom{\dos}} \left( \Delta \; {\textstyle \frac{1}{2}}\vV\rab\dos - \Delta \; \vA\rab^{\vphantom{\dos}} \cdot \vR\rab^{\vphantom{\dos}} \right) & = & 0
\end{eqnarray*}

\end{document}

