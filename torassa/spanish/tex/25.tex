
\documentclass[10pt]{article}
%\documentclass[a4paper,10pt]{article}
%\documentclass[letterpaper,10pt]{article}

\usepackage[dvips]{geometry}
\geometry{papersize={127.0mm,140.0mm}}
\geometry{totalwidth=106.6mm,totalheight=104.0mm}

\usepackage[spanish]{babel}
\usepackage[latin1]{inputenc}
\usepackage[T1]{fontenc}
\usepackage{mathptmx}

\frenchspacing

\usepackage{hyperref}
\hypersetup{colorlinks=true,linkcolor=black}
\hypersetup{bookmarksnumbered=true,pdfstartview=FitH,pdfpagemode=UseNone}
\hypersetup{pdftitle={Principio de Conservaci�n}}
\hypersetup{pdfauthor={Alejandro A. Torassa}}

\setlength{\arraycolsep}{1.74pt}

\newcommand{\mM}{m}
\newcommand{\mW}{W}
\newcommand{\ra}{_a}
\newcommand{\ri}{_i}
\newcommand{\vR}{\mathbf{r}}
\newcommand{\vV}{\mathbf{v}}
\newcommand{\vD}{\mathbf{d}}
\newcommand{\vF}{\mathbf{F}}
\newcommand{\vM}{\mathbf{M}}
\newcommand{\med}{\raise.5ex\hbox{$\scriptstyle 1$}\kern-.15em/\kern-.15em\lower.25ex\hbox{$\scriptstyle 2$}}

\begin{document}

\begin{center}

{\LARGE Principio de Conservaci�n}

\bigskip \medskip

Alejandro A. Torassa

\bigskip \medskip

\footnotesize

Licencia Creative Commons Atribuci�n 3.0

(2013) Buenos Aires, Argentina

atorassa@gmail.com

\bigskip \smallskip

\small

{\bf Resumen}

\bigskip

\parbox{81mm}{En mec�nica cl�sica, este trabajo presenta un nuevo principio de conservaci�n para choques el�sticos frontales, que puede ser aplicado en cualquier sistema de referencia inercial.}

\end{center}

\normalsize

\vspace{-0.30em}

{\centering\subsubsection*{Principio de Conservaci�n}}

\vspace{+0.75em}

\par En un sistema aislado de $N$ part�culas, el nuevo principio de conservaci�n para choques el�sticos frontales, est� dado por:
\begin{eqnarray*}
\sum^{\scriptscriptstyle N}_{\scriptscriptstyle i = 1} \hspace{+0.24em} \med \hspace{+0.24em} \mM\ri \hspace{+0.12em} (\vR\ri \times \vV\ri)^{\hspace{+0.03em} 2} \hspace{+0.06em} = \hspace{+0.09em} constante
\end{eqnarray*}
\noindent donde $\mM\ri$ es la masa de la \textit{i}-�sima part�cula, $\vR\ri$ es la posici�n de la \textit{i}-�sima part�cula y $\vV\ri$ es la velocidad de la \textit{i}-�sima part�cula.

\newpage

{\centering\subsubsection*{Ap�ndice}}

\vspace{+0.75em}

{\centering\subsubsection*{Trabajo Angular}}

\vspace{+0.75em}

\par El trabajo angular $\mW\ra$ realizado por un momento constante $\vM\ra$ que act�a sobre una part�cula A, est� dado por:
\vspace{-0.15em}
\begin{eqnarray*}
\mW\ra = \vM\ra \cdot (\vR\ra \times \vD\ra)
\end{eqnarray*}
\noindent donde $\vR\ra$ es la posici�n de la part�cula A, $\vD\ra$ es el vector desplazamiento de la part�cula A y $\vF\ra$ es la fuerza constante que act�a sobre la part�cula A $[\hspace{+0.12em} \vM\ra = (\vR\ra \times \vF\ra) \hspace{+0.12em}]$

\vspace{+1.50em}

{\centering\subsubsection*{Energ�a Cin�tica Angular}}

\vspace{+0.75em}

\par El trabajo angular realizado por el momento resultante que act�a sobre una part�cula A es igual a la variaci�n de la energ�a cin�tica angular de la part�cula A.
\vspace{-0.45em}
\begin{eqnarray*}
\mW\ra = \hspace{+0.09em} \Delta \hspace{+0.24em} \med \hspace{+0.24em} \mM\ra \hspace{+0.12em} (\vR\ra \times \vV\ra)^{\hspace{+0.03em} 2}
\end{eqnarray*}
\par Por lo tanto, si el momento resultante que act�a sobre la part�cula A no realiza trabajo angular entonces la energ�a cin�tica angular de la part�cula A permanece constante.

\end{document}

