
\documentclass[10pt]{article}
%\documentclass[a4paper,10pt]{article}
%\documentclass[letterpaper,10pt]{article}

\usepackage[dvips]{geometry}
\geometry{papersize={123.0mm,162.0mm}}
\geometry{totalwidth=102.0mm,totalheight=126.0mm}

\usepackage[spanish]{babel}
\usepackage[latin1]{inputenc}
\usepackage[T1]{fontenc}
\usepackage{mathptmx}

\frenchspacing

\usepackage{hyperref}
\hypersetup{colorlinks=true,linkcolor=black}
\hypersetup{bookmarksnumbered=true,pdfstartview=FitH,pdfpagemode=UseNone}
\hypersetup{pdftitle={Principio de M�nima Acci�n Angular}}
\hypersetup{pdfauthor={Alejandro A. Torassa}}

\setlength{\arraycolsep}{1.74pt}

\newcommand{\mM}{m}
\newcommand{\mL}{L}
\newcommand{\mT}{T}
\newcommand{\mV}{V}
\newcommand{\ra}{_a}
\newcommand{\vR}{\mathbf{r}}
\newcommand{\vV}{\mathbf{v}}
\newcommand{\vF}{\mathbf{F}}
\newcommand{\med}{\raise.5ex\hbox{$\scriptstyle 1$}\kern-.15em/\kern-.15em\lower.25ex\hbox{$\scriptstyle 2$}}

\begin{document}

\begin{center}

{\Large Principio de M�nima Acci�n Angular}

\bigskip \medskip

Alejandro A. Torassa

\bigskip \medskip

\footnotesize

Licencia Creative Commons Atribuci�n 3.0

(2014) Buenos Aires, Argentina

atorassa@gmail.com

\bigskip \smallskip

\small

{\bf Resumen}

\bigskip

Este trabajo presenta el principio de m�nima acci�n angular.

\end{center}

\normalsize

\vspace{-0.30em}

{\centering\subsubsection*{Principio de M�nima Acci�n Angular}}

\vspace{+0.75em}

\par Si consideramos una part�cula A de masa $\mM\ra$ entonces el principio de m�nima acci�n angular, est� dado por:
\begin{eqnarray*}
\delta \int_{t_1}^{t_2} \med \hspace{+0.24em} \mM\ra \hspace{+0.06em} (\hspace{+0.03em} \vR \times \vV\ra)^{\hspace{+0.03em} 2} \hspace{+0.12em} dt \hspace{+0.12em} + \int_{t_1}^{t_2} (\hspace{+0.03em} \vR \times \vF\ra) \cdot \delta (\hspace{+0.03em} \vR \times \vR\ra) \hspace{+0.18em} dt = 0
\end{eqnarray*}
\noindent donde $\vR$ es un vector posici�n que es constante en magnitud y direcci�n, $\vV\ra$ es la velocidad de la part�cula A, $\vF\ra$ es la fuerza resultante que act�a sobre la part�cula A y $\vR\ra$ es la posici�n de la part�cula A.
\medskip
\par Si $- \hspace{+0.03em} \delta \hspace{+0.09em} \mV\ra = (\hspace{+0.03em} \vR \times \vF\ra) \cdot \delta (\hspace{+0.03em} \vR \times \vR\ra)$ y como $\mT\ra = \med \hspace{+0.24em} \mM\ra \hspace{+0.06em} (\hspace{+0.03em} \vR \times \vV\ra)^{\hspace{+0.03em} 2}$, entonces:
\begin{eqnarray*}
\delta \int_{t_1}^{t_2} (\mT\ra - \mV\ra) \hspace{+0.18em} dt = 0
\end{eqnarray*}
\par Y como $\mL\ra = \mT\ra - \mV\ra$, entonces se obtiene:
\begin{eqnarray*}
\delta \int_{t_1}^{t_2} \mL\ra \hspace{+0.24em} dt = 0
\end{eqnarray*}

\end{document}

