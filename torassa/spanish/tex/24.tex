
\documentclass[10pt]{article}
%\documentclass[a4paper,10pt]{article}
%\documentclass[letterpaper,10pt]{article}

\usepackage[dvips]{geometry}
\geometry{papersize={129.0mm,192.0mm}}
\geometry{totalwidth=106.6mm,totalheight=158.0mm}

\usepackage[spanish]{babel}
\usepackage[latin1]{inputenc}
\usepackage[T1]{fontenc}
\usepackage{mathptmx}

\frenchspacing

\usepackage{hyperref}
\hypersetup{colorlinks=true,linkcolor=black}
\hypersetup{bookmarksnumbered=true,pdfstartview=FitH,pdfpagemode=UseNone}
\hypersetup{pdftitle={Principio de Relatividad}}
\hypersetup{pdfauthor={Alejandro A. Torassa}}

\setlength{\unitlength}{0.57pt}
\setlength{\arraycolsep}{1.74pt}

\newcommand{\mX}{x}
\newcommand{\mY}{y}
\newcommand{\mZ}{z}
\newcommand{\mT}{t}
\newcommand{\rt}{'}
\newcommand{\rot}{_{o'}}
\newcommand{\vR}{\mathbf{r}}
\newcommand{\vV}{\mathbf{v}}
\newcommand{\vA}{\mathbf{a}}

\begin{document}

\begin{center}

{\LARGE Principio de Relatividad}

\bigskip \medskip

Alejandro A. Torassa

\bigskip \medskip

\footnotesize

Licencia Creative Commons Atribuci�n 3.0

(2013) Buenos Aires, Argentina

atorassa@gmail.com

\bigskip \smallskip

\small

{\bf Resumen}

\bigskip

\parbox{81mm}{Este trabajo presenta un principio de relatividad, que establece que las leyes de la f�sica deben solamente tener la misma forma en todos los sistemas de referencia no rotantes.}

\end{center}

\normalsize

\vspace{-0.30em}

{\centering\subsubsection*{Principio de Relatividad}}

\vspace{+0.75em}

\par Cualquier sistema de referencia debe estar fijo a un cuerpo.
\medskip
\par Un sistema de referencia rotante no puede representar en todos los puntos del espacio la velocidad angular de rotaci�n de un cuerpo que gira.
\medskip
\par Cualquier sistema de referencia es un cuerpo r�gido ideal y, seg�n la teor�a de relatividad, ning�n cuerpo puede superar la velocidad de la luz.
\medskip
\par Por lo tanto, ning�n sistema de referencia rotante puede tener la misma velocidad angular de rotaci�n en todos los puntos del espacio, debido a que su velocidad tangencial no puede superar la velocidad de la luz.
\medskip
\par Sin embargo, es posible convenir que cualquier sistema de referencia fijo a un cuerpo debe ser no rotante.
\medskip
\par Las leyes de la f�sica deben ser las mismas para todos los observadores.
\medskip
\par Por lo tanto, seg�n este trabajo, las leyes de la f�sica deben solamente tener la misma forma en todos los sistemas de referencia no rotantes.
\medskip
\par En adici�n, varias leyes de la f�sica adoptar�an una forma m�s simple si ning�n sistema de referencia fuese un sistema de referencia rotante.
\medskip
\par Por �ltimo, todo cuerpo es un observador posible. Por lo tanto, seg�n este trabajo, todo cuerpo es tambi�n un sistema de referencia no rotante posible.

\newpage

{\centering\subsubsection*{Mec�nica Cl�sica}}

\vspace{+0.75em}

\par Si cualquier sistema de referencia es un sistema de referencia no rotante, entonces los ejes de dos sistemas de referencia S y S' permanecer�n siempre fijos entre s�. Por lo tanto, se puede convenir, para facilitar los c�lculos, que los ejes de los sistemas de referencia S y S' tengan la misma orientaci�n entre s�, seg�n como muestra la siguiente figura:

\vspace{+1.2em}

\begin{center}
\begin{picture}(228,198)
\multiput(75,75)(45,18){2}{\vector(1,0){90}}
\multiput(75,75)(45,18){2}{\vector(0,1){90}}
\multiput(75,75)(45,18){2}{\vector(-1,-1){60}}
\put(72,171){$\mZ$}\put(117,189){$\mZ\rt$}
\put(171,72){$\mX$}\put(216,90){$\mX\rt$}
\put(3,3){$\mY$}\put(45,18){$\mY\rt$}
\put(78,78){$O$}\put(123,96){$O\rt$}
\put(24,96){S}\put(162,141){S'}
\end{picture}
\end{center}

\smallskip

\par Se puede pasar de las coordenadas $\mX$, $\mY$, $\mZ$, $\mT$ del sistema de \hbox {referencia S} a las coordenadas $\mX\rt$, $\mY\rt$, $\mZ\rt$, $\mT\rt$ del sistema de referencia S' cuyo origen de coordenadas $O\rt$ se encuentra en la posici�n $\mX\rot$, $\mY\rot$, $\mZ\rot$ respecto al sistema de referencia S, aplicando las siguientes ecuaciones:
\begin{eqnarray*}
\mX\rt & = & \mX - \mX\rot \\*
\mY\rt & = & \mY - \mY\rot \\*
\mZ\rt & = & \mZ - \mZ\rot \\*
\mT\rt & = & \mT
\end{eqnarray*}
\par Desde estas ecuaciones, es posible transformar en forma vectorial las posiciones, las velocidades y las aceleraciones del sistema de referencia S al sistema de referencia S', aplicando las siguientes ecuaciones:
\begin{eqnarray*}
\vR\hspace{+0.06em}\rt & = & \vR - \vR\rot \\
\vV\hspace{+0.06em}\rt & = & \vV - \vV\rot \\
\vA\hspace{+0.06em}\rt & = & \vA - \vA\rot
\end{eqnarray*}
\noindent donde $\vR\rot$, $\vV\rot$ y $\vA\rot$ son la posici�n, la velocidad y la aceleraci�n del sistema de referencia S' respecto al sistema de referencia S.

\end{document}

