
\documentclass[10pt]{article}
%\documentclass[a4paper,10pt]{article}
%\documentclass[letterpaper,10pt]{article}

\usepackage[dvips]{geometry}
\geometry{papersize={130.0mm,168.0mm}}
\geometry{totalwidth=109.0mm,totalheight=132.0mm}

\usepackage[spanish]{babel}
\usepackage[latin1]{inputenc}
\usepackage[T1]{fontenc}
\usepackage{mathptmx}

\frenchspacing

\usepackage{hyperref}
\hypersetup{colorlinks=true,linkcolor=black}
\hypersetup{bookmarksnumbered=true,pdfstartview=FitH,pdfpagemode=UseNone}
\hypersetup{pdftitle={Din�mica Cl�sica de Part�culas}}
\hypersetup{pdfauthor={Alejandro A. Torassa}}

\setlength{\arraycolsep}{1.74pt}

\newcommand{\mT}{t}
\newcommand{\mN}{m}
\newcommand{\mM}{m\,}
\newcommand{\til}{\breve}
\newcommand{\dos}{^{\,2}}
\newcommand{\vR}{\mathbf{r}}
\newcommand{\vV}{\mathbf{v}}
\newcommand{\vA}{\mathbf{a}}
\newcommand{\vF}{\mathbf{F}}
\newcommand{\med}{\raise.5ex\hbox{$\scriptstyle 1$}\kern-.15em/\kern-.15em\lower.25ex\hbox{$\scriptstyle 2$}\:}

\begin{document}

\begin{center}

{\LARGE Din�mica Cl�sica de Part�culas}

\bigskip \medskip

Alejandro A. Torassa

\bigskip \medskip

\footnotesize

Licencia Creative Commons Atribuci�n 3.0

(2013) Buenos Aires, Argentina

atorassa@gmail.com

\bigskip \smallskip

\small

{\bf Resumen}

\bigskip

\parbox{84mm}{Este trabajo presenta una din�mica cl�sica de part�culas, que puede ser aplicada en cualquier sistema de referencia inercial.}

\end{center}

\normalsize

\vspace{-0.30em}

{\centering\subsubsection*{Definiciones}}

\vspace{+1.20em}

\begin{center}
\begin{tabular}{ll}
$\vR$ = posici�n & $\til\vR$ = posici�n no cin�tica \\ \\
$\vV$ = velocidad & $\til\vV$ = velocidad no cin�tica \\ \\
$\vA$ = aceleraci�n & $\til\vA$ = aceleraci�n no cin�tica
\end{tabular}
\end{center}

{\centering\subsubsection*{Relaciones}}

\vspace{-0.60em}

\begin{eqnarray*}
\til\vA = \vF/\mN \;\;\;\;\;\; \rightarrow \;\;\;\;\;\; \til\vA\dos = (\vF/\mN)\dos
\end{eqnarray*}

\begin{eqnarray*}
\til\vV = \int \til\vA \; d\mT \;\;\;\;\;\; \rightarrow \;\;\;\;\;\; \til\vV = \int (\vF/\mN) \; d\mT
\end{eqnarray*}

\begin{eqnarray*}
\med\til\vV\dos = \int \til\vA \; d\til\vR \;\;\;\;\;\; \rightarrow \;\;\;\;\;\; \med\til\vV\dos = \int (\vF/\mN) \; d\til\vR
\end{eqnarray*}

\newpage

{\centering\subsubsection*{Principios}}

\vspace{+0.60em}

\begin{center}
\begin{tabular}{ccccc}
{\makebox(6,30){(1)}} & {\framebox(115,30){$\mM\vR - \mM\til\vR = 0$}} & {\makebox(6,30){$\rightarrow$}} & {\framebox(115,30){$\med\mM\vR\dos - \med\mM\til\vR\dos = 0$}} & {\makebox(6,30){(2)}} \\
& {\makebox(115,21){$\downarrow$}} & & {\makebox(115,21){$\downarrow$}} & \\
{\makebox(6,30){(3)}} & {\framebox(115,30){$\mM\vV - \mM\til\vV = 0$}} & {\makebox(6,30){$\rightarrow$}} & {\framebox(115,30){$\med\mM\vV\dos - \med\mM\til\vV\dos = 0$}} & {\makebox(6,30){(4)}} \\
& {\makebox(115,21){$\downarrow$}} & {\makebox(6,21){$\nearrow$}} & {\makebox(115,21){$\downarrow$}} & \\
{\makebox(6,30){(5)}} & {\framebox(115,30){$\mM\vA - \mM\til\vA = 0$}} & {\makebox(6,30){$\rightarrow$}} & {\framebox(115,30){$\med\mM\vA\dos - \med\mM\til\vA\dos = 0$}} & {\makebox(6,30){(6)}}
\end{tabular}
\end{center}

\vspace{+0.60em}

\par \hspace{+0.69em} Sustituyendo las relaciones en los principios, se obtiene:

\vspace{+0.90em}

\begin{center}
\begin{tabular}{ccccc}
{\makebox(6,30){(1)}} & {\framebox(115,30){$\mM\vR - \mM\til\vR = 0$}} & {\makebox(6,30){$\rightarrow$}} & {\framebox(115,30){$\med\mM\vR\dos - \med\mM\til\vR\dos = 0$}} & {\makebox(6,30){(2)}} \\
& {\makebox(115,21){$\downarrow$}} & & {\makebox(115,21){$\downarrow$}} & \\
{\makebox(6,30){(3)}} & {\framebox(115,30){$\mM\vV - \int \vF \; d\mT = 0$}} & {\makebox(6,30){$\rightarrow$}} & {\framebox(115,30){$\med\mM\vV\dos - \int \vF \; d\til\vR = 0$}} & {\makebox(6,30){(4)}} \\
& {\makebox(115,21){$\downarrow$}} & {\makebox(6,21){$\nearrow$}} & {\makebox(115,21){$\downarrow$}} & \\
{\makebox(6,30){(5)}} & {\framebox(115,30){$\mM\vA - \vF = 0$}} & {\makebox(6,30){$\rightarrow$}} & {\framebox(115,30){$\med\mM\vA\dos - \med(\vF\dos/\mN) = 0$}} & {\makebox(6,30){(6)}}
\end{tabular}
\end{center}

\newpage

{\centering\subsubsection*{Observaciones}}

\vspace{+0.60em}

\par La ecuaci�n (1) est� relacionada con el centro de masa.
\bigskip
\par La ecuaci�n (2) est� relacionada con el momento de inercia.
\bigskip
\par La ecuaci�n (3) est� relacionada con el impulso y el momentum lineal.
\bigskip
\par La ecuaci�n (4) est� relacionada con el trabajo y la energ�a.
\bigskip
\par La ecuaci�n (5) est� relacionada con las fuerzas (en forma vectorial)
\bigskip
\par La ecuaci�n (6) est� relacionada con las fuerzas (en forma escalar)
\bigskip
\par Por �ltimo, desde la ecuaci�n (5) se deduce que la aceleraci�n $\vA$ de una part�cula, est� dada por:
\begin{eqnarray*}
\vA = \vF/\mN
\end{eqnarray*}
\noindent donde $\vF$ es la fuerza resultante que act�a sobre la part�cula y $\mN$ es la masa de la part�cula.

\vspace{+0.60em}

{\centering\subsubsection*{Bibliograf�a}}

\vspace{+0.30em}

\par \textbf{A. Einstein}, Sobre la Teor�a de la Relatividad Especial y General.
\bigskip
\par \textbf{E. Mach}, La Ciencia de la Mec�nica.
\bigskip
\par \textbf{R. Resnick y D. Halliday}, F�sica.
\bigskip
\par \textbf{J. Kane y M. Sternheim}, F�sica.
\bigskip
\par \textbf{H. Goldstein}, Mec�nica Cl�sica.
\bigskip
\par \textbf{L. Landau y E. Lifshitz}, Mec�nica.

\end{document}

