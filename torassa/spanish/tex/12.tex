
\documentclass[10pt]{article}
%\documentclass[a4paper,10pt]{article}
%\documentclass[letterpaper,10pt]{article}

\usepackage[dvips]{geometry}
\geometry{papersize={127.0mm,165.1mm}}
\geometry{totalwidth=106.6mm,totalheight=129.6mm}

\usepackage[spanish]{babel}
\usepackage[latin1]{inputenc}
\usepackage[T1]{fontenc}
\usepackage{mathptmx}

\frenchspacing

\usepackage{hyperref}
\hypersetup{colorlinks=true,linkcolor=black}
\hypersetup{bookmarksnumbered=true,pdfstartview=FitH,pdfpagemode=UseNone}
\hypersetup{pdftitle={Principio de Relatividad}}
\hypersetup{pdfauthor={Alejandro A. Torassa}}

\begin{document}

\begin{center}

{\LARGE Principio de Relatividad}

\bigskip \medskip

{\large Alejandro A. Torassa}

\bigskip \medskip

\footnotesize

Licencia Creative Commons Atribuci�n 3.0

(2011) Buenos Aires, Argentina

atorassa@gmail.com

\bigskip \smallskip

\small

{\bf Resumen}

\bigskip

\parbox{81mm}{Este trabajo presenta un principio de relatividad, que establece que las leyes de la f�sica solamente deben tener la misma forma en todos los sistemas de referencia no rotantes.}

\end{center}

\normalsize

\vspace{-0.60em}

{\centering\subsection*{Primera Parte}}

\par Cualquier sistema de referencia debe estar fijo a un cuerpo.
\medskip
\par Es posible convenir que cualquier sistema de referencia fijo a un cuerpo debe ser no rotante.
\medskip
\par Por lo tanto, las leyes de la f�sica solamente deben tener la misma forma en todos los sistemas de referencia no rotantes.

\vspace{+0.99em}

{\centering\subsection*{Segunda Parte}}

\par Un sistema de referencia rotante no puede representar en todos los puntos del espacio la velocidad angular de rotaci�n de un cuerpo que gira.
\medskip
\par Cualquier sistema de referencia es un cuerpo r�gido ideal y, seg�n la teor�a de relatividad, ning�n cuerpo puede superar la velocidad de la luz.
\medskip
\par Por lo tanto, ning�n sistema de referencia rotante puede tener la misma velocidad angular de rotaci�n en todos los puntos del espacio, debido a que su velocidad tangencial no puede superar la velocidad de la luz.

\end{document}

