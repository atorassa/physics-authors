
\documentclass[10pt]{article}
%\documentclass[a4paper,10pt]{article}
%\documentclass[letterpaper,10pt]{article}

\usepackage[dvips]{geometry}
\geometry{papersize={138.0mm,165.0mm}}
\geometry{totalwidth=117.0mm,totalheight=129.0mm}

\usepackage[spanish]{babel}
\usepackage[latin1]{inputenc}
\usepackage[T1]{fontenc}
\usepackage{mathptmx}

\frenchspacing

\usepackage{hyperref}
\hypersetup{colorlinks=true,linkcolor=black}
\hypersetup{bookmarksnumbered=true,pdfstartview=FitH,pdfpagemode=UseNone}
\hypersetup{pdftitle={Una Terna de Ecuaciones Invariantes}}
\hypersetup{pdfauthor={Alejandro A. Torassa}}

\setlength{\arraycolsep}{1.74pt}

\newcommand{\ra}{_a}
\newcommand{\rb}{_b}
\newcommand{\vR}{\mathbf{r}}
\newcommand{\vV}{\mathbf{v}}
\newcommand{\vA}{\mathbf{a}}

\begin{document}

\begin{center}

{\LARGE Una Terna de Ecuaciones Invariantes}

\bigskip \medskip

Alejandro A. Torassa

\bigskip \medskip

\footnotesize

Licencia Creative Commons Atribuci�n 3.0

(2014) Buenos Aires, Argentina

atorassa@gmail.com

\bigskip \smallskip

\small

{\bf Resumen}

\bigskip

\parbox{69mm}{En mec�nica cl�sica, este trabajo presenta una terna de ecuaciones, que son invariantes bajo transformaciones entre sistemas de referencia.}

\end{center}

\normalsize

\vspace{-0.30em}

{\centering\subsubsection*{Terna de Ecuaciones Invariantes}}

\vspace{+1.20em}

\par Dadas dos part�culas A y B entonces la terna de ecuaciones invariantes es:
\vspace{+0.60em}
\begin{eqnarray*}
\hspace{-4.20em} (\vR\ra - \vR\rb) \cdot (\vR\ra - \vR\rb) = invariante \\ \\
\hspace{-4.20em} (\vR\ra - \vR\rb) \cdot (\vV\ra - \vV\rb) = invariante \\ \\
\hspace{-4.20em} (\vV\ra - \vV\rb) \cdot (\vV\ra - \vV\rb) + (\vR\ra - \vR\rb) \cdot (\vA\ra - \vA\rb) = invariante
\end{eqnarray*}
\vspace{-0.30em}
\par \noindent donde $\vR\ra$ y $\vR\rb$ son las posiciones de las part�culas A y B, $\vV\ra$ y $\vV\rb$ son las velocidades de las part�culas A y B y $\vA\ra$ y $\vA\rb$ son las aceleraciones de las part�culas A y B.

\end{document}

