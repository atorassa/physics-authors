
\documentclass[10pt]{article}
%\documentclass[a4paper,10pt]{article}
%\documentclass[letterpaper,10pt]{article}

\usepackage[dvips]{geometry}
\geometry{papersize={138.0mm,174.0mm}}
\geometry{totalwidth=117.0mm,totalheight=138.0mm}

\usepackage[spanish]{babel}
\usepackage[latin1]{inputenc}
\usepackage[T1]{fontenc}
\usepackage{mathptmx}

\frenchspacing

\usepackage{hyperref}
\hypersetup{colorlinks=true,linkcolor=black}
\hypersetup{bookmarksnumbered=true,pdfstartview=FitH,pdfpagemode=UseNone}
\hypersetup{pdftitle={Energ�a Cin�tica Angular}}
\hypersetup{pdfauthor={Alejandro A. Torassa}}

\setlength{\arraycolsep}{1.74pt}

\newcommand{\mM}{m}
\newcommand{\mI}{I}
\newcommand{\ra}{_a}
\newcommand{\vR}{\mathbf{r}}
\newcommand{\vV}{\mathbf{v}}
\newcommand{\aV}{\mathbf{\omega}}
\newcommand{\med}{\raise.5ex\hbox{$\scriptstyle 1$}\kern-.15em/\kern-.15em\lower.25ex\hbox{$\scriptstyle 2$}}

\begin{document}

\begin{center}

{\LARGE Energ�a Cin�tica Angular}

\bigskip \medskip

Alejandro A. Torassa

\bigskip \medskip

\footnotesize

Licencia Creative Commons Atribuci�n 3.0

(2014) Buenos Aires, Argentina

atorassa@gmail.com

\bigskip \smallskip

\small

{\bf Resumen}

\bigskip

\parbox{87.3mm}{Este trabajo presenta una ecuaci�n alternativa para calcular la energ�a cin�tica angular de una part�cula que describe un movimiento circular.}

\end{center}

\normalsize

\vspace{-0.60em}

{\centering\subsubsection*{Energ�a Cin�tica Angular}}

\vspace{+0.60em}

\par La energ�a cin�tica angular de una part�cula A de masa $\mM\ra$, est� dada por:
\begin{eqnarray*}
\med \hspace{+0.24em} \mM\ra \hspace{+0.06em} (\hspace{+0.03em} \vR \times \vV\ra)^{\hspace{+0.03em} 2}
\end{eqnarray*}
\noindent donde $\vR$ es un vector posici�n que es constante en magnitud y direcci�n y $\vV\ra$ es la velocidad de la part�cula A.
\medskip
\par Si la part�cula A tiene una velocidad angular $\aV\ra$ y como $\vV\ra = \aV\ra \times \vR\ra$, se obtiene:
\begin{eqnarray*}
\med \hspace{+0.24em} \mM\ra \hspace{+0.06em} (\hspace{+0.03em} \vR \times (\aV\ra \times \vR\ra))^{\hspace{+0.03em} 2}
\end{eqnarray*}
\par Si el vector posici�n $\vR$ es paralelo a la velocidad angular $\aV\ra$, entonces se deduce:
\begin{eqnarray*}
\med \hspace{+0.24em} \mM\ra^{\vphantom{2}} \hspace{+0.06em} \vR\ra^{\hspace{+0.09em} 2} \hspace{+0.09em} (\hspace{+0.03em} \vR \cdot \aV\ra^{\vphantom{2}})^{\hspace{+0.03em} 2}
\end{eqnarray*}
\par Finalmente, como $\mM\ra^{\vphantom{2}} \hspace{+0.06em} \vR\ra^{\hspace{+0.09em} 2}$ es el momento de inercia $\mI\ra$ de la part�cula A, se tiene:
\begin{eqnarray*}
\med \hspace{+0.24em} \mI\ra \hspace{+0.06em} (\hspace{+0.03em} \vR \cdot \aV\ra)^{\hspace{+0.03em} 2}
\end{eqnarray*}

\end{document}

