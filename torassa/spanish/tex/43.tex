
\documentclass[10pt]{article}
%\documentclass[a4paper,10pt]{article}
%\documentclass[letterpaper,10pt]{article}

\usepackage[dvips]{geometry}
\geometry{papersize={148.2mm,204.0mm}}
\geometry{totalwidth=127.2mm,totalheight=168.0mm}

\usepackage[spanish]{babel}
\usepackage[latin1]{inputenc}
\usepackage[T1]{fontenc}
\usepackage{mathptmx}

\usepackage{hyperref}
\hypersetup{colorlinks=true,linkcolor=black}
\hypersetup{bookmarksnumbered=true,pdfstartview=FitH,pdfpagemode=UseNone}
\hypersetup{pdftitle={Un Principio de Conservaci�n de la Energ�a Relacional}}
\hypersetup{pdfauthor={Alejandro A. Torassa}}

\setlength{\arraycolsep}{1.74pt}

\begin{document}

\begin{center}

{\Large Un Principio de Conservaci�n de la Energ�a Relacional}

\bigskip \medskip

Alejandro A. Torassa

\bigskip \medskip

\footnotesize

Licencia Creative Commons Atribuci�n 3.0

(2014) Buenos Aires, Argentina

atorassa@gmail.com

\bigskip \smallskip

\small

{\bf Resumen}

\bigskip

\parbox{90mm}{En mec�nica cl�sica, este trabajo presenta un principio de conservaci�n de la energ�a relacional que puede ser aplicado en cualquier sistema de referencia sin necesidad de introducir fuerzas ficticias.}

\end{center}

\normalsize

\vspace{-0.30em}

{\centering\subsubsection*{Principio de Conservaci�n}}

\vspace{+0.90em}

\par La energ�a cin�tica $K$ de un sistema de {\small N} part�culas de masa total $M$, est� dada por:
\vspace{+0.45em}
\begin{eqnarray*}
K = \hspace{+0.06em} \sum_{i={\scriptscriptstyle 1}}^{\mathrm N} \hspace{+0.18em} \sum_{j>i}^{\mathrm N} \hspace{+0.12em} \frac{m_i \hspace{+0.09em} m_j}{M} \hspace{+0.18em} ( \hspace{+0.12em} \dot{r}_{ij} \, \dot{r}_{ij} + \ddot{r}_{ij} \, r_{ij} \hspace{+0.09em} )
\end{eqnarray*}
\vspace{+0.15em}
\par El principio de conservaci�n de la energ�a relacional establece que en un sistema aislado de {\small N} part�culas que est� sujeto solamente a fuerzas conservativas (de acci�n y reacci�n y proporcionales a $1/r^2$) la energ�a relacional del sistema de part�culas permanece constante.
\vspace{-0.60em}
\begin{eqnarray*}
K + U = constante
\end{eqnarray*}
\vspace{-0.45em}
\par \noindent donde $r_{ij}=|\vec{r}_i - \vec{r}_j|$, $\dot{r}_{ij}=d|\vec{r}_i - \vec{r}_j|/dt$, $\ddot{r}_{ij}=d^2|\vec{r}_i - \vec{r}_j|/dt^2$, $\vec{r}_i$ y $\vec{r}_j$ son las posiciones de las part�culas \textit{i}-�sima y \textit{j}-�sima, $m_i$ y $m_j$ son las masas de las part�culas \textit{i}-�sima y \textit{j}-�sima. $U$ es la energ�a potencial interna del sistema aislado de part�culas.

\vspace{+1.50em}

{\centering\subsubsection*{Bibliograf�a}}

\vspace{+1.20em}

{\small

\par \hspace{-0.36em} E. Schr\"{o}dinger. Die Erf\"{u}llbarkeit der Relativit\"{a}tsforderung in der klassischen Mechanik. Annalen der Physik, 1925.
\medskip
\par \hspace{-0.36em} A. K. T. Assis. Relational Mechanics and Implementation of Mach's Principle with Weber's Gravitational Force. Apeiron, 2014.
\medskip
\par \hspace{-0.36em} V. W. Hughes, H. G. Robinson, and V. Beltran-Lopez. Upper Limit for the Anisotropy of Inertial Mass from Nuclear Resonance Experiments. Physical Review Letters, 1960.

}

\end{document}

