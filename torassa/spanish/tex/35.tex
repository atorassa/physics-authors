
\documentclass[10pt]{article}
%\documentclass[a4paper,10pt]{article}
%\documentclass[letterpaper,10pt]{article}

\usepackage[dvips]{geometry}
\geometry{papersize={135.0mm,186.0mm}}
\geometry{totalwidth=114.0mm,totalheight=150.0mm}

\usepackage[spanish]{babel}
\usepackage[latin1]{inputenc}
\usepackage[T1]{fontenc}
\usepackage{mathptmx}

\frenchspacing

\usepackage{hyperref}
\hypersetup{colorlinks=true,linkcolor=black}
\hypersetup{bookmarksnumbered=true,pdfstartview=FitH,pdfpagemode=UseNone}
\hypersetup{pdftitle={Una Definici�n de Trabajo}}
\hypersetup{pdfauthor={Alejandro A. Torassa}}

\setlength{\arraycolsep}{1.74pt}

\newcommand{\mM}{m}
\newcommand{\mW}{W}
\newcommand{\mK}{K}
\newcommand{\mU}{U}
\newcommand{\ra}{_a}
\newcommand{\rb}{_b}
\newcommand{\rab}{_{ab}}
\newcommand{\vR}{\mathbf{r}}
\newcommand{\vV}{\mathbf{v}}
\newcommand{\vA}{\mathbf{a}}
\newcommand{\vF}{\mathbf{F}}

\begin{document}

\begin{center}

{\LARGE Una Definici�n de Trabajo}

\bigskip \medskip

Alejandro A. Torassa

\bigskip \medskip

\footnotesize

Licencia Creative Commons Atribuci�n 3.0

(2014) Buenos Aires, Argentina

atorassa@gmail.com

\bigskip \smallskip

\small

{\bf Resumen}

\bigskip

\parbox{90mm}{En mec�nica cl�sica, este trabajo presenta una definici�n de trabajo, que puede ser usada en cualquier sistema de referencia (rotante o no rotante) (inercial o no inercial) sin necesidad de introducir fuerzas ficticias.}

\end{center}

\normalsize

\vspace{-0.60em}

{\centering\subsubsection*{Definici�n de Trabajo}}

\vspace{+0.60em}

\par Si consideramos dos part�culas A y B entonces el trabajo total $\mW\rab$ realizado por las fuerzas $\vF\ra$ y $\vF\rb$ que act�an sobre las part�culas A y B respectivamente es:
\par \vspace{-0.30em}
{\fontsize{8}{8}\selectfont\begin{eqnarray*}
\mW\rab = \frac{1}{2} \hspace{+0.24em} \mM\ra\mM\rb \left[ \hspace{+0.12em} 2 \hspace{-0.12em} \int_1^{\hspace{+0.09em} 2} \hspace{-0.12em} \left(\frac{\vF\ra}{\mM\ra} - \frac{\vF\rb}{\mM\rb}\right) \cdot d(\vR\ra - \vR\rb) + \hspace{+0.03em} \Delta \left(\frac{\vF\ra}{\mM\ra} - \frac{\vF\rb}{\mM\rb}\right) \cdot (\vR\ra - \vR\rb) \hspace{+0.12em} \right]
\end{eqnarray*}}
\vspace{-0.60em}
\par \noindent donde $\mM\ra$ y $\mM\rb$ son las masas de las part�culas A y B y $\vR\ra$ y $\vR\rb$ son las posiciones de las part�culas A y B.
\medskip
\par El trabajo total $\mW\rab$ es igual al cambio en la energ�a cin�tica.
\vspace{+0.03em}
{\fontsize{8}{8}\selectfont\begin{eqnarray*}
\mW\rab = \Delta \hspace{+0.24em} \frac{1}{2} \hspace{+0.24em} \mM\ra\mM\rb \left[(\vV\ra - \vV\rb)^{\hspace{+0.03em} 2} + (\vA\ra - \vA\rb) \cdot (\vR\ra - \vR\rb)\right]
\end{eqnarray*}}
\vspace{-0.90em}
\par \noindent donde $\vV\ra$ y $\vV\rb$ son las velocidades de las part�culas A y B y $\vA\ra$ y $\vA\rb$ son las aceleraciones de las part�culas A y B.
\medskip
\par Por lo tanto, la energ�a cin�tica $\mK\rab$ de las part�culas A y B es:
{\fontsize{8}{8}\selectfont\begin{eqnarray*}
\mK\rab = \frac{1}{2} \hspace{+0.24em} \mM\ra\mM\rb \left[(\vV\ra - \vV\rb)^{\hspace{+0.03em} 2} + (\vA\ra - \vA\rb) \cdot (\vR\ra - \vR\rb)\right]
\end{eqnarray*}}
\vspace{-0.60em}
\medskip
\par Y la energ�a potencial $\mU\rab$ de las part�culas A y B es:
\vspace{+0.30em}
{\fontsize{8}{8}\selectfont\begin{eqnarray*}
\Delta \hspace{+0.24em} \mU\rab = - \frac{1}{2} \hspace{+0.24em} \mM\ra\mM\rb \left[ \hspace{+0.12em} 2 \hspace{-0.12em} \int_1^{\hspace{+0.09em} 2} \hspace{-0.12em} \left(\frac{\vF\ra}{\mM\ra} - \frac{\vF\rb}{\mM\rb}\right) \cdot d(\vR\ra - \vR\rb) + \hspace{+0.03em} \Delta \left(\frac{\vF\ra}{\mM\ra} - \frac{\vF\rb}{\mM\rb}\right) \cdot (\vR\ra - \vR\rb) \hspace{+0.12em} \right]
\end{eqnarray*}}

\end{document}

