
\documentclass[12pt]{article}
%\documentclass[a4paper,12pt]{article}
%\documentclass[letterpaper,12pt]{article}

\usepackage[dvips]{geometry}
\geometry{papersize={153mm,198mm}}
\geometry{totalwidth=132mm,totalheight=162mm}

\usepackage[spanish]{babel}
\usepackage[latin1]{inputenc}
\usepackage[T1]{fontenc}
\usepackage{mathptmx}

\frenchspacing

\usepackage{hyperref}
\hypersetup{colorlinks=true,linkcolor=black}
\hypersetup{bookmarksnumbered=true,pdfstartview=FitH,pdfpagemode=UseNone}
\hypersetup{pdftitle={� Es el principio de energ�a una tautolog�a ?}}
\hypersetup{pdfauthor={Alejandro A. Torassa}}

\setlength{\arraycolsep}{4.5pt}

\newcommand{\mM}{m}
\newcommand{\mT}{T}
\newcommand{\mV}{V}
\newcommand{\dos}{^{\,2}}
\newcommand{\ct}{constante}
\newcommand{\vR}{\mathbf{r}}
\newcommand{\vV}{\mathbf{v}}
\newcommand{\vA}{\mathbf{a}}
\newcommand{\ep}{\hspace{+3.6em}}
\newcommand{\eq}{\hspace{+4.2em}}
\newcommand{\ra}{_{\scriptscriptstyle \mathrm A}}

\begin{document}

\begin{center}

{\fontsize{18}{18}\selectfont � Es el principio de energ�a una tautolog�a ?}

\bigskip \bigskip

{\fontsize{12}{12}\selectfont Alejandro A. Torassa}

\bigskip \bigskip

\footnotesize

Licencia Creative Commons Atribuci�n 3.0

(2011) Buenos Aires, Argentina

atorassa@gmail.com

\bigskip \bigskip

\small

{\bf Resumen}

\bigskip

\parbox{102mm}{Este trabajo muestra que es posible obtener el principio de energ�a a partir de la aceleraci�n de una part�cula.}

\end{center}

\normalsize

\bigskip \bigskip

\noindent En mec�nica cl�sica, si consideramos un campo de fuerzas (uniforme o no uniforme) en el que la aceleraci�n $\vA\ra$ de una part�cula A es constante, entonces

\vspace{-0.3em}

\begin{eqnarray*}
\ep {\vphantom{\int}} \vA\ra & = & \vA\ra \\
\ep \int_a^b \vA\ra \cdot d\vR\ra & = & \int_a^b \vA\ra \cdot d\vR\ra \\
\ep {\vphantom{\int}} \Delta \; {\textstyle \frac{1}{2}} \; \vV\ra\dos & = & \Delta \; \vA\ra \cdot \vR\ra \\
\ep {\vphantom{\int}} \Delta \; {\textstyle \frac{1}{2}} \; \vV\ra\dos - \, \Delta \; \vA\ra^{\vphantom{\dos}} \cdot \vR\ra^{\vphantom{\dos}} & = & 0 \\
\ep {\vphantom{\int}} \mM\ra^{\vphantom{\dos}} \left( \Delta \; {\textstyle \frac{1}{2}} \; \vV\ra\dos - \, \Delta \; \vA\ra^{\vphantom{\dos}} \cdot \vR\ra^{\vphantom{\dos}} \right) & = & 0 \\
\ep {\vphantom{\int}} \Delta \; \mT\ra + \, \Delta \; \mV\ra & = & 0 \eq {\hphantom{\ct}} \mT\ra = {\textstyle \frac{1}{2}} \; \mM\ra^{\vphantom{\dos}}\vV\ra\dos \\
\ep {\vphantom{\int}} \mT\ra + \, \mV\ra & = & \ct \eq {\hphantom{0}} \mV\ra = - \; \mM\ra^{\vphantom{\dos}} \; \vA\ra^{\vphantom{\dos}} \cdot \vR\ra^{\vphantom{\dos}}
\end{eqnarray*}

\vspace{+0.9em}

\noindent Si $\vA\ra$ no es constante pero $\vA\ra$ es funci�n de $\vR\ra$ entonces se obtiene el mismo resultado, aun si la segunda ley de Newton no fuese v�lida.

\end{document}

