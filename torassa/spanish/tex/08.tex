
\documentclass[11pt]{article}
%\documentclass[a4paper,11pt]{article}
%\documentclass[letterpaper,11pt]{article}

\usepackage[dvips]{geometry}
\geometry{papersize={127.0mm,165.1mm}}
\geometry{totalwidth=106.5mm,totalheight=126.0mm}

\usepackage[spanish]{babel}
\usepackage[latin1]{inputenc}
\usepackage[T1]{fontenc}
\usepackage{mathptmx}

\frenchspacing

\usepackage{hyperref}
\hypersetup{colorlinks=true,linkcolor=black}
\hypersetup{bookmarksnumbered=true,pdfstartview=FitH,pdfpagemode=UseNone}
\hypersetup{pdftitle={Sobre Nuevas Leyes de Movimiento para un Cuerpo Puntual en Mec�nica Cl�sica}}
\hypersetup{pdfauthor={Alejandro A. Torassa}}

\setlength{\arraycolsep}{2pt}

\newcommand{\rt}{'}
\newcommand{\mm}{m}
\newcommand{\rot}{_{o'}}
\newcommand{\vA}{\mathbf{a}}
\newcommand{\vF}{\mathbf{F}}
\newcommand{\ra}{_{\mbox {\scriptsize A}}}
\newcommand{\rb}{_{\mbox {\scriptsize B}}}
\newcommand{\rs}{_{\mbox {\scriptsize S}}}

\begin{document}

\begin{center}

{\Large Sobre Nuevas Leyes de Movimiento para un \\ Cuerpo Puntual en Mec�nica Cl�sica}

\bigskip \medskip

{\large Alejandro A. Torassa}

\bigskip \medskip

\scriptsize

Licencia Creative Commons Atribuci�n 3.0

(2010) Buenos Aires, Argentina

atorassa@gmail.com

\bigskip \medskip

\small

{\bf Resumen}

\bigskip

\parbox{89mm}{Este trabajo presenta nuevas leyes de movimiento para un cuerpo puntual en mec�nica cl�sica, que pueden ser aplicadas en cualquier sistema de referencia no rotante (inercial o no inercial) sin necesidad de introducir fuerzas ficticias.}

\vspace{+0.18em}

\end{center}

\normalsize

{\centering\subsection*{Introducci�n}}

\par Es sabido que la primera y segunda ley de Newton s�lo pueden ser aplicadas en un sistema de referencia no inercial si se introducen fuerzas ficticias. Pero, a diferencia de las fuerzas reales, las fuerzas ficticias no son causadas por la interacci�n entre los cuerpos.
\smallskip
\par Sin embargo, este trabajo presenta nuevas leyes de movimiento para un cuerpo puntual en mec�nica cl�sica, que pueden ser aplicadas en cualquier sistema de referencia no rotante (inercial o no inercial) sin necesidad de introducir fuerzas ficticias.
\smallskip
\par En este trabajo se asume que las fuerzas pueden actuar sobre un sistema de referencia debido a que cualquier sistema de referencia est� fijo a un cuerpo.

\newpage \baselineskip=13.5pt \enlargethispage{+0.36em}

{\centering\subsection*{Leyes de Movimiento}}

\par Primera nueva ley de movimiento: Las fuerzas que act�an sobre un cuerpo puntual A y las fuerzas que act�an sobre un sistema de referencia S pueden cambiar el estado de movimiento del cuerpo \hbox {puntual A} con respecto al sistema de referencia S.
\smallskip
\par Segunda nueva ley de movimiento: La aceleraci�n $\vA\ra$ de un cuerpo puntual A con respecto a un sistema de referencia S (no rotante) fijo a un cuerpo puntual S, est� dada por la siguiente ecuaci�n:
\begin{eqnarray*}
\vA\ra = \frac{\sum \vF\ra}{\mm\ra} - \frac{\sum \vF\rs}{\mm\rs}
\end{eqnarray*}
\noindent donde $\sum \vF\ra$ es la suma de las fuerzas que act�an sobre el cuerpo puntual A, $\mm\ra$ es la masa del cuerpo puntual A, $\sum \vF\rs$ es la suma de las fuerzas que act�an sobre el cuerpo puntual S y $\mm\rs$ es la masa del cuerpo puntual S.

\vspace{+0.15em}

{\centering\subsection*{Observaciones}}

\par En contradicci�n con la primera y segunda ley de Newton, de la ecuaci�n anterior se deduce que el cuerpo puntual A puede estar acelerado aun si sobre el cuerpo puntual A no act�a fuerza alguna y tambi�n que el cuerpo puntual A puede no estar acelerado (estado de reposo o de movimiento rectil�neo uniforme) aun si sobre el cuerpo puntual A act�a una fuerza no equilibrada.
\smallskip
\par Finalmente, de la ecuaci�n anterior se deduce que la primera y segunda ley de Newton son v�lidas en el sistema de referencia S \hbox {solamente} si la suma de las fuerzas que act�an sobre el sistema de referencia S (cuerpo puntual S) es igual a cero.

\newpage \baselineskip=14.7pt

{\centering\subsection*{Ap�ndice}}

{\centering\subsection*{Comportamiento Din�mico de los Cuerpos Puntuales}}

\par Seg�n la segunda ley de Newton el comportamiento de dos cuerpos puntuales A y B est� determinado para un sistema de referen- \hbox {cia S} (inercial) por las siguientes ecuaciones:
\begin{eqnarray}
\sum \vF\ra = \mm\ra\vA\ra \\
\sum \vF\rb = \mm\rb\vA\rb
\end{eqnarray}
\noindent o sea:
\begin{eqnarray}
\frac{\sum \vF\ra}{\mm\ra} - \vA\ra = 0 \\
\frac{\sum \vF\rb}{\mm\rb} - \vA\rb = 0
\end{eqnarray}
\par Igualando las ecuaciones (3) y (4), se obtiene:
\begin{eqnarray}
\frac{\sum \vF\ra}{\mm\ra} - \vA\ra = \frac{\sum \vF\rb}{\mm\rb} - \vA\rb
\end{eqnarray}
\par Por lo tanto, el comportamiento de los cuerpos puntuales A y B est� determinado ahora para el sistema de referencia S por la ecua- \hbox {ci�n (5)}.
\smallskip
\par Ahora, si se pasa la ecuaci�n (5) del sistema de referencia S a otro sistema de referencia no rotante S' (inercial o no inercial), utilizando la transformaci�n de la cinem�tica: ($\vA\rt = \vA - \vA\rot$) y las transformaciones de la din�mica: ($\vF\rt = \vF$) y ($\mm\rt = \mm$), se deduce:
\begin{eqnarray}
\frac{\sum {\vF\ra}\rt}{{\mm\ra}\rt} - {\vA\ra}\rt = \frac{\sum {\vF\rb}\rt}{{\mm\rb}\rt} - {\vA\rb}\rt
\end{eqnarray}
\par Como la ecuaci�n (6) tiene la misma forma que la ecuaci�n (5) entonces se puede establecer que el comportamiento de los cuerpos puntuales A y B est� determinado para cualquier sistema de referencia no rotante (inercial o no inercial) por la ecuaci�n (5).
\smallskip
\par Ahora, si aplicamos la ecuaci�n (5) a un cuerpo puntual A y un sistema de referencia no rotante S (inercial o no inercial) fijo a un cuerpo puntual S, entonces:
\begin{eqnarray}
\frac{\sum \vF\ra}{\mm\ra} - \vA\ra = \frac{\sum \vF\rs}{\mm\rs} - \vA\rs
\end{eqnarray}
\par Despejando $\vA\ra$ y como la aceleraci�n $\vA\rs$ del cuerpo puntual S con respecto al sistema de referencia no rotante S siempre es igual a cero, resulta:
\begin{eqnarray}
\vA\ra = \frac{\sum \vF\ra}{\mm\ra} - \frac{\sum \vF\rs}{\mm\rs}
\end{eqnarray}
\par Finalmente obtenemos la ecuaci�n (8), que es la ecuaci�n b�sica para generar las nuevas leyes de movimiento para un cuerpo puntual en mec�nica cl�sica, que pueden ser aplicadas en cualquier sistema de referencia no rotante (inercial o no inercial) sin necesidad de introducir fuerzas ficticias.

\end{document}

