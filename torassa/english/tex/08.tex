
\documentclass[12pt]{article}
%\documentclass[a4paper,12pt]{article}
%\documentclass[letterpaper,12pt]{article}

\usepackage[dvips]{geometry}
\geometry{papersize={127.0mm,165.1mm}}
\geometry{totalwidth=106.0mm,totalheight=129.0mm}

\usepackage[english]{babel}
\usepackage{mathptmx}

\usepackage{hyperref}
\hypersetup{colorlinks=true,linkcolor=black}
\hypersetup{bookmarksnumbered=true,pdfstartview=FitH,pdfpagemode=UseNone}
\hypersetup{pdftitle={On New Laws of Motion for a Particle in Classical Mechanics}}
\hypersetup{pdfauthor={Alejandro A. Torassa}}

\setlength{\arraycolsep}{2pt}

\newcommand{\rt}{'}
\newcommand{\mm}{m}
\newcommand{\rot}{_{o'}}
\newcommand{\vA}{\mathbf{a}}
\newcommand{\vF}{\mathbf{F}}
\newcommand{\ra}{_{\mbox {\scriptsize A}}}
\newcommand{\rb}{_{\mbox {\scriptsize B}}}
\newcommand{\rs}{_{\mbox {\scriptsize S}}}

\begin{document}

\begin{center}

{\Large \makebox[78.9mm][s]{On New Laws of Motion for a}\\ Particle in Classical Mechanics}

\bigskip \medskip

{\normalsize Alejandro A. Torassa}

\bigskip \medskip

\scriptsize

Creative Commons Attribution 3.0 License

(2010) Buenos Aires, Argentina

atorassa@gmail.com

\bigskip \medskip

\small

{\bf Abstract}

\bigskip

\parbox{81mm}{This work presents new laws of motion for a particle in classical mechanics, which can be applied in any non-rotating reference frame (inertial or non-inertial) without the necessity of introducing fictitious forces.}

\vspace{+0.03em}

\end{center}

\normalsize

{\centering\subsection*{Introduction}}

\par It is known that Newton's first and second laws can only be applied in a non-inertial reference frame if fictitious forces are introduced. But, unlike real forces, fictitious forces are not caused by the interaction between bodies.
\smallskip
\par \makebox[99.6mm][s]{However, this work presents new laws of motion for a}\\ particle in classical mechanics, which can be applied in any non-rotating reference frame (inertial or non-inertial) without the necessity of introducing fictitious forces.
\smallskip
\par In this work, it is assumed that forces can act on a reference frame because any reference frame is fixed to a body.

\newpage

{\centering\subsection*{Laws of Motion}}

\par First new law of motion: The forces acting on a particle A and the forces acting on a reference frame S can change the state of motion of particle A relative to the reference frame S.
\smallskip
\par Second new law of motion: The acceleration $\vA\ra$ of a particle A relative to a reference frame S (non-rotating) fixed to a particle S is given by the following equation:
\begin{eqnarray*}
\vA\ra = \frac{\sum \vF\ra}{\mm\ra} - \frac{\sum \vF\rs}{\mm\rs}
\end{eqnarray*}
\noindent where $\sum \vF\ra$ is the sum of the forces acting on particle A, $\mm\ra$ is the mass of particle A, $\sum \vF\rs$ is the sum of the forces acting on particle S, and $\mm\rs$ is the mass of particle S.

\vspace{+0.36em}

{\centering\subsection*{Observations}}

\par In contradiction with Newton's first and second laws, from the above equation it follows that particle A can have non-zero acceleration even if there is no force acting on particle A, and also that particle A can have zero acceleration (state of rest or of uniform linear motion) even if there is an unbalanced force acting on particle A.
\smallskip
\par Finally, from the above equation it follows that Newton's first and second laws are valid in the reference frame S only if the sum of the forces acting on the reference frame S (parti- \hbox {cle S)} is equal to zero.

\newpage

{\centering\subsection*{Appendix}}

{\centering\subsection*{Dynamical Behavior of Particles}}

\par The behavior of two particles A and B which follow Newton's second law is determined from a reference frame S (inertial) by the equations:
\begin{eqnarray}
\sum \vF\ra = \mm\ra\vA\ra \\
\sum \vF\rb = \mm\rb\vA\rb
\end{eqnarray}
\noindent that is
\begin{eqnarray}
\frac{\sum \vF\ra}{\mm\ra} - \vA\ra = 0 \\
\frac{\sum \vF\rb}{\mm\rb} - \vA\rb = 0
\end{eqnarray}
\par Combining the equations (3) and (4) yields
\begin{eqnarray}
\frac{\sum \vF\ra}{\mm\ra} - \vA\ra = \frac{\sum \vF\rb}{\mm\rb} - \vA\rb
\end{eqnarray}
\par Therefore, the behavior of particles A and B is now determined from the reference frame S by the equation (5).
\smallskip
\par Now, if the equation (5) is transformed from the reference frame S to another non-rotating reference frame S' (inertial or non-inertial) using the transformations of kinematics and \hbox {dynamics:} ($\vA\rt = \vA - \vA\rot$), ($\vF\rt = \vF$) and ($\mm\rt = \mm$), it follows that
\begin{eqnarray}
\frac{\sum {\vF\ra}\rt}{{\mm\ra}\rt} - {\vA\ra}\rt = \frac{\sum {\vF\rb}\rt}{{\mm\rb}\rt} - {\vA\rb}\rt
\end{eqnarray}
\par Considering that the equation (6) have the same form as the equation (5), then it can be assumed that the behavior of particles A and B is determined from any non-rotating reference frame (inertial or non-inertial) by the equation (5).
\smallskip
\par Now, if the equation (5) is applied to a particle A and a non-rotating reference frame S (inertial or non-inertial) fixed to a particle S, then
\begin{eqnarray}
\frac{\sum \vF\ra}{\mm\ra} - \vA\ra = \frac{\sum \vF\rs}{\mm\rs} - \vA\rs
\end{eqnarray}
\par Since the acceleration $\vA\rs$ of particle S relative to the non-rotating reference frame S equals zero always, $\vA\ra$ may be obtained from the equation (7) as follows:
\begin{eqnarray}
\vA\ra = \frac{\sum \vF\ra}{\mm\ra} - \frac{\sum \vF\rs}{\mm\rs}
\end{eqnarray}
\par Finally we obtain the equation (8), which is the basic equation to generate the new laws of motion for a particle in classical mechanics, which can be applied in any non-rotating reference frame (inertial or non-inertial) without the necessity of introducing fictitious forces.

\end{document}

