
\documentclass[10pt]{article}
%\documentclass[a4paper,10pt]{article}
%\documentclass[letterpaper,10pt]{article}

\usepackage[dvips]{geometry}
\geometry{papersize={126.0mm,165.0mm}}
\geometry{totalwidth=104.7mm,totalheight=129.0mm}

\usepackage[english]{babel}
\usepackage{mathptmx}

\usepackage{hyperref}
\hypersetup{colorlinks=true,linkcolor=black}
\hypersetup{bookmarksnumbered=true,pdfstartview=FitH,pdfpagemode=UseNone}
\hypersetup{pdftitle={Linear Magnitudes}}
\hypersetup{pdfauthor={Alejandro A. Torassa}}

\setlength{\arraycolsep}{1.74pt}

\newcommand{\mM}{m}
\newcommand{\mY}{Y}
\newcommand{\mP}{P}
\newcommand{\mF}{F}
\newcommand{\mW}{W}
\newcommand{\mU}{U}
\newcommand{\mE}{E}
\newcommand{\mL}{L}
\newcommand{\mT}{T}
\newcommand{\mV}{V}
\newcommand{\ra}{_a}
\newcommand{\vR}{\mathbf{r}}
\newcommand{\vV}{\mathbf{v}}
\newcommand{\vA}{\mathbf{a}}
\newcommand{\vF}{\mathbf{F}}
\newcommand{\med}{\raise.5ex\hbox{$\scriptstyle 1$}\kern-.15em/\kern-.15em\lower.25ex\hbox{$\scriptstyle 2$}}

\begin{document}

\begin{center}

{\LARGE Linear Magnitudes}

\bigskip \medskip

Alejandro A. Torassa

\bigskip \medskip

\footnotesize

Creative Commons Attribution 3.0 License

(2014) Buenos Aires, Argentina

atorassa@gmail.com

\bigskip \smallskip

\small

{\bf Abstract}

\bigskip

\parbox{81mm}{In classical mechanics, this paper presents definitions of linear magnitudes from vector magnitudes.}

\end{center}

\normalsize

\vspace{-0.60em}

{\centering\subsubsection*{Linear Magnitudes}}

\vspace{+0.60em}

\par The linear magnitudes for a particle A of mass $\mM\ra$ are defined with respect to a position vector $\vR$ which is constant in magnitude and direction.

\vspace{+1.20em}

\begin{center}
\begin{tabular}{ll}
Linear Mass & \hspace{+1.20em} $\mY\ra = \mM\ra \hspace{+0.12em} (\hspace{+0.03em} \vR \cdot \vR\ra)$ \vspace{+0.90em} \\
Linear Momentum & \hspace{+1.20em} $\mP\ra = \mM\ra \hspace{+0.12em} (\hspace{+0.03em} \vR \cdot \vV\ra)$ \vspace{+0.90em} \\
Linear Force & \hspace{+1.20em} $\mF\ra = \mM\ra \hspace{+0.12em} (\hspace{+0.03em} \vR \cdot \vA\ra)$ \vspace{+0.90em} \\
Linear Work & \hspace{+1.20em} $\mW\ra = \int \hspace{+0.18em} \mF\ra \hspace{+0.27em} d(\hspace{+0.03em} \vR \cdot \vR\ra)$ \vspace{+0.90em} \\
Theorem & \hspace{+1.20em} $\mW\ra = \hspace{+0.09em} \Delta \hspace{+0.24em} \med \hspace{+0.24em} \mM\ra \hspace{+0.12em} (\hspace{+0.03em} \vR \cdot \vV\ra)^{\hspace{+0.03em} 2}$
\end{tabular}
\end{center}

\vspace{+0.60em}

\par Where $\vR\ra$, $\vV\ra$, and $\vA\ra$ are the position, the velocity, and the acceleration of particle A.
\medskip
\par The linear magnitudes for a system of particles are also defined with respect to a position vector $\vR$ which is constant in magnitude and direction.

\newpage

{\centering\subsubsection*{Linear Potential Energy}}

\vspace{+0.75em}

\par The linear potential energy $\mU\ra$ of a particle A on which a resultant force $\vF\ra$ acts, is given by:
\begin{eqnarray*}
\mU\ra = - \int (\hspace{+0.03em} \vR \cdot \vF\ra) \hspace{+0.27em} d(\hspace{+0.03em} \vR \cdot \vR\ra)
\end{eqnarray*}
\noindent where $\vR$ is a position vector which is constant in magnitude and direction, and $\vR\ra$ is the position of particle A.
\medskip
\par If $\vF\ra$ is constant and since $\vF\ra = \mM\ra \hspace{+0.09em} \vA\ra$, it follows that:
\begin{eqnarray*}
\mU\ra = - \hspace{+0.18em} \mM\ra \hspace{+0.06em} (\hspace{+0.03em} \vR \cdot \vA\ra) \hspace{+0.06em} (\hspace{+0.03em} \vR \cdot \vR\ra)
\end{eqnarray*}
\noindent where $\mM\ra$ is the mass of particle A, and $\vA\ra$ is the constant acceleration of particle A.

\vspace{+0.75em}

{\centering\subsubsection*{Linear Mechanical Energy}}

\vspace{+0.75em}

\par The linear mechanical energy $\mE\ra$ of a particle A of mass $\mM\ra$ which moves in a uniform force field, is given by:
\begin{eqnarray*}
\mE\ra = \med \hspace{+0.24em} \mM\ra \hspace{+0.06em} (\hspace{+0.03em} \vR \cdot \vV\ra)^{\hspace{+0.03em} 2} - \hspace{+0.09em} \mM\ra \hspace{+0.06em} (\hspace{+0.03em} \vR \cdot \vA\ra) \hspace{+0.06em} (\hspace{+0.03em} \vR \cdot \vR\ra)
\end{eqnarray*}
\noindent where $\vR$ is a position vector which is constant in magnitude and direction, and $\vV\ra$, $\vA\ra$ and $\vR\ra$ are the velocity, the constant acceleration and the position of particle A.
\medskip
\par The principle of conservation of the linear mechanical energy establishes that if a particle A moves in a uniform force field then the linear mechanical energy of particle A remains constant.

\newpage

{\centering\subsubsection*{Principle of Least Linear Action}}

\vspace{+0.75em}

\par If we consider a single particle A of mass $\mM\ra$ then the principle of least linear action, is given by:
\begin{eqnarray*}
\delta \int_{t_1}^{t_2} \med \hspace{+0.24em} \mM\ra \hspace{+0.06em} (\hspace{+0.03em} \vR \cdot \vV\ra)^{\hspace{+0.03em} 2} \hspace{+0.12em} dt \hspace{+0.12em} + \int_{t_1}^{t_2} (\hspace{+0.03em} \vR \cdot \vF\ra) \hspace{+0.12em} \delta (\hspace{+0.03em} \vR \cdot \vR\ra) \hspace{+0.18em} dt = 0
\end{eqnarray*}
\noindent where $\vR$ is a position vector which is constant in magnitude and direction, $\vV\ra$ is the velocity of particle A, $\vF\ra$ is the net force acting on particle A, and $\vR\ra$ is the position of particle A.
\medskip
\par If $- \hspace{+0.03em} \delta \hspace{+0.09em} \mV\ra = (\hspace{+0.03em} \vR \cdot \vF\ra) \hspace{+0.12em} \delta (\hspace{+0.03em} \vR \cdot \vR\ra)$ and since $\mT\ra = \med \hspace{+0.24em} \mM\ra \hspace{+0.06em} (\hspace{+0.03em} \vR \cdot \vV\ra)^{\hspace{+0.03em} 2}$, then:
\begin{eqnarray*}
\delta \int_{t_1}^{t_2} (\mT\ra - \mV\ra) \hspace{+0.18em} dt = 0
\end{eqnarray*}
\par And since $\mL\ra = \mT\ra - \mV\ra$, then we obtain:
\begin{eqnarray*}
\delta \int_{t_1}^{t_2} \mL\ra \hspace{+0.24em} dt = 0
\end{eqnarray*}

\vspace{+1.20em}

{\centering\subsubsection*{Bibliography}}

\vspace{+0.75em}

\par \textbf{A. Einstein}, Relativity: The Special and General Theory.
\bigskip
\par \textbf{E. Mach}, The Science of Mechanics.
\bigskip
\par \textbf{R. Resnick and D. Halliday}, Physics.
\bigskip
\par \textbf{J. Kane and M. Sternheim}, Physics.
\bigskip
\par \textbf{H. Goldstein}, Classical Mechanics.
\bigskip
\par \textbf{L. Landau and E. Lifshitz}, Mechanics.

\end{document}

