
\documentclass[10pt,fleqn]{article}
%\documentclass[a4paper,10pt]{article}
%\documentclass[letterpaper,10pt]{article}

\usepackage[dvips]{geometry}
\geometry{papersize={157.0mm,210.0mm}}
\geometry{totalwidth=136.0mm,totalheight=168.0mm}

\usepackage[english]{babel}
\usepackage{mathptmx}

\usepackage{hyperref}
\hypersetup{colorlinks=true,linkcolor=black}
\hypersetup{bookmarksnumbered=true,pdfstartview=FitH,pdfpagemode=UseNone}
\hypersetup{pdftitle={A New Principle of Least Action}}
\hypersetup{pdfauthor={Alejandro A. Torassa}}

\setlength{\arraycolsep}{1.74pt}

\newcommand{\mM}{m}
\newcommand{\mL}{L}
\newcommand{\mT}{T}
\newcommand{\mV}{V}
\newcommand{\ri}{_i}
\newcommand{\vR}{\mathbf{r}}
\newcommand{\vV}{\mathbf{v}}
\newcommand{\vA}{\mathbf{a}}
\newcommand{\vF}{\mathbf{F}}
\newcommand{\Hs}{\hspace{+0.90em}}
\newcommand{\rj}{_{\hspace{-0.081em}j}}
\newcommand{\rij}{_{i\hspace{-0.081em}j}}
\newcommand{\med}{\raise.5ex\hbox{$\scriptstyle 1$}\kern-.15em/\kern-.15em\lower.25ex\hbox{$\scriptstyle 2$}}

\begin{document}

\begin{center}

{\LARGE A New Principle of Least Action}

\bigskip \medskip

Alejandro A. Torassa

\bigskip \medskip

\footnotesize

Creative Commons Attribution 3.0 License

(2014) Buenos Aires, Argentina

atorassa@gmail.com

\bigskip \smallskip

\small

{\bf Abstract}

\bigskip

\parbox{102mm}{In classical mechanics, this paper presents a new principle of least action which is invariant under transformations between reference frames and which can be applied in any reference frame (rotating or non-rotating) (inertial or non-inertial) without the necessity of introducing fictitious forces.}

\end{center}

\normalsize

\vspace{-0.30em}

{\centering\subsubsection*{The New Principle of Least Action}}

\vspace{+0.90em}

\par If we consider two particles i and j then the new principle of least action is given by:
\vspace{+0.90em}
\begin{eqnarray*}
\Hs \delta \int_{t_1}^{t_2} \mL\rij \hspace{+0.24em} dt = 0
\end{eqnarray*}
\vspace{+0.30em}
\begin{eqnarray*}
\Hs \delta \int_{t_1}^{t_2} (\mT\rij - \mV\rij) \hspace{+0.24em} dt = 0
\end{eqnarray*}
\vspace{+0.30em}
\begin{eqnarray*}
\Hs \mT\rij = + \hspace{+0.18em} \med \hspace{+0.15em} \mM\ri\hspace{+0.09em}\mM\rj \hspace{+0.12em} \left[ \hspace{+0.12em} (\vV\ri - \vV\rj) \cdot (\vV\ri - \vV\rj) + (\vA\ri - \vA\rj) \cdot (\vR\ri - \vR\rj) \hspace{+0.12em} \right]
\end{eqnarray*}
\vspace{+0.30em}
\begin{eqnarray*}
\Hs \mV\rij = - \hspace{+0.18em} \med \hspace{+0.15em} \mM\ri\hspace{+0.09em}\mM\rj \hspace{+0.03em} \left[ \hspace{+0.12em} 2 \hspace{-0.12em} \int \hspace{-0.12em} \left(\frac{\vF\ri}{\mM\ri} - \frac{\vF\rj}{\mM\rj}\right) \cdot d(\vR\ri - \vR\rj) + \left(\frac{\vF\ri}{\mM\ri} - \frac{\vF\rj}{\mM\rj}\right) \cdot (\vR\ri - \vR\rj) \hspace{+0.12em} \right]
\end{eqnarray*}
\vspace{+0.60em}
\par \noindent where $\mM\ri$ and $\mM\rj$ are the masses of particles i and j, $\vR\ri$, $\vR\rj$, $\vV\ri$, $\vV\rj$, $\vA\ri$ and $\vA\rj$ are the positions, the velocities and the accelerations of particles i and j, and $\vF\ri$ and $\vF\rj$ are the net (conservative) forces acting on particles i and j.
\bigskip
\par The Lagrangian $\mL\rij$ is invariant under transformations between reference frames.
\bigskip
\par The Lagrangian $\mL\rij$ can be applied in any reference frame (rotating or non-rotating) (inertial or non-inertial) without the necessity of introducing fictitious forces.

\end{document}

