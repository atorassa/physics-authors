
\documentclass[10pt]{article}
%\documentclass[a4paper,10pt]{article}
%\documentclass[letterpaper,10pt]{article}

\usepackage[dvips]{geometry}
\geometry{papersize={147.0mm,192.0mm}}
\geometry{totalwidth=126.0mm,totalheight=156.0mm}

\usepackage[english]{babel}
\usepackage{mathptmx}

\usepackage{hyperref}
\hypersetup{colorlinks=true,linkcolor=black}
\hypersetup{bookmarksnumbered=true,pdfstartview=FitH,pdfpagemode=UseNone}
\hypersetup{pdftitle={A New Principle of Conservation of Energy}}
\hypersetup{pdfauthor={Alejandro A. Torassa}}

\setlength{\arraycolsep}{1.74pt}

\newcommand{\mM}{m}
\newcommand{\mW}{W}
\newcommand{\mK}{K}
\newcommand{\mU}{U}
\newcommand{\ri}{_i}
\newcommand{\rcm}{_{cm}}
\newcommand{\vR}{\mathbf{r}}
\newcommand{\vV}{\mathbf{v}}
\newcommand{\vA}{\mathbf{a}}
\newcommand{\vF}{\mathbf{F}}
\newcommand{\rj}{_{\hspace{-0.081em}j}}

\begin{document}

\begin{center}

{\Large A New Principle of Conservation of Energy}

\bigskip \medskip

Alejandro A. Torassa

\bigskip \medskip

\footnotesize

Creative Commons Attribution 3.0 License

(2014) Buenos Aires, Argentina

atorassa@gmail.com

\bigskip \smallskip

\small

{\bf Abstract}

\bigskip

\parbox{96mm}{In classical mechanics, this paper presents a new principle of conservation of energy which is invariant under transformations between reference frames and which can be applied in any reference frame (rotating or non-rotating) (inertial or non-inertial) without the necessity of introducing fictitious forces.}

\end{center}

\normalsize

\vspace{-0.60em}

{\centering\subsubsection*{Definitions of Work, K and U}}

\vspace{+0.90em}

\par If we consider a system of N particles then the total work $\mW$ done by the forces acting on the system of particles, the total kinetic energy $\mK$ of the system of particles and the total potential energy $\mU$ of the system of particles, are as follows:
\medskip
\begin{eqnarray*}
\mW \hspace{+0.12em} = \hspace{+0.12em} \sum_{i={\scriptscriptstyle 1}}^{\mathrm N} \hspace{+0.36em} + \left( \hspace{+0.06em} \int_1^{\hspace{+0.09em} 2} \hspace{+0.12em} \vF\ri \cdot d\bar\vR\ri \hspace{+0.12em} + \hspace{+0.12em} \Delta \hspace{+0.24em} \frac{1}{2} \hspace{+0.24em} \vF\ri \cdot \bar\vR\ri \hspace{+0.06em} \right)
\end{eqnarray*}
\begin{eqnarray*}
\hspace{-0.45em} \Delta \hspace{+0.18em} \mK \hspace{+0.12em} = \hspace{+0.12em} \sum_{i={\scriptscriptstyle 1}}^{\mathrm N} \hspace{+0.36em} \Delta \hspace{+0.09em} \left( \hspace{+0.06em} \frac{1}{2} \hspace{+0.24em} \mM\ri \hspace{+0.24em} \bar\vV\ri \cdot \bar\vV\ri \hspace{+0.12em} + \hspace{+0.12em} \frac{1}{2} \hspace{+0.24em} \mM\ri \hspace{+0.24em} \bar\vA\ri \cdot \bar\vR\ri \hspace{+0.06em} \right)
\end{eqnarray*}
\begin{eqnarray*}
\hspace{-0.60em} \Delta \hspace{+0.18em} \mU \hspace{+0.12em} = \hspace{+0.12em} \sum_{i={\scriptscriptstyle 1}}^{\mathrm N} \hspace{+0.36em} - \left( \hspace{+0.06em} \int_1^{\hspace{+0.09em} 2} \hspace{+0.12em} \vF\ri \cdot d\bar\vR\ri \hspace{+0.12em} + \hspace{+0.12em} \Delta \hspace{+0.24em} \frac{1}{2} \hspace{+0.24em} \vF\ri \cdot \bar\vR\ri \hspace{+0.06em} \right)
\end{eqnarray*}
\smallskip
\par \noindent where $\bar\vR\ri = \vR\ri - \vR\rcm$, $\bar\vV\ri = \vV\ri - \vV\rcm$, $\bar\vA\ri = \vA\ri - \vA\rcm$, $\vR\ri$, $\vV\ri$ and $\vA\ri$ are the position, the velocity and the acceleration of the \textit{i}-th particle, $\vR\rcm$, $\vV\rcm$ and $\vA\rcm$ are the position, the velocity and the acceleration of the center of mass of the system of particles, $\mM\ri$ is the mass of the \textit{i}-th particle, and $\vF\ri$ is the net force acting on the \textit{i}-th particle.

\newpage

{\centering\subsubsection*{Theorems of K and U}}

\vspace{+0.90em}

\par In a system of N particles, the total work $\mW$ done by the forces acting on the system of particles is equal to the change in the total kinetic energy $\mK$ of the system of particles.
\begin{eqnarray*}
\mW = + \hspace{+0.09em} \Delta \hspace{+0.18em} \mK
\end{eqnarray*}
\par In a system of N particles, the total work $\mW$ done by the conservative forces acting on the system of particles is equal and opposite in sign to the change in the total potential energy $\mU$ of the system of particles.
\begin{eqnarray*}
\mW = - \hspace{+0.09em} \Delta \hspace{+0.18em} \mU
\end{eqnarray*}

\vspace{+0.90em}

{\centering\subsubsection*{Conservation of Energy}}

\vspace{+0.90em}

\par In a system of N particles, if the non-conservative forces acting on the system of \hbox {particles} do not perform work then the total (mechanical) energy of the system of \hbox {particles} remains constant.
\begin{eqnarray*}
\mK + \mU = \hspace{+0.09em} constant
\end{eqnarray*}

\vspace{+0.90em}

{\centering\subsubsection*{General Observations}}

\vspace{+0.90em}

\par The new principle of conservation of energy is invariant under transformations between reference frames.
\bigskip
\par The new principle of conservation of energy can be applied in any reference frame (rotating or non-rotating) (inertial or non-inertial) without the necessity of introducing fictitious forces.
\bigskip
\par The new principle of conservation of energy would be valid even if Newton's third law of motion were false in an inertial reference frame.
\bigskip
\par The new principle of conservation of energy would be valid even if Newton's three laws of motion were false in a non-inertial reference frame.

\newpage

{\centering\subsubsection*{Annex}}

\vspace{+0.90em}

{\centering\subsubsection*{Work and Potential Energy}}

\vspace{+0.90em}

\par If we consider an isolated system of N particles and if Newton's third law of motion is valid then the total work $\mW$ done by the forces acting on the system of particles and the total potential energy $\mU$ of the system of particles, are as follows:
\medskip
\begin{eqnarray*}
\mW \hspace{+0.12em} = \hspace{+0.12em} \sum_{i={\scriptscriptstyle 1}}^{\mathrm N} \hspace{+0.36em} + \left( \hspace{+0.06em} \int_1^{\hspace{+0.09em} 2} \hspace{+0.12em} \vF\ri \cdot d\vR\ri \hspace{+0.12em} + \hspace{+0.12em} \Delta \hspace{+0.24em} \frac{1}{2} \hspace{+0.24em} \vF\ri \cdot \vR\ri \hspace{+0.06em} \right)
\end{eqnarray*}
\begin{eqnarray*}
\hspace{-0.60em} \Delta \hspace{+0.18em} \mU \hspace{+0.12em} = \hspace{+0.12em} \sum_{i={\scriptscriptstyle 1}}^{\mathrm N} \hspace{+0.36em} - \left( \hspace{+0.06em} \int_1^{\hspace{+0.09em} 2} \hspace{+0.12em} \vF\ri \cdot d\vR\ri \hspace{+0.12em} + \hspace{+0.12em} \Delta \hspace{+0.24em} \frac{1}{2} \hspace{+0.24em} \vF\ri \cdot \vR\ri \hspace{+0.06em} \right)
\end{eqnarray*}
\smallskip
\par \noindent where $\vR\ri$ is the position of the \textit{i}-th particle, and $\vF\ri$ is the net force acting on the \textit{i}-th particle.

\vspace{+1.50em}

{\centering\subsubsection*{Kinetic Energy}}

\vspace{+0.90em}

\par If we consider a system of N particles then the total kinetic energy $\mK$ of the system of particles can also be expressed as follows:
\medskip
\begin{eqnarray*}
\hspace{-0.18em} \mK \hspace{+0.12em} = \hspace{+0.12em} \sum_{i={\scriptscriptstyle 1}}^{\mathrm N} \hspace{+0.27em} \frac{1}{2} \hspace{+0.24em} \mM\ri \hspace{+0.24em} \vV\ri \cdot \vV\ri \hspace{+0.12em} - \hspace{+0.12em} \frac{1}{2} \hspace{+0.24em} \mM\rcm \hspace{+0.24em} \vV\rcm \cdot \vV\rcm \hspace{+0.12em} + \hspace{+0.12em} \sum_{i={\scriptscriptstyle 1}}^{\mathrm N} \hspace{+0.27em} \frac{1}{2} \hspace{+0.24em} \mM\ri \hspace{+0.24em} \vA\ri \cdot \vR\ri \hspace{+0.12em} - \hspace{+0.12em} \frac{1}{2} \hspace{+0.24em} \mM\rcm \hspace{+0.24em} \vA\rcm \cdot \vR\rcm
\end{eqnarray*}
\smallskip
\par \noindent that is:
\medskip
\begin{eqnarray*}
\mK \hspace{+0.12em} = \hspace{+0.12em} \sum_{i={\scriptscriptstyle 1}}^{\mathrm N} \hspace{+0.36em} \sum_{j>i}^{\mathrm N} \hspace{+0.36em} \left( \hspace{+0.06em} \frac{1}{2} \hspace{+0.24em} \frac{\mM\ri \hspace{+0.12em} \mM\rj}{\mM\rcm} \hspace{+0.24em} (\vV\ri - \vV\rj) \cdot (\vV\ri - \vV\rj) \hspace{+0.18em} + \hspace{+0.18em} \frac{1}{2} \hspace{+0.24em} \frac{\mM\ri \hspace{+0.12em} \mM\rj}{\mM\rcm} \hspace{+0.24em} (\vA\ri - \vA\rj) \cdot (\vR\ri - \vR\rj) \hspace{+0.06em} \right)
\end{eqnarray*}
\smallskip
\par \noindent where $\vR\ri$, $\vV\ri$, $\vA\ri$, $\vR\rj$, $\vV\rj$, $\vA\rj$, $\vR\rcm$, $\vV\rcm$, $\vA\rcm$ are the positions, the velocities and the accelerations of the \textit{i}-th particle, of the \textit{j}-th particle and of the center of mass of the system of particles, and $\mM\ri$, $\mM\rj$, $\mM\rcm$ are the masses of the \textit{i}-th particle, of the \textit{j}-th particle and of the center of mass of the system of particles.

\end{document}

