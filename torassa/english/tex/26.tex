
\documentclass[10pt]{article}
%\documentclass[a4paper,10pt]{article}
%\documentclass[letterpaper,10pt]{article}

\usepackage[dvips]{geometry}
\geometry{papersize={128.0mm,165.0mm}}
\geometry{totalwidth=106.6mm,totalheight=129.0mm}

\usepackage[english]{babel}
\usepackage{mathptmx}

\usepackage{hyperref}
\hypersetup{colorlinks=true,linkcolor=black}
\hypersetup{bookmarksnumbered=true,pdfstartview=FitH,pdfpagemode=UseNone}
\hypersetup{pdftitle={Angular Magnitudes}}
\hypersetup{pdfauthor={Alejandro A. Torassa}}

\setlength{\arraycolsep}{1.74pt}

\newcommand{\mM}{m}
\newcommand{\mW}{W}
\newcommand{\ra}{_a}
\newcommand{\vR}{\mathbf{r}}
\newcommand{\vV}{\mathbf{v}}
\newcommand{\vA}{\mathbf{a}}
\newcommand{\vK}{\mathbf{K}}
\newcommand{\vL}{\mathbf{L}}
\newcommand{\vM}{\mathbf{M}}
\newcommand{\med}{\raise.5ex\hbox{$\scriptstyle 1$}\kern-.15em/\kern-.15em\lower.25ex\hbox{$\scriptstyle 2$}}

\begin{document}

\begin{center}

{\LARGE Angular Magnitudes}

\bigskip \medskip

Alejandro A. Torassa

\bigskip \medskip

\footnotesize

Creative Commons Attribution 3.0 License

(2013) Buenos Aires, Argentina

atorassa@gmail.com

\bigskip \smallskip

\small

{\bf Abstract}

\bigskip

\parbox{81mm}{In classical mechanics, this paper presents alternative definitions of angular magnitudes.}

\end{center}

\normalsize

\vspace{-0.60em}

{\centering\subsubsection*{Angular Magnitudes}}

\vspace{+0.60em}

\par The angular magnitudes for a particle A of mass $\mM\ra$ are defined with respect to a position vector $\vR$ which is constant in magnitude and direction.

\vspace{+1.20em}

\begin{center}
\begin{tabular}{ll}
Mass Moment & \hspace{+1.20em} $\vK\ra = \mM\ra \hspace{+0.12em} (\hspace{+0.03em} \vR \times \vR\ra)$ \vspace{+0.90em} \\
Angular Momentum & \hspace{+1.20em} $\vL\ra = \mM\ra \hspace{+0.12em} (\hspace{+0.03em} \vR \times \vV\ra)$ \vspace{+0.90em} \\
Dynamic Moment & \hspace{+1.20em} $\vM\ra = \mM\ra \hspace{+0.12em} (\hspace{+0.03em} \vR \times \vA\ra)$ \vspace{+0.90em} \\
Angular Work & \hspace{+1.20em} $\mW\ra = \int \hspace{+0.18em} \vM\ra \cdot d(\hspace{+0.03em} \vR \times \vR\ra)$ \vspace{+0.90em} \\
Theorem & \hspace{+1.20em} $\mW\ra = \hspace{+0.09em} \Delta \hspace{+0.24em} \med \hspace{+0.24em} \mM\ra \hspace{+0.12em} (\hspace{+0.03em} \vR \times \vV\ra)^{\hspace{+0.03em} 2}$
\end{tabular}
\end{center}

\vspace{+0.60em}

\par Where $\vR\ra$, $\vV\ra$, and $\vA\ra$ are the position, the velocity, and the acceleration of particle A.
\medskip
\par The angular magnitudes for a system of particles are also defined with respect to a position vector $\vR$ which is constant in magnitude and direction.

\end{document}

