
\documentclass[10pt]{article}
%\documentclass[a4paper,10pt]{article}
%\documentclass[letterpaper,10pt]{article}

\usepackage[dvips]{geometry}
\geometry{papersize={128.4mm,165.0mm}}
\geometry{totalwidth=107.4mm,totalheight=129.0mm}

\usepackage[english]{babel}
\usepackage{mathptmx}

\usepackage{hyperref}
\hypersetup{colorlinks=true,linkcolor=black}
\hypersetup{bookmarksnumbered=true,pdfstartview=FitH,pdfpagemode=UseNone}
\hypersetup{pdftitle={A Scalar Equation of Motion}}
\hypersetup{pdfauthor={Alejandro A. Torassa}}

\setlength{\arraycolsep}{1.74pt}

\newcommand{\mM}{m}
\newcommand{\mE}{E}
\newcommand{\ra}{_a}
\newcommand{\rb}{_b}
\newcommand{\ri}{_i}
\newcommand{\vR}{\mathbf{r}}
\newcommand{\vV}{\mathbf{v}}
\newcommand{\vA}{\mathbf{a}}
\newcommand{\vF}{\mathbf{F}}
\newcommand{\Hs}{\hspace{+2.10em}}
\newcommand{\He}{\hspace{+1.20em}}
\newcommand{\rj}{_{\hspace{-0.081em}j}}
\newcommand{\rij}{_{i\hspace{-0.081em}j}}

\begin{document}

\begin{center}

{\Large A Scalar Equation of Motion}

\bigskip \medskip

Alejandro A. Torassa

\bigskip \medskip

\footnotesize

Creative Commons Attribution 3.0 License

(2014) Buenos Aires, Argentina

atorassa@gmail.com

\bigskip \smallskip

\small

{\bf Abstract}

\bigskip

\parbox{74mm}{In classical mechanics, this paper presents a scalar equation of motion, which can be applied in any reference frame (rotating or non-rotating) (inertial or non-inertial) without the necessity of introducing fictitious forces.}

\end{center}

\normalsize

\vspace{-0.60em}

{\centering\subsubsection*{Scalar Equation of Motion}}

\vspace{+1.20em}

\par If we consider two particles A and B of mass $\mM\ra$ and $\mM\rb$ respectively, then the scalar equation of motion, is given by:
{\fontsize{6}{6}\selectfont\begin{eqnarray*}
\Hs \frac{1}{2} \hspace{+0.24em} \mM\ra\mM\rb \left[(\vV\ra - \vV\rb)^{\hspace{+0.03em} 2} + (\vA\ra - \vA\rb) \cdot (\vR\ra - \vR\rb)\right] = \frac{1}{2} \hspace{+0.24em} \mM\ra\mM\rb \left[ \hspace{+0.12em} 2 \hspace{-0.12em} \int \hspace{-0.12em} \left(\frac{\vF\ra}{\mM\ra} - \frac{\vF\rb}{\mM\rb}\right) \cdot d(\vR\ra - \vR\rb) + \left(\frac{\vF\ra}{\mM\ra} - \frac{\vF\rb}{\mM\rb}\right) \cdot (\vR\ra - \vR\rb) \hspace{+0.12em} \right]
\end{eqnarray*}}
\vspace{-0.90em}
\par \noindent where $\vV\ra$ and $\vV\rb$ are the velocities of particles A and B, $\vA\ra$ and $\vA\rb$ are the accelerations of particles A and B, $\vR\ra$ and $\vR\rb$ are the positions of particles A and B, and $\vF\ra$ and $\vF\rb$ are the net forces acting on particles A and B.
\medskip
\par This scalar equation of motion can be applied in any reference frame (rotating or non-rotating) (inertial or non-inertial) without the necessity of introducing fictitious forces. In addition, this scalar equation of motion is invariant under transformations between reference frames.
\medskip
\par On the other hand, this scalar equation of motion would be valid even if Newton's three laws of motion were false.

\newpage

{\centering\subsubsection*{Annex}}

\vspace{+0.90em}

{\centering\subsubsection*{Conservation of Energy}}

\vspace{+0.90em}

\par A system of particles forms a system of biparticles. For example, the system of particles A, B, C and D forms the system of biparticles AB, AC, AD, BC, BD and CD.
\medskip
\par In this paper, the total energy $\mE\rij$ of a system of biparticles is:
{\fontsize{6}{6}\selectfont\begin{eqnarray*}
\He \mE\rij \hspace{+0.12em} = \hspace{+0.12em} \sum_{\scriptscriptstyle i} \hspace{+0.12em} \sum_{\scriptscriptstyle j>i} \hspace{+0.24em} \frac{1}{2} \hspace{+0.24em} \mM\ri\mM\rj \left[ \hspace{+0.12em} (\vV\ri - \vV\rj)^{\hspace{+0.03em} 2} + (\vA\ri - \vA\rj) \cdot (\vR\ri - \vR\rj) - \hspace{+0.12em} 2 \hspace{-0.12em} \int \hspace{-0.12em} \left(\frac{\vF\ri}{\mM\ri} - \frac{\vF\rj}{\mM\rj}\right) \cdot d(\vR\ri - \vR\rj) - \left(\frac{\vF\ri}{\mM\ri} - \frac{\vF\rj}{\mM\rj}\right) \cdot (\vR\ri - \vR\rj) \hspace{+0.12em} \right]
\end{eqnarray*}}
\vspace{-0.90em}
\par \noindent where $\mM\ri$ and $\mM\rj$ are the masses of the \textit{i}-th and \textit{j}-th particles, $\vV\ri$ and $\vV\rj$ are the velocities of the \textit{i}-th and \textit{j}-th particles, $\vA\ri$ and $\vA\rj$ are the accelerations of the \textit{i}-th and \textit{j}-th particles, $\vR\ri$ and $\vR\rj$ are the positions of the \textit{i}-th and \textit{j}-th particles, and $\vF\ri$ and $\vF\rj$ are the net forces acting on the \textit{i}-th and \textit{j}-th particles.
\medskip
\par Therefore, from the scalar equation of motion it follows that the total energy $\mE\rij$ of a system of biparticles is always in equilibrium.

\vspace{+1.50em}

{\centering\subsubsection*{General Equation of Motion}}

\vspace{+1.20em}

\par The scalar equation of motion can be obtained from the following general equation of motion:
{\fontsize{6}{6}\selectfont\begin{eqnarray*}
\sum_{\scriptscriptstyle i} \hspace{+0.24em} \sum_{\scriptscriptstyle j>i} \hspace{+0.36em} \mM\ri\mM\rj \left[ \hspace{+0.03em} \frac{(\vR\ri - \vR\rj)}{|\vR\ri - \vR\rj|} \cdot (\vR\ri - \vR\rj) \hspace{+0.06em} - \frac{(\vR\ri - \vR\rj)}{|\vR\ri - \vR\rj|} \cdot \hspace{-0.12em} \int \hspace{-0.12em} \int \hspace{-0.12em} \left(\frac{\vF\ri}{\mM\ri} - \frac{\vF\rj}{\mM\rj}\right) \hspace{+0.12em} dt \hspace{+0.12em} dt \hspace{+0.12em} \right] = \hspace{+0.15em} 0
\end{eqnarray*}}
\vspace{-0.90em}
\par \noindent where $\mM\ri$ and $\mM\rj$ are the masses of the \textit{i}-th and \textit{j}-th particles, $\vR\ri$ and $\vR\rj$ are the positions of the \textit{i}-th and \textit{j}-th particles, and $\vF\ri$ and $\vF\rj$ are the net forces acting on the \textit{i}-th and \textit{j}-th particles.

\end{document}

