
\documentclass[10pt]{article}
%\documentclass[a4paper,10pt]{article}
%\documentclass[letterpaper,10pt]{article}

\usepackage[dvips]{geometry}
\geometry{papersize={123.0mm,150.0mm}}
\geometry{totalwidth=102.0mm,totalheight=114.0mm}

\usepackage[english]{babel}
\usepackage{mathptmx}

\usepackage{hyperref}
\hypersetup{colorlinks=true,linkcolor=black}
\hypersetup{bookmarksnumbered=true,pdfstartview=FitH,pdfpagemode=UseNone}
\hypersetup{pdftitle={Angular Potential Energy}}
\hypersetup{pdfauthor={Alejandro A. Torassa}}

\setlength{\arraycolsep}{1.74pt}

\newcommand{\mM}{m}
\newcommand{\mU}{U}
\newcommand{\ra}{_a}
\newcommand{\vR}{\mathbf{r}}
\newcommand{\vA}{\mathbf{a}}
\newcommand{\vF}{\mathbf{F}}

\begin{document}

\begin{center}

{\LARGE Angular Potential Energy}

\bigskip \medskip

Alejandro A. Torassa

\bigskip \medskip

\footnotesize

Creative Commons Attribution 3.0 License

(2014) Buenos Aires, Argentina

atorassa@gmail.com

\bigskip \smallskip

\small

{\bf Abstract}

\bigskip \smallskip

\parbox{84mm}{This paper presents an equation to calculate the angular potential energy of a particle.}

\end{center}

\normalsize

\vspace{-0.30em}

{\centering\subsubsection*{Angular Potential Energy}}

\vspace{+0.75em}

\par The angular potential energy $\mU\ra$ of a particle A on which a resultant force $\vF\ra$ acts, is given by:
\begin{eqnarray*}
\mU\ra = - \int (\hspace{+0.03em} \vR \times \vF\ra) \cdot d(\hspace{+0.03em} \vR \times \vR\ra)
\end{eqnarray*}
\noindent where $\vR$ is a position vector which is constant in magnitude and direction, and $\vR\ra$ is the position of particle A.
\medskip
\par If $\vF\ra$ is constant and since $\vF\ra = \mM\ra \hspace{+0.09em} \vA\ra$, it follows that:
\begin{eqnarray*}
\mU\ra = - \hspace{+0.18em} \mM\ra \hspace{+0.06em} (\hspace{+0.03em} \vR \times \vA\ra) \cdot (\hspace{+0.03em} \vR \times \vR\ra)
\end{eqnarray*}
\noindent where $\mM\ra$ is the mass of particle A, and $\vA\ra$ is the constant acceleration of particle A.

\end{document}

