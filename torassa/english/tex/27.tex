
\documentclass[10pt]{article}
%\documentclass[a4paper,10pt]{article}
%\documentclass[letterpaper,10pt]{article}

\usepackage[dvips]{geometry}
\geometry{papersize={138.0mm,174.0mm}}
\geometry{totalwidth=117.0mm,totalheight=138.0mm}

\usepackage[english]{babel}
\usepackage{mathptmx}

\usepackage{hyperref}
\hypersetup{colorlinks=true,linkcolor=black}
\hypersetup{bookmarksnumbered=true,pdfstartview=FitH,pdfpagemode=UseNone}
\hypersetup{pdftitle={Angular Kinetic Energy}}
\hypersetup{pdfauthor={Alejandro A. Torassa}}

\setlength{\arraycolsep}{1.74pt}

\newcommand{\mM}{m}
\newcommand{\mI}{I}
\newcommand{\ra}{_a}
\newcommand{\vR}{\mathbf{r}}
\newcommand{\vV}{\mathbf{v}}
\newcommand{\aV}{\mathbf{\omega}}
\newcommand{\med}{\raise.5ex\hbox{$\scriptstyle 1$}\kern-.15em/\kern-.15em\lower.25ex\hbox{$\scriptstyle 2$}}

\begin{document}

\begin{center}

{\LARGE Angular Kinetic Energy}

\bigskip \medskip

Alejandro A. Torassa

\bigskip \medskip

\footnotesize

Creative Commons Attribution 3.0 License

(2014) Buenos Aires, Argentina

atorassa@gmail.com

\bigskip \smallskip

\small

{\bf Abstract}

\bigskip

\parbox{87.3mm}{This paper presents an alternative equation to calculate the angular kinetic energy of a particle which describes a circular motion.}

\end{center}

\normalsize

\vspace{-0.60em}

{\centering\subsubsection*{Angular Kinetic Energy}}

\vspace{+0.60em}

\par The angular kinetic energy of a particle A of mass $\mM\ra$, is given by:
\begin{eqnarray*}
\med \hspace{+0.24em} \mM\ra \hspace{+0.06em} (\hspace{+0.03em} \vR \times \vV\ra)^{\hspace{+0.03em} 2}
\end{eqnarray*}
\noindent where $\vR$ is a position vector which is constant in magnitude and direction, and $\vV\ra$ is the velocity of particle A.
\medskip
\par If particle A has an angular velocity $\aV\ra$ and since $\vV\ra = \aV\ra \times \vR\ra$, then we have:
\begin{eqnarray*}
\med \hspace{+0.24em} \mM\ra \hspace{+0.06em} (\hspace{+0.03em} \vR \times (\aV\ra \times \vR\ra))^{\hspace{+0.03em} 2}
\end{eqnarray*}
\par If the position vector $\vR$ is parallel to the angular velocity $\aV\ra$, it follows that:
\begin{eqnarray*}
\med \hspace{+0.24em} \mM\ra^{\vphantom{2}} \hspace{+0.06em} \vR\ra^{\hspace{+0.09em} 2} \hspace{+0.09em} (\hspace{+0.03em} \vR \cdot \aV\ra^{\vphantom{2}})^{\hspace{+0.03em} 2}
\end{eqnarray*}
\par Finally, since $\mM\ra^{\vphantom{2}} \hspace{+0.06em} \vR\ra^{\hspace{+0.09em} 2}$ is the moment of inertia $\mI\ra$ of particle A, we obtain:
\begin{eqnarray*}
\med \hspace{+0.24em} \mI\ra \hspace{+0.06em} (\hspace{+0.03em} \vR \cdot \aV\ra)^{\hspace{+0.03em} 2}
\end{eqnarray*}

\end{document}

