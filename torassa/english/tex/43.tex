
\documentclass[10pt]{article}
%\documentclass[a4paper,10pt]{article}
%\documentclass[letterpaper,10pt]{article}

\usepackage[dvips]{geometry}
\geometry{papersize={141.0mm,201.0mm}}
\geometry{totalwidth=120.0mm,totalheight=165.0mm}

\usepackage[english]{babel}
\usepackage{mathptmx}

\usepackage{hyperref}
\hypersetup{colorlinks=true,linkcolor=black}
\hypersetup{bookmarksnumbered=true,pdfstartview=FitH,pdfpagemode=UseNone}
\hypersetup{pdftitle={A Principle of Conservation of Relational Energy}}
\hypersetup{pdfauthor={Alejandro A. Torassa}}

\setlength{\arraycolsep}{1.74pt}

\begin{document}

\begin{center}

{\Large A Principle of Conservation of Relational Energy}

\bigskip \medskip

Alejandro A. Torassa

\bigskip \medskip

\footnotesize

Creative Commons Attribution 3.0 License

(2014) Buenos Aires, Argentina

atorassa@gmail.com

\bigskip \smallskip

\small

{\bf Abstract}

\bigskip

\parbox{90mm}{In classical mechanics, this paper presents a principle of conservation of relational energy which can be applied in any reference frame without the necessity of introducing fictitious forces.}

\end{center}

\normalsize

\vspace{-0.30em}

{\centering\subsubsection*{The Principle of Conservation}}

\vspace{+0.90em}

\par The kinetic energy $K$ of a system of {\small N} particles of total mass $M$, is given by:
\vspace{+0.60em}
\begin{eqnarray*}
K = \hspace{+0.06em} \sum_{i={\scriptscriptstyle 1}}^{\mathrm N} \hspace{+0.18em} \sum_{j>i}^{\mathrm N} \hspace{+0.12em} \frac{m_i \hspace{+0.09em} m_j}{M} \hspace{+0.18em} ( \hspace{+0.12em} \dot{r}_{ij} \, \dot{r}_{ij} + \ddot{r}_{ij} \, r_{ij} \hspace{+0.09em} )
\end{eqnarray*}
\vspace{+0.15em}
\par The principle of conservation of relational energy states that in an isolated system of {\small N} particles that is only subject to conservative forces (of action and reaction and proportional to $1/r^2$) the relational energy of the system of particles remains constant.
\vspace{-0.60em}
\begin{eqnarray*}
K + U = constant
\end{eqnarray*}
\vspace{-0.45em}
\par \noindent where $r_{ij}=|\vec{r}_i - \vec{r}_j|$, $\dot{r}_{ij}=d|\vec{r}_i - \vec{r}_j|/dt$, $\ddot{r}_{ij}=d^2|\vec{r}_i - \vec{r}_j|/dt^2$, $\vec{r}_i$ and $\vec{r}_j$ are the positions of the \textit{i}-th and \textit{j}-th particles, $m_i$ and $m_j$ are the masses of the \textit{i}-th and \textit{j}-th particles. \hbox {$U$ is} the internal potential energy of the isolated system of particles.

\vspace{+1.50em}

{\centering\subsubsection*{Bibliography}}

\vspace{+1.20em}

{\small

\par \hspace{-0.36em} E. Schr\"{o}dinger. Die Erf\"{u}llbarkeit der Relativit\"{a}tsforderung in der klassischen Mechanik. Annalen der Physik, 1925.
\medskip
\par \hspace{-0.36em} A. K. T. Assis. Relational Mechanics and Implementation of Mach's Principle with Weber's Gravitational Force. Apeiron, 2014.
\medskip
\par \hspace{-0.36em} V. W. Hughes, H. G. Robinson, and V. Beltran-Lopez. Upper Limit for the Anisotropy of Inertial Mass from Nuclear Resonance Experiments. Physical Review Letters, 1960.

}

\end{document}

