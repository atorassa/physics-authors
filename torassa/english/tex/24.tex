
\documentclass[10pt]{article}
%\documentclass[a4paper,10pt]{article}
%\documentclass[letterpaper,10pt]{article}

\usepackage[dvips]{geometry}
\geometry{papersize={121.0mm,188.0mm}}
\geometry{totalwidth=99.9mm,totalheight=154.0mm}

\usepackage[english]{babel}
\usepackage{mathptmx}

\usepackage{hyperref}
\hypersetup{colorlinks=true,linkcolor=black}
\hypersetup{bookmarksnumbered=true,pdfstartview=FitH,pdfpagemode=UseNone}
\hypersetup{pdftitle={Principle of Relativity}}
\hypersetup{pdfauthor={Alejandro A. Torassa}}

\setlength{\unitlength}{0.51pt}
\setlength{\arraycolsep}{1.74pt}

\newcommand{\mX}{x}
\newcommand{\mY}{y}
\newcommand{\mZ}{z}
\newcommand{\mT}{t}
\newcommand{\rt}{'}
\newcommand{\rot}{_{o'}}
\newcommand{\vR}{\mathbf{r}}
\newcommand{\vV}{\mathbf{v}}
\newcommand{\vA}{\mathbf{a}}

\begin{document}

\begin{center}

{\LARGE Principle of Relativity}

\bigskip \medskip

Alejandro A. Torassa

\bigskip \medskip

\footnotesize

Creative Commons Attribution 3.0 License

(2013) Buenos Aires, Argentina

atorassa@gmail.com

\bigskip \smallskip

\small

{\bf Abstract}

\bigskip

\parbox{75mm}{This paper presents a principle of relativity, which states that the laws of physics should only have the same form in all non-rotating reference frames.}

\end{center}

\normalsize

\vspace{-0.45em}

{\centering\subsubsection*{Principle of Relativity}}

\vspace{+0.75em}

\par Any reference frame should be fixed to a body.
\medskip
\par A rotating reference frame cannot represent at all points of space the angular velocity of rotation of a rotating body.
\medskip
\par Any reference frame is an ideal rigid body and, according to the theory of relativity, no body can exceed the speed of light.
\medskip
\par Therefore, no rotating reference frame can have the same angular velocity of rotation at all points of space, since its tangential velocity cannot exceed the speed of light.
\medskip
\par However, it is possible to agree that any reference frame fixed to a body should be non-rotating.
\medskip
\par The laws of physics should be the same for all observers.
\medskip
\par Therefore, according to this paper, the laws of physics should only have the same form in all non-rotating reference frames.
\medskip
\par In addition, several laws of physics would take a simpler form if no reference frame were a rotating reference frame.
\medskip
\par Finally, every body is a possible observer. Therefore, according to this paper, every body is also a possible non-rotating reference frame.

\newpage

{\centering\subsubsection*{Classical Mechanics}}

\vspace{+0.75em}

\par If any reference frame is a non-rotating reference frame, then each coordinate axis of a reference frame S will remain at a fixed angle to the corresponding coordinate axis of another reference frame S'. Therefore, to simplify calculations it will be assumed that each axis of S is parallel to the corresponding axis of S', as shown in the following figure:

\vspace{+1.2em}

\begin{center}
\begin{picture}(228,198)
\multiput(75,75)(45,18){2}{\vector(1,0){90}}
\multiput(75,75)(45,18){2}{\vector(0,1){90}}
\multiput(75,75)(45,18){2}{\vector(-1,-1){60}}
\put(72,171){$\mZ$}\put(117,189){$\mZ\rt$}
\put(171,72){$\mX$}\put(216,90){$\mX\rt$}
\put(3,3){$\mY$}\put(45,18){$\mY\rt$}
\put(78,78){$O$}\put(123,96){$O\rt$}
\put(24,96){S}\put(162,141){S'}
\end{picture}
\end{center}

\smallskip

\par A change of coordinates $\mX$, $\mY$, $\mZ$, $\mT$ from reference frame S to coordinates $\mX\rt$, $\mY\rt$, $\mZ\rt$, $\mT\rt$ from reference frame S' whose origin $O\rt$ has coordinates $\mX\rot$, $\mY\rot$, $\mZ\rot$ measured from S, can be carried out by means of the following equations:
\begin{eqnarray*}
\mX\rt & = & \mX - \mX\rot \\*
\mY\rt & = & \mY - \mY\rot \\*
\mZ\rt & = & \mZ - \mZ\rot \\*
\mT\rt & = & \mT
\end{eqnarray*}
\par From these equations, the transformation of position, velocity and acceleration from reference frame S to reference frame S' may be carried out, and expressed in vector form as follows:
\begin{eqnarray*}
\vR\hspace{+0.06em}\rt & = & \vR - \vR\rot \\
\vV\hspace{+0.06em}\rt & = & \vV - \vV\rot \\
\vA\hspace{+0.06em}\rt & = & \vA - \vA\rot
\end{eqnarray*}
\noindent where $\vR\rot$, $\vV\rot$ and $\vA\rot$ are the position, the velocity and the acceleration, of reference frame S' relative to reference frame S.

\end{document}

