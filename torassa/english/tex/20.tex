
\documentclass[10pt]{article}
%\documentclass[a4paper,10pt]{article}
%\documentclass[letterpaper,10pt]{article}

\usepackage[dvips]{geometry}
\geometry{papersize={138.0mm,168.0mm}}
\geometry{totalwidth=117.0mm,totalheight=132.0mm}

\usepackage[english]{babel}
\usepackage{mathptmx}
\usepackage{chngpage}

\usepackage{hyperref}
\hypersetup{colorlinks=true,linkcolor=black}
\hypersetup{bookmarksnumbered=true,pdfstartview=FitH,pdfpagemode=UseNone}
\hypersetup{pdftitle={Classical Dynamics of Biparticles}}
\hypersetup{pdfauthor={Alejandro A. Torassa}}

\setlength{\arraycolsep}{1.74pt}

\newcommand{\mT}{t}
\newcommand{\mM}{m}
\newcommand{\EM}{M}
\newcommand{\ra}{_a}
\newcommand{\rb}{_b}
\newcommand{\rc}{_c}
\newcommand{\rs}{_s}
\newcommand{\ri}{_i}
\newcommand{\rT}{_T}
\newcommand{\rab}{_{ab}}
\newcommand{\rac}{_{ac}}
\newcommand{\rbc}{_{bc}}
\newcommand{\rcm}{_{cm}}
\newcommand{\til}{\breve}
\newcommand{\dos}{^{\,2}}
\newcommand{\vR}{\mathbf{r}}
\newcommand{\vV}{\mathbf{v}}
\newcommand{\vA}{\mathbf{a}}
\newcommand{\vF}{\mathbf{F}}
\newcommand{\VR}{\mathbf{R}}
\newcommand{\VV}{\mathbf{V}}
\newcommand{\VA}{\mathbf{A}}
\newcommand{\ER}{\mathrm{R}}
\newcommand{\EV}{\mathrm{V}}
\newcommand{\EA}{\mathrm{A}}
\newcommand{\des}{\mathring}
\newcommand{\nK}{{\scriptstyle K}}
\newcommand{\rt}{\hspace{-0.30em}'}
\newcommand{\rij}{_{i\hspace{-0.081em}j}}
\newcommand{\daa}{\hspace{-0.36em}^2\hspace{+0.06em}}
\newcommand{\dab}{\hspace{-0.72em}^2\hspace{+0.36em}}
\newcommand{\med}{\raise.5ex\hbox{$\scriptstyle 1$}\kern-.15em/\kern-.15em\lower.25ex\hbox{$\scriptstyle 2$}\,}

\begin{document}

\begin{center}

{\LARGE Classical Dynamics of Biparticles}

\bigskip \medskip

Alejandro A. Torassa

\bigskip \medskip

\footnotesize

Creative Commons Attribution 3.0 License

(2013) Buenos Aires, Argentina

atorassa@gmail.com

\bigskip \smallskip

\small

{\bf Abstract}

\bigskip

\parbox{84mm}{This paper presents a classical dynamics of biparticles, which can be applied in any non-rotating reference frame (inertial or non-inertial) without the necessity of introducing fictitious forces.}

\end{center}

\normalsize

\vspace{-0.30em}

{\centering\subsubsection*{Definitions}}

\vspace{+1.20em}

\begin{center}
\begin{tabular}{ll}
$\vR\rab = (\vR\ra - \vR\rb)$ = position & $\til\vR\rab = (\til\vR\ra - \til\vR\rb)$ = non-kinetic position \\ \\
$\vV\rab = (\vV\ra - \vV\rb)$ = velocity & $\til\vV\rab = (\til\vV\ra - \til\vV\rb)$ = non-kinetic velocity \\ \\
$\vA\rab = (\vA\ra - \vA\rb)$ = acceleration & $\til\vA\rab = (\til\vA\ra - \til\vA\rb)$ = non-kinetic acceleration
\end{tabular}
\end{center}

{\centering\subsubsection*{Relations}}

\vspace{+1.20em}

\begin{center}
\begin{tabular}{ccc}
\hspace{+0.30em} $\til\vA\rab = \vF\rab/\mM\rab$ & $\rightarrow$ & $\til\vA\rab\dab = (\vF\rab/\mM\rab)\dos$ \\ \\
\hspace{+0.30em} $\til\vV\rab = \int \til\vA\rab \; d\mT$ & $\rightarrow$ & $\til\vV\rab = \int (\vF\rab/\mM\rab) \; d\mT$ \\ \\
\hspace{+0.30em} $\med\til\vV\rab\dab = \int \til\vA\rab \; d\til\vR\rab$ & $\rightarrow$ & $\med\til\vV\rab\dab = \int (\vF\rab/\mM\rab) \; d\til\vR\rab$ \\ \\
\hspace{+0.30em} $\mM\rab = \mM\ra\mM\rb$ & & $\vF\rab = (\vF\ra\mM\rb - \vF\rb\mM\ra)$
\end{tabular}
\end{center}

\newpage

{\centering\subsubsection*{Principles}}

\vspace{+1.20em}

\begin{adjustwidth}{-1.80mm}{-1.80mm}

\begin{center}
\begin{tabular}{ccccc}
{\makebox(6,30){(1)}} & {\framebox(132,30){$\mM\rab\vR\rab - \mM\rab\til\vR\rab = 0$}} & {\makebox(6,30){$\rightarrow$}} & {\framebox(132,30){$\med\mM\rab\vR\rab\dab - \med\mM\rab\til\vR\rab\dab = 0$}} & {\makebox(6,30){(2)}} \\
& {\makebox(132,21){$\downarrow$}} & & {\makebox(132,21){$\downarrow$}} & \\
{\makebox(6,30){(3)}} & {\framebox(132,30){$\mM\rab\vV\rab - \mM\rab\til\vV\rab = 0$}} & {\makebox(6,30){$\rightarrow$}} & {\framebox(132,30){$\med\mM\rab\vV\rab\dab - \med\mM\rab\til\vV\rab\dab = 0$}} & {\makebox(6,30){(4)}} \\
& {\makebox(132,21){$\downarrow$}} & {\makebox(6,21){$\nearrow$}} & {\makebox(132,21){$\downarrow$}} & \\
{\makebox(6,30){(5)}} & {\framebox(132,30){$\mM\rab\vA\rab - \mM\rab\til\vA\rab = 0$}} & {\makebox(6,30){$\rightarrow$}} & {\framebox(132,30){$\med\mM\rab\vA\rab\dab - \med\mM\rab\til\vA\rab\dab = 0$}} & {\makebox(6,30){(6)}}
\end{tabular}
\end{center}

\vspace{+1.50em}

\par \hspace{+0.69em} Substituting the relations into the principles, we obtain:

\vspace{+1.80em}

\begin{center}
\begin{tabular}{ccccc}
{\makebox(6,30){(1)}} & {\framebox(132,30){$\mM\rab\vR\rab - \mM\rab\til\vR\rab = 0$}} & {\makebox(6,30){$\rightarrow$}} & {\framebox(132,30){$\med\mM\rab\vR\rab\dab - \med\mM\rab\til\vR\rab\dab = 0$}} & {\makebox(6,30){(2)}} \\
& {\makebox(132,21){$\downarrow$}} & & {\makebox(132,21){$\downarrow$}} & \\
{\makebox(6,30){(3)}} & {\framebox(132,30){$\mM\rab\vV\rab - \int \vF\rab \; d\mT = 0$}} & {\makebox(6,30){$\rightarrow$}} & {\framebox(132,30){$\med\mM\rab\vV\rab\dab - \int \vF\rab \; d\til\vR\rab = 0$}} & {\makebox(6,30){(4)}} \\
& {\makebox(132,21){$\downarrow$}} & {\makebox(6,21){$\nearrow$}} & {\makebox(132,21){$\downarrow$}} & \\
{\makebox(6,30){(5)}} & {\framebox(132,30){$\mM\rab\vA\rab - \vF\rab = 0$}} & {\makebox(6,30){$\rightarrow$}} & {\framebox(132,30){$\med\mM\rab\vA\rab\dab - \med(\vF\rab\dab/\mM\rab) = 0$}} & {\makebox(6,30){(6)}}
\end{tabular}
\end{center}

\end{adjustwidth}

\newpage

{\centering\subsubsection*{Observations}}

\vspace{+0.60em}

\par A system of particles forms a system of biparticles. For example, the system of particles A, B, C and D forms the system of biparticles AB, AC, AD, BC, BD \hbox {and CD}.
\medskip
\par The dynamics of particles is obtained from the dynamics of biparticles if we only consider the biparticles that have the same particle (Annex A)
\medskip
\par The principles are the transformation equations between a reference frame S and a non-kinetic reference frame $\til\mathrm S$. According to this paper, an observer S uses a reference frame S and a non-kinetic reference frame $\til\mathrm S$.
\medskip
\par The non-kinetic acceleration is related to the non-kinetic forces of interaction (gravitational force, electromagnetic force, etc.) However, the acceleration is related to the kinetic force of interaction (Annex B)
\medskip
\par Finally, from equation (5) it follows that the acceleration $\vA\ra$ of a particle A relative to a reference frame S fixed to a particle S, is given by:
\begin{eqnarray*}
\vA\ra = \frac{\vF\ra}{\mM\ra} - \frac{\vF\rs}{\mM\rs}
\end{eqnarray*}
\noindent where $\vF\ra$ is the net (non-kinetic) force acting on particle A, $\mM\ra$ is the mass of \hbox {particle A}, $\vF\rs$ is the net (non-kinetic) force acting on particle S, and $\mM\rs$ is the mass of particle S.

\vspace{+0.90em}

{\centering\subsubsection*{Bibliography}}

\vspace{+0.30em}

\par \textbf{A. Einstein}, Relativity: The Special and General Theory.
\bigskip
\par \textbf{E. Mach}, The Science of Mechanics.
\bigskip
\par \textbf{H. Goldstein}, Classical Mechanics.

\newpage

{\centering\subsubsection*{Annex A}}

\vspace{+0.60em}

\par If we consider a system of biparticles AB, AC and BC, we have:
\begin{eqnarray*}
\mM\rab\vA\rab + \mM\rac\vA\rac + \mM\rbc\vA\rbc - \vF\rab - \vF\rac - \vF\rbc = 0
\end{eqnarray*}
\par Considering only the biparticles that have particle C, then it follows:
\begin{eqnarray*}
\mM\rac\vA\rac + \mM\rbc\vA\rbc - \vF\rac - \vF\rbc = 0
\end{eqnarray*}
\par Substituting the definitions and the relations into the above equation, we obtain:
\begin{eqnarray*}
\mM\ra\mM\rc(\vA\ra - \vA\rc) + \mM\rb\mM\rc(\vA\rb - \vA\rc) - (\vF\ra\mM\rc - \vF\rc\mM\ra) - (\vF\rb\mM\rc - \vF\rc\mM\rb) = 0
\end{eqnarray*}
\par Dividing by $\mM\rc$, and assuming that observer C fixed to particle C ($\vA\rc = 0$ relative to observer C) is inertial $(\vF\rc = 0)$, finally yields:
\begin{eqnarray*}
\mM\ra\vA\ra + \mM\rb\vA\rb - \vF\ra - \vF\rb = 0
\end{eqnarray*}

\vspace{+0.60em}

{\centering\subsubsection*{Annex B}}

\vspace{+0.60em}

\par The kinetic force $\vF\nK\rab$ exerted on a particle A by another particle B, caused by the interaction between particle A and particle B, is given by:
\begin{eqnarray*}
\vF\nK\rab = \frac{\mM\ra\mM\rb}{\EM\rT}(\vA\ra - \vA\rb)
\end{eqnarray*}
\noindent where $\mM\ra$ is the mass of particle A, $\mM\rb$ is the mass of particle B, $\vA\ra$ is the acceleration of particle A, $\vA\rb$ is the acceleration of particle B, and $\EM\rT$ is the total mass of the Universe.
\par From the above equation it follows that the net kinetic force $\vF\nK\ra$ acting on a particle A, is given by:
\begin{eqnarray*}
\vF\nK\ra = \mM\ra(\vA\ra - \vA\rcm)
\end{eqnarray*}
\noindent where $\mM\ra$ is the mass of particle A, $\vA\ra$ is the acceleration of particle A, and $\vA\rcm$ is the acceleration of the center of mass of the Universe.

\newpage

\begin{center}
{\Large Classical Dynamics of Biparticles II}
\end{center}

\vspace{-0.30em}

{\centering\subsubsection*{Definitions}}

\vspace{+1.20em}

\begin{center}
\begin{tabular}{l}
$\vR\rab = (\vR\ra - \vR\rb)$ = kinetic position \\
$\vV\rab = (\vV\ra - \vV\rb)$ = kinetic velocity \\
$\vA\rab = (\vA\ra - \vA\rb)$ = kinetic acceleration \\ \\
$\til\vR\rab = (\til\vR\ra - \til\vR\rb)$ = non-kinetic position \\
$\til\vV\rab = (\til\vV\ra - \til\vV\rb)$ = non-kinetic velocity \\
$\til\vA\rab = (\til\vA\ra - \til\vA\rb)$ = non-kinetic acceleration \\ \\
$\des\vR\rab = (\vR\rab - \til\vR\rab)$ = total position \\
$\des\vV\rab = (\vV\rab - \til\vV\rab)$ = total velocity \\
$\des\vA\rab = (\vA\rab - \til\vA\rab)$ = total acceleration
\end{tabular}
\end{center}

{\centering\subsubsection*{Relations}}

\vspace{+1.20em}

\begin{center}
\begin{tabular}{ccc}
\hspace{+0.30em} $\des\vA\rab = \des\vF\rab/\mM\rab$ & $\rightarrow$ & $\des\vA\rab\dab = (\des\vF\rab/\mM\rab)\dos$ \\ \\
\hspace{+0.30em} $\des\vV\rab = \int \des\vA\rab \; d\mT$ & $\rightarrow$ & $\des\vV\rab = \int (\des\vF\rab/\mM\rab) \; d\mT$ \\ \\
\hspace{+0.30em} $\med\des\vV\rab\dab = \int \des\vA\rab \; d\des\vR\rab$ & $\rightarrow$ & $\med\des\vV\rab\dab = \int (\des\vF\rab/\mM\rab) \; d\des\vR\rab$ \\ \\
\hspace{+0.30em} $\mM\rab = \mM\ra\mM\rb$ & & $\des\vF\rab = (\vF\rab - \til\vF\rab)$ \\ \\
\hspace{+0.30em} $\vF\rab = (\vF\ra\mM\rb - \vF\rb\mM\ra)$ & & $\vF$ = net kinetic force \\ \\
\hspace{+0.30em} $\til\vF\rab = (\til\vF\ra\mM\rb - \til\vF\rb\mM\ra)$ & & $\til\vF$ = net non-kinetic force
\end{tabular}
\end{center}

\newpage

{\centering\subsubsection*{Principles}}

\vspace{+1.20em}

\begin{adjustwidth}{-1.80mm}{-1.80mm}

\begin{center}
\begin{tabular}{ccccc}
{\makebox(6,30){(1)}} & {\framebox(132,30){$\mM\rab\des\vR\rab = 0$}} & {\makebox(6,30){$\rightarrow$}} & {\framebox(132,30){$\med\mM\rab\des\vR\rab\dab = 0$}} & {\makebox(6,30){(2)}} \\
& {\makebox(132,21){$\downarrow$}} & & {\makebox(132,21){$\downarrow$}} & \\
{\makebox(6,30){(3)}} & {\framebox(132,30){$\mM\rab\des\vV\rab = 0$}} & {\makebox(6,30){$\rightarrow$}} & {\framebox(132,30){$\med\mM\rab\des\vV\rab\dab = 0$}} & {\makebox(6,30){(4)}} \\
& {\makebox(132,21){$\downarrow$}} & {\makebox(6,21){$\nearrow$}} & {\makebox(132,21){$\downarrow$}} & \\
{\makebox(6,30){(5)}} & {\framebox(132,30){$\mM\rab\des\vA\rab = 0$}} & {\makebox(6,30){$\rightarrow$}} & {\framebox(132,30){$\med\mM\rab\des\vA\rab\dab = 0$}} & {\makebox(6,30){(6)}}
\end{tabular}
\end{center}

\vspace{+1.50em}

\par \hspace{+0.69em} Substituting the relations into the principles, we obtain:

\vspace{+1.80em}

\begin{center}
\begin{tabular}{ccccc}
{\makebox(6,30){(1)}} & {\framebox(132,30){$\mM\rab\des\vR\rab = 0$}} & {\makebox(6,30){$\rightarrow$}} & {\framebox(132,30){$\med\mM\rab\des\vR\rab\dab = 0$}} & {\makebox(6,30){(2)}} \\
& {\makebox(132,21){$\downarrow$}} & & {\makebox(132,21){$\downarrow$}} & \\
{\makebox(6,30){(3)}} & {\framebox(132,30){$\int \des\vF\rab \; d\mT = 0$}} & {\makebox(6,30){$\rightarrow$}} & {\framebox(132,30){$\int \des\vF\rab \; d\des\vR\rab = 0$}} & {\makebox(6,30){(4)}} \\
& {\makebox(132,21){$\downarrow$}} & {\makebox(6,21){$\nearrow$}} & {\makebox(132,21){$\downarrow$}} & \\
{\makebox(6,30){(5)}} & {\framebox(132,30){$\des\vF\rab = 0$}} & {\makebox(6,30){$\rightarrow$}} & {\framebox(132,30){$\med(\des\vF\rab\dab/\mM\rab) = 0$}} & {\makebox(6,30){(6)}}
\end{tabular}
\end{center}

\end{adjustwidth}

\newpage

\begin{center}
{\Large Classical Dynamics of Particles II}
\end{center}

\vspace{-0.30em}

{\centering\subsubsection*{Definitions}}

\vspace{+1.20em}

\begin{center}
\begin{tabular}{l}
$\vR\ra\rt = (\vR\ra - \vR\rs)$ = kinetic position \\
$\vV\ra\rt = (\vV\ra - \vV\rs)$ = kinetic velocity \\
$\vA\ra\rt = (\vA\ra - \vA\rs)$ = kinetic acceleration \\ \\
$\til\vR\ra\rt = (\til\vR\ra - \til\vR\rs)$ = non-kinetic position \\
$\til\vV\ra\rt = (\til\vV\ra - \til\vV\rs)$ = non-kinetic velocity \\
$\til\vA\ra\rt = (\til\vA\ra - \til\vA\rs)$ = non-kinetic acceleration \\ \\
$\des\vR\ra = (\vR\ra\rt - \til\vR\ra\rt)$ = total position \\
$\des\vV\ra = (\vV\ra\rt - \til\vV\ra\rt)$ = total velocity \\
$\des\vA\ra = (\vA\ra\rt - \til\vA\ra\rt)$ = total acceleration
\end{tabular}
\end{center}

{\centering\subsubsection*{Relations}}

\vspace{+1.20em}

\begin{center}
\begin{tabular}{ccc}
\hspace{+0.30em} $\des\vA\ra = \des\vF\ra/\mM\ra$ & $\rightarrow$ & $\des\vA\ra\daa = (\des\vF\ra/\mM\ra)\dos$ \\ \\
\hspace{+0.30em} $\des\vV\ra = \int \des\vA\ra \; d\mT$ & $\rightarrow$ & $\des\vV\ra = \int (\des\vF\ra/\mM\ra) \; d\mT$ \\ \\
\hspace{+0.30em} $\med\des\vV\ra\daa = \int \des\vA\ra \; d\des\vR\ra$ & $\rightarrow$ & $\med\des\vV\ra\daa = \int (\des\vF\ra/\mM\ra) \; d\des\vR\ra$ \\ \\
\hspace{+0.30em} $\des\vF\ra = (\vF\ra\rt - \til\vF\ra\rt)$ & & S = reference frame \\ \\
\hspace{+0.30em} $\vF\ra\rt = (\vF\ra\mM\rs - \vF\rs\mM\ra)/\mM\rs$ & & $\vF$ = net kinetic force \\ \\
\hspace{+0.30em} $\til\vF\ra\rt = (\til\vF\ra\mM\rs - \til\vF\rs\mM\ra)/\mM\rs$ & & $\til\vF$ = net non-kinetic force
\end{tabular}
\end{center}

\newpage

{\centering\subsubsection*{Principles}}

\vspace{+1.20em}

\begin{adjustwidth}{-1.80mm}{-1.80mm}

\begin{center}
\begin{tabular}{ccccc}
{\makebox(6,30){(1)}} & {\framebox(132,30){$\mM\ra\des\vR\ra = 0$}} & {\makebox(6,30){$\rightarrow$}} & {\framebox(132,30){$\med\mM\ra\des\vR\ra\daa = 0$}} & {\makebox(6,30){(2)}} \\
& {\makebox(132,21){$\downarrow$}} & & {\makebox(132,21){$\downarrow$}} & \\
{\makebox(6,30){(3)}} & {\framebox(132,30){$\mM\ra\des\vV\ra = 0$}} & {\makebox(6,30){$\rightarrow$}} & {\framebox(132,30){$\med\mM\ra\des\vV\ra\daa = 0$}} & {\makebox(6,30){(4)}} \\
& {\makebox(132,21){$\downarrow$}} & {\makebox(6,21){$\nearrow$}} & {\makebox(132,21){$\downarrow$}} & \\
{\makebox(6,30){(5)}} & {\framebox(132,30){$\mM\ra\des\vA\ra = 0$}} & {\makebox(6,30){$\rightarrow$}} & {\framebox(132,30){$\med\mM\ra\des\vA\ra\daa = 0$}} & {\makebox(6,30){(6)}}
\end{tabular}
\end{center}

\vspace{+1.50em}

\par \hspace{+0.69em} Substituting the relations into the principles, we obtain:

\vspace{+1.80em}

\begin{center}
\begin{tabular}{ccccc}
{\makebox(6,30){(1)}} & {\framebox(132,30){$\mM\ra\des\vR\ra = 0$}} & {\makebox(6,30){$\rightarrow$}} & {\framebox(132,30){$\med\mM\ra\des\vR\ra\daa = 0$}} & {\makebox(6,30){(2)}} \\
& {\makebox(132,21){$\downarrow$}} & & {\makebox(132,21){$\downarrow$}} & \\
{\makebox(6,30){(3)}} & {\framebox(132,30){$\int \des\vF\ra \; d\mT = 0$}} & {\makebox(6,30){$\rightarrow$}} & {\framebox(132,30){$\int \des\vF\ra \; d\des\vR\ra = 0$}} & {\makebox(6,30){(4)}} \\
& {\makebox(132,21){$\downarrow$}} & {\makebox(6,21){$\nearrow$}} & {\makebox(132,21){$\downarrow$}} & \\
{\makebox(6,30){(5)}} & {\framebox(132,30){$\des\vF\ra = 0$}} & {\makebox(6,30){$\rightarrow$}} & {\framebox(132,30){$\med(\des\vF\ra\daa/\mM\ra) = 0$}} & {\makebox(6,30){(6)}}
\end{tabular}
\end{center}

\end{adjustwidth}

\newpage

\begin{center}
{\Large General Principles}
\end{center}

\vspace{+0.60em}

\begin{center}
\begin{tabular}{lll}
Definitions & Particles & Biparticles \hspace{-0.6em} \vspace{+0.9em} \\
Mass & $\hspace{-0.06em}\EM\ri = \sum_{\scriptscriptstyle i} \; \mM\ri$ & $\hspace{-0.06em}\EM\rij = \sum_{\scriptscriptstyle i} \, \sum_{\scriptscriptstyle j>i} \; \mM\rij$ \hspace{-0.6em} \vspace{+0.9em} \\
Total vector position & $\des\VR\ri = \sum_{\scriptscriptstyle i} \; \mM\ri\des\vR\ri \, / \EM\ri$ & $\des\VR\rij = \sum_{\scriptscriptstyle i} \, \sum_{\scriptscriptstyle j>i} \; \mM\rij\des\vR\rij \, / \EM\rij$ \hspace{-0.6em} \vspace{+0.3em} \\
Total vector velocity & $\des\VV\ri = \sum_{\scriptscriptstyle i} \; \mM\ri\des\vV\ri \, / \EM\ri$ & $\des\VV\rij = \sum_{\scriptscriptstyle i} \, \sum_{\scriptscriptstyle j>i} \; \mM\rij\des\vV\rij \, / \EM\rij$ \hspace{-0.6em} \vspace{+0.3em} \\
Total vector acceleration & $\des\VA\ri = \sum_{\scriptscriptstyle i} \; \mM\ri\des\vA\ri \, / \EM\ri$ & $\des\VA\rij = \sum_{\scriptscriptstyle i} \, \sum_{\scriptscriptstyle j>i} \; \mM\rij\des\vA\rij \, / \EM\rij$ \hspace{-0.6em} \vspace{+0.9em} \\
Total scalar position & $\des\ER\ri^{\vphantom{\dos}} = \sum_{\scriptscriptstyle i} \; \med\mM\ri^{\vphantom{\dos}}\des\vR\ri\dos / \EM\ri^{\vphantom{\dos}}$ & $\des\ER\rij^{\vphantom{\dos}} = \sum_{\scriptscriptstyle i} \, \sum_{\scriptscriptstyle j>i} \; \med\mM\rij^{\vphantom{\dos}}\des\vR\rij\dos \, / \EM\rij^{\vphantom{\dos}}$ \hspace{-0.6em} \vspace{+0.3em} \\
Total scalar velocity & $\des\EV\ri^{\vphantom{\dos}} = \sum_{\scriptscriptstyle i} \; \med\mM\ri^{\vphantom{\dos}}\des\vV\ri\dos / \EM\ri^{\vphantom{\dos}}$ & $\des\EV\rij^{\vphantom{\dos}} = \sum_{\scriptscriptstyle i} \, \sum_{\scriptscriptstyle j>i} \; \med\mM\rij^{\vphantom{\dos}}\des\vV\rij\dos \, / \EM\rij^{\vphantom{\dos}}$ \hspace{-0.6em} \vspace{+0.3em} \\
Total scalar acceleration & $\des\EA\ri^{\vphantom{\dos}} = \sum_{\scriptscriptstyle i} \; \med\mM\ri^{\vphantom{\dos}}\des\vA\ri\dos / \EM\ri^{\vphantom{\dos}}$ & $\des\EA\rij^{\vphantom{\dos}} = \sum_{\scriptscriptstyle i} \, \sum_{\scriptscriptstyle j>i} \; \med\mM\rij^{\vphantom{\dos}}\des\vA\rij\dos \, / \EM\rij^{\vphantom{\dos}}$ \hspace{-0.6em} \vspace{+0.9em} \\
\end{tabular}
\end{center}

\vspace{-0.90em}

{\centering\subsubsection*{General Principles}}

\vspace{+1.50em}

\begin{adjustwidth}{-1.80mm}{-1.80mm}

\begin{center}
\begin{tabular}{ccccccc}
{\framebox(54,30){$\des\VR\ri = 0$}} & {\makebox(9,30){$\rightarrow$}} & {\framebox(54,30){$\des\ER\ri = 0$}} & \hspace{+1.20em} & {\framebox(54,30){$\des\VR\rij = 0$}} & {\makebox(9,30){$\rightarrow$}} & {\framebox(54,30){$\des\ER\rij = 0$}} \\
{\makebox(54,21){$\downarrow$}} & & {\makebox(54,21){$\downarrow$}} & \hspace{+1.20em} & {\makebox(54,21){$\downarrow$}} & & {\makebox(54,21){$\downarrow$}} \\
{\framebox(54,30){$\des\VV\ri = 0$}} & {\makebox(9,30){$\rightarrow$}} & {\framebox(54,30){$\des\EV\ri = 0$}} & \hspace{+1.20em} & {\framebox(54,30){$\des\VV\rij = 0$}} & {\makebox(9,30){$\rightarrow$}} & {\framebox(54,30){$\des\EV\rij = 0$}} \\
{\makebox(54,21){$\downarrow$}} & {\makebox(9,21){$\nearrow$}} & {\makebox(54,21){$\downarrow$}} & \hspace{+1.20em} & {\makebox(54,21){$\downarrow$}} & {\makebox(9,21){$\nearrow$}} & {\makebox(54,21){$\downarrow$}} \\
{\framebox(54,30){$\des\VA\ri = 0$}} & {\makebox(9,30){$\rightarrow$}} & {\framebox(54,30){$\des\EA\ri = 0$}} & \hspace{+1.20em} & {\framebox(54,30){$\des\VA\rij = 0$}} & {\makebox(9,30){$\rightarrow$}} & {\framebox(54,30){$\des\EA\rij = 0$}}
\end{tabular}
\end{center}

\end{adjustwidth}

\end{document}

