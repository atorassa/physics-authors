
\documentclass[10pt]{article}
%\documentclass[a4paper,10pt]{article}
%\documentclass[letterpaper,10pt]{article}

\usepackage[dvips]{geometry}
\geometry{papersize={150.0mm,210.0mm}}
\geometry{totalwidth=129.0mm,totalheight=174.0mm}

\usepackage[english]{babel}
\usepackage{mathptmx}

\usepackage{hyperref}
\hypersetup{colorlinks=true,linkcolor=black}
\hypersetup{bookmarksnumbered=true,pdfstartview=FitH,pdfpagemode=UseNone}
\hypersetup{pdftitle={Universal Reference Frame}}
\hypersetup{pdfauthor={Alejandro A. Torassa}}

\setlength{\arraycolsep}{1.74pt}

\newcommand{\mT}{t}
\newcommand{\mM}{m}
\newcommand{\ra}{_a}
\newcommand{\rb}{_b}
\newcommand{\rs}{_s}
\newcommand{\rS}{_S}
\newcommand{\dos}{^{2}}
\newcommand{\rab}{_{ab}}
\newcommand{\rcm}{_{cm}}
\newcommand{\vR}{\mathbf{r}}
\newcommand{\vV}{\mathbf{v}}
\newcommand{\vA}{\mathbf{a}}
\newcommand{\vF}{\mathbf{F}}
\newcommand{\vK}{\mathbf{K}}
\newcommand{\til}{\mathring}
\newcommand{\aV}{\mathbf{\omega}}
\newcommand{\aA}{\mathbf{\alpha}}
\newcommand{\daa}{\hspace{-0.36em}^2\hspace{+0.06em}}
\newcommand{\med}{\raise.5ex\hbox{$\scriptstyle 1$}\kern-.15em/\kern-.15em\lower.25ex\hbox{$\scriptstyle 2$}\:}

\begin{document}

\begin{center}

{\huge Universal Reference Frame}

\bigskip \bigskip

{\large Alejandro A. Torassa}

\bigskip \bigskip

\small

Creative Commons Attribution 3.0 License

(2013) Buenos Aires, Argentina

atorassa@gmail.com

\bigskip \medskip

{\bf Abstract}

\bigskip

In classical mechanics, this paper presents the universal reference frame.

\end{center}

\normalsize

\vspace{-0.30em}

{\centering\subsection*{Universal Reference Frame}}

\vspace{+1.20em}

\par The universal reference frame is a reference frame fixed to the center of mass of the universe.
\medskip
\par The position $\til\vR\ra$, the velocity $\til\vV\ra$, and the acceleration $\til\vA\ra$ of a particle A of mass $\mM\ra$ relative to the universal reference frame $\til\mathrm S$, are given by:
\bigskip\smallskip
\begin{center}
\begin{tabular}{l}
$\til\vR\ra = \int\int \hspace{+0.12em} (\vF\ra/\mM\ra) \; d\mT \; d\mT$ \vspace{+1.20em} \\
$\til\vV\ra = \int \hspace{+0.12em} (\vF\ra/\mM\ra) \; d\mT$ \vspace{+1.20em} \\
$\til\vA\ra = (\vF\ra/\mM\ra)$
\end{tabular}
\end{center}
\medskip\smallskip
\noindent where $\vF\ra$ is the net force acting on particle A.
\bigskip
\par From the above equations the following equations are obtained:
\bigskip\smallskip
\begin{center}
\begin{tabular}{ccc}
\hspace {-2.31em} {\framebox(108,21){$\mM\ra\til\vR\ra - \int\int \hspace{+0.12em} \vF\ra \; d\mT \; d\mT = 0$}} & {\makebox(3,21){$\rightarrow$}} & {\framebox(180,21){$\med\mM\ra\til\vR\ra\daa - \med\mM\ra(\hspace{+0.03em} \int\int \hspace{+0.12em} (\vF\ra/\mM\ra) \; d\mT \; d\mT)\dos = 0$}} \\
\hspace {-2.31em} {\makebox(108,12){$\downarrow$}} & & {\makebox(180,12){$\downarrow$}} \\
\hspace {-2.31em} {\framebox(108,21){$\mM\ra\til\vV\ra - \int \hspace{+0.12em} \vF\ra \; d\mT = 0$}} & {\makebox(3,21){$\rightarrow$}} & {\framebox(180,21){$\med\mM\ra\til\vV\ra\daa - \int \hspace{+0.12em} \vF\ra \; d\til\vR\ra = 0$}} \\
\hspace {-2.31em} {\makebox(108,12){$\downarrow$}} & {\makebox(3,12){$\nearrow$}} & {\makebox(180,12){$\downarrow$}} \\
\hspace {-2.31em} {\framebox(108,21){$\mM\ra\til\vA\ra - \vF\ra = 0$}} & {\makebox(3,21){$\rightarrow$}} & {\framebox(180,21){$\med\mM\ra\til\vA\ra\daa - \med\mM\ra(\vF\ra/\mM\ra)\dos = 0$}}
\end{tabular}
\end{center}
\medskip\smallskip
\noindent where $\; \med\til\vV\ra\daa=\int\hspace{+0.12em}\til\vA\ra\;d\til\vR\ra \;\rightarrow\; \med\mM\ra\til\vV\ra\daa=\int\hspace{+0.12em}\mM\ra\,\til\vA\ra\;d\til\vR\ra \;\rightarrow\; \med\mM\ra\til\vV\ra\daa=\int\hspace{+0.12em}\vF\ra\;d\til\vR\ra$

\newpage

{\centering\subsection*{Reference Frame}}

\vspace{+1.20em}

\par The position $\til\vR\ra$, the velocity $\til\vV\ra$, and the acceleration $\til\vA\ra$ of a particle A of mass $\mM\ra$ relative to a reference frame S, are given by:
\bigskip
\begin{center}
\begin{tabular}{l}
$\til\vR\ra = \vR\ra + \til\vR\rS$ \vspace{+1.20em} \\
$\til\vV\ra = \vV\ra + \til\aV\rS \times \vR\ra + \til\vV\rS$ \vspace{+1.20em} \\
$\til\vA\ra = \vA\ra + 2 \hspace{+0.12em} \til\aV\rS \times \vV\ra + \til\aV\rS \times (\til\aV\rS \times \vR\ra) + \til\aA\rS \times \vR\ra + \til\vA\rS$
\end{tabular}
\end{center}
\medskip
\noindent where $\vR\ra$, $\vV\ra$, and $\vA\ra$ are the position, the velocity, and the acceleration of particle A relative to the reference frame S; $\til\vR\rS$, $\til\vV\rS$, $\til\vA\rS$, $\til\aV\rS$, and $\til\aA\rS$ are the position, the velocity, the acceleration, the angular velocity, and the angular acceleration of the reference frame S relative to the universal reference frame $\til\mathrm S$.
\medskip
\par The position $\til\vR\rS$, the velocity $\til\vV\rS$, the acceleration $\til\vA\rS$, the angular velocity $\til\aV\rS$, and the angular acceleration $\til\aA\rS$ of a reference frame S fixed to a particle S relative to the universal reference frame $\til\mathrm S$, are given by:
\medskip
\begin{center}
\begin{tabular}{l}
\hspace {-5.7em} $\til\vR\rS = \int\int \hspace{+0.12em} (\vF_0/\mM\rs) \; d\mT \; d\mT$ \vspace{+1.20em} \\
\hspace {-5.7em} $\til\vV\rS = \int \hspace{+0.12em} (\vF_0/\mM\rs) \; d\mT$ \vspace{+1.20em} \\
\hspace {-5.7em} $\til\vA\rS = (\vF_0/\mM\rs)$ \vspace{+1.20em} \\
\hspace {-5.7em} $\til\aV\rS = \big|(\vF_1/\mM\rs - \vF_0/\mM\rs)/(\vR_1 - \vR_0)\big|^{1/2}$ \vspace{+1.20em} \\
\hspace {-5.7em} $\til\aA\rS = d(\til\aV\rS)/d\mT$
\end{tabular}
\end{center}
\medskip
\noindent where $\vF_0$ is the net force acting on the reference frame S in a point 0, $\vF_1$ is the net force acting on the reference frame S in a point 1, $\vR_0$ is the position of the point 0 relative to the reference frame S (the point 0 is the center of mass of particle S and the origin of the reference frame S) $\vR_1$ is the position of the point 1 relative to the reference frame S (the point 1 does not belong to the axis of rotation) and $\mM\rs$ is the mass of particle S \hspace{+0.09em}(the vector $\til\aV\rS$ is along the axis of rotation)
\medskip
\par On the other hand, the position $\til\vR\rS$, the velocity $\til\vV\rS$, and the acceleration $\til\vA\rS$ of a reference frame S relative to the universal reference frame $\til\mathrm S$ are related to the position $\vR\rcm$, the velocity $\vV\rcm$, and the acceleration $\vA\rcm$ of the center of mass of the universe relative to the reference frame S.

\newpage

{\centering\subsection*{Kinetic Force}}

\vspace{+1.20em}

\par The kinetic force $\vK\rab$ exerted on a particle A of mass $\mM\ra$ by another particle B of mass $\mM\rb$, caused by the interaction between particle A and particle B, is given by:
\begin{eqnarray*}
\vK\rab = \frac{\mM\ra\mM\rb}{\mM\rcm}(\til\vA\ra - \til\vA\rb)
\end{eqnarray*}
\noindent where $\mM\rcm$ is the mass of the center of mass of the universe, $\til\vA\ra$ and $\til\vA\rb$ are the accelerations of particles A and B relative to the universal reference frame $\til\mathrm S$.
\medskip
\par From the above equation it follows that the net kinetic force $\vK\ra$ acting on a particle A of mass $\mM\ra$, is given by:
\begin{eqnarray*}
\vK\ra = \mM\ra\til\vA\ra
\end{eqnarray*}
\noindent where $\til\vA\ra$ is the acceleration of particle A relative to the universal reference frame $\til\mathrm S$.
\medskip
\par From page [1], we have:
\begin{eqnarray*}
\mM\ra\til\vA\ra - \vF\ra = 0
\end{eqnarray*}
\par That is:
\begin{eqnarray*}
\vK\ra - \vF\ra = 0
\end{eqnarray*}
\par Therefore, the total force $(\vK\ra - \vF\ra)$ acting on a particle A is always in equilibrium.

\vspace{+1.50em}

{\centering\subsection*{Bibliography}}

\vspace{+1.20em}

\par \textbf{A. Einstein}, Relativity: The Special and General Theory.
\bigskip
\par \textbf{E. Mach}, The Science of Mechanics.
\bigskip
\par \textbf{R. Resnick and D. Halliday}, Physics.
\bigskip
\par \textbf{J. Kane and M. Sternheim}, Physics.
\bigskip
\par \textbf{H. Goldstein}, Classical Mechanics.
\bigskip
\par \textbf{L. Landau and E. Lifshitz}, Mechanics.

\end{document}

