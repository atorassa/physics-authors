
\documentclass[10pt]{article}
%\documentclass[a4paper,10pt]{article}
%\documentclass[letterpaper,10pt]{article}

\usepackage[dvips]{geometry}
\geometry{papersize={167.1mm,188.7mm}}
\geometry{totalwidth=146.1mm,totalheight=146.7mm}

\usepackage[english]{babel}
\usepackage{mathptmx}

\usepackage{hyperref}
\hypersetup{colorlinks=true,linkcolor=black}
\hypersetup{bookmarksnumbered=true,pdfstartview=FitH,pdfpagemode=UseNone}
\hypersetup{pdftitle={General Classical Mechanics}}
\hypersetup{pdfauthor={Alejandro A. Torassa}}

\setlength{\arraycolsep}{1.74pt}

\newcommand{\mT}{t}
\newcommand{\mM}{m}
\newcommand{\ri}{_i}
\newcommand{\bre}{\breve}
\newcommand{\vR}{\mathbf{r}}
\newcommand{\vV}{\mathbf{v}}
\newcommand{\vA}{\mathbf{a}}
\newcommand{\vF}{\mathbf{F}}
\newcommand{\rj}{_{\hspace{-0.081em}j}}
\newcommand{\rij}{_{i\hspace{-0.081em}j}}
\newcommand{\med}{\raise.5ex\hbox{$\scriptstyle 1$}\kern-.15em/\kern-.15em\lower.25ex\hbox{$\scriptstyle 2$}\:}

\begin{document}

\begin{center}

{\huge General Classical Mechanics}

\bigskip \bigskip

{\large Alejandro A. Torassa}

\bigskip \bigskip

\small

Creative Commons Attribution 3.0 License

(2014) Buenos Aires, Argentina

atorassa@gmail.com

\bigskip \medskip

{\bf Abstract}

\bigskip

\parbox{96mm}{This paper presents a general classical mechanics which is invariant under transformations between reference frames and which can be applied in any reference frame (rotating or non-rotating) (inertial or non-inertial) without the necessity of introducing fictitious forces.}

\end{center}

\normalsize

\vspace{-0.30em}

{\centering\subsection*{Introduction}}

\vspace{+1.20em}

\par The position $\vR\ri$, the velocity $\vV\ri$, and the acceleration $\vA\ri$ of a particle i of mass $\mM\ri$, are given by:
\bigskip
\begin{center}
\begin{tabular}{l}
\hspace{-3.09em} $\vR\ri = (\vR\ri)$ \vspace{+1.20em} \\
\hspace{-3.09em} $\vV\ri = d(\vR\ri)/d\mT$ \vspace{+1.20em} \\
\hspace{-3.09em} $\vA\ri = d^2(\vR\ri)/d\mT^2$
\end{tabular}
\end{center}
\bigskip
\noindent where $\vR\ri$ is the position vector of particle i.
\medskip
\par And the dynamic position $\bre\vR\ri$, the dynamic velocity $\bre\vV\ri$, and the dynamic acceleration $\bre\vA\ri$, are given by:
\medskip
\begin{center}
\begin{tabular}{l}
$\bre\vR\ri = \int\int \hspace{+0.12em} (\vF\ri/\mM\ri) \; d\mT \; d\mT$ \vspace{+1.20em} \\
$\bre\vV\ri = \int \hspace{+0.12em} (\vF\ri/\mM\ri) \; d\mT$ \vspace{+1.20em} \\
$\bre\vA\ri = (\vF\ri/\mM\ri)$
\end{tabular}
\end{center}
\medskip
\noindent where $\vF\ri$ is the net force acting on particle i.

\newpage

{\centering\subsection*{Equations of Motion}}

\vspace{+1.20em}

\par If we consider two particles i and j then for a reference frame S the equations of motion are:
\vspace{+1.20em}
\par $\med \hspace{+0.15em} \mM\ri\hspace{+0.09em}\mM\rj \hspace{+0.12em} \big[ \hspace{+0.12em} (\vR \hspace{+0.18em}{\S}\hspace{+0.18em} \vR\rij) \cdot (\vR \hspace{+0.18em}{\S}\hspace{+0.18em} \vR\rij) - (\vR \hspace{+0.18em}{\S}\hspace{+0.18em} \bre\vR\rij) \cdot (\vR \hspace{+0.18em}{\S}\hspace{+0.18em} \bre\vR\rij) \hspace{+0.12em} \big] = \hspace{+0.12em} 0$ \\
\vspace{+0.30em}
\par $\med \hspace{+0.15em} \mM\ri\hspace{+0.09em}\mM\rj \hspace{+0.12em} \big[ \hspace{+0.12em} (\vR \hspace{+0.18em}{\S}\hspace{+0.18em} \vR\rij) \cdot (\vR \hspace{+0.18em}{\S}\hspace{+0.18em} \vV\rij) - (\vR \hspace{+0.18em}{\S}\hspace{+0.18em} \vR\rij) \cdot \int (\vR \hspace{+0.18em}{\S}\hspace{+0.18em} \bre\vA\rij) \hspace{+0.18em} dt \hspace{+0.12em} \big] = \hspace{+0.12em} 0$ \\
\vspace{+0.30em}
\par $\med \hspace{+0.15em} \mM\ri\hspace{+0.09em}\mM\rj \hspace{+0.12em} \big[ \hspace{+0.12em} (\vR \hspace{+0.18em}{\S}\hspace{+0.18em} \vV\rij) \cdot (\vR \hspace{+0.18em}{\S}\hspace{+0.18em} \vV\rij) + (\vR \hspace{+0.18em}{\S}\hspace{+0.18em} \vA\rij) \cdot (\vR \hspace{+0.18em}{\S}\hspace{+0.18em} \vR\rij) - 2 \int (\vR \hspace{+0.18em}{\S}\hspace{+0.18em} \bre\vA\rij) \cdot d(\vR \hspace{+0.18em}{\S}\hspace{+0.18em} \vR\rij) - (\vR \hspace{+0.18em}{\S}\hspace{+0.18em} \bre\vA\rij) \cdot (\vR \hspace{+0.18em}{\S}\hspace{+0.18em} \vR\rij) \hspace{+0.12em} \big] = \hspace{+0.12em} 0$
\vspace{+1.20em}
\par \noindent where $\vR\rij = \vR\ri - \vR\rj$, $\vV\rij = \vV\ri - \vV\rj$, $\vA\rij = \vA\ri - \vA\rj$, $\bre\vR\rij = \bre\vR\ri - \bre\vR\rj$, $\bre\vA\rij = \bre\vA\ri - \bre\vA\rj$, $\mM\ri$ and $\mM\rj$ are the masses of particles i and j, $\vR\ri$, $\vR\rj$, $\vV\ri$, $\vV\rj$, $\vA\ri$ and $\vA\rj$ are the positions, the velocities and the accelerations of particles i and j, and $\bre\vR\ri$, $\bre\vR\rj$, $\bre\vA\ri$ and $\bre\vA\rj$ are the dynamic positions and the dynamic accelerations of particles i and j.
\vspace{+1.20em}
\par \noindent $\vR$ is a position vector defined by two fixed points 1 and 2 of the reference frame S ($\vR = \vR_1 - \vR_2$) in which the dynamic acceleration of the point 1 is equal to the dynamic acceleration of the point 2 ($\bre\vA_1 = \bre\vA_2$)
\vspace{+0.90em}
\par \noindent $\vR$ is invariant under transformations between reference frames.
\vspace{+0.90em}
\par \noindent ${\S}$ can be changed by the following operators:
\vspace{+0.90em}
\par\hspace{-1.20em}\begin{tabular}{cl}
\vspace{+0.90em}
\par $\ast$ & First excluded product: $\vec{A} \ast \vec{B} = \vec{B}$ \\
\vspace{+1.20em}
\par $\cdot$ & Dot product: $\vec{A} \cdot \vec{B} = |\vec{A}| |\vec{B}| \cos \theta$ \\
\vspace{+1.20em}
\par $\times$ & Cross product: $\vec{A} \times \vec{B} = |\vec{A}| |\vec{B}| \sin \theta \hspace{+0.12em} \hat{n}$ \hspace{+0.12em} (direction: right-hand rule) \\
\vspace{+1.20em}
\par $:$ & Vector dot product: $\vec{A} : \vec{B} = |\vec{A}| |\vec{B}| \cos \theta \hspace{+0.12em} \hat{n}$ \hspace{+0.12em} (direction: same as $\vec{A}$) \\
\vspace{+1.20em}
\par $\dot{\times}$ & Scalar cross product: $\vec{A} \hspace{+0.12em}\dot{\times}\hspace{+0.12em} \vec{B} = |\vec{A}| |\vec{B}| \sin \theta$ \hspace{+0.12em} (sign: right-hand rule)
\end{tabular}
\vspace{-0.12em}
\par \noindent \textbf{Note}: In this paper the following rule is valid:
\vspace{+0.90em}
\par \noindent (scalar)\hspace{+0.09em}$\cdot$\hspace{+0.09em}(scalar)\hspace{+0.27em}=\hspace{+0.27em}(scalar)\hspace{+0.27em}(scalar)

\newpage

{\centering\subsection*{Reference Frames}}

\vspace{+1.20em}

\par The magnitudes $\bre\vR\ri$, $\bre\vV\ri$ and $\bre\vA\ri$ are invariant under transformations between reference frames.
\bigskip
\par In any reference frame $\vR\rij = \bre\vR\rij$. Therefore, $\vR\rij$ is invariant under transformations between reference frames.
\bigskip
\par In any non-rotating reference frame $\vV\rij = \bre\vV\rij$ and $\vA\rij = \bre\vA\rij$. Therefore, $\vV\rij$ and $\vA\rij$ are invariant under transformations between non-rotating reference frames.
\bigskip
\par In any inertial reference frame $\vA\ri = \bre\vA\ri$. Therefore, $\vA\ri$ is invariant under transformations between inertial reference frames. Any inertial reference frame is a non-rotating reference frame.
\bigskip
\par In the universal reference frame $\vR\ri = \bre\vR\ri$, $\vV\ri = \bre\vV\ri$ and $\vA\ri = \bre\vA\ri$. Therefore, the universal reference frame is an inertial reference frame.
\bigskip
\par The universal reference frame is a reference frame fixed to the center of mass of the universe (if the net force acting on the center of mass of the universe is always zero)

\vspace{+1.50em}

{\centering\subsection*{Observations}}

\vspace{+1.20em}

\par The equations of motion are also conservation equations.
\bigskip
\par The equations of motion are invariant under transformations between reference frames.
\bigskip
\par The equations of motion can be applied in any reference frame (rotating or non-rotating) (inertial or non-inertial) without the necessity of introducing fictitious forces.
\bigskip
\par The equations of motion would be valid even if Newton's third law of motion were false in an inertial reference frame.
\bigskip
\par The equations of motion would be valid even if Newton's three laws of motion were false in a non-inertial reference frame.
\bigskip
\par The equations of motion are transformation equations between reference frames, and they can be obtained from the general equation of motion ( \hspace{-0.45em} \textbf{A. Torassa}, General Equation of Motion \hspace{-0.45em} )

\newpage

{\centering\subsection*{Annex}}

\vspace{+1.20em}

{\centering\subsection*{\rm Work, K and U}}

\vspace{+1.50em}

\par $W\rij = \med \hspace{+0.15em} \mM\ri\hspace{+0.09em}\mM\rj \hspace{+0.12em} \big[ \hspace{+0.12em} 2 \int_{\scriptscriptstyle \hspace{-0.03em} 1}^{\scriptscriptstyle 2} (\vR \hspace{+0.18em}{\S}\hspace{+0.18em} \bre\vA\rij) \cdot d(\vR \hspace{+0.18em}{\S}\hspace{+0.18em} \vR\rij) + \Delta \hspace{+0.18em} (\vR \hspace{+0.18em}{\S}\hspace{+0.18em} \bre\vA\rij) \cdot (\vR \hspace{+0.18em}{\S}\hspace{+0.18em} \vR\rij) \hspace{+0.12em} \big]$
\vspace{+1.20em}
\par $W\rij = \Delta \hspace{+0.24em} K\rij$
\vspace{+1.20em}
\par $\Delta \hspace{+0.24em} K\rij = \hspace{+0.09em} \Delta \hspace{+0.24em} \med \hspace{+0.15em} \mM\ri\hspace{+0.09em}\mM\rj \hspace{+0.12em} \big[ \hspace{+0.12em} (\vR \hspace{+0.18em}{\S}\hspace{+0.18em} \vV\rij) \cdot (\vR \hspace{+0.18em}{\S}\hspace{+0.18em} \vV\rij) + (\vR \hspace{+0.18em}{\S}\hspace{+0.18em} \vA\rij) \cdot (\vR \hspace{+0.18em}{\S}\hspace{+0.18em} \vR\rij) \hspace{+0.12em} \big]$
\vspace{+1.20em}
\par $\Delta \hspace{+0.24em} U\rij = - \hspace{+0.18em} \med \hspace{+0.15em} \mM\ri\hspace{+0.09em}\mM\rj \hspace{+0.12em} \big[ \hspace{+0.12em} 2 \int_{\scriptscriptstyle \hspace{-0.03em} 1}^{\scriptscriptstyle 2} (\vR \hspace{+0.18em}{\S}\hspace{+0.18em} \bre\vA\rij) \cdot d(\vR \hspace{+0.18em}{\S}\hspace{+0.18em} \vR\rij) + \Delta \hspace{+0.18em} (\vR \hspace{+0.18em}{\S}\hspace{+0.18em} \bre\vA\rij) \cdot (\vR \hspace{+0.18em}{\S}\hspace{+0.18em} \vR\rij) \hspace{+0.12em} \big]$

\vspace{+2.10em}

{\centering\subsection*{\rm Principle of Least Action}}

\vspace{+1.50em}

\par $\delta \int_{\scriptscriptstyle \hspace{-0.03em} 1}^{\scriptscriptstyle 2} \hspace{+0.18em} L\rij \hspace{+0.24em} dt = 0$
\vspace{+1.20em}
\par $\delta \int_{\scriptscriptstyle \hspace{-0.03em} 1}^{\scriptscriptstyle 2} \hspace{+0.18em} (T\rij - V\rij) \hspace{+0.24em} dt = 0$
\vspace{+1.20em}
\par $T\rij = + \hspace{+0.18em} \med \hspace{+0.15em} \mM\ri\hspace{+0.09em}\mM\rj \hspace{+0.12em} \big[ \hspace{+0.12em} (\vR \hspace{+0.18em}{\S}\hspace{+0.18em} \vV\rij) \cdot (\vR \hspace{+0.18em}{\S}\hspace{+0.18em} \vV\rij) + (\vR \hspace{+0.18em}{\S}\hspace{+0.18em} \vA\rij) \cdot (\vR \hspace{+0.18em}{\S}\hspace{+0.18em} \vR\rij) \hspace{+0.12em} \big]$
\vspace{+1.20em}
\par $V\rij = - \hspace{+0.18em} \med \hspace{+0.15em} \mM\ri\hspace{+0.09em}\mM\rj \hspace{+0.12em} \big[ \hspace{+0.12em} 2 \int (\vR \hspace{+0.18em}{\S}\hspace{+0.18em} \bre\vA\rij) \cdot d(\vR \hspace{+0.18em}{\S}\hspace{+0.18em} \vR\rij) + (\vR \hspace{+0.18em}{\S}\hspace{+0.18em} \bre\vA\rij) \cdot (\vR \hspace{+0.18em}{\S}\hspace{+0.18em} \vR\rij) \hspace{+0.12em} \big]$

\vspace{+2.10em}

{\centering\subsection*{\rm Group of Invariant Equations}}

\vspace{+1.50em}

\par $\med \hspace{+0.15em} \mM\ri\hspace{+0.09em}\mM\rj \hspace{+0.12em} \big[ \hspace{+0.12em} (\vR \hspace{+0.18em}{\S}\hspace{+0.18em} \vR\rij) \cdot (\vR \hspace{+0.18em}{\S}\hspace{+0.18em} \vR\rij) \hspace{+0.12em} \big] = \hspace{+0.12em}$ Invariant from S to S'
\vspace{+1.20em}
\par $\med \hspace{+0.15em} \mM\ri\hspace{+0.09em}\mM\rj \hspace{+0.12em} \big[ \hspace{+0.12em} (\vR \hspace{+0.18em}{\S}\hspace{+0.18em} \vR\rij) \cdot (\vR \hspace{+0.18em}{\S}\hspace{+0.18em} \vV\rij) \hspace{+0.12em} \big] = \hspace{+0.12em}$ Invariant from S to S'
\vspace{+1.20em}
\par $\med \hspace{+0.15em} \mM\ri\hspace{+0.09em}\mM\rj \hspace{+0.12em} \big[ \hspace{+0.12em} (\vR \hspace{+0.18em}{\S}\hspace{+0.18em} \vV\rij) \cdot (\vR \hspace{+0.18em}{\S}\hspace{+0.18em} \vV\rij) + (\vR \hspace{+0.18em}{\S}\hspace{+0.18em} \vA\rij) \cdot (\vR \hspace{+0.18em}{\S}\hspace{+0.18em} \vR\rij) \hspace{+0.12em} \big] = \hspace{+0.12em}$ Invariant from S to S'

\end{document}

