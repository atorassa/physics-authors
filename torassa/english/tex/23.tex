
\documentclass[10pt]{article}
%\documentclass[a4paper,10pt]{article}
%\documentclass[letterpaper,10pt]{article}

\usepackage[dvips]{geometry}
\geometry{papersize={167.1mm,213.9mm}}
\geometry{totalwidth=146.1mm,totalheight=177.9mm}

\usepackage[english]{babel}
\usepackage{mathptmx}

\usepackage{hyperref}
\hypersetup{colorlinks=true,linkcolor=black}
\hypersetup{bookmarksnumbered=true,pdfstartview=FitH,pdfpagemode=UseNone}
\hypersetup{pdftitle={Alternative Classical Mechanics}}
\hypersetup{pdfauthor={Alejandro A. Torassa}}

\setlength{\arraycolsep}{1.74pt}

\newcommand{\mT}{t}
\newcommand{\mM}{m}
\newcommand{\ra}{_a}
\newcommand{\rb}{_b}
\newcommand{\rs}{_s}
\newcommand{\rS}{_S}
\newcommand{\dos}{^{2}}
\newcommand{\rcm}{_{cm}}
\newcommand{\bre}{\breve}
\newcommand{\til}{\tilde}
\newcommand{\uni}{\mathring}
\newcommand{\vR}{\mathbf{r}}
\newcommand{\vV}{\mathbf{v}}
\newcommand{\vA}{\mathbf{a}}
\newcommand{\vF}{\mathbf{F}}
\newcommand{\vK}{\mathbf{K}}
\newcommand{\aV}{\mathbf{\omega}}
\newcommand{\aA}{\mathbf{\alpha}}
\newcommand{\daa}{\hspace{-0.36em}^2\hspace{+0.06em}}
\newcommand{\med}{\raise.5ex\hbox{$\scriptstyle 1$}\kern-.15em/\kern-.15em\lower.25ex\hbox{$\scriptstyle 2$}\:}

\begin{document}

\begin{center}

{\huge Alternative Classical Mechanics}

\bigskip \bigskip

{\large Alejandro A. Torassa}

\bigskip \bigskip

\small

Creative Commons Attribution 3.0 License

(2013) Buenos Aires, Argentina

atorassa@gmail.com

\bigskip \medskip

{\bf Abstract}

\bigskip

\parbox{96mm}{This paper presents an alternative classical mechanics, which can be applied in any reference frame (rotating or non-rotating) (inertial or non-inertial) without the necessity of introducing fictitious forces.}

\end{center}

\normalsize

\vspace{-0.30em}

{\centering\subsection*{Universal Reference Frame}}

\vspace{+1.20em}

\par The universal reference frame is a reference frame fixed to the center of mass of the universe.
\medskip
\par The universal position $\uni\vR\ra$, the universal velocity $\uni\vV\ra$, and the universal acceleration $\uni\vA\ra$ of a particle A relative to the universal reference frame $\uni\mathrm S$, are given by:
\bigskip
\begin{center}
\begin{tabular}{l}
\hspace{-3.09em} $\uni\vR\ra = (\vR\ra)$ \vspace{+1.20em} \\
\hspace{-3.09em} $\uni\vV\ra = d(\vR\ra)/d\mT$ \vspace{+1.20em} \\
\hspace{-3.09em} $\uni\vA\ra = d^2(\vR\ra)/d\mT^2$
\end{tabular}
\end{center}
\medskip
\noindent where $\vR\ra$ is the position of particle A relative to the universal reference frame $\uni\mathrm S$.
\medskip
\par The dynamic position $\bre\vR\ra$, the dynamic velocity $\bre\vV\ra$, and the dynamic acceleration $\bre\vA\ra$ of a particle A of mass $\mM\ra$, are given by:
\medskip
\begin{center}
\begin{tabular}{l}
$\bre\vR\ra = \int\int \hspace{+0.12em} (\vF\ra/\mM\ra) \; d\mT \; d\mT$ \vspace{+1.20em} \\
$\bre\vV\ra = \int \hspace{+0.12em} (\vF\ra/\mM\ra) \; d\mT$ \vspace{+1.20em} \\
$\bre\vA\ra = (\vF\ra/\mM\ra)$
\end{tabular}
\end{center}
\medskip
\noindent where $\vF\ra$ is the net force acting on particle A.

\newpage

{\centering\subsection*{General Principle}}

\vspace{+1.20em}

\par The total position $\til\vR\ra$ of a particle A, is given by:
\begin{eqnarray*}
\hspace{+0.60em} \til\vR\ra = \uni\vR\ra - \bre\vR\ra
\end{eqnarray*}
\par The general principle establishes that the total position $\til\vR\ra$ of a particle A is always in equilibrium.
\begin{eqnarray*}
\hspace{+0.45em} \til\vR\ra = 0
\end{eqnarray*}

\vspace{+0.60em}

{\centering\subsection*{Observations}}

\vspace{+1.20em}

\par Applying the general principle to a particle A, it follows:
\bigskip\smallskip
\begin{center}
\begin{tabular}{ccc}
{\framebox(96,27){$\mM\ra\uni\vR\ra - \mM\ra\bre\vR\ra = 0$}} & {\makebox(3,27){$\rightarrow$}} & {\framebox(120,27){$\med\mM\ra\uni\vR\ra\daa - \med\mM\ra\bre\vR\ra\daa = 0$}} \\
{\makebox(96,12){$\downarrow$}} & & {\makebox(120,12){$\downarrow$}} \\
{\framebox(96,27){$\mM\ra\uni\vV\ra - \mM\ra\bre\vV\ra = 0$}} & {\makebox(3,27){$\rightarrow$}} & {\framebox(120,27){$\med\mM\ra\uni\vV\ra\daa - \med\mM\ra\bre\vV\ra\daa = 0$}} \\
{\makebox(96,12){$\downarrow$}} & {\makebox(3,12){$\nearrow$}} & {\makebox(120,12){$\downarrow$}} \\
{\framebox(96,27){$\mM\ra\uni\vA\ra - \mM\ra\bre\vA\ra = 0$}} & {\makebox(3,27){$\rightarrow$}} & {\framebox(120,27){$\med\mM\ra\uni\vA\ra\daa - \med\mM\ra\bre\vA\ra\daa = 0$}}
\end{tabular}
\end{center}
\medskip\smallskip
\par Substituting $\bre\vR\ra$, $\bre\vV\ra$ and $\bre\vA\ra$ from page [1] into the above equations, we obtain:
\bigskip\smallskip
\begin{center}
\begin{tabular}{ccc}
{\framebox(123,27){$\mM\ra\uni\vR\ra - \int\int \hspace{+0.12em} \vF\ra \; d\mT \; d\mT = 0$}} & {\makebox(3,27){$\rightarrow$}} & {\framebox(192,27){$\med\mM\ra\uni\vR\ra\daa - \med\mM\ra(\hspace{+0.03em} \int\int \hspace{+0.12em} (\vF\ra/\mM\ra) \; d\mT \; d\mT)\dos = 0$}} \\
{\makebox(123,12){$\downarrow$}} & & {\makebox(192,12){$\downarrow$}} \\
{\framebox(123,27){$\mM\ra\uni\vV\ra - \int \hspace{+0.12em} \vF\ra \; d\mT = 0$}} & {\makebox(3,27){$\rightarrow$}} & {\framebox(192,27){$\med\mM\ra\uni\vV\ra\daa - \int \hspace{+0.12em} \vF\ra \; d\uni\vR\ra = 0$}} \\
{\makebox(123,12){$\downarrow$}} & {\makebox(3,12){$\nearrow$}} & {\makebox(192,12){$\downarrow$}} \\
{\framebox(123,27){$\mM\ra\uni\vA\ra - \vF\ra = 0$}} & {\makebox(3,27){$\rightarrow$}} & {\framebox(192,27){$\med\mM\ra\uni\vA\ra\daa - \med\mM\ra(\vF\ra/\mM\ra)\dos = 0$}}
\end{tabular}
\end{center}
\medskip\smallskip
\par Where $\; \med\bre\vV\ra\daa=\int\hspace{+0.12em}\bre\vA\ra\;d\bre\vR\ra \;\rightarrow\; \med\mM\ra\bre\vV\ra\daa=\int\hspace{+0.12em}\mM\ra\,\bre\vA\ra\;d\bre\vR\ra \;\rightarrow\; \med\mM\ra\bre\vV\ra\daa=\int\hspace{+0.12em}\vF\ra\;d\bre\vR\ra \;\;\; (\bre\vR\ra=\uni\vR\ra)$

\newpage

{\centering\subsection*{Transformations}}

\vspace{+1.20em}

\par The universal position $\uni\vR\ra$, the universal velocity $\uni\vV\ra$, and the universal acceleration $\uni\vA\ra$ of a particle A relative to a reference frame S, are given by:
\bigskip
\begin{center}
\begin{tabular}{l}
$\uni\vR\ra = \vR\ra + \bre\vR\rS$ \vspace{+1.20em} \\
$\uni\vV\ra = \vV\ra + \bre\aV\rS \times \vR\ra + \bre\vV\rS$ \vspace{+1.20em} \\
$\uni\vA\ra = \vA\ra + 2 \hspace{+0.06em} \bre\aV\rS \times \vV\ra + \bre\aV\rS \times (\bre\aV\rS \times \vR\ra) + \bre\aA\rS \times \vR\ra + \bre\vA\rS$
\end{tabular}
\end{center}
\medskip
\noindent where $\vR\ra$, $\vV\ra$, and $\vA\ra$ are the position, the velocity, and the acceleration of particle A relative to the reference frame S; $\bre\vR\rS$, $\bre\vV\rS$, $\bre\vA\rS$, $\bre\aV\rS$, and $\bre\aA\rS$ are the dynamic position, the dynamic velocity, the dynamic acceleration, the dynamic angular velocity and the dynamic angular acceleration of the reference \hbox {frame S}.
\vspace{-0.75em}
\par The dynamic position $\bre\vR\rS$, the dynamic velocity $\bre\vV\rS$, the dynamic acceleration $\bre\vA\rS$, the dynamic angular velocity $\bre\aV\rS$, and the dynamic angular acceleration $\bre\aA\rS$ of a reference frame S fixed to a particle S, are given by:
\medskip
\begin{center}
\begin{tabular}{l}
\hspace {-5.10em} $\bre\vR\rS = \int\int \hspace{+0.12em} (\vF_0/\mM\rs) \; d\mT \; d\mT$ \vspace{+1.20em} \\
\hspace {-5.10em} $\bre\vV\rS = \int \hspace{+0.12em} (\vF_0/\mM\rs) \; d\mT$ \vspace{+1.20em} \\
\hspace {-5.10em} $\bre\vA\rS = (\vF_0/\mM\rs)$ \vspace{+1.20em} \\
\hspace {-5.10em} $\bre\aV\rS = \pm \hspace{+0.12em} \big|(\vF_1/\mM\rs - \vF_0/\mM\rs) \cdot (\vR_1 - \vR_0)/(\vR_1 - \vR_0)^2\big|^{1/2}$ \vspace{+1.20em} \\
\hspace {-5.10em} $\bre\aA\rS = d(\bre\aV\rS)/d\mT$
\end{tabular}
\end{center}
\medskip
\noindent where $\vF_0$ and $\vF_1$ are the net forces acting on the reference frame S in the points 0 and 1, $\vR_0$ and $\vR_1$ are the positions of the points 0 and 1 relative to the reference frame S, and $\mM\rs$ is the mass of \hbox {particle S} \hbox {(the point 0} is the origin of the reference frame S and the center of mass of particle S) (the point 0 belongs to the axis of dynamic rotation, and the segment 01 is perpendicular to the axis of dynamic rotation) \hbox {(the vector} $\bre\aV\rS$ is along the axis of dynamic rotation)
\medskip
\par The magnitudes $\bre\vR$, $\bre\vV$, $\bre\vA$, $\bre\aV$, and $\bre\aA$ are invariant under transformations between reference frames.

\vspace{+0.30em}

{\centering\subsection*{Inertial Reference Frame}}

\vspace{+1.20em}

\par A reference frame S is inertial if $\bre\aV\rS=0$ and $\bre\vA\rS=0$, but it is non-inertial if $\bre\aV\rS \ne 0$ or $\bre\vA\rS \ne 0$.
\medskip
\par This paper considers that the universal reference frame $\uni\mathrm S$ is always inertial.

\newpage

{\centering\subsection*{Equation of Motion}}

\vspace{+1.20em}

\par From the general principle and the transformations it follows that the acceleration $\vA\ra$ of a particle A of mass $\mM\ra$ relative to a reference frame S fixed to a particle S of mass $\mM\rs$, is given by:
\begin{eqnarray*}
\vA\ra = \vF\ra/\mM\ra - 2 \hspace{+0.06em} \bre\aV\rS \times \vV\ra - \bre\aV\rS \times (\bre\aV\rS \times \vR\ra) - \bre\aA\rS \times \vR\ra - \vF\rs/\mM\rs
\end{eqnarray*}
\noindent where $\vF\ra$ is the net force acting on particle A, and $\vF\rs$ is the net force acting on particle S.
\medskip
\par In contradiction with Newton's first and second laws, from the last equation it follows that \hbox {particle A} can have non-zero acceleration even if there is no force acting on particle A, and also that particle A can have zero acceleration (state of rest or of uniform linear motion) even if there is an unbalanced force acting on particle A.
\medskip
\par In order to apply Newton's first and second laws in a non-inertial reference frame, it is necessary to introduce fictitious forces.
\medskip
\par However, this paper considers that Newton's first and second laws are false. Therefore, in this paper there is no need to introduce fictitious forces.

\vspace{+1.50em}

{\centering\subsection*{Kinetic Force}}

\vspace{+1.20em}

\par The kinetic force $\vK_{a|b}$ exerted on a particle A of mass $\mM\ra$ by another particle B of mass $\mM\rb$, caused by the interaction between particle A and particle B, is given by:
\begin{eqnarray*}
\vK_{a|b} = \frac{\mM\ra\mM\rb}{\mM\rcm}(\uni\vA\ra - \uni\vA\rb)
\end{eqnarray*}
\noindent where $\mM\rcm$ is the mass of the center of mass of the universe, $\uni\vA\ra$ and $\uni\vA\rb$ are the universal accelerations of particles A and B.
\medskip
\par From the above equation it follows that the net kinetic force $\vK\ra$ acting on a particle A of mass $\mM\ra$, is given by:
\begin{eqnarray*}
\vK\ra = \mM\ra\uni\vA\ra
\end{eqnarray*}
\noindent where $\uni\vA\ra$ is the universal acceleration of particle A.
\medskip
\par Now from page [2], we have:
\begin{eqnarray*}
\mM\ra\uni\vA\ra - \vF\ra = 0
\end{eqnarray*}
\par That is:
\begin{eqnarray*}
\vK\ra - \vF\ra = 0
\end{eqnarray*}
\par Therefore, the total force $(\vK\ra - \vF\ra)$ acting on a particle A is always in equilibrium.

\end{document}

