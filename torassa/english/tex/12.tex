
\documentclass[10pt]{article}
%\documentclass[a4paper,10pt]{article}
%\documentclass[letterpaper,10pt]{article}

\usepackage[dvips]{geometry}
\geometry{papersize={127.0mm,165.1mm}}
\geometry{totalwidth=99.5mm,totalheight=125.9mm}

\usepackage[english]{babel}
\usepackage{mathptmx}

\usepackage{hyperref}
\hypersetup{colorlinks=true,linkcolor=black}
\hypersetup{bookmarksnumbered=true,pdfstartview=FitH,pdfpagemode=UseNone}
\hypersetup{pdftitle={Principle of Relativity}}
\hypersetup{pdfauthor={Alejandro A. Torassa}}

\begin{document}

\begin{center}

{\LARGE Principle of Relativity}

\bigskip \medskip

{\large Alejandro A. Torassa}

\bigskip \medskip

\footnotesize

Creative Commons Attribution 3.0 License

(2011) Buenos Aires, Argentina

atorassa@gmail.com

\bigskip \smallskip

\small

{\bf Abstract}

\bigskip

\parbox{75mm}{This paper presents a principle of relativity, which states that the laws of physics should only have the same form in all non-rotating reference frames.}

\end{center}

\normalsize

\vspace{-0.60em}

{\centering\subsection*{Part One}}

\par Any reference frame should be fixed to a body.
\medskip
\par It is possible to agree that any reference frame fixed to a body should be non-rotating.
\medskip
\par Therefore, the laws of physics should only have the same form in all non-rotating reference frames.

\vspace{+0.99em}

{\centering\subsection*{Part Two}}

\par A rotating reference frame cannot represent at all points of space the angular velocity of rotation of a rotating body.
\medskip
\par Any reference frame is an ideal rigid body and, according to the theory of relativity, no body can exceed the speed of light.
\medskip
\par Therefore, no rotating reference frame can have the same angular velocity of rotation at all points of space, since its tangential velocity cannot exceed the speed of light.

\end{document}

