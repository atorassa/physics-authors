
\documentclass[10pt]{article}
%\documentclass[a4paper,10pt]{article}
%\documentclass[letterpaper,10pt]{article}

\usepackage[dvips]{geometry}
\geometry{papersize={164.1mm,223.5mm}}
\geometry{totalwidth=143.1mm,totalheight=187.5mm}

\usepackage[english]{babel}
\usepackage{mathptmx}

\usepackage{hyperref}
\hypersetup{colorlinks=true,linkcolor=black}
\hypersetup{bookmarksnumbered=true,pdfstartview=FitH,pdfpagemode=UseNone}
\hypersetup{pdftitle={A Reformulation of Classical Mechanics}}
\hypersetup{pdfauthor={Alejandro A. Torassa}}

\setlength{\unitlength}{0.84pt}
\setlength{\arraycolsep}{1.74pt}

\newcommand{\mM}{m}
\newcommand{\mW}{W}
\newcommand{\mK}{K}
\newcommand{\mU}{U}
\newcommand{\mE}{E}
\newcommand{\ra}{_a}
\newcommand{\rb}{_b}
\newcommand{\rs}{_s}
\newcommand{\ri}{_i}
\newcommand{\rS}{_S}
\newcommand{\yya}{30}
\newcommand{\xxa}{36}
\newcommand{\xxb}{42}
\newcommand{\xxc}{45}
\newcommand{\xxd}{48}
\newcommand{\rab}{_{ab}}
\newcommand{\ris}{_{is}}
\newcommand{\rcm}{_{cm}}
\newcommand{\bre}{\breve}
\newcommand{\ricm}{_{icm}}
\newcommand{\vR}{\mathbf{r}}
\newcommand{\vV}{\mathbf{v}}
\newcommand{\vA}{\mathbf{a}}
\newcommand{\vF}{\mathbf{F}}
\newcommand{\vP}{\mathbf{P}}
\newcommand{\vL}{\mathbf{L}}
\newcommand{\aV}{\mathbf{\omega}}
\newcommand{\aA}{\mathbf{\alpha}}
\newcommand{\rt}{\hspace{+0.03em}'}
\newcommand{\rj}{_{\hspace{-0.081em}j}}
\newcommand{\rij}{_{i\hspace{-0.081em}j}}
\newcommand{\med}{\raise.5ex\hbox{$\scriptstyle 1$}\kern-.15em/\kern-.12em\lower.45ex\hbox{$\scriptstyle 2$}\;}

\begin{document}

\begin{center}

{\LARGE A Reformulation of Classical Mechanics}

\bigskip \bigskip

{\large Alejandro A. Torassa}

\bigskip \bigskip

\small

Creative Commons Attribution 3.0 License

(2014) Buenos Aires, Argentina

atorassa@gmail.com

\bigskip \medskip

{\fontsize{9.27}{11.01}\selectfont\textbf{Abstract}}

\bigskip

\parbox{99.12mm}{\fontsize{9.27}{11.01}\selectfont This paper presents a reformulation of classical mechanics which is invariant under transformations between reference frames and which can be applied in any reference frame (rotating or non-rotating) (inertial or non-inertial) \hbox {without} the necessity of introducing fictitious forces.}

\end{center}

\normalsize

\vspace{-0.30em}

{\centering\subsection*{Introduction}}

\vspace{+1.20em}

\par The reformulation of classical mechanics presented in this paper is obtained starting from a general equation of motion. This paper considers that any observer S uses a reference frame S and a dynamic reference frame $\bre\mathrm S$. The general equation of motion is a transformation equation between the reference frame S and the dynamic reference frame $\bre\mathrm S$.
\medskip
\par The dynamic position $\bre\vR\ra$, the dynamic velocity $\bre\vV\ra$, and the dynamic acceleration $\bre\vA\ra$ of a particle A of mass $\mM\ra$ relative to the dynamic reference frame $\bre\mathrm S$ are given by:
\vspace{+0.81em}
\par \begin{tabular}{l}
$\hspace{+6.00em} \bre\vR\ra = \int \int \hspace{+0.12em} (\vF\ra/\mM\ra) \; dt \; dt$ \vspace{+1.20em} \\
$\hspace{+6.00em} \bre\vV\ra = \int \hspace{+0.12em} (\vF\ra/\mM\ra) \; dt$ \vspace{+1.20em} \\
$\hspace{+6.00em} \bre\vA\ra = (\vF\ra/\mM\ra)$
\end{tabular}
\vspace{+0.81em}
\par \noindent where $\vF\ra$ is the net force acting on particle A.
\medskip
\par The dynamic angular velocity $\bre\aV\rS$ and the dynamic angular acceleration $\bre\aA\rS$ of the reference \hbox {frame S} fixed to a particle S relative to the dynamic reference frame $\bre\mathrm S$ are given by:
\vspace{+0.81em}
\par \begin{tabular}{l}
$\hspace{+6.00em} \bre\aV\rS = \pm \hspace{+0.12em} \big| (\vF_{\scriptscriptstyle 1}/\mM\rs - \vF_{\scriptscriptstyle 0}/\mM\rs) \cdot (\vR_{\scriptscriptstyle 1} - \vR_{\scriptscriptstyle 0})/(\vR_{\scriptscriptstyle 1} - \vR_{\scriptscriptstyle 0})^2 \big|^{1/2}$ \vspace{+1.20em} \\
$\hspace{+6.00em} \bre\aA\rS = d(\bre\aV\rS)/dt$
\end{tabular}
\vspace{+0.90em}
\par \noindent where $\vF_{\scriptscriptstyle 0}$ and $\vF_{\scriptscriptstyle 1}$ are the net forces acting on the reference frame S in the points 0 and 1, $\vR_{\scriptscriptstyle 0}$ and $\vR_{\scriptscriptstyle 1}$ are the positions of the points 0 and 1 relative to the reference frame S, and $\mM\rs$ is the mass of \hbox {particle S} \hbox {(the point 0} is the origin of the reference frame S and the center of mass of particle S) (the point 0 belongs to the axis of dynamic rotation, and the segment 01 is perpendicular to the axis of dynamic rotation) (the vector $\bre\aV\rS$ is along the axis of dynamic rotation)

\newpage

{\centering\subsection*{General Equation of Motion}}

\vspace{+1.20em}

\par The general equation of motion for two particles A and B relative to an observer S is:
\begin{eqnarray*}
\mM\ra \, \mM\rb \big[ \hspace{+0.045em} \vR\ra - \vR\rb \big] - \mM\ra \, \mM\rb \big[ \hspace{+0.045em} \bre\vR\ra - \bre\vR\rb \big] = 0
\end{eqnarray*}
\noindent where $\mM\ra$ and $\mM\rb$ are the masses of particles A and B, $\vR\ra$ and $\vR\rb$ are the positions of particles A and B, $\bre\vR\ra$ and $\bre\vR\rb$ are the dynamic positions of particles A and B.
\medskip
\par Differentiating the above equation with respect to time, we obtain:
\begin{eqnarray*}
\mM\ra \, \mM\rb \big[ (\vV\ra - \vV\rb) + \bre\aV\rS \times (\vR\ra - \vR\rb) \big] - \mM\ra \, \mM\rb \big[ \hspace{+0.036em} \bre\vV\ra - \bre\vV\rb \big] = 0
\end{eqnarray*}
\par Differentiating again with respect to time, we obtain:
\begin{eqnarray*}
\mM\ra \, \mM\rb \big[ (\vA\ra - \vA\rb) + 2 \hspace{+0.06em} \bre\aV\rS \times (\vV\ra - \vV\rb) + \bre\aV\rS \times (\bre\aV\rS \times (\vR\ra - \vR\rb)) + \bre\aA\rS \times (\vR\ra - \vR\rb) \big] - \mM\ra \, \mM\rb \big[ \hspace{+0.045em} \bre\vA\ra - \bre\vA\rb \big] = 0
\end{eqnarray*}

\vspace{+1.20em}

{\centering\subsection*{Reference Frames}}

\vspace{+1.50em}

\par Applying the above equation to two particles A and S, we have:
\begin{eqnarray*}
\mM\ra \, \mM\rs \big[ (\vA\ra - \vA\rs) + 2 \hspace{+0.06em} \bre\aV\rS \times (\vV\ra - \vV\rs) + \bre\aV\rS \times (\bre\aV\rS \times (\vR\ra - \vR\rs)) + \bre\aA\rS \times (\vR\ra - \vR\rs) \big] - \mM\ra \, \mM\rs \big[ \hspace{+0.045em} \bre\vA\ra - \bre\vA\rs \big] = 0
\end{eqnarray*}
\par If we divide by $\mM\rs$ and if the reference frame S fixed to particle S $(\vR\rs=0, \vV\rs=0$ and $\vA\rs=0)$ is rotating relative to the dynamic reference frame $\bre\mathrm S$ $(\bre\aV\rS \neq 0)$ then we obtain:
\begin{eqnarray*}
\mM\ra \big[ \vA\ra + 2 \hspace{+0.06em} \bre\aV\rS \times \vV\ra + \bre\aV\rS \times (\bre\aV\rS \times \vR\ra) + \bre\aA\rS \times \vR\ra \big] - \mM\ra \big[ \hspace{+0.045em} \bre\vA\ra - \bre\vA\rs \big] = 0
\end{eqnarray*}
\par If the reference frame S is non-rotating relative to the dynamic reference frame $\bre\mathrm S$ $(\bre\aV\rS=0)$ then \hbox {we obtain}:
\begin{eqnarray*}
\mM\ra \, \vA\ra - \mM\ra \big[ \hspace{+0.045em} \bre\vA\ra - \bre\vA\rs \big] = 0
\end{eqnarray*}
\par If the reference frame S is inertial relative to the dynamic reference frame $\bre\mathrm S$ $(\bre\aV\rS=0$ and $\bre\vA\rs=0)$ then we obtain:
\begin{eqnarray*}
\mM\ra \, \vA\ra - \mM\ra \, \bre\vA\ra = 0
\end{eqnarray*}
\noindent that is:
\begin{eqnarray*}
\mM\ra \, \vA\ra - \vF\ra = 0
\end{eqnarray*}
\noindent or else:
\begin{eqnarray*}
\vF\ra = \mM\ra \, \vA\ra
\end{eqnarray*}
\par \vspace{+0.45em} \noindent where this equation is Newton's second law.

\newpage

{\centering\subsection*{Equation of Motion}}

\vspace{+1.20em}

\par From the general equation of motion it follows that the acceleration $\vA\ra$ of a particle A of mass $\mM\ra$ relative to a reference frame S fixed to a particle S of mass $\mM\rs$ is given by:
\begin{eqnarray*}
\vA\ra = \frac{\vF\ra}{\mM\ra} - 2 \hspace{+0.06em} \bre\aV\rS \times \vV\ra - \bre\aV\rS \times (\bre\aV\rS \times \vR\ra) - \bre\aA\rS \times \vR\ra - \frac{\vF\rs}{\mM\rs}
\end{eqnarray*}
\noindent where $\vF\ra$ is the net force acting on particle A, $\bre\aV\rS$ is the dynamic angular velocity of the reference frame S, $\vV\ra$ is the velocity of particle A, $\vR\ra$ is the position of particle A, $\bre\aA\rS$ is the dynamic angular acceleration of the reference frame S, and $\vF\rs$ is the net force acting on particle S.
\medskip
\par In contradiction with Newton's first and second laws, from the above equation it follows that particle A can have a non-zero acceleration even if there is no force acting on particle A, and also that particle A can have zero acceleration (state of rest or of uniform linear motion) even if there is \hbox {an unbalanced} force acting on particle A.
\medskip
\par Therefore, in order to apply Newton's first and second laws in a non-inertial reference frame it is necessary to introduce fictitious forces.
\medskip
\par However, this paper considers that Newton's first and second laws are false. Therefore, in this paper there is no need to introduce fictitious forces.

\vspace{+1.50em}

{\centering\subsection*{System of Equations}}

\vspace{+1.20em}

\par If we consider a system of N particles (of total mass M and center of mass CM) and a single \hbox {particle J} relative to a reference frame S (fixed to a particle S) then from the general equation of motion the following equations are obtained:

\vspace{+2.10em}

\begin{center}
\begin{tabular}{ccccccc}
& & {\framebox(\xxa,\yya){[1]}} & {\makebox(\xxd,\yya){$\rightarrow \int \hspace{+0.03em} d\bre\vR\rij \rightarrow$}} & {\framebox(\xxa,\yya){[6]}} & {\makebox(\xxb,\yya){$\rightarrow \frac{1}{2} \; dt \rightarrow$}} & {\framebox(\xxa,\yya){[8]}} \\
& & {\makebox(\xxa,\yya){$\downarrow$ $dt$ $\downarrow$}} & & & & {\makebox(\xxa,\yya){$\downarrow$ $dt$ $\downarrow$}} \\
{\framebox(\xxa,\yya){[4]}} & {\makebox(\xxc,\yya){$\leftarrow \times \: \bre\vR\rij \leftarrow$}} & {\framebox(\xxa,\yya){[2]}} & {\makebox(\xxd,\yya){$\rightarrow \int \hspace{+0.03em} d\bre\vV\rij \rightarrow$}} & {\framebox(\xxa,\yya){[7]}} & & {\framebox(\xxa,\yya){[9]}} \\
{\makebox(\xxa,\yya){$\downarrow$ $dt$ $\downarrow$}} & & {\makebox(\xxa,\yya){$\downarrow$ $dt$ $\downarrow$}} & {\makebox(\xxd,\yya){$\nearrow \hspace{-0.001em} \int \hspace{+0.03em} d\bre\vR\rij \hspace{+0.001em} \nearrow$}} \\
{\framebox(\xxa,\yya){[5]}} & {\makebox(\xxc,\yya){$\leftarrow \times \: \bre\vR\rij \leftarrow$}} & {\framebox(\xxa,\yya){[3]}}
\end{tabular}
\end{center}

\vspace{+1.20em}

\par The equations [1, 2, 3, 4 and 5] are vector equations, and the equations [6, 7, 8 and 9] are scalar equations. The principles of conservation are obtained from the equations [2, 4, 7 and 9]

\newpage

\par {\fontsize{11}{11}\selectfont\textbf{Equation [1]}}
\bigskip
\par \hspace{+1.20em} $\sum_{i=1}^N \mM\ri \hspace{+0.03em} \big[ (\vR\rij) - (\bre\vR\rij) \big] = 0$
\bigskip
\par {\fontsize{11}{11}\selectfont\textbf{Equation [2]}}
\bigskip
\par \hspace{+1.20em} $\sum_{i=1}^N \mM\ri \hspace{+0.03em} \big[ (\vV\rij + \bre\aV\rS \times \vR\rij) - (\bre\vV\rij) \big] = 0$
\bigskip
\par {\fontsize{11}{11}\selectfont\textbf{Equation [3]}}
\bigskip
\par \hspace{+1.20em} $\sum_{i=1}^N \mM\ri \hspace{+0.03em} \big[ (\vA\rij + 2 \hspace{+0.06em} \bre\aV\rS \times \vV\rij + \bre\aV\rS \times (\bre\aV\rS \times \vR\rij) + \bre\aA\rS \times \vR\rij) - (\bre\vA\rij) \big] = 0$
\bigskip
\par {\fontsize{11}{11}\selectfont\textbf{Equation [4]}}
\bigskip
\par \hspace{+1.20em} $\sum_{i=1}^N \mM\ri \hspace{+0.03em} \big[ (\vV\rij + \bre\aV\rS \times \vR\rij) \times \vR\rij - (\bre\vV\rij) \times \bre\vR\rij \big] = 0$
\bigskip
\par {\fontsize{11}{11}\selectfont\textbf{Equation [5]}}
\bigskip
\par \hspace{+1.20em} $\sum_{i=1}^N \mM\ri \hspace{+0.03em} \big[ (\vA\rij + 2 \hspace{+0.06em} \bre\aV\rS \times \vV\rij + \bre\aV\rS \times (\bre\aV\rS \times \vR\rij) + \bre\aA\rS \times \vR\rij) \times \vR\rij - (\bre\vA\rij) \times \bre\vR\rij \big] = 0$
\bigskip
\par {\fontsize{11}{11}\selectfont\textbf{Equation [6]}}
\bigskip
\par \hspace{+1.20em} $\sum_{i=1}^N \hspace{-0.09em} \med \mM\ri \hspace{+0.03em} \big[ (\vR\rij)^2 - (\bre\vR\rij)^2 \big] = 0$
\bigskip
\par {\fontsize{11}{11}\selectfont\textbf{Equation [7]}}
\bigskip
\par \hspace{+1.20em} $\sum_{i=1}^N \hspace{-0.09em} \med \mM\ri \hspace{+0.03em} \big[ (\vV\rij + \bre\aV\rS \times \vR\rij)^2 - (\bre\vV\rij)^2 \big] = 0$
\bigskip
\par {\fontsize{11}{11}\selectfont\textbf{Equation [8]}}
\bigskip
\par \hspace{+1.20em} $\sum_{i=1}^N \hspace{-0.09em} \med \mM\ri \hspace{+0.03em} \big[ (\vR\rij \cdot \vV\rij) - (\bre\vR\rij \cdot \bre\vV\rij) \big] = 0$
\bigskip
\par {\fontsize{11}{11}\selectfont\textbf{Equation [9]}}
\bigskip
\par \hspace{+1.20em} $\sum_{i=1}^N \hspace{-0.09em} \med \mM\ri \hspace{+0.03em} \big[ (\vV\rij \cdot \vV\rij + \vA\rij \cdot \vR\rij) - (\bre\vV\rij \cdot \bre\vV\rij + \bre\vA\rij \cdot \bre\vR\rij) \big] = 0$

\vspace{+1.20em}

\par \rule{131.40mm}{0.10mm}

\vspace{+1.50em}

\par {\fontsize{10.00}{10.00}\selectfont The \textit{i}-th particle (of mass $\mM\ri$) relative to particle J, to particle S, and to the center of mass CM}

\vspace{+1.50em}

{\fontsize{10.26}{10.26}\selectfont\begin{tabular}{llllll}
\hspace{-0.81em} $\vR\rij = \vR\ri - \vR\rj$ & \hspace{-0.60em} $\bre\vR\rij = \bre\vR\ri - \bre\vR\rj$ & $\vR\ris = \vR\ri - \vR\rs$ & \hspace{-0.60em} $\bre\vR\ris = \bre\vR\ri - \bre\vR\rs$ & $\vR\ricm = \vR\ri - \vR\rcm$ & \hspace{-0.60em} $\bre\vR\ricm = \bre\vR\ri - \bre\vR\rcm$ \vspace{+1.20em} \\
\hspace{-0.81em} $\vV\rij = \vV\ri - \vV\rj$ & \hspace{-0.60em} $\bre\vV\rij = \bre\vV\ri - \bre\vV\rj$ & $\vV\ris = \vV\ri - \vV\rs$ & \hspace{-0.60em} $\bre\vV\ris = \bre\vV\ri - \bre\vV\rs$ & $\vV\ricm = \vV\ri - \vV\rcm$ & \hspace{-0.60em} $\bre\vV\ricm = \bre\vV\ri - \bre\vV\rcm$ \vspace{+1.20em} \\
\hspace{-0.81em} $\vA\rij = \vA\ri - \vA\rj$ & \hspace{-0.60em} $\bre\vA\rij = \bre\vA\ri - \bre\vA\rj$ & $\vA\ris = \vA\ri - \vA\rs$ & \hspace{-0.60em} $\bre\vA\ris = \bre\vA\ri - \bre\vA\rs$ & $\vA\ricm = \vA\ri - \vA\rcm$ & \hspace{-0.60em} $\bre\vA\ricm = \bre\vA\ri - \bre\vA\rcm$
\end{tabular}}

\newpage

\par {\fontsize{11}{11}\selectfont\textbf{{\large $\Delta$} Equation [2]}}
\bigskip
\par \hspace{+1.20em} $\sum_{i=1}^N \Delta \; \mM\ri \hspace{+0.03em} \big[ (\vV\rij + \bre\aV\rS \times \vR\rij) - (\bre\vV\rij) \big] = 0$
\bigskip
\par Now, replacing particle J by particle S and distributing $(\Delta \; \mM\ri)$ we have:
\bigskip
\par \hspace{+1.20em} $\sum_{i=1}^N \hspace{-0.12em} \big[ \hspace{+0.045em} \Delta \; \mM\ri \, (\vV\ris + \bre\aV\rS \times \vR\ris) - \Delta \; \mM\ri \, (\bre\vV\ris) \big] = 0$
\bigskip
\par If the reference frame S $(\vV\rs=0)$ is inertial $(\bre\aV\rS=0$ and $\bre\vV\rs=\mathrm{constant})$ then:
\bigskip
\par \hspace{+1.20em} $\sum_{i=1}^N \hspace{-0.12em} \big[ \hspace{+0.045em} \Delta \; \mM\ri \, \vV\ri - \Delta \; \mM\ri \, \bre\vV\ri \big] = 0$
\bigskip
\par Since \hspace{-0.36em} $\big[ \hspace{+0.045em} \Delta \; \mM\ri \, \bre\vV\ri = \int_{\scriptscriptstyle 1}^{\scriptscriptstyle 2} \mM\ri \, \bre\vA\ri \: dt = \int_{\scriptscriptstyle 1}^{\scriptscriptstyle 2} \hspace{+0.045em} \vF\ri \: dt \big]$ \hspace{-0.36em} we obtain:
\bigskip
\par \hspace{+1.20em} $\sum_{i=1}^N \hspace{-0.12em} \big[ \hspace{+0.045em} \Delta \; \mM\ri \, \vV\ri - \int_{\scriptscriptstyle 1}^{\scriptscriptstyle 2} \hspace{+0.045em} \vF\ri \: dt \big] = 0$
\bigskip
\par If the system of particles is isolated and if the internal forces obey Newton's third law in its weak form $( \hspace{+0.09em} \sum_{i=1}^N \vF\ri = 0 \hspace{+0.03em} )$ then:
\bigskip
\par \hspace{+1.20em} $\sum_{i=1}^N \mM\ri \, \vV\ri = \vP = \mathrm{constant}$
\bigskip
\par Therefore, if the system of particles is isolated and if the internal forces obey Newton's third law in its weak form then the total linear momentum $\vP$ of the system of particles remains constant relative to an inertial reference frame.
\bigskip
\par {\fontsize{11}{11}\selectfont\textbf{{\large $\Delta$} Equation [4]}}
\bigskip
\par \hspace{+1.20em} $\sum_{i=1}^N \Delta \; \mM\ri \hspace{+0.03em} \big[ (\vV\rij + \bre\aV\rS \times \vR\rij) \times \vR\rij - (\bre\vV\rij) \times \bre\vR\rij \big] = 0$
\bigskip
\par Now, replacing particle J by the center of mass CM and distributing $(\Delta \; \mM\ri)$ we have:
\bigskip
\par \hspace{+1.20em} $\sum_{i=1}^N \hspace{-0.12em} \big[ \hspace{+0.045em} \Delta \; \mM\ri \, (\vV\ricm + \bre\aV\rS \times \vR\ricm) \times \vR\ricm - \Delta \; \mM\ri \, (\bre\vV\ricm) \times \bre\vR\ricm \big] = 0$
\bigskip
\par Since \hspace{-0.36em} $\big[ \hspace{+0.045em} \Delta \; \mM\ri \, (\bre\vV\ricm) \times \bre\vR\ricm = \Delta \; \mM\ri \, \bre\vV\ricm \times \bre\vR\ricm = \int_{\scriptscriptstyle 1}^{\scriptscriptstyle 2} (\mM\ri \, \bre\vA\ricm \times \bre\vR\ricm) \, dt = \int_{\scriptscriptstyle 1}^{\scriptscriptstyle 2} (\mM\ri \, \bre\vA\ricm \times \vR\ricm) \, dt \big]$ \hspace{-0.36em} we obtain:
\bigskip \vspace{-1.20em}
\par \hspace{+1.20em} $\sum_{i=1}^N \hspace{-0.12em} \big[ \hspace{+0.045em} \Delta \; \mM\ri \, (\vV\ricm + \bre\aV\rS \times \vR\ricm) \times \vR\ricm - \int_{\scriptscriptstyle 1}^{\scriptscriptstyle 2} (\mM\ri \, \bre\vA\ricm \times \vR\ricm) \, dt \big] = 0$
\bigskip
\par Given that \hspace{-0.36em} $\big[ \sum_{i=1}^N \int_{\scriptscriptstyle 1}^{\scriptscriptstyle 2} (\mM\ri \, \bre\vA\ricm \times \vR\ricm) \, dt = \sum_{i=1}^N \int_{\scriptscriptstyle 1}^{\scriptscriptstyle 2} (\mM\ri \, \bre\vA\ri \times \vR\ricm) \, dt = \sum_{i=1}^N \int_{\scriptscriptstyle 1}^{\scriptscriptstyle 2} (\vF\ri \times \vR\ricm) \, dt \big]$ \hspace{-0.36em} we get:
\bigskip
\par \hspace{+1.20em} $\sum_{i=1}^N \hspace{-0.12em} \big[ \hspace{+0.045em} \Delta \; \mM\ri \, (\vV\ricm + \bre\aV\rS \times \vR\ricm) \times \vR\ricm - \int_{\scriptscriptstyle 1}^{\scriptscriptstyle 2} (\vF\ri \times \vR\ricm) \, dt \big] = 0$
\bigskip
\par If the system of particles is isolated and if the internal forces obey Newton's third law in its strong form $( \hspace{+0.09em} \sum_{i=1}^N \vF\ri \times \vR\ricm = 0 \hspace{+0.03em} )$ then:
\bigskip
\par \hspace{+1.20em} $\sum_{i=1}^N \mM\ri \, (\vV\ricm + \bre\aV\rS \times \vR\ricm) \times \vR\ricm = \vL = \mathrm{constant}$
\bigskip
\par Therefore, if the system of particles is isolated and if the internal forces obey Newton's third law in its strong form then the total angular momentum $\vL$ of the system of particles remains constant.

\newpage

\par {\fontsize{11}{11}\selectfont\textbf{{\large $\Delta$} Equation [7]}}
\bigskip
\par \hspace{+1.20em} $\sum_{i=1}^N \Delta \, \med \mM\ri \hspace{+0.03em} \big[ (\vV\rij + \bre\aV\rS \times \vR\rij)^2 - (\bre\vV\rij)^2 \big] = 0$
\bigskip
\par Now, replacing particle J by the center of mass CM and distributing $(\Delta \, \med \mM\ri)$ we have:
\bigskip
\par \hspace{+1.20em} $\sum_{i=1}^N \hspace{-0.12em} \big[ \hspace{+0.045em} \Delta \, \med \mM\ri \, (\vV\ricm + \bre\aV\rS \times \vR\ricm)^2 - \Delta \, \med \mM\ri \, (\bre\vV\ricm)^2 \big] = 0$
\bigskip
\par \hypertarget{eq_a}{} Since \hspace{-0.36em} $\big[ \hspace{+0.045em} \Delta \, \med \mM\ri \, (\bre\vV\ricm)^2 = \Delta \, \med \mM\ri \, \bre\vV\ricm \cdot \bre\vV\ricm = \int_{\scriptscriptstyle 1}^{\scriptscriptstyle 2} \mM\ri \, \bre\vA\ricm \cdot d\bre\vR\ricm = \int_{\scriptscriptstyle 1}^{\scriptscriptstyle 2} \mM\ri \, \bre\vA\ricm \cdot d\vR\ricm \big]$ \hspace{-0.51em} $\big[$ \hspace{-0.48em} {\fontsize{9.75}{9.75}\selectfont Eq. \hspace{-0.42em} A} \hspace{-0.51em} $\big]$ \hspace{-0.36em} we obtain:
\bigskip \vspace{-1.20em}
\par \hspace{+1.20em} $\sum_{i=1}^N \hspace{-0.12em} \big[ \hspace{+0.045em} \Delta \, \med \mM\ri \, (\vV\ricm + \bre\aV\rS \times \vR\ricm)^2 - \int_{\scriptscriptstyle 1}^{\scriptscriptstyle 2} \mM\ri \, \bre\vA\ricm \cdot d\vR\ricm \big] = 0$
\bigskip
\par \hypertarget{eq_b}{} Given that \hspace{-0.36em} $\big[ \sum_{i=1}^N \int_{\scriptscriptstyle 1}^{\scriptscriptstyle 2} \mM\ri \, \bre\vA\ricm \cdot d\vR\ricm = \sum_{i=1}^N \int_{\scriptscriptstyle 1}^{\scriptscriptstyle 2} \mM\ri \, \bre\vA\ri \cdot d\vR\ricm = \sum_{i=1}^N \int_{\scriptscriptstyle 1}^{\scriptscriptstyle 2} \hspace{+0.045em} \vF\ri \cdot d\vR\ricm \big]$ \hspace{-0.51em} $\big[$ \hspace{-0.48em} {\fontsize{9.75}{9.75}\selectfont Eq. \hspace{-0.42em} B} \hspace{-0.51em} $\big]$ \hspace{-0.36em} we get:
\bigskip
\par \hspace{+1.20em} $\sum_{i=1}^N \hspace{-0.12em} \big[ \hspace{+0.045em} \Delta \, \med \mM\ri \, (\vV\ricm + \bre\aV\rS \times \vR\ricm)^2 - \int_{\scriptscriptstyle 1}^{\scriptscriptstyle 2} \hspace{+0.045em} \vF\ri \cdot d\vR\ricm \big] = 0$
\bigskip
\par Therefore, we can consider that the total work $\mW$ done by the forces acting on the system of particles, the total kinetic energy $\mK$ of the system of particles and the total potential energy $\mU$ of the system of particles are as follows:
\bigskip
\par \hspace{+1.20em} $\mW = \sum_{i=1}^N \int_{\scriptscriptstyle 1}^{\scriptscriptstyle 2} \hspace{+0.045em} \vF\ri \cdot d\vR\ricm$
\bigskip
\par \hspace{+1.20em} $\Delta \, \mK = \sum_{i=1}^N \Delta \, \med \mM\ri \, (\vV\ricm + \bre\aV\rS \times \vR\ricm)^2$
\bigskip
\par \hspace{+1.20em} $\Delta \, \mU = \sum_{i=1}^N - \int_{\scriptscriptstyle 1}^{\scriptscriptstyle 2} \hspace{+0.045em} \vF\ri \cdot d\vR\ricm$
\bigskip
\par If the system of particles is isolated and if the internal forces obey Newton's third law in its weak form $( \hspace{+0.09em} \sum_{i=1}^N \vF\ri = 0 \hspace{+0.03em} )$ then:
\bigskip
\par \hspace{+1.20em} $\mW = \sum_{i=1}^N \int_{\scriptscriptstyle 1}^{\scriptscriptstyle 2} \hspace{+0.045em} \vF\ri \cdot d\vR\ri$
\bigskip
\par \hspace{+1.20em} $\Delta \, \mU = \sum_{i=1}^N - \int_{\scriptscriptstyle 1}^{\scriptscriptstyle 2} \hspace{+0.045em} \vF\ri \cdot d\vR\ri$
\bigskip
\par The total work $\mW$ done by the forces acting on the system of particles is equal to the change in the total kinetic energy $\mK$ of the system of particles.
\bigskip
\par \hspace{+1.20em} $\mW = \Delta \, \mK$
\bigskip
\par The total work $\mW$ done by the conservative forces acting on the system of particles is equal and opposite in sign to the change in the total potential energy $\mU$ of the system of particles.
\bigskip
\par \hspace{+1.20em} $\mW = - \, \Delta \, \mU$
\bigskip
\par Therefore, if the system of particles is exclusively subject to conservative forces then the total mechanical energy $\mE$ of the system of particles remains constant.
\bigskip
\par \hspace{+1.20em} $\mE = \mK + \mU = \mathrm{constant}$

\newpage

\par {\fontsize{11}{11}\selectfont\textbf{{\large $\Delta$} Equation [9]}}
\bigskip
\par \hspace{+1.20em} $\sum_{i=1}^N \Delta \, \med \mM\ri \hspace{+0.03em} \big[ (\vV\rij \cdot \vV\rij + \vA\rij \cdot \vR\rij) - (\bre\vV\rij \cdot \bre\vV\rij + \bre\vA\rij \cdot \bre\vR\rij) \big] = 0$
\bigskip
\par Now, replacing particle J by the center of mass CM and distributing $(\Delta \, \med \mM\ri)$ we have:
\bigskip
\par \hspace{+1.20em} $\sum_{i=1}^N \hspace{-0.12em} \big[ \hspace{+0.045em} \Delta \, \med \mM\ri \, (\vV\ricm \cdot \vV\ricm + \vA\ricm \cdot \vR\ricm) - (\Delta \, \med \mM\ri \, \bre\vV\ricm \cdot \bre\vV\ricm + \Delta \, \med \mM\ri \, \bre\vA\ricm \cdot \bre\vR\ricm) \big] = 0$
\bigskip
\par Since \hspace{-0.36em} $\big[$ \hspace{-0.48em} \hyperlink{eq_a}{\fontsize{9.75}{9.75}\selectfont Eq. \hspace{-0.42em} A} \hspace{-0.51em} $\big]$ \hspace{-0.36em} and \hspace{-0.36em} $\big[ \hspace{+0.045em} \Delta \, \med \mM\ri \, \bre\vA\ricm \cdot \bre\vR\ricm = \Delta \, \med \mM\ri \, \bre\vA\ricm \cdot \vR\ricm \big]$ \hspace{-0.36em} we obtain:
\bigskip
\par \hspace{+1.20em} $\sum_{i=1}^N \hspace{-0.12em} \big[ \hspace{+0.045em} \Delta \, \med \mM\ri \, (\vV\ricm \cdot \vV\ricm + \vA\ricm \cdot \vR\ricm) - (\int_{\scriptscriptstyle 1}^{\scriptscriptstyle 2} \mM\ri \, \bre\vA\ricm \cdot d\vR\ricm + \Delta \, \med \mM\ri \, \bre\vA\ricm \cdot \vR\ricm) \big] = 0$
\bigskip
\par Given that \hspace{-0.36em} $\big[$ \hspace{-0.48em} \hyperlink{eq_b}{\fontsize{9.75}{9.75}\selectfont Eq. \hspace{-0.42em} B} \hspace{-0.51em} $\big]$ \hspace{-0.36em} and \hspace{-0.36em} $\big[ \sum_{i=1}^N \Delta \, \med \mM\ri \, \bre\vA\ricm \cdot \vR\ricm = \sum_{i=1}^N \Delta \, \med \mM\ri \, \bre\vA\ri \cdot \vR\ricm = \sum_{i=1}^N \Delta \, \med \vF\ri \cdot \vR\ricm \big]$ \hspace{-0.36em} we get:
\bigskip
\par \hspace{+1.20em} $\sum_{i=1}^N \hspace{-0.12em} \big[ \hspace{+0.045em} \Delta \, \med \mM\ri \, (\vV\ricm \cdot \vV\ricm + \vA\ricm \cdot \vR\ricm) - (\int_{\scriptscriptstyle 1}^{\scriptscriptstyle 2} \hspace{+0.045em} \vF\ri \cdot d\vR\ricm + \Delta \, \med \vF\ri \cdot \vR\ricm) \big] = 0$
\bigskip
\par Therefore, we can consider that the total work $\mW\rt$ done by the forces acting on the system of particles, the total kinetic energy $\mK\rt$ of the system of particles and the total potential energy $\mU\rt$ of the system of particles are as follows:
\bigskip
\par \hspace{+1.20em} $\mW\rt = \sum_{i=1}^N \hspace{+0.036em} (\int_{\scriptscriptstyle 1}^{\scriptscriptstyle 2} \hspace{+0.045em} \vF\ri \cdot d\vR\ricm + \Delta \, \med \vF\ri \cdot \vR\ricm)$
\bigskip
\par \hspace{+1.20em} $\Delta \, \mK\rt = \sum_{i=1}^N \Delta \, \med \mM\ri \, (\vV\ricm \cdot \vV\ricm + \vA\ricm \cdot \vR\ricm)$
\bigskip
\par \hspace{+1.20em} $\Delta \, \mU\rt = \sum_{i=1}^N - \, (\int_{\scriptscriptstyle 1}^{\scriptscriptstyle 2} \hspace{+0.045em} \vF\ri \cdot d\vR\ricm + \Delta \, \med \vF\ri \cdot \vR\ricm)$
\bigskip
\par If the system of particles is isolated and if the internal forces obey Newton's third law in its weak form $( \hspace{+0.09em} \sum_{i=1}^N \vF\ri = 0 \hspace{+0.03em} )$ then:
\bigskip
\par \hspace{+1.20em} $\mW\rt = \sum_{i=1}^N \hspace{+0.036em} (\int_{\scriptscriptstyle 1}^{\scriptscriptstyle 2} \hspace{+0.045em} \vF\ri \cdot d\vR\ri + \Delta \, \med \vF\ri \cdot \vR\ri)$
\bigskip
\par \hspace{+1.20em} $\Delta \, \mU\rt = \sum_{i=1}^N - \, (\int_{\scriptscriptstyle 1}^{\scriptscriptstyle 2} \hspace{+0.045em} \vF\ri \cdot d\vR\ri + \Delta \, \med \vF\ri \cdot \vR\ri)$
\bigskip
\par The total work $\mW\rt$ done by the forces acting on the system of particles is equal to the change in the total kinetic energy $\mK\rt$ of the system of particles.
\bigskip
\par \hspace{+1.20em} $\mW\rt = \Delta \, \mK\rt$
\bigskip
\par The total work $\mW\rt$ done by the conservative forces acting on the system of particles is equal and opposite in sign to the change in the total potential energy $\mU\rt$ of the system of particles.
\bigskip
\par \hspace{+1.20em} $\mW\rt = - \, \Delta \, \mU\rt$
\bigskip
\par Therefore, if the system of particles is exclusively subject to conservative forces then the total mechanical energy $\mE\rt$ of the system of particles remains constant.
\bigskip
\par \hspace{+1.20em} $\mE\rt = \mK\rt + \mU\rt = \mathrm{constant}$

\newpage

{\centering\subsection*{General Observations}}

\vspace{+1.20em}

\par The magnitudes $\bre\vR$, $\bre\vV$, $\bre\vA$, $\bre\aV$ and $\bre\aA$ are invariant under transformations between reference frames.
\bigskip
\par In any reference frame $\vR\rij = \bre\vR\rij$. Therefore, $\vR\rij$ is invariant under transformations between reference frames.
\bigskip
\par In any non-rotating reference frame $\vV\rij = \bre\vV\rij$ and $\vA\rij = \bre\vA\rij$. Therefore, $\vV\rij$ and $\vA\rij$ are invariant under transformations between non-rotating reference frames.
\bigskip
\par In any inertial reference frame $\vA\ri = \bre\vA\ri$. Therefore, $\vA\ri$ is invariant under transformations between inertial reference frames. Any inertial reference frame is a non-rotating reference frame.
\bigskip
\par In the universal reference frame $\vR\ri = \bre\vR\ri$, $\vV\ri = \bre\vV\ri$ and $\vA\ri = \bre\vA\ri$. Therefore, the universal reference frame is an inertial reference frame.
\bigskip
\par The total angular momentum $\vL$ of a system of particles is invariant under transformations between reference frames.
\bigskip
\par The total kinetic energy $\mK$ and the total potential energy $\mU$ of a system of particles are invariant under transformations between reference frames. Therefore, the total mechanical energy $\mE$ of a system of particles is invariant under transformations between reference frames.
\bigskip
\par The total kinetic energy $\mK\rt$ and the total potential energy $\mU\rt$ of a system of particles are invariant under transformations between reference frames. Therefore, the total mechanical energy $\mE\rt$ of a system of particles is invariant under transformations between reference frames.
\bigskip
\par The total mechanical energy $\mE$ of a system of particles is equal to the total mechanical energy $\mE\rt$ of the system of particles $(\hspace{+0.01em}\mE=\mE\rt\hspace{+0.03em})$

\vspace{+1.50em}

{\centering\subsection*{Bibliography}}

\vspace{+1.20em}

\par \textbf{A. Einstein}, Relativity: The Special and General Theory.
\bigskip
\par \textbf{E. Mach}, The Science of Mechanics.
\bigskip
\par \textbf{R. Resnick and D. Halliday}, Physics.
\bigskip
\par \textbf{J. Kane and M. Sternheim}, Physics.
\bigskip
\par \textbf{H. Goldstein}, Classical Mechanics.
\bigskip
\par \textbf{L. Landau and E. Lifshitz}, Mechanics.

\newpage

{\centering\subsection*{Appendix}}

\vspace{+1.20em}

\par \hspace{+0.81em} {\fontsize{10.50}{10.50}\selectfont\textbf{Definitions and Relations}}

\vspace{+1.02em}

\begin{tabular}{lll}
\hspace{+0.30em} $\vR\ri = \vR\ri$ & \hspace{+8.70em} & $\vR\rij = \vR\ri - \vR\rj$ \vspace{+0.90em} \\
\hspace{+0.30em} $\vV\ri = d\vR\ri/dt$ & \hspace{+8.70em} & $\vV\rij = d\vR\rij/dt$ \vspace{+0.90em} \\
\hspace{+0.30em} $\vA\ri = d\vV\ri/dt$ & \hspace{+8.70em} & $\vA\rij = d\vV\rij/dt$ \vspace{+0.90em} \\
\hspace{+0.30em} $\vV\ri = \int \vA\ri \; dt$ & \hspace{+8.70em} & $\vV\rij = \int \vA\rij \; dt$ \vspace{+0.90em} \\
\hspace{+0.30em} $\Delta \, \vV\ri = \int_{\scriptscriptstyle 1}^{\scriptscriptstyle 2} \vA\ri \; dt$ & \hspace{+8.70em} & $\Delta \, \vV\rij = \int_{\scriptscriptstyle 1}^{\scriptscriptstyle 2} \vA\rij \; dt$ \vspace{+0.90em} \\
\hspace{+0.30em} $\med \vV\ri \cdot \vV\ri = \int \vA\ri \cdot d\vR\ri$ & \hspace{+8.70em} & $\med \vV\rij \cdot \vV\rij = \int \vA\rij \cdot d\vR\rij$ \vspace{+0.90em} \\
\hspace{+0.30em} $\Delta \, \med \vV\ri \cdot \vV\ri = \int_{\scriptscriptstyle 1}^{\scriptscriptstyle 2} \vA\ri \cdot d\vR\ri$ & \hspace{+8.70em} & $\Delta \, \med \vV\rij \cdot \vV\rij = \int_{\scriptscriptstyle 1}^{\scriptscriptstyle 2} \vA\rij \cdot d\vR\rij$ \vspace{+0.90em} \\
\hspace{+0.30em} $\vV\ri \times \vR\ri = \int (\vA\ri \times \vR\ri) \; dt$ & \hspace{+8.70em} & $\vV\rij \times \vR\rij = \int (\vA\rij \times \vR\rij) \; dt$ \vspace{+0.90em} \\
\hspace{+0.30em} $\Delta \, \vV\ri \times \vR\ri = \int_{\scriptscriptstyle 1}^{\scriptscriptstyle 2} (\vA\ri \times \vR\ri) \; dt$ & \hspace{+8.70em} & $\Delta \, \vV\rij \times \vR\rij = \int_{\scriptscriptstyle 1}^{\scriptscriptstyle 2} (\vA\rij \times \vR\rij) \; dt$
\end{tabular}

\vspace{+1.20em}

\par \hspace{+0.81em} {\fontsize{10.50}{10.50}\selectfont\textbf{Invariant Equations}}

\vspace{+1.20em}

\par \hspace{+0.90em} $\vR\rij \cdot \vR\rij = \acute\vR\rij \cdot \acute\vR\rij$
\bigskip
\par \hspace{+0.90em} $\vR\rij \cdot \vV\rij = \acute\vR\rij \cdot \acute\vV\rij$
\bigskip
\par \hspace{+0.90em} $\vV\rij \cdot \vV\rij + \vA\rij \cdot \vR\rij = \acute\vV\rij \cdot \acute\vV\rij + \acute\vA\rij \cdot \acute\vR\rij$
\bigskip
\par \hspace{+0.90em} $\vR\rij = \acute\vR\rij$
\bigskip
\par \hspace{+0.90em} $\vV\rij + \bre\aV_{S \vphantom{\acute S}} \times \vR\rij = \acute\vV\rij + \bre\aV_{\acute S} \times \acute\vR\rij$
\bigskip
\par \hspace{+0.90em} $\vA\rij + 2 \hspace{+0.06em} \bre\aV_{S \vphantom{\acute S}} \times \vV\rij + \bre\aV_{S \vphantom{\acute S}} \times (\bre\aV_{S \vphantom{\acute S}} \times \vR\rij) + \bre\aA_{S \vphantom{\acute S}} \times \vR\rij = \acute\vA\rij + 2 \hspace{+0.06em} \bre\aV_{\acute S} \times \acute\vV\rij + \bre\aV_{\acute S} \times (\bre\aV_{\acute S} \times \acute\vR\rij) + \bre\aA_{\acute S} \times \acute\vR\rij$

\vspace{+1.20em}

\par \hspace{+0.81em} {\fontsize{10.50}{10.50}\selectfont\textbf{Alternative Equations}}

\vspace{+1.20em}

\par \hspace{+0.90em} $\vL = \sum_{i=1}^N \mM\ri \, (\vV\ri + \bre\aV\rS \times \vR\ri) \times \vR\ri - {\mathrm M} \, (\vV\rcm + \bre\aV\rS \times \vR\rcm) \times \vR\rcm$
\bigskip
\par \hspace{+0.90em} $\vL = \sum_{i=1}^N \sum_{j{\scriptscriptstyle >}i}^N \, \mM\ri \, \mM\rj \, {\mathrm M^{\scriptscriptstyle -1}} (\vV\rij + \bre\aV\rS \times \vR\rij) \times \vR\rij$
\bigskip
\par \hspace{+0.90em} $\mK = \sum_{i=1}^N \hspace{-0.09em} \med \mM\ri \, (\vV\ri + \bre\aV\rS \times \vR\ri)^2 - \med {\mathrm M} \, (\vV\rcm + \bre\aV\rS \times \vR\rcm)^2$
\bigskip
\par \hspace{+0.90em} $\mK = \sum_{i=1}^N \sum_{j{\scriptscriptstyle >}i}^N \hspace{+0.045em} \med \mM\ri \, \mM\rj \, {\mathrm M^{\scriptscriptstyle -1}} (\vV\rij + \bre\aV\rS \times \vR\rij)^2$
\bigskip
\par \hspace{+0.90em} $\mK\rt \hspace{-0.06em} = \sum_{i=1}^N \hspace{-0.09em} \med \mM\ri \, (\vV\ri \cdot \vV\ri + \vA\ri \cdot \vR\ri) - \med {\mathrm M} \, (\vV\rcm \cdot \vV\rcm + \vA\rcm \cdot \vR\rcm)$
\bigskip
\par \hspace{+0.90em} $\mK\rt \hspace{-0.06em} = \sum_{i=1}^N \sum_{j{\scriptscriptstyle >}i}^N \hspace{+0.045em} \med \mM\ri \, \mM\rj \, {\mathrm M^{\scriptscriptstyle -1}} (\vV\rij \cdot \vV\rij + \vA\rij \cdot \vR\rij)$

\end{document}

