
\documentclass[12pt]{article}
%\documentclass[a4paper,12pt]{article}
%\documentclass[letterpaper,12pt]{article}

\usepackage[dvips]{geometry}
\geometry{papersize={153mm,198mm}}
\geometry{totalwidth=132mm,totalheight=162mm}

\usepackage[english]{babel}
\usepackage{mathptmx}

\usepackage{hyperref}
\hypersetup{colorlinks=true,linkcolor=black}
\hypersetup{bookmarksnumbered=true,pdfstartview=FitH,pdfpagemode=UseNone}
\hypersetup{pdftitle={Is the principle of energy a tautology ?}}
\hypersetup{pdfauthor={Alejandro A. Torassa}}

\setlength{\arraycolsep}{4.5pt}

\newcommand{\mM}{m}
\newcommand{\mT}{T}
\newcommand{\mV}{V}
\newcommand{\dos}{^{\,2}}
\newcommand{\ct}{constant}
\newcommand{\vR}{\mathbf{r}}
\newcommand{\vV}{\mathbf{v}}
\newcommand{\vA}{\mathbf{a}}
\newcommand{\ep}{\hspace{+3.6em}}
\newcommand{\eq}{\hspace{+4.5em}}
\newcommand{\ra}{_{\scriptscriptstyle \mathrm A}}

\begin{document}

\begin{center}

{\fontsize{18}{18}\selectfont Is the principle of energy a tautology ?}

\bigskip \bigskip

{\fontsize{12}{12}\selectfont Alejandro A. Torassa}

\bigskip \bigskip

\footnotesize

Creative Commons Attribution 3.0 License

(2011) Buenos Aires, Argentina

atorassa@gmail.com

\bigskip \bigskip

\small

{\bf Abstract}

\bigskip

\parbox{102mm}{This paper shows that it is possible to obtain the principle of energy starting from the acceleration of a particle.}

\end{center}

\normalsize

\bigskip \bigskip

\noindent In classical mechanics, if we consider a force field (uniform or non-uniform) in which the acceleration $\vA\ra$ of a particle A is constant, then

\vspace{-0.3em}

\begin{eqnarray*}
\ep {\vphantom{\int}} \vA\ra & = & \vA\ra \\
\ep \int_a^b \vA\ra \cdot d\vR\ra & = & \int_a^b \vA\ra \cdot d\vR\ra \\
\ep {\vphantom{\int}} \Delta \; {\textstyle \frac{1}{2}} \; \vV\ra\dos & = & \Delta \; \vA\ra \cdot \vR\ra \\
\ep {\vphantom{\int}} \Delta \; {\textstyle \frac{1}{2}} \; \vV\ra\dos - \, \Delta \; \vA\ra^{\vphantom{\dos}} \cdot \vR\ra^{\vphantom{\dos}} & = & 0 \\
\ep {\vphantom{\int}} \mM\ra^{\vphantom{\dos}} \left( \Delta \; {\textstyle \frac{1}{2}} \; \vV\ra\dos - \, \Delta \; \vA\ra^{\vphantom{\dos}} \cdot \vR\ra^{\vphantom{\dos}} \right) & = & 0 \\
\ep {\vphantom{\int}} \Delta \; \mT\ra + \, \Delta \; \mV\ra & = & 0 \eq {\hphantom{\ct}} \mT\ra = {\textstyle \frac{1}{2}} \; \mM\ra^{\vphantom{\dos}}\vV\ra\dos \\
\ep {\vphantom{\int}} \mT\ra + \, \mV\ra & = & \ct \eq {\hphantom{0}} \mV\ra = - \; \mM\ra^{\vphantom{\dos}} \; \vA\ra^{\vphantom{\dos}} \cdot \vR\ra^{\vphantom{\dos}}
\end{eqnarray*}

\vspace{+0.9em}

\noindent If $\vA\ra$ is not constant but $\vA\ra$ is function of $\vR\ra$ then the same result is obtained, even if Newton's second law were not valid.

\end{document}

