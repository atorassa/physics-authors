
\documentclass[10pt,fleqn]{article}
%\documentclass[a4paper,10pt]{article}
%\documentclass[letterpaper,10pt]{article}

\usepackage[dvips]{geometry}
\geometry{papersize={168.0mm,228.0mm}}
\geometry{totalwidth=147.0mm,totalheight=192.0mm}

\usepackage[english]{babel}
\usepackage{mathptmx}

\usepackage{hyperref}
\hypersetup{colorlinks=true,linkcolor=black}
\hypersetup{bookmarksnumbered=true,pdfstartview=FitH,pdfpagemode=UseNone}
\hypersetup{pdftitle={A New System of Equations in Classical Mechanics}}
\hypersetup{pdfauthor={Alejandro A. Torassa}}

\setlength{\arraycolsep}{1.74pt}

\newcommand{\mM}{m}
\newcommand{\mI}{I}
\newcommand{\mY}{Y}
\newcommand{\mK}{K}
\newcommand{\mP}{P}
\newcommand{\mW}{W}
\newcommand{\mU}{U}
\newcommand{\ri}{_i}
\newcommand{\rcm}{_{cm}}
\newcommand{\vR}{\mathbf{r}}
\newcommand{\vV}{\mathbf{v}}
\newcommand{\vA}{\mathbf{a}}
\newcommand{\vF}{\mathbf{F}}
\newcommand{\rj}{_{\hspace{-0.081em}j}}

\begin{document}

\begin{center}

{\Large A New System of Equations in Classical Mechanics}

\bigskip \medskip

Alejandro A. Torassa

\bigskip \medskip

\footnotesize

Creative Commons Attribution 3.0 License

(2014) Buenos Aires, Argentina

atorassa@gmail.com

\bigskip \smallskip

\small

{\bf Abstract}

\bigskip

\parbox{123mm}{In classical mechanics, this paper presents a new system of equations which is invariant under transformations between reference frames and which can be applied in any reference frame (rotating or non-rotating) (inertial or non-inertial) without the necessity of introducing fictitious forces.}

\end{center}

\normalsize

\vspace{-0.60em}

{\centering\subsubsection*{Definitions}}

\vspace{+0.60em}

\par If we consider a system of N particles then the total moment of inertia $\mI$ of the system of particles, the total kinetic liveliness $\mY$ of the system of particles, and the total kinetic energy $\mK$ of the system of particles, are as follows:
\medskip
\begin{eqnarray*}
\hspace{+1.50em} \mI \hspace{+0.12em} = \hspace{+0.12em} \sum_{i={\scriptscriptstyle 1}}^{\mathrm N} \hspace{+0.24em} \left( \hspace{+0.12em} \mM\ri \hspace{+0.24em} \bar\vR\ri \cdot \bar\vR\ri \hspace{+0.12em} \right)
\end{eqnarray*}
\begin{eqnarray*}
\hspace{+1.50em} \mY \hspace{+0.12em} = \hspace{+0.12em} \sum_{i={\scriptscriptstyle 1}}^{\mathrm N} \hspace{+0.24em} \left( \hspace{+0.12em} \mM\ri \hspace{+0.24em} \bar\vR\ri \cdot \bar\vV\ri \hspace{+0.12em} \right)
\end{eqnarray*}
\begin{eqnarray*}
\hspace{+1.50em} \mK \hspace{+0.12em} = \hspace{+0.12em} \sum_{i={\scriptscriptstyle 1}}^{\mathrm N} \hspace{+0.24em} \left( \hspace{+0.12em} \mM\ri \hspace{+0.24em} \bar\vV\ri \cdot \bar\vV\ri \hspace{+0.12em} + \hspace{+0.12em} \mM\ri \hspace{+0.24em} \bar\vA\ri \cdot \bar\vR\ri \hspace{+0.12em} \right)
\end{eqnarray*}
\smallskip
\par \noindent where $\bar\vR\ri = \vR\ri - \vR\rcm$, $\bar\vV\ri = \vV\ri - \vV\rcm$, $\bar\vA\ri = \vA\ri - \vA\rcm$, $\vR\ri$, $\vV\ri$ and $\vA\ri$ are the position, the velocity and the acceleration of the \textit{i}-th particle, $\vR\rcm$, $\vV\rcm$ and $\vA\rcm$ are the position, the velocity and the acceleration of the center of mass of the system of particles, and $\mM\ri$ is the mass of the \textit{i}-th particle.
\smallskip
\par If we consider a system of N particles then the total push $\mP$ done by the sub-forces acting on the system of particles, the total work $\mW$ done by the forces acting on the system of particles, and the total potential energy $\mU$ of the system of particles, are as follows:
\medskip
\begin{eqnarray*}
\hspace{+1.50em} \mP \hspace{+0.12em} = \hspace{+0.12em} \sum_{i={\scriptscriptstyle 1}}^{\mathrm N} \hspace{+0.24em} \Delta \hspace{+0.03em} \left( \hspace{+0.06em} \bar\vR\ri \hspace{+0.09em} \cdot \int \hspace{+0.12em} \vF\ri \hspace{+0.36em} dt \hspace{+0.06em} \right)
\end{eqnarray*}
\begin{eqnarray*}
\hspace{+1.50em} \mW \hspace{+0.12em} = \hspace{+0.12em} \sum_{i={\scriptscriptstyle 1}}^{\mathrm N} \hspace{+0.24em} + \left( \hspace{+0.06em} 2 \int_1^{\hspace{+0.09em} 2} \hspace{+0.12em} \vF\ri \cdot d\bar\vR\ri \hspace{+0.12em} + \hspace{+0.12em} \Delta \hspace{+0.24em} \vF\ri \cdot \bar\vR\ri \hspace{+0.06em} \right)
\end{eqnarray*}
\begin{eqnarray*}
\hspace{+1.50em} \Delta \hspace{+0.18em} \mU \hspace{+0.12em} = \hspace{+0.12em} \sum_{i={\scriptscriptstyle 1}}^{\mathrm N} \hspace{+0.24em} - \left( \hspace{+0.06em} 2 \int_1^{\hspace{+0.09em} 2} \hspace{+0.12em} \vF\ri \cdot d\bar\vR\ri \hspace{+0.12em} + \hspace{+0.12em} \Delta \hspace{+0.24em} \vF\ri \cdot \bar\vR\ri \hspace{+0.06em} \right)
\end{eqnarray*}
\smallskip
\par \noindent where $\bar\vR\ri = \vR\ri - \vR\rcm$, $\vR\ri$ is the position of the \textit{i}-th particle, $\vR\rcm$ is the position of the center of mass of the system of particles, $\vF\ri$ is the net force acting on the \textit{i}-th particle, and $t$ is the time.

\newpage

{\centering\subsubsection*{Theorems}}

\vspace{+1.20em}

\par In a system of N particles, the total push $\mP$ done by the sub-forces acting on the system of particles is equal to the change in the total kinetic liveliness $\mY$ of the system of particles.
\begin{eqnarray*}
\hspace{+1.50em} \mP = + \hspace{+0.09em} \Delta \hspace{+0.18em} \mY
\end{eqnarray*}
\par In a system of N particles, the total work $\mW$ done by the forces acting on the system of particles is equal to the change in the total kinetic energy $\mK$ of the system of particles.
\begin{eqnarray*}
\hspace{+1.50em} \mW = + \hspace{+0.09em} \Delta \hspace{+0.18em} \mK
\end{eqnarray*}
\par In a system of N particles, the total work $\mW$ done by the conservative forces acting on the system of particles is equal and opposite in sign to the change in the total potential energy $\mU$ of the system of particles.
\begin{eqnarray*}
\hspace{+1.50em} \mW = - \hspace{+0.09em} \Delta \hspace{+0.18em} \mU
\end{eqnarray*}

\vspace{+0.90em}

{\centering\subsubsection*{Principles}}

\vspace{+1.20em}

\par In a system of N particles, if the sub-forces acting on the system of particles do not perform push then the total kinetic liveliness of the system of particles remains constant.
\begin{eqnarray*}
\hspace{+1.50em} \mY = \hspace{+0.09em} constant
\end{eqnarray*}
\par In a system of N particles, if the non-conservative forces acting on the system of particles do not perform work then the total (mechanical) energy of the system of particles remains constant.
\begin{eqnarray*}
\hspace{+1.50em} \mK + \mU = \hspace{+0.09em} constant
\end{eqnarray*}

\vspace{+0.90em}

{\centering\subsubsection*{Observations}}

\vspace{+1.20em}

\par The new system of equations is invariant under transformations between reference frames.
\bigskip
\par The new system of equations can be applied in any reference frame (rotating or non-rotating) (inertial or non-inertial) without the necessity of introducing fictitious forces.
\bigskip
\par The new system of equations would be valid even if Newton's third law of motion were false in an inertial reference frame.
\bigskip
\par The new system of equations would be valid even if Newton's three laws of motion were false in a non-inertial reference frame.
\bigskip
\par On the other hand, the new system of equations can be obtained from the general equation of motion ( \hspace{-0.45em} \textbf{A. Torassa}, General Equation of Motion \hspace{-0.45em} )

\newpage

{\centering\subsubsection*{Annex}}

\vspace{+0.90em}

\par If we consider an isolated system of N particles and if Newton's third law of motion is valid then the total work $\mW$ done by the forces acting on the system of particles, and the total potential energy $\mU$ of the system of particles, are as follows:
\medskip
\begin{eqnarray*}
\hspace{+1.50em} \mW \hspace{+0.12em} = \hspace{+0.12em} \sum_{i={\scriptscriptstyle 1}}^{\mathrm N} \hspace{+0.24em} + \left( \hspace{+0.06em} 2 \int_1^{\hspace{+0.09em} 2} \hspace{+0.12em} \vF\ri \cdot d\vR\ri \hspace{+0.12em} + \hspace{+0.12em} \Delta \hspace{+0.24em} \vF\ri \cdot \vR\ri \hspace{+0.06em} \right)
\end{eqnarray*}
\begin{eqnarray*}
\hspace{+1.50em} \Delta \hspace{+0.18em} \mU \hspace{+0.12em} = \hspace{+0.12em} \sum_{i={\scriptscriptstyle 1}}^{\mathrm N} \hspace{+0.24em} - \left( \hspace{+0.06em} 2 \int_1^{\hspace{+0.09em} 2} \hspace{+0.12em} \vF\ri \cdot d\vR\ri \hspace{+0.12em} + \hspace{+0.12em} \Delta \hspace{+0.24em} \vF\ri \cdot \vR\ri \hspace{+0.06em} \right)
\end{eqnarray*}
\smallskip
\par \noindent where $\vR\ri$ is the position of the \textit{i}-th particle, and $\vF\ri$ is the net force acting on the \textit{i}-th particle.
\bigskip
\par If we consider a system of N particles then the total moment of inertia $\mI$ of the system of particles, the total kinetic liveliness $\mY$ of the system of particles, and the total kinetic energy $\mK$ of the system of particles, can also be expressed as follows:
\medskip
\begin{eqnarray*}
\hspace{+1.50em} \mI \hspace{+0.12em} = \hspace{+0.12em} \sum_{i={\scriptscriptstyle 1}}^{\mathrm N} \hspace{+0.24em} \left( \hspace{+0.12em} \mM\ri \hspace{+0.24em} \vR\ri \cdot \vR\ri \hspace{+0.12em} \right) \hspace{+0.12em} - \hspace{+0.12em} \mM\rcm \hspace{+0.24em} \vR\rcm \cdot \vR\rcm
\end{eqnarray*}
\begin{eqnarray*}
\hspace{+1.50em} \mY \hspace{+0.12em} = \hspace{+0.12em} \sum_{i={\scriptscriptstyle 1}}^{\mathrm N} \hspace{+0.24em} \left( \hspace{+0.12em} \mM\ri \hspace{+0.24em} \vR\ri \cdot \vV\ri \hspace{+0.12em} \right) \hspace{+0.12em} - \hspace{+0.12em} \mM\rcm \hspace{+0.24em} \vR\rcm \cdot \vV\rcm
\end{eqnarray*}
\begin{eqnarray*}
\hspace{+1.50em} \mK \hspace{+0.12em} = \hspace{+0.12em} \sum_{i={\scriptscriptstyle 1}}^{\mathrm N} \hspace{+0.24em} \left( \hspace{+0.12em} \mM\ri \hspace{+0.24em} \vV\ri \cdot \vV\ri \hspace{+0.12em} + \hspace{+0.12em} \mM\ri \hspace{+0.24em} \vA\ri \cdot \vR\ri \hspace{+0.12em} \right) \hspace{+0.12em} - \hspace{+0.12em} \mM\rcm \hspace{+0.24em} \vV\rcm \cdot \vV\rcm \hspace{+0.12em} - \hspace{+0.12em} \mM\rcm \hspace{+0.24em} \vA\rcm \cdot \vR\rcm
\end{eqnarray*}
\begin{eqnarray*}
\hspace{+1.50em} \mI \hspace{+0.12em} = \hspace{+0.12em} \sum_{i={\scriptscriptstyle 1}}^{\mathrm N} \hspace{+0.36em} \sum_{j>i}^{\mathrm N} \hspace{+0.24em} \left( \hspace{+0.06em} \frac{\mM\ri \hspace{+0.12em} \mM\rj}{\mM\rcm} \hspace{+0.24em} (\vR\ri - \vR\rj) \cdot (\vR\ri - \vR\rj) \hspace{+0.06em} \right)
\end{eqnarray*}
\begin{eqnarray*}
\hspace{+1.50em} \mY \hspace{+0.12em} = \hspace{+0.12em} \sum_{i={\scriptscriptstyle 1}}^{\mathrm N} \hspace{+0.36em} \sum_{j>i}^{\mathrm N} \hspace{+0.24em} \left( \hspace{+0.06em} \frac{\mM\ri \hspace{+0.12em} \mM\rj}{\mM\rcm} \hspace{+0.24em} (\vR\ri - \vR\rj) \cdot (\vV\ri - \vV\rj) \hspace{+0.06em} \right)
\end{eqnarray*}
\begin{eqnarray*}
\hspace{+1.50em} \mK \hspace{+0.12em} = \hspace{+0.12em} \sum_{i={\scriptscriptstyle 1}}^{\mathrm N} \hspace{+0.36em} \sum_{j>i}^{\mathrm N} \hspace{+0.24em} \left( \hspace{+0.06em} \frac{\mM\ri \hspace{+0.12em} \mM\rj}{\mM\rcm} \hspace{+0.24em} (\vV\ri - \vV\rj) \cdot (\vV\ri - \vV\rj) \hspace{+0.18em} + \hspace{+0.18em} \frac{\mM\ri \hspace{+0.12em} \mM\rj}{\mM\rcm} \hspace{+0.24em} (\vA\ri - \vA\rj) \cdot (\vR\ri - \vR\rj) \hspace{+0.06em} \right)
\end{eqnarray*}
\smallskip
\par \noindent where $\vR\ri$, $\vV\ri$, $\vA\ri$, $\vR\rj$, $\vV\rj$, $\vA\rj$, $\vR\rcm$, $\vV\rcm$, $\vA\rcm$ are the positions, the velocities and the accelerations of the \textit{i}-th particle, of the \textit{j}-th particle and of the center of mass of the system of particles, and $\mM\ri$, $\mM\rj$, $\mM\rcm$ are the masses of the \textit{i}-th particle, of the \textit{j}-th particle and of the center of mass of the system of particles.

\end{document}

