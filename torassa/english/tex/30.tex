
\documentclass[10pt]{article}
%\documentclass[a4paper,10pt]{article}
%\documentclass[letterpaper,10pt]{article}

\usepackage[dvips]{geometry}
\geometry{papersize={123.0mm,162.0mm}}
\geometry{totalwidth=102.0mm,totalheight=126.0mm}

\usepackage[english]{babel}
\usepackage{mathptmx}

\usepackage{hyperref}
\hypersetup{colorlinks=true,linkcolor=black}
\hypersetup{bookmarksnumbered=true,pdfstartview=FitH,pdfpagemode=UseNone}
\hypersetup{pdftitle={Principle of Least Angular Action}}
\hypersetup{pdfauthor={Alejandro A. Torassa}}

\setlength{\arraycolsep}{1.74pt}

\newcommand{\mM}{m}
\newcommand{\mL}{L}
\newcommand{\mT}{T}
\newcommand{\mV}{V}
\newcommand{\ra}{_a}
\newcommand{\vR}{\mathbf{r}}
\newcommand{\vV}{\mathbf{v}}
\newcommand{\vF}{\mathbf{F}}
\newcommand{\med}{\raise.5ex\hbox{$\scriptstyle 1$}\kern-.15em/\kern-.15em\lower.25ex\hbox{$\scriptstyle 2$}}

\begin{document}

\begin{center}

{\Large Principle of Least Angular Action}

\bigskip \medskip

Alejandro A. Torassa

\bigskip \medskip

\footnotesize

Creative Commons Attribution 3.0 License

(2014) Buenos Aires, Argentina

atorassa@gmail.com

\bigskip \smallskip

\small

{\bf Abstract}

\bigskip

This paper presents the principle of least angular action.

\end{center}

\normalsize

\vspace{-0.30em}

{\centering\subsubsection*{Principle of Least Angular Action}}

\vspace{+0.75em}

\par If we consider a single particle A of mass $\mM\ra$ then the principle of least angular action, is given by:
\begin{eqnarray*}
\delta \int_{t_1}^{t_2} \med \hspace{+0.24em} \mM\ra \hspace{+0.06em} (\hspace{+0.03em} \vR \times \vV\ra)^{\hspace{+0.03em} 2} \hspace{+0.12em} dt \hspace{+0.12em} + \int_{t_1}^{t_2} (\hspace{+0.03em} \vR \times \vF\ra) \cdot \delta (\hspace{+0.03em} \vR \times \vR\ra) \hspace{+0.18em} dt = 0
\end{eqnarray*}
\noindent where $\vR$ is a position vector which is constant in magnitude and direction, $\vV\ra$ is the velocity of particle A, $\vF\ra$ is the net force acting on particle A, and $\vR\ra$ is the position of particle A.
\medskip
\par If $- \hspace{+0.03em} \delta \hspace{+0.09em} \mV\ra = (\hspace{+0.03em} \vR \times \vF\ra) \cdot \delta (\hspace{+0.03em} \vR \times \vR\ra)$ and since $\mT\ra = \med \hspace{+0.24em} \mM\ra \hspace{+0.06em} (\hspace{+0.03em} \vR \times \vV\ra)^{\hspace{+0.03em} 2}$, then:
\begin{eqnarray*}
\delta \int_{t_1}^{t_2} (\mT\ra - \mV\ra) \hspace{+0.18em} dt = 0
\end{eqnarray*}
\par And since $\mL\ra = \mT\ra - \mV\ra$, then we obtain:
\begin{eqnarray*}
\delta \int_{t_1}^{t_2} \mL\ra \hspace{+0.24em} dt = 0
\end{eqnarray*}

\end{document}

