
\documentclass[11pt]{article}
%\documentclass[a4paper,11pt]{article}
%\documentclass[letterpaper,11pt]{article}
\usepackage[totalwidth=108mm,totalheight=174mm]{geometry}

\usepackage{graphicx}

\usepackage[english]{babel}
\usepackage{mathptmx}

\parindent=3mm

\usepackage{hyperref}
\hypersetup{colorlinks=true,linkcolor=black}
\hypersetup{bookmarksnumbered=true,pdfstartview=FitH,pdfpagemode=UseNone}
\hypersetup{pdftitle={On Reference Frames}}
\hypersetup{pdfauthor={Author: A. Torassa - Editor: W. Babin}}

\setlength{\unitlength}{0.57pt}
\setlength{\arraycolsep}{1.74pt}

\newcommand{\vV}{\mathbf{v}}
\newcommand{\vA}{\mathbf{a}}
\newcommand{\mX}{x}
\newcommand{\mY}{y}
\newcommand{\mZ}{z}
\newcommand{\mT}{t}
\newcommand{\rt}{'}
\newcommand{\rot}{_{o'}}
\newcommand{\tf}{Figure}

\begin{document}

\begin{figure}
\includegraphics{logo6.jpg}
%\includegraphics{logo6.eps}
\end{figure}

\enlargethispage{0em}

\begin{center}

\ \vspace{-1.5em}

{\LARGE On Reference Frames}

\vspace{+1.8em}

\normalsize

Alejandro A. Torassa

\medskip

\footnotesize

Buenos Aires, Argentina, E-mail: atorassa@gmail.com

\medskip

Creative Commons Attribution 3.0 License

\medskip

(Copyright 2009)

\vspace{+1.2em}

\end{center}

\begin{abstract}

\noindent In this work it is established, on one hand, that any reference frame should be fixed to a body and, on the other hand, that it is possible to agree that any reference frame fixed to a body should be non-rotating.

\medskip

\noindent Keywords: classical mechanics, observer, body, center of mass, rotating reference frame, non-rotating reference frame.

\vspace{+0.3em}

\end{abstract}

\normalsize

{\centering\subsection*{Part One}}

\par It is known from the observations of a reference frame S that using the appropriate transformation laws it is possible to learn the observations of a different reference frame S'.
\medskip
\par However, observations of reference frame S' obtained from the previous method are hypothetical, since the real observations of reference frame S' are the ones performed on the reference frame S' itself.
\medskip
\par According to this work, in order to learn the real observations of a reference frame it is necessary for the reference frame to exist; and for the reference frame to exist it is necessary that it be fixed to a body.
\medskip
\par Therefore, any reference frame should be fixed to a body.

\newpage \enlargethispage{0em}

\ \vspace{-0.9em}

{\centering\subsection*{Part Two}}

\par How should a reference frame be fixed to a body?
\medskip
\par According to this work, if we consider that any body is a particle or a system of particles, then the origin of any reference frame should be fixed to the center of mass of a body.
\medskip
\par Since the center of mass of any body is a point in space without rotation, then it is possible that the origin of a non-rotating reference frame be always fixed to the center of mass of any body. However, it is not possible that the origin of an inertial reference frame be always fixed to the center of mass of any body.
\medskip
\par Therefore, if the origin of any reference frame should be fixed to the center of mass of a body, then it is possible to agree that any reference frame fixed to a body should be non-rotating.

\bigskip

{\centering\subsection*{Conclusions}}

\par According to this work, every non-rotating reference frame is defined by an origin that is fixed to the center of mass of a body, and three perpendicular coordinate axes, where each axis is parallel to the corresponding axis of a universal reference frame defined by four widely separated stars.
\medskip
\par On the other hand, several laws of physics would take a simpler form if no reference frame were a rotating reference frame. However, the rotation of a body would be absolute; for example, the rotation of the Earth would be absolute. But, in the theory of relativity, the speed of light is also absolute.
\medskip
\par In addition, according to this work, every body is a possible non-rotating reference frame. Therefore, every body is also a possible observer.
\medskip
\par Finally, the laws of physics should be the same for all observers. Therefore, according to this work, the laws of physics should have the same form in all non-rotating reference frames.

\newpage \enlargethispage{+0.3em}

{\centering\subsection*{Appendix}}

{\centering\subsection*{Transformations of Classical Mechanics}}

\par If any reference frame is a non-rotating reference frame, then each coordinate axis of a reference frame S will remain at a fixed angle to the corresponding coordinate axis of another reference frame S'. Therefore, to simplify calculations it will be assumed that each axis \hbox {of S} is parallel to the corresponding axis of S', as shown in Figure 1.

\vspace{+1.2em}

\begin{center}
\begin{picture}(228,198)
\multiput(75,75)(45,18){2}{\vector(1,0){90}}
\multiput(75,75)(45,18){2}{\vector(0,1){90}}
\multiput(75,75)(45,18){2}{\vector(-1,-1){60}}
\put(72,171){$\mZ$}\put(117,189){$\mZ\rt$}
\put(171,72){$\mX$}\put(216,90){$\mX\rt$}
\put(3,3){$\mY$}\put(45,18){$\mY\rt$}
\put(78,78){$O$}\put(123,96){$O\rt$}
\put(24,96){S}\put(162,141){S'}
\end{picture}
\\* \tf \ 1
\end{center}

\smallskip

\par A change of coordinates $\mX$, $\mY$, $\mZ$, $\mT$ from reference frame S to coordinates $\mX\rt$, $\mY\rt$, $\mZ\rt$, $\mT\rt$ from reference frame S' whose origin $O\rt$ has coordinates $\mX\rot$, $\mY\rot$, $\mZ\rot$ measured from S, can be carried out by means of the following equations:
\begin{eqnarray*}
\mX\rt & = & \mX - \mX\rot \\*
\mY\rt & = & \mY - \mY\rot \\*
\mZ\rt & = & \mZ - \mZ\rot \\*
\mT\rt & = & \mT
\end{eqnarray*}
\par From these equations, the transformation of velocity and acceleration from reference frame S to reference frame S' may be carried out, and expressed in vector form as follows:
\begin{eqnarray*}
\vV\rt & = & \vV - \vV\rot \\*
\vA\rt & = & \vA - \vA\rot
\end{eqnarray*}
\noindent where $\vV\rot$ and $\vA\rot$ are the velocity and acceleration respectively, of reference frame S' relative to S.

\end{document}

