
\documentclass[10pt]{article}
%\documentclass[a4paper,10pt]{article}
%\documentclass[letterpaper,10pt]{article}

\usepackage[dvips]{geometry}
\geometry{papersize={130.0mm,168.0mm}}
\geometry{totalwidth=109.0mm,totalheight=132.0mm}

\usepackage[english]{babel}
\usepackage{mathptmx}

\usepackage{hyperref}
\hypersetup{colorlinks=true,linkcolor=black}
\hypersetup{bookmarksnumbered=true,pdfstartview=FitH,pdfpagemode=UseNone}
\hypersetup{pdftitle={Classical Dynamics of Particles}}
\hypersetup{pdfauthor={Alejandro A. Torassa}}

\setlength{\arraycolsep}{1.74pt}

\newcommand{\mT}{t}
\newcommand{\mN}{m}
\newcommand{\mM}{m\,}
\newcommand{\til}{\breve}
\newcommand{\dos}{^{\,2}}
\newcommand{\vR}{\mathbf{r}}
\newcommand{\vV}{\mathbf{v}}
\newcommand{\vA}{\mathbf{a}}
\newcommand{\vF}{\mathbf{F}}
\newcommand{\med}{\raise.5ex\hbox{$\scriptstyle 1$}\kern-.15em/\kern-.15em\lower.25ex\hbox{$\scriptstyle 2$}\:}

\begin{document}

\begin{center}

{\LARGE Classical Dynamics of Particles}

\bigskip \medskip

Alejandro A. Torassa

\bigskip \medskip

\footnotesize

Creative Commons Attribution 3.0 License

(2013) Buenos Aires, Argentina

atorassa@gmail.com

\bigskip \smallskip

\small

{\bf Abstract}

\bigskip

\parbox{84mm}{This paper presents a classical dynamics of particles, which can be applied in any inertial reference frame.}

\end{center}

\normalsize

\vspace{-0.30em}

{\centering\subsubsection*{Definitions}}

\vspace{+1.20em}

\begin{center}
\begin{tabular}{ll}
$\vR$ = position & $\til\vR$ = non-kinetic position \\ \\
$\vV$ = velocity & $\til\vV$ = non-kinetic velocity \\ \\
$\vA$ = acceleration & $\til\vA$ = non-kinetic acceleration
\end{tabular}
\end{center}

{\centering\subsubsection*{Relations}}

\vspace{-0.60em}

\begin{eqnarray*}
\til\vA = \vF/\mN \;\;\;\;\;\; \rightarrow \;\;\;\;\;\; \til\vA\dos = (\vF/\mN)\dos
\end{eqnarray*}

\begin{eqnarray*}
\til\vV = \int \til\vA \; d\mT \;\;\;\;\;\; \rightarrow \;\;\;\;\;\; \til\vV = \int (\vF/\mN) \; d\mT
\end{eqnarray*}

\begin{eqnarray*}
\med\til\vV\dos = \int \til\vA \; d\til\vR \;\;\;\;\;\; \rightarrow \;\;\;\;\;\; \med\til\vV\dos = \int (\vF/\mN) \; d\til\vR
\end{eqnarray*}

\newpage

{\centering\subsubsection*{Principles}}

\vspace{+0.60em}

\begin{center}
\begin{tabular}{ccccc}
{\makebox(6,30){(1)}} & {\framebox(115,30){$\mM\vR - \mM\til\vR = 0$}} & {\makebox(6,30){$\rightarrow$}} & {\framebox(115,30){$\med\mM\vR\dos - \med\mM\til\vR\dos = 0$}} & {\makebox(6,30){(2)}} \\
& {\makebox(115,21){$\downarrow$}} & & {\makebox(115,21){$\downarrow$}} & \\
{\makebox(6,30){(3)}} & {\framebox(115,30){$\mM\vV - \mM\til\vV = 0$}} & {\makebox(6,30){$\rightarrow$}} & {\framebox(115,30){$\med\mM\vV\dos - \med\mM\til\vV\dos = 0$}} & {\makebox(6,30){(4)}} \\
& {\makebox(115,21){$\downarrow$}} & {\makebox(6,21){$\nearrow$}} & {\makebox(115,21){$\downarrow$}} & \\
{\makebox(6,30){(5)}} & {\framebox(115,30){$\mM\vA - \mM\til\vA = 0$}} & {\makebox(6,30){$\rightarrow$}} & {\framebox(115,30){$\med\mM\vA\dos - \med\mM\til\vA\dos = 0$}} & {\makebox(6,30){(6)}}
\end{tabular}
\end{center}

\vspace{+0.60em}

\par \hspace{+0.69em} Substituting the relations into the principles, we obtain:

\vspace{+0.90em}

\begin{center}
\begin{tabular}{ccccc}
{\makebox(6,30){(1)}} & {\framebox(115,30){$\mM\vR - \mM\til\vR = 0$}} & {\makebox(6,30){$\rightarrow$}} & {\framebox(115,30){$\med\mM\vR\dos - \med\mM\til\vR\dos = 0$}} & {\makebox(6,30){(2)}} \\
& {\makebox(115,21){$\downarrow$}} & & {\makebox(115,21){$\downarrow$}} & \\
{\makebox(6,30){(3)}} & {\framebox(115,30){$\mM\vV - \int \vF \; d\mT = 0$}} & {\makebox(6,30){$\rightarrow$}} & {\framebox(115,30){$\med\mM\vV\dos - \int \vF \; d\til\vR = 0$}} & {\makebox(6,30){(4)}} \\
& {\makebox(115,21){$\downarrow$}} & {\makebox(6,21){$\nearrow$}} & {\makebox(115,21){$\downarrow$}} & \\
{\makebox(6,30){(5)}} & {\framebox(115,30){$\mM\vA - \vF = 0$}} & {\makebox(6,30){$\rightarrow$}} & {\framebox(115,30){$\med\mM\vA\dos - \med(\vF\dos/\mN) = 0$}} & {\makebox(6,30){(6)}}
\end{tabular}
\end{center}

\newpage

{\centering\subsubsection*{Observations}}

\vspace{+0.60em}

\par Equation (1) is related to the center of mass.
\bigskip
\par Equation (2) is related to the moment of inertia.
\bigskip
\par Equation (3) is related to the impulse and the linear momentum.
\bigskip
\par Equation (4) is related to the work and the energy.
\bigskip
\par Equation (5) is related to the forces (in vector form)
\bigskip
\par Equation (6) is related to the forces (in scalar form)
\bigskip
\par Finally, from equation (5) it follows that the acceleration $\vA$ of a particle, is given by:
\begin{eqnarray*}
\vA = \vF/\mN
\end{eqnarray*}
\noindent where $\vF$ is the net force acting on the particle, and $\mN$ is the mass of the particle.

\vspace{+0.90em}

{\centering\subsubsection*{Bibliography}}

\vspace{+0.30em}

\par \textbf{A. Einstein}, Relativity: The Special and General Theory.
\bigskip
\par \textbf{E. Mach}, The Science of Mechanics.
\bigskip
\par \textbf{R. Resnick and D. Halliday}, Physics.
\bigskip
\par \textbf{J. Kane and M. Sternheim}, Physics.
\bigskip
\par \textbf{H. Goldstein}, Classical Mechanics.
\bigskip
\par \textbf{L. Landau and E. Lifshitz}, Mechanics.

\end{document}

