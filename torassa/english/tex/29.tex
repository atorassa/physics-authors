
\documentclass[10pt]{article}
%\documentclass[a4paper,10pt]{article}
%\documentclass[letterpaper,10pt]{article}

\usepackage[dvips]{geometry}
\geometry{papersize={128.7mm,150.0mm}}
\geometry{totalwidth=108.6mm,totalheight=114.0mm}

\usepackage[english]{babel}
\usepackage{mathptmx}

\usepackage{hyperref}
\hypersetup{colorlinks=true,linkcolor=black}
\hypersetup{bookmarksnumbered=true,pdfstartview=FitH,pdfpagemode=UseNone}
\hypersetup{pdftitle={Angular Mechanical Energy}}
\hypersetup{pdfauthor={Alejandro A. Torassa}}

\setlength{\arraycolsep}{1.74pt}

\newcommand{\mM}{m}
\newcommand{\mE}{E}
\newcommand{\ra}{_a}
\newcommand{\vR}{\mathbf{r}}
\newcommand{\vV}{\mathbf{v}}
\newcommand{\vA}{\mathbf{a}}
\newcommand{\med}{\raise.5ex\hbox{$\scriptstyle 1$}\kern-.15em/\kern-.15em\lower.25ex\hbox{$\scriptstyle 2$}}

\begin{document}

\begin{center}

{\LARGE Angular Mechanical Energy}

\bigskip \medskip

Alejandro A. Torassa

\bigskip \medskip

\footnotesize

Creative Commons Attribution 3.0 License

(2014) Buenos Aires, Argentina

atorassa@gmail.com

\bigskip \smallskip

\small

{\bf Abstract}

\bigskip \smallskip

\parbox{72mm}{This paper presents the principle of conservation of the angular mechanical energy for a particle which moves in a uniform force field.}

\end{center}

\normalsize

\vspace{-0.30em}

{\centering\subsubsection*{Angular Mechanical Energy}}

\vspace{+0.75em}

\par The angular mechanical energy $\mE\ra$ of a particle A of mass $\mM\ra$ which moves in a uniform force field, is given by:
\begin{eqnarray*}
\mE\ra = \med \hspace{+0.24em} \mM\ra \hspace{+0.06em} (\hspace{+0.03em} \vR \times \vV\ra)^{\hspace{+0.03em} 2} - \hspace{+0.09em} \mM\ra \hspace{+0.06em} (\hspace{+0.03em} \vR \times \vA\ra) \cdot (\hspace{+0.03em} \vR \times \vR\ra)
\end{eqnarray*}
\noindent where $\vR$ is a position vector which is constant in magnitude and direction, and $\vV\ra$, $\vA\ra$ and $\vR\ra$ are the velocity, the constant acceleration and the position of particle A.
\medskip
\par The principle of conservation of the angular mechanical energy establishes that if a particle A moves in a uniform force field then the angular mechanical energy of particle A remains constant.

\end{document}

