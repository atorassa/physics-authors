
\documentclass[10pt]{article}
%\documentclass[a4paper,10pt]{article}
%\documentclass[letterpaper,10pt]{article}

\usepackage[dvips]{geometry}
\geometry{papersize={147.0mm,183.0mm}}
\geometry{totalwidth=126.0mm,totalheight=147.0mm}

\usepackage[english]{babel}
\usepackage{mathptmx}

\usepackage{hyperref}
\hypersetup{colorlinks=true,linkcolor=black}
\hypersetup{bookmarksnumbered=true,pdfstartview=FitH,pdfpagemode=UseNone}
\hypersetup{pdftitle={The Principle of Conservation of Energy}}
\hypersetup{pdfauthor={Alejandro A. Torassa}}

\setlength{\arraycolsep}{1.74pt}

\newcommand{\mM}{m}
\newcommand{\mW}{W}
\newcommand{\mK}{K}
\newcommand{\mU}{U}
\newcommand{\ri}{_i}
\newcommand{\rcm}{_{cm}}
\newcommand{\bre}{\breve}
\newcommand{\vR}{\mathbf{r}}
\newcommand{\vV}{\mathbf{v}}
\newcommand{\vA}{\mathbf{a}}
\newcommand{\vF}{\mathbf{F}}
\newcommand{\rj}{_{\hspace{-0.081em}j}}

\begin{document}

\begin{center}

{\LARGE The Principle of Conservation of Energy}

\bigskip \medskip

Alejandro A. Torassa

\bigskip \medskip

\footnotesize

Creative Commons Attribution 3.0 License

(2014) Buenos Aires, Argentina

atorassa@gmail.com

\bigskip \smallskip

\small

{\bf Abstract}

\bigskip

\parbox{96mm}{In classical mechanics, this paper presents a new principle of conservation of energy which is invariant under transformations between reference frames and which can be applied in any reference frame (rotating or non-rotating) (inertial or non-inertial) without the necessity of introducing fictitious forces.}

\end{center}

\normalsize

\vspace{-0.60em}

{\centering\subsubsection*{Introduction}}

\vspace{+0.90em}

\par \hspace{-0.24em} If we consider a system of two particles i and j then the following relations are obtained:
\vspace{-0.30em}
\begin{eqnarray*}
\bre\vR\ri - \bre\vR\rj = \vR\ri - \vR\rj
\end{eqnarray*}
\vspace{-0.60em}
\begin{eqnarray*}
\bre\vA\ri - \bre\vA\rj = \frac{\vF\ri}{\mM\ri} - \frac{\vF\rj}{\mM\rj}
\end{eqnarray*}
\vspace{-0.30em}
\begin{eqnarray*}
\Delta \hspace{+0.18em} (\bre\vV\ri - \bre\vV\rj) \cdot (\bre\vV\ri - \bre\vV\rj) = \hspace{+0.06em} 2 \hspace{-0.12em} \int_1^{\hspace{+0.09em} 2} (\bre\vA\ri - \bre\vA\rj) \cdot d(\bre\vR\ri - \bre\vR\rj)
\end{eqnarray*}
\vspace{-0.30em}
\begin{eqnarray*}
(\bre\vV\ri - \bre\vV\rj) \cdot (\bre\vV\ri - \bre\vV\rj) + (\bre\vA\ri - \bre\vA\rj) \cdot (\bre\vR\ri - \bre\vR\rj) = (\vV\ri - \vV\rj) \cdot (\vV\ri - \vV\rj) + (\vA\ri - \vA\rj) \cdot (\vR\ri - \vR\rj)
\end{eqnarray*}
\vspace{-0.30em} \smallskip
\par \noindent where $\bre\vR\ri, \bre\vV\ri, \bre\vA\ri, \bre\vR\rj, \bre\vV\rj, \bre\vA\rj$ are the positions, the velocities and the accelerations of particles i and j relative to an inertial reference frame $\bre S$, $\vR\ri, \vV\ri, \vA\ri, \vR\rj, \vV\rj, \vA\rj$ are the positions, the velocities and the accelerations of particles i and j relative to an inertial or non-inertial reference frame $S$, $\mM\ri, \mM\rj$ are the masses of particles i and j, and $\vF\ri, \vF\rj$ are the net forces acting on particles i and j.

\newpage

{\centering\subsubsection*{Work, K and U}}

\vspace{+0.90em}

\par If we consider a system of N particles then the total work $\mW$ done by the forces acting on the system of particles, is given by:
\vspace{+0.75em}
\begin{eqnarray*}
\mW = \sum_{i={\scriptscriptstyle 1}}^{\mathrm N} \hspace{+0.36em} \sum_{j>i}^{\mathrm N} \hspace{+0.36em} \frac{1}{2} \hspace{+0.24em} \frac{\mM\ri\hspace{+0.09em}\mM\rj}{M} \left[ \hspace{+0.12em} 2 \hspace{-0.12em} \int_1^{\hspace{+0.09em} 2} \hspace{-0.12em} \left(\frac{\vF\ri}{\mM\ri} - \frac{\vF\rj}{\mM\rj}\right) \cdot d(\vR\ri - \vR\rj) + \hspace{+0.03em} \Delta \left(\frac{\vF\ri}{\mM\ri} - \frac{\vF\rj}{\mM\rj}\right) \cdot (\vR\ri - \vR\rj) \hspace{+0.12em} \right]
\end{eqnarray*}
\medskip
\par In a system of N particles, the total work $\mW$ done by the forces acting on the system of particles is equal to the change in the total kinetic energy $\mK$ of the system of particles.
\vspace{+0.75em}
\begin{eqnarray*}
\Delta \hspace{+0.18em} \mK = \Delta \hspace{+0.24em} \sum_{i={\scriptscriptstyle 1}}^{\mathrm N} \hspace{+0.36em} \sum_{j>i}^{\mathrm N} \hspace{+0.36em} \frac{1}{2} \hspace{+0.24em} \frac{\mM\ri\hspace{+0.09em}\mM\rj}{M} \Big[ (\vV\ri - \vV\rj) \cdot (\vV\ri - \vV\rj) + (\vA\ri - \vA\rj) \cdot (\vR\ri - \vR\rj) \Big]
\end{eqnarray*}
\medskip
\par In a system of N particles, the total work $\mW$ done by the conservative forces acting on the system of particles is equal and opposite in sign to the change in the total potential energy $\mU$ of the system of particles.
\vspace{+0.75em}
\begin{eqnarray*}
- \hspace{+0.12em} \Delta \hspace{+0.18em} \mU = \sum_{i={\scriptscriptstyle 1}}^{\mathrm N} \hspace{+0.36em} \sum_{j>i}^{\mathrm N} \hspace{+0.36em} \frac{1}{2} \hspace{+0.24em} \frac{\mM\ri\hspace{+0.09em}\mM\rj}{M} \left[ \hspace{+0.12em} 2 \hspace{-0.12em} \int_1^{\hspace{+0.09em} 2} \hspace{-0.12em} \left(\frac{\vF\ri}{\mM\ri} - \frac{\vF\rj}{\mM\rj}\right) \cdot d(\vR\ri - \vR\rj) + \hspace{+0.03em} \Delta \left(\frac{\vF\ri}{\mM\ri} - \frac{\vF\rj}{\mM\rj}\right) \cdot (\vR\ri - \vR\rj) \hspace{+0.12em} \right]
\end{eqnarray*}
\medskip
\par In an isolated system of N particles, if Newton's third law of motion is valid then the total potential energy $\mU$ of the system of particles, is given by:
\vspace{+0.75em}
\begin{eqnarray*}
- \hspace{+0.12em} \Delta \hspace{+0.18em} \mU = \sum_{i={\scriptscriptstyle 1}}^{\mathrm N} \hspace{+0.12em} \left( \hspace{+0.06em} \int_1^{\hspace{+0.09em} 2} \hspace{+0.12em} \vF\ri \cdot d\vR\ri \hspace{+0.12em} + \hspace{+0.12em} \Delta \hspace{+0.24em} \frac{1}{2} \hspace{+0.24em} \vF\ri \cdot \vR\ri \hspace{+0.06em} \right)
\end{eqnarray*}
\medskip
\par \noindent where $\vR\ri, \vV\ri, \vA\ri, \vR\rj, \vV\rj, \vA\rj$ are the positions, the velocities and the accelerations of the \textit{i}-th and \textit{j}-th particles (relative to an inertial or non-inertial reference frame $S$) $\mM\ri, \mM\rj$ are the masses of the \textit{i}-th and \textit{j}-th particles, $\vF\ri, \vF\rj$ are the net forces acting on the \textit{i}-th and \textit{j}-th particles, and $M$ is the total mass of the system of particles.

\newpage

{\centering\subsubsection*{The Principle}}

\vspace{+0.90em}

\par The new principle of conservation of energy establishes that in a system of N particles that is only subject to conservative forces the total (mechanical) energy of the system of particles remains constant.
\smallskip
\begin{eqnarray*}
\mK + \mU = \hspace{+0.09em} constant
\end{eqnarray*}
\vspace{-0.30em}
\par \noindent where $K$ is the total kinetic energy of the system of particles, and $U$ is the total potential energy of the system of particles.

\vspace{+1.20em}

{\centering\subsubsection*{Observations}}

\vspace{+0.90em}

\par The total kinetic energy of a system of particles is invariant under transformations between reference frames.
\bigskip
\par The total potential energy of a system of particles is invariant under transformations between reference frames.
\bigskip
\par The new principle of conservation of energy is invariant under transformations between reference frames.
\bigskip
\par The new principle of conservation of energy can be applied in any reference frame (rotating or non-rotating) (inertial or non-inertial) without the necessity of introducing fictitious forces.

\vspace{+1.50em}

{\centering\subsubsection*{Bibliography}}

\vspace{+1.20em}

\par \textbf{A. Einstein}, Relativity: The Special and General Theory.
\bigskip
\par \textbf{E. Mach}, The Science of Mechanics.
\bigskip
\par \textbf{H. Goldstein}, Classical Mechanics.

\newpage

{\centering\subsubsection*{Annex}}

\vspace{+0.90em}

{\centering\subsubsection*{Work (cm)}}

\vspace{+0.90em}

\par If we consider a system of N particles then the total work $\mW$ done by the forces acting on the system of particles can also be expressed as follows:
\smallskip
\begin{eqnarray*}
\mW = \sum_{i={\scriptscriptstyle 1}}^{\mathrm N} \hspace{+0.12em} \left( \hspace{+0.06em} \int_1^{\hspace{+0.09em} 2} \hspace{+0.12em} \vF\ri \cdot d\bar\vR\ri \hspace{+0.12em} + \hspace{+0.12em} \Delta \hspace{+0.24em} \frac{1}{2} \hspace{+0.24em} \vF\ri \cdot \bar\vR\ri \hspace{+0.06em} \right)
\end{eqnarray*}
\smallskip
\par \noindent where $\bar\vR\ri = \vR\ri - \vR\rcm$, $\vR\ri$ is the position of the \textit{i}-th particle, $\vR\rcm$ is the position of the center of mass of the system of particles, and $\vF\ri$ is the net force acting on the \textit{i}-th particle.

\vspace{+1.50em}

{\centering\subsubsection*{Kinetic Energy}}

\vspace{+0.90em}

\par If we consider a system of N particles then the total kinetic energy $\mK$ of the system of particles can also be expressed as follows:
\smallskip
\begin{eqnarray*}
\mK = \sum_{i={\scriptscriptstyle 1}}^{\mathrm N} \hspace{+0.36em} \frac{1}{2} \hspace{+0.24em} \mM\ri \hspace{+0.12em} \Big( \hspace{+0.06em} \bar\vV\ri \cdot \bar\vV\ri \hspace{+0.12em} + \hspace{+0.12em} \bar\vA\ri \cdot \bar\vR\ri \hspace{+0.06em} \Big)
\end{eqnarray*}
\smallskip
\begin{eqnarray*}
\mK = \sum_{i={\scriptscriptstyle 1}}^{\mathrm N} \hspace{+0.36em} \sum_{j>i}^{\mathrm N} \hspace{+0.36em} \frac{1}{2} \hspace{+0.24em} \frac{\mM\ri\hspace{+0.09em}\mM\rj}{M} \hspace{+0.12em} \Big( \hspace{+0.06em} \dot{r}_{ij} \; \dot{r}_{ij} \hspace{+0.12em} + \hspace{+0.12em} \ddot{r}_{ij} \; r_{ij} \hspace{+0.06em} \Big)
\end{eqnarray*}
\smallskip
\begin{eqnarray*}
\mK = \sum_{i={\scriptscriptstyle 1}}^{\mathrm N} \hspace{+0.27em} \frac{1}{2} \hspace{+0.24em} \mM\ri \hspace{+0.24em} \vV\ri \cdot \vV\ri \hspace{+0.12em} - \hspace{+0.12em} \frac{1}{2} \hspace{+0.24em} M \hspace{+0.24em} \vV\rcm \cdot \vV\rcm \hspace{+0.12em} + \hspace{+0.12em} \sum_{i={\scriptscriptstyle 1}}^{\mathrm N} \hspace{+0.27em} \frac{1}{2} \hspace{+0.24em} \mM\ri \hspace{+0.24em} \vA\ri \cdot \vR\ri \hspace{+0.12em} - \hspace{+0.12em} \frac{1}{2} \hspace{+0.24em} M \hspace{+0.24em} \vA\rcm \cdot \vR\rcm
\end{eqnarray*}
\smallskip
\par \noindent where $r_{ij}=|\vR\ri - \vR\rj|$, $\dot{r}_{ij}=d|\vR\ri - \vR\rj|/dt$, $\ddot{r}_{ij}=d^2|\vR\ri - \vR\rj|/dt^2$, $\bar\vR\ri = \vR\ri - \vR\rcm$, $\bar\vV\ri = \vV\ri - \vV\rcm$, $\bar\vA\ri = \vA\ri - \vA\rcm$, $\vR\ri$, $\vV\ri$, $\vA\ri$, $\vR\rj$, $\vV\rj$, $\vA\rj$, $\vR\rcm$, $\vV\rcm$, $\vA\rcm$ are the positions, the velocities and the accelerations of the \textit{i}-th particle, of the \textit{j}-th particle and of the center of mass of the system of particles, $\mM\ri$, $\mM\rj$ are the masses of the \textit{i}-th particle and of the \textit{j}-th particle, and $M$ is the total mass of the system of particles.

\end{document}

