
\documentclass[10pt]{article}
%\documentclass[a4paper,10pt]{article}
%\documentclass[letterpaper,10pt]{article}

\usepackage[dvips]{geometry}
\geometry{papersize={135.0mm,165.0mm}}
\geometry{totalwidth=114.0mm,totalheight=129.0mm}

\usepackage[english]{babel}
\usepackage{mathptmx}

\usepackage{hyperref}
\hypersetup{colorlinks=true,linkcolor=black}
\hypersetup{bookmarksnumbered=true,pdfstartview=FitH,pdfpagemode=UseNone}
\hypersetup{pdftitle={A Group of Invariant Equations}}
\hypersetup{pdfauthor={Alejandro A. Torassa}}

\setlength{\arraycolsep}{1.74pt}

\newcommand{\ra}{_a}
\newcommand{\rb}{_b}
\newcommand{\vR}{\mathbf{r}}
\newcommand{\vV}{\mathbf{v}}
\newcommand{\vA}{\mathbf{a}}

\begin{document}

\begin{center}

{\LARGE A Group of Invariant Equations}

\bigskip \medskip

Alejandro A. Torassa

\bigskip \medskip

\footnotesize

Creative Commons Attribution 3.0 License

(2014) Buenos Aires, Argentina

atorassa@gmail.com

\bigskip \smallskip

\small

{\bf Abstract}

\bigskip

\parbox{69mm}{In classical mechanics, this paper presents a group of equations, which are invariant under transformations between reference frames.}

\end{center}

\normalsize

\vspace{-0.30em}

{\centering\subsubsection*{Group of Invariant Equations}}

\vspace{+1.20em}

\par If we consider two particles A and B then the group of invariant equations is:
\vspace{+0.90em}
\begin{eqnarray*}
\hspace{-4.50em} (\vR\ra - \vR\rb) \cdot (\vR\ra - \vR\rb) = invariant \\ \\
\hspace{-4.50em} (\vR\ra - \vR\rb) \cdot (\vV\ra - \vV\rb) = invariant \\ \\
\hspace{-4.50em} (\vV\ra - \vV\rb) \cdot (\vV\ra - \vV\rb) + (\vR\ra - \vR\rb) \cdot (\vA\ra - \vA\rb) = invariant
\end{eqnarray*}
\vspace{+0.09em}
\par \noindent where $\vR\ra$ and $\vR\rb$ are the positions of particles A and B, $\vV\ra$ and $\vV\rb$ are the velocities of particles A and B, and $\vA\ra$ and $\vA\rb$ are the accelerations of particles A and B.

\end{document}

