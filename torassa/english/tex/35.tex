
\documentclass[10pt]{article}
%\documentclass[a4paper,10pt]{article}
%\documentclass[letterpaper,10pt]{article}

\usepackage[dvips]{geometry}
\geometry{papersize={135.0mm,186.0mm}}
\geometry{totalwidth=114.0mm,totalheight=150.0mm}

\usepackage[english]{babel}
\usepackage{mathptmx}

\usepackage{hyperref}
\hypersetup{colorlinks=true,linkcolor=black}
\hypersetup{bookmarksnumbered=true,pdfstartview=FitH,pdfpagemode=UseNone}
\hypersetup{pdftitle={A Definition of Work}}
\hypersetup{pdfauthor={Alejandro A. Torassa}}

\setlength{\arraycolsep}{1.74pt}

\newcommand{\mM}{m}
\newcommand{\mW}{W}
\newcommand{\mK}{K}
\newcommand{\mU}{U}
\newcommand{\ra}{_a}
\newcommand{\rb}{_b}
\newcommand{\rab}{_{ab}}
\newcommand{\vR}{\mathbf{r}}
\newcommand{\vV}{\mathbf{v}}
\newcommand{\vA}{\mathbf{a}}
\newcommand{\vF}{\mathbf{F}}

\begin{document}

\begin{center}

{\LARGE A Definition of Work}

\bigskip \medskip

Alejandro A. Torassa

\bigskip \medskip

\footnotesize

Creative Commons Attribution 3.0 License

(2014) Buenos Aires, Argentina

atorassa@gmail.com

\bigskip \smallskip

\small

{\bf Abstract}

\bigskip

\parbox{87mm}{In classical mechanics, this paper presents a definition of work, which can be used in any reference frame (rotating or non-rotating) (inertial or non-inertial) without the necessity of introducing fictitious forces.}

\end{center}

\normalsize

\vspace{-0.60em}

{\centering\subsubsection*{Definition of Work}}

\vspace{+0.60em}

\par If we consider two particles A and B then the definition of the total work $\mW\rab$ done by the forces $\vF\ra$ and $\vF\rb$ acting on particles A and B respectively is:
\par \vspace{-0.30em}
{\fontsize{8}{8}\selectfont\begin{eqnarray*}
\mW\rab = \frac{1}{2} \hspace{+0.24em} \mM\ra\mM\rb \left[ \hspace{+0.12em} 2 \hspace{-0.12em} \int_1^{\hspace{+0.09em} 2} \hspace{-0.12em} \left(\frac{\vF\ra}{\mM\ra} - \frac{\vF\rb}{\mM\rb}\right) \cdot d(\vR\ra - \vR\rb) + \hspace{+0.03em} \Delta \left(\frac{\vF\ra}{\mM\ra} - \frac{\vF\rb}{\mM\rb}\right) \cdot (\vR\ra - \vR\rb) \hspace{+0.12em} \right]
\end{eqnarray*}}
\vspace{-0.60em}
\par \noindent where $\mM\ra$ and $\mM\rb$ are the masses of particles A and B, and $\vR\ra$ and $\vR\rb$ are the positions of particles A and B.
\medskip
\par The total work $\mW\rab$ is equal to the change in kinetic energy.
\vspace{+0.03em}
{\fontsize{8}{8}\selectfont\begin{eqnarray*}
\mW\rab = \Delta \hspace{+0.24em} \frac{1}{2} \hspace{+0.24em} \mM\ra\mM\rb \left[(\vV\ra - \vV\rb)^{\hspace{+0.03em} 2} + (\vA\ra - \vA\rb) \cdot (\vR\ra - \vR\rb)\right]
\end{eqnarray*}}
\vspace{-0.90em}
\par \noindent where $\vV\ra$ and $\vV\rb$ are the velocities of particles A and B, and $\vA\ra$ and $\vA\rb$ are the accelerations of particles A and B.
\medskip
\par Therefore, the kinetic energy $\mK\rab$ of particles A and B is:
{\fontsize{8}{8}\selectfont\begin{eqnarray*}
\mK\rab = \frac{1}{2} \hspace{+0.24em} \mM\ra\mM\rb \left[(\vV\ra - \vV\rb)^{\hspace{+0.03em} 2} + (\vA\ra - \vA\rb) \cdot (\vR\ra - \vR\rb)\right]
\end{eqnarray*}}
\vspace{-0.60em}
\medskip
\par And the potential energy $\mU\rab$ of particles A and B is:
\vspace{+0.30em}
{\fontsize{8}{8}\selectfont\begin{eqnarray*}
\Delta \hspace{+0.24em} \mU\rab = - \frac{1}{2} \hspace{+0.24em} \mM\ra\mM\rb \left[ \hspace{+0.12em} 2 \hspace{-0.12em} \int_1^{\hspace{+0.09em} 2} \hspace{-0.12em} \left(\frac{\vF\ra}{\mM\ra} - \frac{\vF\rb}{\mM\rb}\right) \cdot d(\vR\ra - \vR\rb) + \hspace{+0.03em} \Delta \left(\frac{\vF\ra}{\mM\ra} - \frac{\vF\rb}{\mM\rb}\right) \cdot (\vR\ra - \vR\rb) \hspace{+0.12em} \right]
\end{eqnarray*}}

\end{document}

