
\documentclass[10pt]{article}
%\documentclass[a4paper,10pt]{article}
%\documentclass[letterpaper,10pt]{article}

\usepackage[dvips]{geometry}
\geometry{papersize={127.0mm,128.0mm}}
\geometry{totalwidth=106.6mm,totalheight=92.0mm}

\usepackage[english]{babel}
\usepackage{mathptmx}

\usepackage{hyperref}
\hypersetup{colorlinks=true,linkcolor=black}
\hypersetup{bookmarksnumbered=true,pdfstartview=FitH,pdfpagemode=UseNone}
\hypersetup{pdftitle={Principle of Conservation}}
\hypersetup{pdfauthor={Alejandro A. Torassa}}

\setlength{\arraycolsep}{1.74pt}

\newcommand{\mM}{m}
\newcommand{\mW}{W}
\newcommand{\ra}{_a}
\newcommand{\ri}{_i}
\newcommand{\vR}{\mathbf{r}}
\newcommand{\vV}{\mathbf{v}}
\newcommand{\vD}{\mathbf{d}}
\newcommand{\vF}{\mathbf{F}}
\newcommand{\vM}{\mathbf{M}}
\newcommand{\med}{\raise.5ex\hbox{$\scriptstyle 1$}\kern-.15em/\kern-.15em\lower.25ex\hbox{$\scriptstyle 2$}}

\begin{document}

\begin{center}

{\LARGE Principle of Conservation}

\bigskip \medskip

Alejandro A. Torassa

\bigskip \medskip

\footnotesize

Creative Commons Attribution 3.0 License

(2013) Buenos Aires, Argentina

atorassa@gmail.com

\bigskip \smallskip

\small

{\bf Abstract}

\bigskip

\parbox{81mm}{In classical mechanics, this paper presents a new principle of conservation for frontal elastic collisions, which can be applied in any inertial reference frame.}

\end{center}

\normalsize

\vspace{-0.30em}

{\centering\subsubsection*{Principle of Conservation}}

\vspace{+0.75em}

\par In an isolated system of $N$ particles, the new principle of conservation for frontal elastic collisions, is given by:
\begin{eqnarray*}
\sum^{\scriptscriptstyle N}_{\scriptscriptstyle i = 1} \hspace{+0.24em} \med \hspace{+0.24em} \mM\ri \hspace{+0.12em} (\vR\ri \times \vV\ri)^{\hspace{+0.03em} 2} \hspace{+0.06em} = \hspace{+0.09em} constant
\end{eqnarray*}
\noindent where $\mM\ri$ is the mass of the \textit{i}-th particle, $\vR\ri$ is the position of the \textit{i}-th particle, and $\vV\ri$ is the velocity of the \textit{i}-th particle.

\newpage

{\centering\subsubsection*{Appendix}}

\vspace{+0.75em}

{\centering\subsubsection*{Angular Work}}

\vspace{+0.75em}

\par The angular work $\mW\ra$ done by a constant moment $\vM\ra$ acting on a parti- \hbox {cle A}, is given by:
\vspace{-0.45em}
\begin{eqnarray*}
\mW\ra = \vM\ra \cdot (\vR\ra \times \vD\ra)
\end{eqnarray*}
\noindent where $\vR\ra$ is the position of particle A, $\vD\ra$ is the displacement vector of particle A, and $\vF\ra$ is the constant force acting on particle A $[\hspace{+0.12em} \vM\ra = (\vR\ra \times \vF\ra) \hspace{+0.12em}]$

\vspace{+1.50em}

{\centering\subsubsection*{Angular Kinetic Energy}}

\vspace{+0.75em}

\par The angular work done by the net moment acting on a particle A is equal to the variation of the angular kinetic energy of particle A.
\vspace{-0.15em}
\begin{eqnarray*}
\mW\ra = \hspace{+0.09em} \Delta \hspace{+0.24em} \med \hspace{+0.24em} \mM\ra \hspace{+0.12em} (\vR\ra \times \vV\ra)^{\hspace{+0.03em} 2}
\end{eqnarray*}
\par Therefore, if the net moment acting on particle A does no angular work then the angular kinetic energy of particle A remains constant.

\end{document}

