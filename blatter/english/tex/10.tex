
\documentclass[10pt]{article}
%\documentclass[a4paper,10pt]{article}
%\documentclass[letterpaper,10pt]{article}

\usepackage[dvips]{geometry}
\geometry{papersize={162.0mm,234.0mm}}
\geometry{totalwidth=141.0mm,totalheight=198.0mm}

\usepackage[english]{babel}
\usepackage{amsfonts}
\usepackage{amsmath,bm}

\usepackage{hyperref}
\hypersetup{colorlinks=true,linkcolor=black,bookmarksopen=true}
\hypersetup{bookmarksnumbered=true,pdfstartview=FitH,pdfpagemode=UseNone}
\hypersetup{pdftitle={A New Theory in Relational Mechanics ( I \& II )}}
\hypersetup{pdfauthor={Antonio A. Blatter}}

\setlength{\arraycolsep}{1.74pt}

\newcommand{\spb}{\hspace{+0.015em}}
\newcommand{\spc}{\hspace{+0.150em}}
\newcommand{\med}{\raise.5ex\hbox{$\scriptstyle 1$}\kern-.15em/\kern-.09em\lower.25ex\hbox{$\scriptstyle 2$}}

\begin{document}

\addcontentsline{toc}{section}{Paper I}

\begin{center}

{\LARGE A New Theory in Relational Mechanics}

\bigskip \medskip

{\large Antonio A. Blatter}

\bigskip \medskip

\small

Creative Commons Attribution 3.0 License

\smallskip

(2015) Buenos Aires, Argentina

\medskip

{\sc ( Paper I )}

\smallskip

\bigskip \medskip

\parbox{107.40mm}{In relational mechanics, a new theory is presented, which is invariant under transformations between inertial and non-inertial reference frames and which can be applied in any reference frame without introducing fictitious forces. In addition to the above, in this paper, we assume that all forces \hbox {always obey} Newton's third law.}

\end{center}

\normalsize

\vspace{-1.20em}

\par \bigskip {\centering\subsection*{Introduction}}\addcontentsline{toc}{subsection}{1. Introduction}

\par \bigskip\smallskip \noindent The new theory in relational mechanics presented in this paper is obtained starting from an auxiliary system of particles (called Universe) that is used to obtain kinematic magnitudes (such as universal position, universal velocity, etc.) that are invariant under transformations between inertial and non-inertial reference frames.

\par \bigskip \noindent The universal position ${\mathbf{r}}_i$, the universal velocity ${\mathbf{v}}_i$ and the universal acceleration ${\mathbf{a}}_{\hspace{+0.045em}i}$ of a \hbox {particle $i$} are given by:

\par \bigskip\smallskip ${\mathbf{r}}_i \,\doteq\, ({\vec{\mathit{r}}}_i - {\vec{\mathit{R}}})$

\par \bigskip ${\mathbf{v}}_i \,\doteq\, ({\vec{\mathit{v}}}_i - \hspace{-0.120em}{\vec{\mathit{V}}}) - {\vec{\omega}} \times ({\vec{\mathit{r}}}_i - {\vec{\mathit{R}}})$

\par \bigskip ${\mathbf{a}}_{\hspace{+0.045em}i} \,\doteq\, ({\vec{\mathit{a}}}_i - {\vec{\mathit{A}}}) - 2 \; {\vec{\omega}} \times ({\vec{\mathit{v}}}_i - \hspace{-0.120em}{\vec{\mathit{V}}}) + {\vec{\omega}} \times [ \, {\vec{\omega}} \times ({\vec{\mathit{r}}}_i - {\vec{\mathit{R}}}) \, ] - {\vec{\alpha}} \times ({\vec{\mathit{r}}}_i - {\vec{\mathit{R}}})$

\par \bigskip\smallskip \noindent $( \, {\mathbf{v}}_i \doteq d(\hspace{+0.090em}{\mathbf{r}}_i\hspace{+0.090em})/dt \, )$ and $( \, {\mathbf{a}}_{\hspace{+0.045em}i} \doteq d^2(\hspace{+0.090em}{\mathbf{r}}_i\hspace{+0.090em})/dt^2 \, )$ where ${\vec{\mathit{r}}}_i$ is the position vector of particle $i$, ${\vec{\mathit{R}}}$ is the position vector of the center of mass of the Universe, and ${\vec{\omega}}$ is the angular velocity vector of the Universe \hyperlink{p1a1}{(\hspace{+0.120em}see Appendix I\hspace{+0.120em})}

\par \bigskip \noindent A reference frame S is non-rotating if the angular velocity ${\vec{\omega}}$ of the Universe relative to S is equal to zero, and the reference frame S is also inertial if the acceleration ${\vec{\mathit{A}}}$ of the center of mass of the Universe relative to S is equal to zero.

\vspace{+1.20em}

\par {\centering\subsection*{The New Dynamics}}\addcontentsline{toc}{subsection}{2. The New Dynamics}

\par \bigskip\smallskip \noindent $[\,1\,]$ A force is always caused by the interaction between two or more particles.

\par \bigskip \noindent $[\,2\,]$ The net force ${\mathbf{F}}_i$ acting on a particle $i$ of mass $m_i$ produces a universal acceleration ${\mathbf{a}}_{\hspace{+0.045em}i}$ according to the following equation: $[ \, {\mathbf{F}}_i \,=\, m_i \, {\mathbf{a}}_{\hspace{+0.045em}i} \, ]$

\par \bigskip \noindent $[\,3\,]$ In this paper, we assume that all forces always obey Newton's third law in its weak form and in its strong form.

\newpage

\par \bigskip {\centering\subsection*{The Definitions}}\addcontentsline{toc}{subsection}{3. The Definitions}

\par \bigskip \noindent For a system of N particles, the following definitions are applicable:

\par \bigskip\bigskip \hspace{-2.40em} \begin{tabular}{lll}
Mass & \hspace{+0.00em} & ${\mathrm{M}} ~\doteq~ \sum_i^{\scriptscriptstyle{\mathrm{N}}} \, m_i$ \vspace{+0.99em} \\
\\
Position {\small CM} 1 & \hspace{+0.00em} & ${\vec{\mathit{R}}}_{cm} ~\doteq~ {\mathrm{M}}^{\scriptscriptstyle -1} \, \sum_i^{\scriptscriptstyle{\mathrm{N}}} \, m_i \, {\vec{\mathit{r}}_{i}}$ \vspace{+0.99em} \\
Velocity {\small CM} 1 & \hspace{+0.00em} & ${\vec{\mathit{V}}}_{cm} ~\doteq~ {\mathrm{M}}^{\scriptscriptstyle -1} \, \sum_i^{\scriptscriptstyle{\mathrm{N}}} \, m_i \, {\vec{\mathit{v}}_{i}}$ \vspace{+0.99em} \\
Acceleration {\small CM} 1 & \hspace{+0.00em} & ${\vec{\mathit{A}}}_{cm} ~\doteq~ {\mathrm{M}}^{\scriptscriptstyle -1} \, \sum_i^{\scriptscriptstyle{\mathrm{N}}} \, m_i \, {\vec{\mathit{a}}_{i}}$ \vspace{+0.99em} \\
\\
Position {\small CM} 2 & \hspace{+0.00em} & ${\mathbf{R}}_{cm} ~\doteq~ {\mathrm{M}}^{\scriptscriptstyle -1} \, \sum_i^{\scriptscriptstyle{\mathrm{N}}} \, m_i \, {\mathbf{r}}_{i}$ \vspace{+0.99em} \\
Velocity {\small CM} 2 & \hspace{+0.00em} & ${\mathbf{V}}_{cm} ~\doteq~ {\mathrm{M}}^{\scriptscriptstyle -1} \, \sum_i^{\scriptscriptstyle{\mathrm{N}}} \, m_i \, {\mathbf{v}}_{i}$ \vspace{+0.99em} \\
Acceleration {\small CM} 2 & \hspace{+0.00em} & ${\mathbf{A}}_{cm} ~\doteq~ {\mathrm{M}}^{\scriptscriptstyle -1} \, \sum_i^{\scriptscriptstyle{\mathrm{N}}} \, m_i \, {\mathbf{a}}_{\hspace{+0.045em}i}$ \vspace{+0.99em} \\
\\
Linear Momentum 1 & \hspace{+0.00em} & ${\mathbf{P}}_1 ~\doteq~ \sum_i^{\scriptscriptstyle{\mathrm{N}}} \, m_i \, {\mathbf{v}}_{i}$ \vspace{+0.99em} \\
Angular Momentum 1 & \hspace{+0.00em} & ${\mathbf{L}}_1 ~\doteq~ \sum_i^{\scriptscriptstyle{\mathrm{N}}} \, m_i \, \big [ \, {\mathbf{r}}_{i} \times {\mathbf{v}}_{i} \, \big ]$ \vspace{+0.99em} \\
Angular Momentum 2 & \hspace{+0.00em} & ${\mathbf{L}}_2 ~\doteq~ \sum_i^{\scriptscriptstyle{\mathrm{N}}} \, m_i \, \big [ \, ({\mathbf{r}}_{i} - {\mathbf{R}}_{cm}) \times ({\mathbf{v}}_{i} - {\mathbf{V}}_{cm}) \, \big ]$ \vspace{+0.99em} \\
\\
Work 1 & \hspace{+0.00em} & ${\mathrm{W}}_1 ~\doteq~ \sum_i^{\scriptscriptstyle{\mathrm{N}}} \int_{\scriptscriptstyle 1}^{\scriptscriptstyle 2} \, {\mathbf{F}}_i \cdot d{\mathbf{r}}_{i} \,=\, \Delta \, {\mathrm{K}}_1$ \vspace{+0.99em} \\
Kinetic Energy 1 & \hspace{+0.00em} & $\Delta \, {\mathrm{K}}_1 ~\doteq~ \sum_i^{\scriptscriptstyle{\mathrm{N}}} \Delta \, \med \; m_i \, ({\mathbf{v}}_{i})^2$ \vspace{+0.99em} \\
Potential Energy 1 & \hspace{+0.00em} & $\Delta \, {\mathrm{U}}_1 ~\doteq~ - \, \sum_i^{\scriptscriptstyle{\mathrm{N}}} \int_{\scriptscriptstyle 1}^{\scriptscriptstyle 2} \, {\mathbf{F}}_i \cdot d{\mathbf{r}}_{i}$ \vspace{+0.99em} \\
Mechanical Energy 1 & \hspace{+0.00em} & ${\mathrm{E}}_1 ~\doteq~ {\mathrm{K}}_1 + {\mathrm{U}}_1$ \vspace{+0.99em} \\
Lagrangian 1 & \hspace{+0.00em} & ${\mathrm{L}}_1 ~\doteq~ {\mathrm{K}}_1 - {\mathrm{U}}_1$ \vspace{+0.99em} \\
\\
Work 2 & \hspace{+0.00em} & ${\mathrm{W}}_2 ~\doteq~ \sum_i^{\scriptscriptstyle{\mathrm{N}}} \int_{\scriptscriptstyle 1}^{\scriptscriptstyle 2} \, {\mathbf{F}}_i \cdot d({\mathbf{r}}_{i} - {\mathbf{R}}_{cm}) \,=\, \Delta \, {\mathrm{K}}_2$ \vspace{+0.99em} \\
Kinetic Energy 2 & \hspace{+0.00em} & $\Delta \, {\mathrm{K}}_2 ~\doteq~ \sum_i^{\scriptscriptstyle{\mathrm{N}}} \Delta \, \med \; m_i \, ({\mathbf{v}}_{i} - {\mathbf{V}}_{cm})^2$ \vspace{+0.99em} \\
Potential Energy 2 & \hspace{+0.00em} & $\Delta \, {\mathrm{U}}_2 ~\doteq~ - \, \sum_i^{\scriptscriptstyle{\mathrm{N}}} \int_{\scriptscriptstyle 1}^{\scriptscriptstyle 2} \, {\mathbf{F}}_i \cdot d({\mathbf{r}}_{i} - {\mathbf{R}}_{cm})$ \vspace{+0.99em} \\
Mechanical Energy 2 & \hspace{+0.00em} & ${\mathrm{E}}_2 ~\doteq~ {\mathrm{K}}_2 + {\mathrm{U}}_2$ \vspace{+0.99em} \\
Lagrangian 2 & \hspace{+0.00em} & ${\mathrm{L}}_2 ~\doteq~ {\mathrm{K}}_2 - {\mathrm{U}}_2$
\end{tabular}

\newpage

\par \bigskip\bigskip \hspace{-2.40em} \begin{tabular}{lll}
Work 3 & \hspace{+0.33em} & ${\mathrm{W}}_3 ~\doteq~ \sum_i^{\scriptscriptstyle{\mathrm{N}}} \Delta \, \med \; {\mathbf{F}}_i \cdot {\mathbf{r}}_{i} \,=\, \Delta \, {\mathrm{K}}_3$ \vspace{+0.99em} \\
Kinetic Energy 3 & \hspace{+0.33em} & $\Delta \, {\mathrm{K}}_3 ~\doteq~ \sum_i^{\scriptscriptstyle{\mathrm{N}}} \Delta \, \med \; m_i \: {\mathbf{a}}_{\hspace{+0.045em}i} \cdot {\mathbf{r}}_{i}$ \vspace{+0.99em} \\
Potential Energy 3 & \hspace{+0.33em} & $\Delta \, {\mathrm{U}}_3 ~\doteq~ - \, \sum_i^{\scriptscriptstyle{\mathrm{N}}} \Delta \, \med \; {\mathbf{F}}_i \cdot {\mathbf{r}}_{i}$ \vspace{+0.99em} \\
Mechanical Energy 3 & \hspace{+0.33em} & ${\mathrm{E}}_3 ~\doteq~ {\mathrm{K}}_3 + {\mathrm{U}}_3$ \vspace{+0.99em} \\
\\
Work 4 & \hspace{+0.33em} & ${\mathrm{W}}_4 ~\doteq~ \sum_i^{\scriptscriptstyle{\mathrm{N}}} \Delta \, \med \; {\mathbf{F}}_i \cdot ({\mathbf{r}}_{i} - {\mathbf{R}}_{cm}) \,=\, \Delta \, {\mathrm{K}}_4$ \vspace{+0.99em} \\
Kinetic Energy 4 & \hspace{+0.33em} & $\Delta \, {\mathrm{K}}_4 ~\doteq~ \sum_i^{\scriptscriptstyle{\mathrm{N}}} \Delta \, \med \; m_i \, \big [ \, ({\mathbf{a}}_{\hspace{+0.045em}i} - {\mathbf{A}}_{cm}) \cdot ({\mathbf{r}}_{i} - {\mathbf{R}}_{cm}) \, \big ]$ \vspace{+0.99em} \\
Potential Energy 4 & \hspace{+0.33em} & $\Delta \, {\mathrm{U}}_4 ~\doteq~ - \, \sum_i^{\scriptscriptstyle{\mathrm{N}}} \Delta \, \med \; {\mathbf{F}}_i \cdot ({\mathbf{r}}_{i} - {\mathbf{R}}_{cm})$ \vspace{+0.99em} \\
Mechanical Energy 4 & \hspace{+0.33em} & ${\mathrm{E}}_4 ~\doteq~ {\mathrm{K}}_4 + {\mathrm{U}}_4$ \vspace{+0.99em} \\
\\
Work 5 & \hspace{+0.33em} & ${\mathrm{W}}_5 ~\doteq~ \sum_i^{\scriptscriptstyle{\mathrm{N}}} \, \big [ \int_{\scriptscriptstyle 1}^{\scriptscriptstyle 2} \, {\mathbf{F}}_i \cdot d({\vec{\mathit{r}}_{i}} - {\vec{\mathit{R}}}) + \Delta \, \med \; {\mathbf{F}}_i \cdot ({\vec{\mathit{r}}_{i}} - {\vec{\mathit{R}}}) \, \big ] \,=\, \Delta \, {\mathrm{K}}_5$ \vspace{+0.99em} \\
Kinetic Energy 5 & \hspace{+0.33em} & $\Delta \, {\mathrm{K}}_5 ~\doteq~ \sum_i^{\scriptscriptstyle{\mathrm{N}}} \Delta \, \med \; m_i \, \big [ \, ({\vec{\mathit{v}}_{i}} - {\vec{\mathit{V}}})^2 + ({\vec{\mathit{a}}_{i}} - {\vec{\mathit{A}}}) \cdot ({\vec{\mathit{r}}_{i}} - {\vec{\mathit{R}}}) \, \big ]$ \vspace{+0.99em} \\
Potential Energy 5 & \hspace{+0.33em} & $\Delta \, {\mathrm{U}}_5 ~\doteq~ - \, \sum_i^{\scriptscriptstyle{\mathrm{N}}} \, \big [ \int_{\scriptscriptstyle 1}^{\scriptscriptstyle 2} \, {\mathbf{F}}_i \cdot d({\vec{\mathit{r}}_{i}} - {\vec{\mathit{R}}}) + \Delta \, \med \; {\mathbf{F}}_i \cdot ({\vec{\mathit{r}}_{i}} - {\vec{\mathit{R}}}) \, \big ]$ \vspace{+0.99em} \\
Mechanical Energy 5 & \hspace{+0.33em} & ${\mathrm{E}}_5 ~\doteq~ {\mathrm{K}}_5 + {\mathrm{U}}_5$ \vspace{+0.99em} \\
\\
Work 6 & \hspace{+0.33em} & ${\mathrm{W}}_6 ~\doteq~ \sum_i^{\scriptscriptstyle{\mathrm{N}}} \, \big [ \int_{\scriptscriptstyle 1}^{\scriptscriptstyle 2} \, {\mathbf{F}}_i \cdot d({\vec{\mathit{r}}_{i}} - {\vec{\mathit{R}}}_{cm}) + \Delta \, \med \; {\mathbf{F}}_i \cdot ({\vec{\mathit{r}}_{i}} - {\vec{\mathit{R}}}_{cm}) \, \big ] \,=\, \Delta \, {\mathrm{K}}_6$ \vspace{+0.99em} \\
Kinetic Energy 6 & \hspace{+0.33em} & $\Delta \, {\mathrm{K}}_6 ~\doteq~ \sum_i^{\scriptscriptstyle{\mathrm{N}}} \Delta \, \med \; m_i \, \big [ \, ({\vec{\mathit{v}}_{i}} - {\vec{\mathit{V}}}_{cm})^2 + ({\vec{\mathit{a}}_{i}} - {\vec{\mathit{A}}}_{cm}) \cdot ({\vec{\mathit{r}}_{i}} - {\vec{\mathit{R}}}_{cm}) \, \big ]$ \vspace{+0.99em} \\
Potential Energy 6 & \hspace{+0.33em} & $\Delta \, {\mathrm{U}}_6 ~\doteq~ - \, \sum_i^{\scriptscriptstyle{\mathrm{N}}} \, \big [ \int_{\scriptscriptstyle 1}^{\scriptscriptstyle 2} \, {\mathbf{F}}_i \cdot d({\vec{\mathit{r}}_{i}} - {\vec{\mathit{R}}}_{cm}) + \Delta \, \med \; {\mathbf{F}}_i \cdot ({\vec{\mathit{r}}_{i}} - {\vec{\mathit{R}}}_{cm}) \, \big ]$ \vspace{+0.99em} \\
Mechanical Energy 6 & \hspace{+0.33em} & ${\mathrm{E}}_6 ~\doteq~ {\mathrm{K}}_6 + {\mathrm{U}}_6$
\end{tabular}

\vspace{+0.60em}

\par {\centering\subsection*{The Relations}}\addcontentsline{toc}{subsection}{4. The Relations}

\par \bigskip\smallskip \noindent From the above definitions, the following relations can be obtained \hyperlink{p1a2}{(\hspace{+0.120em}see Appendix II\hspace{+0.120em})}

\vspace{+1.80em}

\par \noindent ${\mathrm{K}}_1 ~=~ {\mathrm{K}}_2 + \med \; {\mathrm{M}} \: {\mathbf{V}}_{cm}^{\hspace{+0.045em}2}$
\vspace{+0.99em}
\par \noindent ${\mathrm{K}}_3 ~=~ {\mathrm{K}}_4 + \med \; {\mathrm{M}} \: {\mathbf{A}}_{cm} \cdot {\mathbf{R}}_{cm}$
\vspace{+0.99em}
\par \noindent ${\mathrm{K}}_5 ~=~ {\mathrm{K}}_6 + \med \; {\mathrm{M}} \: \big [ \, ({\vec{\mathit{V}}}_{cm} - \hspace{-0.120em}{\vec{\mathit{V}}})^2 + ({\vec{\mathit{A}}}_{cm} - {\vec{\mathit{A}}}) \cdot ({\vec{\mathit{R}}}_{cm} - {\vec{\mathit{R}}}) \, \big ]$
\vspace{+0.99em}
\par \noindent ${\mathrm{K}}_5 ~=~ {\mathrm{K}}_1 + {\mathrm{K}}_3$ $\hspace{+0.540em} \& \hspace{+0.540em}$ ${\mathrm{U}}_5 ~=~ {\mathrm{U}}_1 \hspace{+0.027em}+\hspace{+0.027em} {\mathrm{U}}_3$ $\hspace{+0.630em} \& \hspace{+0.630em}$ ${\mathrm{E}}_5 ~=~ {\mathrm{E}}_1 + {\mathrm{E}}_3$
\vspace{+0.99em}
\par \noindent ${\mathrm{K}}_6 ~=~ {\mathrm{K}}_2 + {\mathrm{K}}_4$ $\hspace{+0.540em} \& \hspace{+0.540em}$ ${\mathrm{U}}_6 ~=~ {\mathrm{U}}_2 \hspace{+0.027em}+\hspace{+0.027em} {\mathrm{U}}_4$ $\hspace{+0.630em} \& \hspace{+0.630em}$ ${\mathrm{E}}_6 ~=~ {\mathrm{E}}_2 + {\mathrm{E}}_4$

\newpage

\par {\centering\subsection*{The Principles}}\addcontentsline{toc}{subsection}{5. The Principles}

\par \bigskip\smallskip \noindent The linear momentum $[ \, {\mathbf{P}}_1 \, ]$ of an isolated system of N particles remains constant if the internal forces obey Newton's third law in its weak form.

\par \bigskip\medskip ${\mathbf{P}}_1 ~=~ {\mathrm{constant}} \hspace{+2.88em} \big [ \; d({\mathbf{P}}_1)/dt ~=~ \sum_i^{\scriptscriptstyle{\mathrm{N}}} \, m_i \, {\mathbf{a}}_{\hspace{+0.045em}i} ~=~ \sum_i^{\scriptscriptstyle{\mathrm{N}}} \, {\mathbf{F}}_i ~=~ 0 \; \big ]$

\par \bigskip\medskip \noindent The angular momentum $[ \, {\mathbf{L}}_1 \, ]$ of an isolated system of N particles remains constant if the internal forces obey Newton's third law in its strong form.

\par \bigskip\medskip ${\mathbf{L}}_1 ~=~ {\mathrm{constant}} \hspace{+2.97em} \big [ \; d({\mathbf{L}}_1)/dt ~=~ \sum_i^{\scriptscriptstyle{\mathrm{N}}} \, m_i \, \big [ \, {\mathbf{r}}_i \times {\mathbf{a}}_{\hspace{+0.045em}i} \, \big ]~=~ \sum_i^{\scriptscriptstyle{\mathrm{N}}} \, {\mathbf{r}}_i \times {\mathbf{F}}_i ~=~ 0 \; \big ]$

\par \bigskip\medskip \noindent The angular momentum $[ \, {\mathbf{L}}_2 \, ]$ of an isolated system of N particles remains constant if the internal forces obey Newton's third law in its strong form.

\par \bigskip\medskip ${\mathbf{L}}_2 ~=~ {\mathrm{constant}} \hspace{+2.97em} \big [ \; d({\mathbf{L}}_2)/dt ~=~ \sum_i^{\scriptscriptstyle{\mathrm{N}}} \, m_i \, \big [ \, ({\mathbf{r}}_i - {\mathbf{R}}_{cm}) \times ({\mathbf{a}}_{\hspace{+0.045em}i} - {\mathbf{A}}_{cm}) \, \big ] ~=~$

\par \bigskip $\hspace{+10.44em} \sum_i^{\scriptscriptstyle{\mathrm{N}}} \, m_i \, \big [ \, ({\mathbf{r}}_i - {\mathbf{R}}_{cm}) \times {\mathbf{a}}_{\hspace{+0.045em}i} \, \big ] ~=~ \sum_i^{\scriptscriptstyle{\mathrm{N}}} \, ({\mathbf{r}}_i - {\mathbf{R}}_{cm}) \times {\mathbf{F}}_i ~=~ 0 \; \big ]$

\par \bigskip\medskip \noindent The mechanical energy $[ \, {\mathrm{E}}_1 \, ]$ and the mechanical energy $[ \, {\mathrm{E}}_2 \, ]$ of a system of N particles remain constant if the system is only subject to conservative forces.

\par \bigskip\medskip ${\mathrm{E}}_1 ~=~ {\mathrm{constant}} \hspace{+3.00em} \big [ \; \Delta \; {\mathrm{E}}_1 ~=~ \Delta \; {\mathrm{K}}_1 + \Delta \; {\mathrm{U}}_1 ~=~ 0 \; \big ]$

\par \bigskip ${\mathrm{E}}_2 ~=~ {\mathrm{constant}} \hspace{+3.00em} \big [ \; \Delta \; {\mathrm{E}}_2 ~=~ \Delta \; {\mathrm{K}}_2 + \Delta \; {\mathrm{U}}_2 ~=~ 0 \; \big ]$

\par \bigskip\medskip \noindent The mechanical energy $[ \, {\mathrm{E}}_3 \, ]$ and the mechanical energy $[ \, {\mathrm{E}}_4 \, ]$ of a system of N particles \hbox {are always zero} (\hspace{+0.180em}and therefore they always remain constant\hspace{+0.180em})

\par \bigskip\medskip ${\mathrm{E}}_3 ~=~ {\mathrm{constant}} \hspace{+3.00em} \big [ \; {\mathrm{E}}_3 ~=~ \sum_i^{\scriptscriptstyle{\mathrm{N}}} \, \med \; \big [ \, m_i \: {\mathbf{a}}_{\hspace{+0.045em}i} \cdot {\mathbf{r}}_{i} - {\mathbf{F}}_i \cdot {\mathbf{r}}_{i} \, \big ] ~=~ 0 \; \big ]$

\par \bigskip ${\mathrm{E}}_4 ~=~ {\mathrm{constant}} \hspace{+3.00em} \big [ \; {\mathrm{E}}_4 ~=~ \sum_i^{\scriptscriptstyle{\mathrm{N}}} \, \med \; \big [ \, m_i \: {\mathbf{a}}_{\hspace{+0.045em}i} \cdot ({\mathbf{r}}_{i} - {\mathbf{R}}_{cm}) - {\mathbf{F}}_i \cdot ({\mathbf{r}}_{i} - {\mathbf{R}}_{cm}) \, \big ] ~=~ 0 \; \big ]$

\par \bigskip $\hspace{+10.44em} \sum_i^{\scriptscriptstyle{\mathrm{N}}} \, \med \; m_i \, \big [ \, ({\mathbf{a}}_{\hspace{+0.045em}i} - {\mathbf{A}}_{cm}) \cdot ({\mathbf{r}}_{i} - {\mathbf{R}}_{cm}) \, \big ] \,=\, \sum_i^{\scriptscriptstyle{\mathrm{N}}} \, \med \; m_i \: {\mathbf{a}}_{\hspace{+0.045em}i} \cdot ({\mathbf{r}}_{i} - {\mathbf{R}}_{cm})$

\par \bigskip\medskip \noindent The mechanical energy $[ \, {\mathrm{E}}_5 \, ]$ and the mechanical energy $[ \, {\mathrm{E}}_6 \, ]$ of a system of N particles remain constant if the system is only subject to conservative forces.

\par \bigskip\medskip ${\mathrm{E}}_5 ~=~ {\mathrm{constant}} \hspace{+3.00em} \big [ \; \Delta \; {\mathrm{E}}_5 ~=~ \Delta \; {\mathrm{K}}_5 + \Delta \; {\mathrm{U}}_5 ~=~ 0 \; \big ]$

\par \bigskip ${\mathrm{E}}_6 ~=~ {\mathrm{constant}} \hspace{+3.00em} \big [ \; \Delta \; {\mathrm{E}}_6 ~=~ \Delta \; {\mathrm{K}}_6 + \Delta \; {\mathrm{U}}_6 ~=~ 0 \; \big ]$

\newpage

\par \bigskip {\centering\subsection*{Observations}}\addcontentsline{toc}{subsection}{6. Observations}

\par \bigskip\smallskip \noindent All equations of this paper can be applied in any inertial reference frame and also in any non-inertial reference frame.

\par \bigskip\smallskip \noindent Additionally, inertial reference frames and non-inertial reference frames must not introduce fictitious forces into ${\mathbf{F}}_i$.

\par \bigskip\smallskip \noindent In this paper, the magnitudes $[ \, {\mathit{m}},\spc {\mathbf{r}},\spc {\mathbf{v}},\spc {\mathbf{a}},\spc {\mathrm{M}},\spc {\mathbf{R}},\spc {\mathbf{V}}\hspace{-0.120em},\spc {\mathbf{A}},\spc {\mathbf{F}},\spc {\mathbf{P}}_1,\spc {\mathbf{L}}_1,\spc {\mathbf{L}}_2,\spc {\mathrm{W}}_1,\spc {\mathrm{K}}_1,\spc {\mathrm{U}}_1,\spc {\mathrm{E}}_1,\spc {\mathrm{L}}_1$, ${\mathrm{W}}_2,\spc {\mathrm{K}}_2,\spc {\mathrm{U}}_2,\spc {\mathrm{E}}_2,\spc {\mathrm{L}}_2,\spc {\mathrm{W}}_3,\spc {\mathrm{K}}_3,\spc {\mathrm{U}}_3,\spc {\mathrm{E}}_3,\spc {\mathrm{W}}_4,\spc {\mathrm{K}}_4,\spc {\mathrm{U}}_4,\spc {\mathrm{E}}_4,\spc {\mathrm{W}}_5,\spc {\mathrm{K}}_5,\spc {\mathrm{U}}_5,\spc {\mathrm{E}}_5,\spc {\mathrm{W}}_6,\spc {\mathrm{K}}_6,\spc {\mathrm{U}}_6$ and ${\mathrm{E}}_6 \, ]$ are invariant under transformations between inertial and non-inertial reference frames.

\par \bigskip\smallskip \noindent The mechanical energy ${\mathrm{E}}_3$ of a system of particles is always zero $[ \, {\mathrm{E}}_3 = {\mathrm{K}}_3 + {\mathrm{U}}_3 = 0 \, ]$

\par \bigskip\smallskip \noindent Therefore, the mechanical energy ${\mathrm{E}}_5$ of a system of particles is always equal to the mechanical energy ${\mathrm{E}}_1$ of the system of particles $[ \, {\mathrm{E}}_5 = {\mathrm{E}}_1 \, ]$

\par \bigskip\smallskip \noindent The mechanical energy ${\mathrm{E}}_4$ of a system of particles is always zero $[ \, {\mathrm{E}}_4 = {\mathrm{K}}_4 + {\mathrm{U}}_4 = 0 \, ]$

\par \bigskip\smallskip \noindent Therefore, the mechanical energy ${\mathrm{E}}_6$ of a system of particles is always equal to the mechanical energy ${\mathrm{E}}_2$ of the system of particles $[ \, {\mathrm{E}}_6 = {\mathrm{E}}_2 \, ]$

\par \bigskip\smallskip \noindent If the potential energy ${\mathrm{U}}_1$ of a system of particles is a homogeneous function of \hbox {degree ${\mathit{k}}$} \hbox {then the} potential energy ${\mathrm{U}}_3$ and the potential energy ${\mathrm{U}}_5$ of the system of particles are \hbox {given by}: $[ \, {\mathrm{U}}_3 = (\frac{{\mathit{k}}}{2}) \, {\mathrm{U}}_1 \, ]$ and $[ \, {\mathrm{U}}_5 = ({\scriptstyle 1 +} \frac{{\mathit{k}}}{2}) \, {\mathrm{U}}_1 \, ]$

\par \bigskip\smallskip \noindent If the potential energy ${\mathrm{U}}_2$ of a system of particles is a homogeneous function of \hbox {degree ${\mathit{k}}$} \hbox {then the} potential energy ${\mathrm{U}}_4$ and the potential energy ${\mathrm{U}}_6$ of the system of particles are \hbox {given by}: $[ \, {\mathrm{U}}_4 = (\frac{{\mathit{k}}}{2}) \, {\mathrm{U}}_2 \, ]$ and $[ \, {\mathrm{U}}_6 = ({\scriptstyle 1 +} \frac{{\mathit{k}}}{2}) \, {\mathrm{U}}_2 \, ]$

\par \bigskip\smallskip \noindent If the potential energy ${\mathrm{U}}_1$ of a system of particles is a homogeneous function of \hbox {degree ${\mathit{k}}$} and if the kinetic energy ${\mathrm{K}}_5$ of the system of particles is equal to zero, then we obtain: $[ \, {\mathrm{K}}_1 = - \, {\mathrm{K}}_3 = {\mathrm{U}}_3 = (\frac{{\mathit{k}}}{2}) \, {\mathrm{U}}_1 = (\frac{{\mathit{k}}}{2 + {\mathit{k}}}) \, {\mathrm{E}}_1 \, ]$

\par \bigskip\smallskip \noindent If the potential energy ${\mathrm{U}}_2$ of a system of particles is a homogeneous function of \hbox {degree ${\mathit{k}}$} and if the kinetic energy ${\mathrm{K}}_6$ of the system of particles is equal to zero, then we obtain: $[ \, {\mathrm{K}}_2 = - \, {\mathrm{K}}_4 = {\mathrm{U}}_4 = (\frac{{\mathit{k}}}{2}) \, {\mathrm{U}}_2 = (\frac{{\mathit{k}}}{2 + {\mathit{k}}}) \, {\mathrm{E}}_2 \, ]$

\par \bigskip\smallskip \noindent If the potential energy ${\mathrm{U}}_1$ of a system of particles is a homogeneous function of \hbox {degree ${\mathit{k}}$} \hbox {and if} the average kinetic energy $\langle {\mathrm{K}}_5 \rangle$ of the system of particles is equal to zero, then we obtain: $[ \, \langle {\mathrm{K}}_1 \rangle = - \, \langle {\mathrm{K}}_3 \rangle = \langle {\mathrm{U}}_3 \rangle = (\frac{{\mathit{k}}}{2}) \, \langle {\mathrm{U}}_1 \rangle = (\frac{{\mathit{k}}}{2 + {\mathit{k}}}) \, \langle {\mathrm{E}}_1 \rangle \, ]$

\par \bigskip\smallskip \noindent If the potential energy ${\mathrm{U}}_2$ of a system of particles is a homogeneous function of \hbox {degree ${\mathit{k}}$} \hbox {and if} the average kinetic energy $\langle {\mathrm{K}}_6 \rangle$ of the system of particles is equal to zero, then we obtain: $[ \, \langle {\mathrm{K}}_2 \rangle = - \, \langle {\mathrm{K}}_4 \rangle = \langle {\mathrm{U}}_4 \rangle = (\frac{{\mathit{k}}}{2}) \, \langle {\mathrm{U}}_2 \rangle = (\frac{{\mathit{k}}}{2 + {\mathit{k}}}) \, \langle {\mathrm{E}}_2 \rangle \, ]$

\newpage

\par \bigskip\smallskip \noindent The average kinetic energy $\langle {\mathrm{K}}_5 \rangle$ and the average kinetic energy $\langle {\mathrm{K}}_6 \rangle$ of a system of particles with bounded motion ( in $\langle {\mathrm{K}}_5 \rangle$ relative to ${\vec{\mathit{R}}}$ \hspace{+0.090em}and\hspace{+0.090em} in $\langle {\mathrm{K}}_6 \rangle$ relative to ${\vec{\mathit{R}}}_{cm}$ ) are always zero.

\par \bigskip\smallskip \noindent The kinetic energy ${\mathrm{K}}_5$ and the kinetic energy ${\mathrm{K}}_6$ of a system of N particles can also \hbox {be expressed} as follows : $[ \; {\mathrm{K}}_5 = \sum_i^{\scriptscriptstyle{\mathrm{N}}} \, \med \; m_i \, ( \, {\dot{\mathit{r}}}_{i} \, {\dot{\mathit{r}}}_{i} + {\ddot{\mathit{r}}}_{i} \, {\mathit{r}}_{i} \, ) \; ]$ where ${\mathit{r}}_{i} \doteq | \, {\vec{\mathit{r}}}_{i} - {\vec{\mathit{R}}} \, |$ and \hbox {$[ \; {\mathrm{K}}_6 = \sum_{j{\scriptscriptstyle >}i}^{\scriptscriptstyle{\mathrm{N}}} \, \med \; m_i \, m_j \, {\mathrm{M}}^{\scriptscriptstyle -1} ( \, {\dot{\mathit{r}}}_{\hspace{+0.060em}ij} \, {\dot{\mathit{r}}}_{\hspace{+0.060em}ij} + {\ddot{\mathit{r}}}_{\hspace{+0.060em}ij} \, {\mathit{r}}_{\hspace{+0.060em}ij} \, ) \; ]$} where ${\mathit{r}}_{\hspace{+0.060em}ij} \,\doteq\, | \: {\vec{\mathit{r}}}_{i} - {\vec{\mathit{r}}}_{j} \: |$

\par \bigskip\smallskip \noindent The kinetic energy ${\mathrm{K}}_5$ and the kinetic energy ${\mathrm{K}}_6$ of a system of N particles can also \hbox {be expressed} as follows : $[ \; {\mathrm{K}}_5 = \sum_i^{\scriptscriptstyle{\mathrm{N}}} \, \med \; m_i \, ( \, {\ddot{\tau}}_{\hspace{+0.045em}i} \, ) \; ]$ where ${\tau}_{i} \doteq \med \; ({\vec{\mathit{r}}}_{i} - {\vec{\mathit{R}}}) \cdot ({\vec{\mathit{r}}}_{i} - {\vec{\mathit{R}}})$ and \hbox {$[ \; {\mathrm{K}}_6 = \sum_{j{\scriptscriptstyle >}i}^{\scriptscriptstyle{\mathrm{N}}} \, \med \; m_i \, m_j \, {\mathrm{M}}^{\scriptscriptstyle -1} ( \, {\ddot{\tau}}_{\hspace{+0.060em}ij} \, ) \; ]$} where ${\tau}_{\hspace{+0.060em}ij} \doteq \med \; ({\vec{\mathit{r}}}_{i} - {\vec{\mathit{r}}}_{j}) \cdot ({\vec{\mathit{r}}}_{i} - {\vec{\mathit{r}}}_{j})$

\par \bigskip\smallskip \noindent The kinetic energy ${\mathrm{K}}_6$ is the only kinetic energy that can be expressed without the necessity of introducing any magnitude that is related to the Universe $[ \:\: $such as\hspace{+0.060em}: ${\mathbf{r}},\: {\mathbf{v}},\: {\mathbf{a}},\: {\vec{\omega}},\: {\vec{\mathit{R}}}$, etc.$ \:\: ]$

\par \bigskip\smallskip \noindent In an isolated system of particles, the potential energy ${\mathrm{U}}_2$ is equal to the potential energy ${\mathrm{U}}_1$ if the internal forces obey Newton's third law in its weak form $[ \, {\mathrm{U}}_2 = {\mathrm{U}}_1 \, ]$

\par \bigskip\smallskip \noindent In an isolated system of particles, the potential energy ${\mathrm{U}}_4$ is equal to the potential energy ${\mathrm{U}}_3$ if the internal forces obey Newton's third law in its weak form $[ \, {\mathrm{U}}_4 = {\mathrm{U}}_3 \, ]$

\par \bigskip\smallskip \noindent In an isolated system of particles, the potential energy ${\mathrm{U}}_6$ is equal to the potential energy ${\mathrm{U}}_5$ if the internal forces obey Newton's third law in its weak form $[ \, {\mathrm{U}}_6 = {\mathrm{U}}_5 \, ]$

\par \bigskip\smallskip \noindent A reference frame S is non-rotating if the angular velocity ${\vec{\omega}}$ of the Universe relative to S is equal to zero, and the reference frame S is also inertial if the acceleration ${\vec{\mathit{A}}}$ of the center of mass of the Universe relative to S is equal to zero.

\par \bigskip\smallskip \noindent If the origin of a non-rotating reference frame S $[ \, {\vec{\omega}} = 0 \, ]$ always coincides with the center of mass of the Universe $[ \, {\vec{\mathit{R}}} = {\vec{\mathit{V}}} = {\vec{\mathit{A}}} = 0 \, ]$ then relative to S: $[ \, {\mathbf{r}}_i = {\vec{\mathit{r}}}_i$, ${\mathbf{v}}_i = {\vec{\mathit{v}}}_i$ and ${\mathbf{a}}_{\hspace{+0.045em}i} = {\vec{\mathit{a}}}_i \, ]$ Therefore, it is easy to see that always: $[ \, {\mathbf{v}}_i = d(\hspace{+0.090em}{\mathbf{r}}_i\hspace{+0.090em})/dt$ $\;$and$\;$ ${\mathbf{a}}_{\hspace{+0.045em}i} = d^2(\hspace{+0.090em}{\mathbf{r}}_i\hspace{+0.090em})/dt^2 \, ]$

\par \bigskip\smallskip \noindent This paper does not contradict Newton's first and second laws since these two laws are valid in all inertial reference frames. The equation $[ \: {\mathbf{F}}_i \,=\, m_i \, {\mathbf{a}}_{\hspace{+0.045em}i} \: ]$ is a simple reformulation of Newton's second law.

\par \bigskip\smallskip \noindent In this paper, the equation $[ \: {\mathbf{F}}_i = m_i \, {\mathbf{a}}_{\hspace{+0.045em}i} \: ]$ would be false in all reference frames (\hspace{+0.180em}inertial \hbox {or non-inertial\hspace{+0.180em})} if a new force were always disobeyed Newton's third law in its strong form or in its weak form.

\vspace{-0.90em}

\par \bigskip {\centering\subsection*{Bibliography}}\addcontentsline{toc}{subsection}{7. Bibliography}

\par \bigskip\smallskip \noindent \textbf{A. Einstein}, Relativity: The Special and General Theory.

\par \bigskip\smallskip \noindent \textbf{A. Torassa}, A Reformulation of Classical Mechanics.

\par \bigskip\smallskip \noindent \textbf{E. Mach}, The Science of Mechanics.

\newpage

\par \bigskip {\centering\subsection*{Appendix I}}

\par \medskip {\centering\subsubsection*{The Universe}}\addcontentsline{toc}{subsection}{Appendix I : The Universe}\hypertarget{p1a1}{}

\par \bigskip \noindent The Universe is a system that contains all particles, that is always free of external forces, and that all internal forces always obey Newton's third law in its weak form and in its \hbox {strong form}.

\par \bigskip \noindent The position ${\vec{\mathit{R}}}$, the velocity ${\vec{\mathit{V}}}$ and the acceleration ${\vec{\mathit{A}}}$ of the center of mass of the Universe relative to a reference frame S (and the angular velocity ${\vec{\omega}}$ and the angular acceleration ${\vec{\alpha}}$ \hbox {of the Universe} relative to the reference frame S) are given by:

\par \bigskip\smallskip \hspace{-2.40em} \begin{tabular}{l}
${\mathrm{M}} ~\doteq~ \sum_i^{\scriptscriptstyle{\mathit{All}}} \, m_i$ \vspace{+1.20em} \\
${\vec{\mathit{R}}} ~\doteq~ {\mathrm{M}}^{\scriptscriptstyle -1} \, \sum_i^{\scriptscriptstyle{\mathit{All}}} \, m_i \, {\vec{\mathit{r}}_{i}}$ \vspace{+1.20em} \\
${\vec{\mathit{V}}} ~\doteq~ {\mathrm{M}}^{\scriptscriptstyle -1} \, \sum_i^{\scriptscriptstyle{\mathit{All}}} \, m_i \, {\vec{\mathit{v}}_{i}}$ \vspace{+1.20em} \\
${\vec{\mathit{A}}} ~\doteq~ {\mathrm{M}}^{\scriptscriptstyle -1} \, \sum_i^{\scriptscriptstyle{\mathit{All}}} \, m_i \, {\vec{\mathit{a}}_{i}}$ \vspace{+1.20em} \\
${\vec{\omega}} ~\doteq~ {\mathit{I}}^{\scriptscriptstyle -1}{\vphantom{\sum_1^2}}^{\hspace{-1.500em}\leftrightarrow}\hspace{+0.600em} \cdot {\vec{\mathit{L}}}$ \vspace{+1.20em} \\
${\vec{\alpha}} ~\doteq~ d({\vec{\omega}})/dt$ \vspace{+1.20em} \\
${\mathit{I}}{\vphantom{\sum_1^2}}^{\hspace{-0.555em}\leftrightarrow}\hspace{-0.210em} ~\doteq~ \sum_i^{\scriptscriptstyle{\mathit{All}}} \, m_i \, [ \, |\hspace{+0.090em}{\vec{\mathit{r}}_{i}} - {\vec{\mathit{R}}}\,|^2 \hspace{+0.309em} {\mathrm{1}}{\vphantom{\sum_1^2}}^{\hspace{-0.639em}\leftrightarrow}\hspace{-0.129em} - ({\vec{\mathit{r}}_{i}} - {\vec{\mathit{R}}}) \otimes ({\vec{\mathit{r}}_{i}} - {\vec{\mathit{R}}}) \, ]$ \vspace{+1.20em} \\
${\vec{\mathit{L}}} ~\doteq~ \sum_i^{\scriptscriptstyle{\mathit{All}}} \, m_i \, ({\vec{\mathit{r}}_{i}} - {\vec{\mathit{R}}}) \times ({\vec{\mathit{v}}_{i}} - \hspace{-0.120em}{\vec{\mathit{V}}})$
\end{tabular}

\par \bigskip \noindent where ${\mathrm{M}}$ is the mass of the Universe, ${\mathit{I}}{\vphantom{\sum_1^2}}^{\hspace{-0.555em}\leftrightarrow}\hspace{-0.300em}$ is the inertia tensor of the Universe (relative \hbox {to ${\vec{\mathit{R}}}$)} and ${\vec{\mathit{L}}}$ is the angular momentum of the Universe relative to the reference frame S.

\vspace{+1.50em}

\par {\centering\subsubsection*{The Transformations}}\addcontentsline{toc}{subsection}{Appendix I : The Transformations}

\par \bigskip\medskip \hspace{-1.80em} $({\vec{\mathit{r}}}_i - {\vec{\mathit{R}}}) ~\doteq~ {\mathbf{r}}_i ~=~ {\mathbf{r}}_i\hspace{-0.300em}'$

\par \bigskip \hspace{-1.80em} $({\vec{\mathit{r}}}_i\hspace{-0.150em}' - {\vec{\mathit{R}}}\hspace{+0.015em}') ~\doteq~ {\mathbf{r}}_i\hspace{-0.300em}' ~=~ {\mathbf{r}}_i$

\par \bigskip \hspace{-1.80em} $({\vec{\mathit{v}}}_i - \hspace{-0.120em}{\vec{\mathit{V}}}) - {\vec{\omega}} \times ({\vec{\mathit{r}}}_i - {\vec{\mathit{R}}}) ~\doteq~ {\mathbf{v}}_i ~=~ {\mathbf{v}}_i\hspace{-0.300em}'$

\par \bigskip \hspace{-1.80em} $({\vec{\mathit{v}}}_i\hspace{-0.150em}' - \hspace{-0.120em}{\vec{\mathit{V}}}\hspace{-0.045em}') - {\vec{\omega}}\hspace{+0.060em}' \times ({\vec{\mathit{r}}}_i\hspace{-0.150em}' - {\vec{\mathit{R}}}\hspace{+0.015em}') ~\doteq~ {\mathbf{v}}_i\hspace{-0.300em}' ~=~ {\mathbf{v}}_i$

\par \bigskip \hspace{-1.80em} $({\vec{\mathit{a}}}_i - {\vec{\mathit{A}}}) - 2 \; {\vec{\omega}} \times ({\vec{\mathit{v}}}_i - \hspace{-0.120em}{\vec{\mathit{V}}}) + {\vec{\omega}} \times [ \, {\vec{\omega}} \times ({\vec{\mathit{r}}}_i - {\vec{\mathit{R}}}) \, ] - {\vec{\alpha}} \times ({\vec{\mathit{r}}}_i - {\vec{\mathit{R}}}) ~\doteq~ {\mathbf{a}}_{\hspace{+0.045em}i} ~=~ {\mathbf{a}}_{\hspace{+0.045em}i}\hspace{-0.360em}'$

\par \bigskip \hspace{-1.80em} $({\vec{\mathit{a}}}_i\hspace{-0.150em}' - {\vec{\mathit{A}}}\hspace{-0.045em}') - 2 \; {\vec{\omega}}\hspace{+0.060em}' \times ({\vec{\mathit{v}}}_i\hspace{-0.150em}' - \hspace{-0.120em}{\vec{\mathit{V}}}\hspace{-0.045em}') + {\vec{\omega}}\hspace{+0.060em}' \times [ \, {\vec{\omega}}\hspace{+0.060em}' \times ({\vec{\mathit{r}}}_i\hspace{-0.150em}' - {\vec{\mathit{R}}}\hspace{+0.015em}') \, ] - {\vec{\alpha}}\hspace{+0.060em}' \times ({\vec{\mathit{r}}}_i\hspace{-0.150em}' - {\vec{\mathit{R}}}\hspace{+0.015em}') ~\doteq~ {\mathbf{a}}_{\hspace{+0.045em}i}\hspace{-0.360em}' ~=~ {\mathbf{a}}_{\hspace{+0.045em}i}$

\newpage

\par \bigskip {\centering\subsection*{Appendix II}}

\par \medskip {\centering\subsubsection*{The Relations}}\addcontentsline{toc}{subsection}{Appendix II : The Relations}\hypertarget{p1a2}{}

\par \bigskip \noindent In a system of particles, these relations can be obtained ( The magnitudes ${\mathbf{R}}_{cm}$, ${\mathbf{V}}_{cm}$, ${\mathbf{A}}_{cm}$, ${\vec{\mathit{R}}}_{cm}$, ${\vec{\mathit{V}}}_{cm}$ and ${\vec{\mathit{A}}}_{cm}$ can be replaced by the magnitudes ${\mathbf{R}}$, ${\mathbf{V}}$, ${\mathbf{A}}$, ${\vec{\mathit{R}}}$, ${\vec{\mathit{V}}}$ and ${\vec{\mathit{A}}}$, or by the magnitudes ${\mathbf{r}}_j$, ${\mathbf{v}}_j$, ${\mathbf{a}}_j$, ${\vec{\mathit{r}}}_j$, ${\vec{\mathit{v}}}_j$ and ${\vec{\mathit{a}}}_j$,{\vphantom{\LARGE t}} respectively. On the other hand, $\: {\mathbf{R}} \:=\: {\mathbf{V}} \:=\: {\mathbf{A}} \:=\: 0 \;$)

\par \bigskip\medskip \noindent ${\mathbf{r}}_i \,\doteq\, ({\vec{\mathit{r}}}_i - {\vec{\mathit{R}}})$

\par \bigskip\smallskip \noindent ${\mathbf{R}}_{cm} \,\doteq\, ({\vec{\mathit{R}}}_{cm} - {\vec{\mathit{R}}})$

\par \bigskip\smallskip \noindent $\longrightarrow \hspace{+0.90em} ({\mathbf{r}}_i - {\mathbf{R}}_{cm}) \,=\, ({\vec{\mathit{r}}}_i - {\vec{\mathit{R}}}_{cm})$

\par \bigskip\smallskip \noindent ${\mathbf{v}}_i \,\doteq\, ({\vec{\mathit{v}}}_i - \hspace{-0.120em}{\vec{\mathit{V}}}) - {\vec{\omega}} \times ({\vec{\mathit{r}}}_i - {\vec{\mathit{R}}})$

\par \bigskip\smallskip \noindent ${\mathbf{V}}_{cm} \,\doteq\, ({\vec{\mathit{V}}}_{cm} - \hspace{-0.120em}{\vec{\mathit{V}}}) - {\vec{\omega}} \times ({\vec{\mathit{R}}}_{cm} - {\vec{\mathit{R}}})$

\par \bigskip\smallskip \noindent $\longrightarrow \hspace{+0.90em} ({\mathbf{v}}_i - {\mathbf{V}}_{cm}) \,=\, ({\vec{\mathit{v}}}_i - \hspace{-0.120em}{\vec{\mathit{V}}}_{cm}) - {\vec{\omega}} \times ({\vec{\mathit{r}}}_i - {\vec{\mathit{R}}}_{cm})$

\par \bigskip\smallskip \noindent $({\mathbf{v}}_i - {\mathbf{V}}_{cm}) \cdot ({\mathbf{v}}_i - {\mathbf{V}}_{cm}) \,=\, \big [ \, ({\vec{\mathit{v}}}_i - \hspace{-0.120em}{\vec{\mathit{V}}}_{cm}) - {\vec{\omega}} \times ({\vec{\mathit{r}}}_i - {\vec{\mathit{R}}}_{cm}) \, \big ] \cdot \big [ \, ({\vec{\mathit{v}}}_i - \hspace{-0.120em}{\vec{\mathit{V}}}_{cm}) - {\vec{\omega}} \times ({\vec{\mathit{r}}}_i - {\vec{\mathit{R}}}_{cm}) \, \big ] \,=$

\par \bigskip\smallskip \noindent $({\vec{\mathit{v}}}_i - \hspace{-0.120em}{\vec{\mathit{V}}}_{cm}) \cdot ({\vec{\mathit{v}}}_i - \hspace{-0.120em}{\vec{\mathit{V}}}_{cm}) - 2 \, ({\vec{\mathit{v}}}_i - \hspace{-0.120em}{\vec{\mathit{V}}}_{cm}) \cdot \big [ \, {\vec{\omega}} \times ({\vec{\mathit{r}}}_i - {\vec{\mathit{R}}}_{cm}) \, \big ] + \big [ \, {\vec{\omega}} \times ({\vec{\mathit{r}}}_i - {\vec{\mathit{R}}}_{cm}) \, \big ] \cdot \big [ \, {\vec{\omega}} \times ({\vec{\mathit{r}}}_i - {\vec{\mathit{R}}}_{cm}) \, \big ] \,=$

\par \bigskip\smallskip \noindent $({\vec{\mathit{v}}}_i - \hspace{-0.120em}{\vec{\mathit{V}}}_{cm}) \cdot ({\vec{\mathit{v}}}_i - \hspace{-0.120em}{\vec{\mathit{V}}}_{cm}) + 2 \, ({\vec{\mathit{r}}}_i - {\vec{\mathit{R}}}_{cm}) \cdot \big [ \, {\vec{\omega}} \times ({\vec{\mathit{v}}}_i - \hspace{-0.120em}{\vec{\mathit{V}}}_{cm}) \, \big ] + \big [ \, {\vec{\omega}} \times ({\vec{\mathit{r}}}_i - {\vec{\mathit{R}}}_{cm}) \, \big ] \cdot \big [ \, {\vec{\omega}} \times ({\vec{\mathit{r}}}_i - {\vec{\mathit{R}}}_{cm}) \, \big ] \,=$

\par \bigskip\smallskip \noindent $({\vec{\mathit{v}}}_i - \hspace{-0.120em}{\vec{\mathit{V}}}_{cm}) \cdot ({\vec{\mathit{v}}}_i - \hspace{-0.120em}{\vec{\mathit{V}}}_{cm}) + \big [ \, 2 \; {\vec{\omega}} \times ({\vec{\mathit{v}}}_i - \hspace{-0.120em}{\vec{\mathit{V}}}_{cm}) \, \big ] \cdot ({\vec{\mathit{r}}}_i - {\vec{\mathit{R}}}_{cm}) + \big [ \, {\vec{\omega}} \times ({\vec{\mathit{r}}}_i - {\vec{\mathit{R}}}_{cm}) \, \big ] \cdot \big [ \, {\vec{\omega}} \times ({\vec{\mathit{r}}}_i - {\vec{\mathit{R}}}_{cm}) \, \big ] \,=$

\par \bigskip\smallskip \noindent $({\vec{\mathit{v}}}_i - \hspace{-0.120em}{\vec{\mathit{V}}}_{cm})^2 + \big [ \, 2 \; {\vec{\omega}} \times ({\vec{\mathit{v}}}_i - \hspace{-0.120em}{\vec{\mathit{V}}}_{cm}) \, \big ] \cdot ({\vec{\mathit{r}}}_i - {\vec{\mathit{R}}}_{cm}) + \big [ \, {\vec{\omega}} \times ({\vec{\mathit{r}}}_i - {\vec{\mathit{R}}}_{cm}) \, \big ]^2$

\par \bigskip\smallskip \noindent $({\mathbf{a}}_{\hspace{+0.045em}i} \,-\, {\mathbf{A}}_{cm}) \cdot ({\mathbf{r}}_i \,-\, {\mathbf{R}}_{cm}) \;=\; \big \{ \, ({\vec{\mathit{a}}}_i \,-\, {\vec{\mathit{A}}}_{cm}) \;-\; 2 \; {\vec{\omega}} \times ({\vec{\mathit{v}}}_i \,-\, \hspace{-0.120em}{\vec{\mathit{V}}}_{cm}) \;+\; {\vec{\omega}} \times [ \, {\vec{\omega}} \times ({\vec{\mathit{r}}}_i \,-\, {\vec{\mathit{R}}}_{cm}) \, ] ~\,-$

\par \bigskip\smallskip \noindent ${\vec{\alpha}} \times ({\vec{\mathit{r}}}_i - {\vec{\mathit{R}}}_{cm}) \, \big \} \cdot ({\vec{\mathit{r}}}_i - {\vec{\mathit{R}}}_{cm}) \,=\, ({\vec{\mathit{a}}}_i - {\vec{\mathit{A}}}_{cm}) \cdot ({\vec{\mathit{r}}}_i - {\vec{\mathit{R}}}_{cm}) - \big [ \, 2 \; {\vec{\omega}} \times ({\vec{\mathit{v}}}_i - \hspace{-0.120em}{\vec{\mathit{V}}}_{cm}) \, \big ] \cdot ({\vec{\mathit{r}}}_i - {\vec{\mathit{R}}}_{cm}) ~\,+$

\par \bigskip\smallskip \noindent $\big \{ \, {\vec{\omega}} \times [ \, {\vec{\omega}} \times ({\vec{\mathit{r}}}_i - {\vec{\mathit{R}}}_{cm}) \, ] \, \big \} \cdot ({\vec{\mathit{r}}}_i - {\vec{\mathit{R}}}_{cm}) - \big [ \, {\vec{\alpha}} \times ({\vec{\mathit{r}}}_i - {\vec{\mathit{R}}}_{cm}) \, \big ] \cdot ({\vec{\mathit{r}}}_i - {\vec{\mathit{R}}}_{cm}) \,=\, ({\vec{\mathit{a}}}_i - {\vec{\mathit{A}}}_{cm}) \cdot ({\vec{\mathit{r}}}_i - {\vec{\mathit{R}}}_{cm}) ~\,-$

\par \bigskip\smallskip \noindent $\big [ \, 2 \; {\vec{\omega}} \times ({\vec{\mathit{v}}}_i - \hspace{-0.120em}{\vec{\mathit{V}}}_{cm}) \, \big ] \cdot ({\vec{\mathit{r}}}_i - {\vec{\mathit{R}}}_{cm}) + \big \{ \, \big [ \, {\vec{\omega}} \cdot ({\vec{\mathit{r}}}_i - {\vec{\mathit{R}}}_{cm}) \, \big ] \: {\vec{\omega}} - ( \, {\vec{\omega}} \cdot {\vec{\omega}} \, ) \: ({\vec{\mathit{r}}}_i - {\vec{\mathit{R}}}_{cm}) \, \big \} \cdot ({\vec{\mathit{r}}}_i - {\vec{\mathit{R}}}_{cm}) \,=$

\par \bigskip\smallskip \noindent $({\vec{\mathit{a}}}_i - {\vec{\mathit{A}}}_{cm}) \cdot ({\vec{\mathit{r}}}_i - {\vec{\mathit{R}}}_{cm}) - \big [ \, 2 \; {\vec{\omega}} \times ({\vec{\mathit{v}}}_i - \hspace{-0.120em}{\vec{\mathit{V}}}_{cm}) \, \big ] \cdot ({\vec{\mathit{r}}}_i - {\vec{\mathit{R}}}_{cm}) + \big [ \, {\vec{\omega}} \hspace{+0.120em}\cdot\hspace{+0.090em} ({\vec{\mathit{r}}}_i - {\vec{\mathit{R}}}_{cm}) \, \big ]^2 -\, ( \, {\vec{\omega}} \, )^2 \: ({\vec{\mathit{r}}}_i - {\vec{\mathit{R}}}_{cm})^2$

\par \bigskip\smallskip \noindent $\longrightarrow \hspace{+0.90em} ({\mathbf{v}}_i - {\mathbf{V}}_{cm})^2 + ({\mathbf{a}}_{\hspace{+0.045em}i} - {\mathbf{A}}_{cm}) \cdot ({\mathbf{r}}_i - {\mathbf{R}}_{cm}) \,=\, ({\vec{\mathit{v}}}_i - \hspace{-0.120em}{\vec{\mathit{V}}}_{cm})^2 + ({\vec{\mathit{a}}}_i - {\vec{\mathit{A}}}_{cm}) \cdot ({\vec{\mathit{r}}}_i - {\vec{\mathit{R}}}_{cm})$

\newpage

\par \bigskip {\centering\subsection*{Appendix III}}

\par \medskip {\centering\subsubsection*{The Magnitudes}}\addcontentsline{toc}{subsection}{Appendix III : The Magnitudes}

\par \bigskip \noindent The magnitudes ${\mathbf{L}}_2$, ${\mathrm{W}}_2$, ${\mathrm{K}}_2$, ${\mathrm{U}}_2$, ${\mathrm{W}}_4$, ${\mathrm{K}}_4$, ${\mathrm{U}}_4$, ${\mathrm{W}}_6$, ${\mathrm{K}}_6$ and ${\mathrm{U}}_6$ of a system of N particles can also be expressed as follows:

\par \bigskip\bigskip \noindent ${\mathbf{L}}_2 ~=~ \sum_{j{\scriptscriptstyle >}i}^{\scriptscriptstyle{\mathrm{N}}} \, m_i \, m_j \, {\mathrm{M}}^{\scriptscriptstyle -1} \big [ \, ({\mathbf{r}}_{i} - {\mathbf{r}}_{j}) \times ({\mathbf{v}}_{i} - {\mathbf{v}}_{j}) \, \big ]$

\par \bigskip\bigskip \noindent ${\mathrm{W}}_2 ~=~ \sum_{j{\scriptscriptstyle >}i}^{\scriptscriptstyle{\mathrm{N}}} \, m_i \, m_j \, {\mathrm{M}}^{\scriptscriptstyle -1} \big [ \int_{\scriptscriptstyle 1}^{\scriptscriptstyle 2} \, ({\mathbf{F}}_i / m_i - {\mathbf{F}}_j / m_j) \cdot d({\mathbf{r}}_{i} - {\mathbf{r}}_{j}) \, \big ]$

\par \bigskip\smallskip \noindent $\Delta \, {\mathrm{K}}_2 ~=~ \sum_{j{\scriptscriptstyle >}i}^{\scriptscriptstyle{\mathrm{N}}} \, \Delta \, \med \; m_i \, m_j \, {\mathrm{M}}^{\scriptscriptstyle -1} \, ({\mathbf{v}}_{i} - {\mathbf{v}}_{j})^2 ~=~ {\mathrm{W}}_2$

\par \bigskip\smallskip \noindent $\Delta \, {\mathrm{U}}_2 ~=~ - \, \sum_{j{\scriptscriptstyle >}i}^{\scriptscriptstyle{\mathrm{N}}} \, m_i \, m_j \, {\mathrm{M}}^{\scriptscriptstyle -1} \big [ \int_{\scriptscriptstyle 1}^{\scriptscriptstyle 2} \, ({\mathbf{F}}_i / m_i - {\mathbf{F}}_j / m_j) \cdot d({\mathbf{r}}_{i} - {\mathbf{r}}_{j}) \, \big ]$

\par \bigskip\bigskip \noindent ${\mathrm{W}}_4 ~=~ \sum_{j{\scriptscriptstyle >}i}^{\scriptscriptstyle{\mathrm{N}}} \, \Delta \, \med \; m_i \, m_j \, {\mathrm{M}}^{\scriptscriptstyle -1} \big [ \, ({\mathbf{F}}_i / m_i - {\mathbf{F}}_j / m_j) \cdot ({\mathbf{r}}_{i} - {\mathbf{r}}_{j}) \, \big ]$

\par \bigskip\smallskip \noindent $\Delta \, {\mathrm{K}}_4 ~=~ \sum_{j{\scriptscriptstyle >}i}^{\scriptscriptstyle{\mathrm{N}}} \, \Delta \, \med \; m_i \, m_j \, {\mathrm{M}}^{\scriptscriptstyle -1} \big [ \, ({\mathbf{a}}_{\hspace{+0.045em}i} - {\mathbf{a}}_{j}) \cdot ({\mathbf{r}}_{i} - {\mathbf{r}}_{j}) \, \big ] ~=~ {\mathrm{W}}_4$

\par \bigskip\smallskip \noindent $\Delta \, {\mathrm{U}}_4 ~=~ - \, \sum_{j{\scriptscriptstyle >}i}^{\scriptscriptstyle{\mathrm{N}}} \, \Delta \, \med \; m_i \, m_j \, {\mathrm{M}}^{\scriptscriptstyle -1} \big [ \, ({\mathbf{F}}_i / m_i - {\mathbf{F}}_j / m_j) \cdot ({\mathbf{r}}_{i} - {\mathbf{r}}_{j}) \, \big ]$

\par \bigskip\bigskip \noindent ${\mathrm{W}}_6 ~=~ \sum_{j{\scriptscriptstyle >}i}^{\scriptscriptstyle{\mathrm{N}}} \, m_i \, m_j \, {\mathrm{M}}^{\scriptscriptstyle -1} \big [ \int_{\scriptscriptstyle 1}^{\scriptscriptstyle 2} \, ({\mathbf{F}}_i / m_i - {\mathbf{F}}_j / m_j) \cdot d({\vec{\mathit{r}}_{i}} - {\vec{\mathit{r}}_{j}}) + \Delta \, \med \; ({\mathbf{F}}_i / m_i - {\mathbf{F}}_j / m_j) \cdot ({\vec{\mathit{r}}_{i}} - {\vec{\mathit{r}}_{j}}) \, \big ]$

\par \bigskip\smallskip \noindent $\Delta \, {\mathrm{K}}_6 ~=~ \sum_{j{\scriptscriptstyle >}i}^{\scriptscriptstyle{\mathrm{N}}} \, \Delta \, \med \; m_i \, m_j \, {\mathrm{M}}^{\scriptscriptstyle -1} \big [ \, ({\vec{\mathit{v}}_{i}} - {\vec{\mathit{v}}_{j}})^2 + ({\vec{\mathit{a}}_{i}} - {\vec{\mathit{a}}_{j}}) \cdot ({\vec{\mathit{r}}_{i}} - {\vec{\mathit{r}}_{j}}) \, \big ] ~=~ {\mathrm{W}}_6$

\par \bigskip\smallskip \noindent $\Delta \, {\mathrm{U}}_6 ~=~ - \, \sum_{j{\scriptscriptstyle >}i}^{\scriptscriptstyle{\mathrm{N}}} \, m_i \, m_j \, {\mathrm{M}}^{\scriptscriptstyle -1} \big [ \int_{\scriptscriptstyle 1}^{\scriptscriptstyle 2} \, ({\mathbf{F}}_i / m_i - {\mathbf{F}}_j / m_j) \cdot d({\vec{\mathit{r}}_{i}} - {\vec{\mathit{r}}_{j}}) + \Delta \, \med \; ({\mathbf{F}}_i / m_i - {\mathbf{F}}_j / m_j) \cdot ({\vec{\mathit{r}}_{i}} - {\vec{\mathit{r}}_{j}}) \, \big ]$

\par \bigskip\bigskip \noindent The magnitudes ${\mathrm{W}}_{(1\;{\mathrm{to}}\;6)}$ and ${\mathrm{U}}_{(1\;{\mathrm{to}}\;6)}$ of an isolated system of N particles, whose internal forces obey Newton's third law in its weak form, can be reduced to:

\par \bigskip\bigskip \noindent ${\mathrm{W}}_1 ~=~ {\mathrm{W}}_2 ~=~ \sum_i^{\scriptscriptstyle{\mathrm{N}}} \int_{\scriptscriptstyle 1}^{\scriptscriptstyle 2} \, {\mathbf{F}}_i \cdot d{\vec{\mathit{r}}_{i}}$

\par \bigskip\smallskip \noindent $\Delta \, {\mathrm{U}}_1 ~=~ \Delta \, {\mathrm{U}}_2 ~=~ - \, \sum_i^{\scriptscriptstyle{\mathrm{N}}} \int_{\scriptscriptstyle 1}^{\scriptscriptstyle 2} \, {\mathbf{F}}_i \cdot d{\vec{\mathit{r}}_{i}}$

\par \bigskip\bigskip \noindent ${\mathrm{W}}_3 ~=~ {\mathrm{W}}_4 ~=~ \sum_i^{\scriptscriptstyle{\mathrm{N}}} \Delta \, \med \; {\mathbf{F}}_i \cdot {\vec{\mathit{r}}_{i}}$

\par \bigskip\smallskip \noindent $\Delta \, {\mathrm{U}}_3 ~=~ \Delta \, {\mathrm{U}}_4 ~=~ - \, \sum_i^{\scriptscriptstyle{\mathrm{N}}} \Delta \, \med \; {\mathbf{F}}_i \cdot {\vec{\mathit{r}}_{i}}$

\par \bigskip\bigskip \noindent ${\mathrm{W}}_5 ~=~ {\mathrm{W}}_6 ~=~ \sum_i^{\scriptscriptstyle{\mathrm{N}}} \big [ \int_{\scriptscriptstyle 1}^{\scriptscriptstyle 2} \, {\mathbf{F}}_i \cdot d{\vec{\mathit{r}}_{i}} + \Delta \, \med \; {\mathbf{F}}_i \cdot {\vec{\mathit{r}}_{i}} \, \big ]$

\par \bigskip\smallskip \noindent $\Delta \, {\mathrm{U}}_5 ~=~ \Delta \, {\mathrm{U}}_6 ~=~ - \, \sum_i^{\scriptscriptstyle{\mathrm{N}}} \big [ \int_{\scriptscriptstyle 1}^{\scriptscriptstyle 2} \, {\mathbf{F}}_i \cdot d{\vec{\mathit{r}}_{i}} + \Delta \, \med \; {\mathbf{F}}_i \cdot {\vec{\mathit{r}}_{i}} \, \big ]$

\newpage

\setcounter{page}{1}

\section*{}\addcontentsline{toc}{section}{Paper II}

\begin{center}

{\LARGE A New Theory in Relational Mechanics}

\bigskip \medskip

{\large Antonio A. Blatter}

\bigskip \medskip

\small

Creative Commons Attribution 3.0 License

\smallskip

(2015) Buenos Aires, Argentina

\medskip

{\sc ( Paper II )}

\smallskip

\bigskip \medskip

\parbox{107.40mm}{In relational mechanics, a new theory is presented, which is invariant under transformations between inertial and non-inertial reference frames, which can be applied in any reference frame without introducing fictitious forces and which establishes the existence of a new universal force of interaction, called kinetic force.}

\end{center}

\normalsize

\vspace{-1.20em}

\par \bigskip {\centering\subsection*{Introduction}}\addcontentsline{toc}{subsection}{1. Introduction}

\par \bigskip\smallskip \noindent The new theory in relational mechanics presented in this paper is obtained starting from an auxiliary system of particles (called Universe) that is used to obtain kinematic magnitudes (such as universal position, universal velocity, etc.) that are invariant under transformations between inertial and non-inertial reference frames.

\par \bigskip \noindent The universal position ${\mathbf{r}}_i$, the universal velocity ${\mathbf{v}}_i$ and the universal acceleration ${\mathbf{a}}_{\hspace{+0.045em}i}$ of a \hbox {particle $i$} are given by:

\par \bigskip\smallskip ${\mathbf{r}}_i \,\doteq\, ({\vec{\mathit{r}}}_i - {\vec{\mathit{R}}})$

\par \bigskip ${\mathbf{v}}_i \,\doteq\, ({\vec{\mathit{v}}}_i - \hspace{-0.120em}{\vec{\mathit{V}}}) - {\vec{\omega}} \times ({\vec{\mathit{r}}}_i - {\vec{\mathit{R}}})$

\par \bigskip ${\mathbf{a}}_{\hspace{+0.045em}i} \,\doteq\, ({\vec{\mathit{a}}}_i - {\vec{\mathit{A}}}) - 2 \; {\vec{\omega}} \times ({\vec{\mathit{v}}}_i - \hspace{-0.120em}{\vec{\mathit{V}}}) + {\vec{\omega}} \times [ \, {\vec{\omega}} \times ({\vec{\mathit{r}}}_i - {\vec{\mathit{R}}}) \, ] - {\vec{\alpha}} \times ({\vec{\mathit{r}}}_i - {\vec{\mathit{R}}})$

\par \bigskip\smallskip \noindent $( \, {\mathbf{v}}_i \doteq d(\hspace{+0.090em}{\mathbf{r}}_i\hspace{+0.090em})/dt \, )$ and $( \, {\mathbf{a}}_{\hspace{+0.045em}i} \doteq d^2(\hspace{+0.090em}{\mathbf{r}}_i\hspace{+0.090em})/dt^2 \, )$ where ${\vec{\mathit{r}}}_i$ is the position vector of particle $i$, ${\vec{\mathit{R}}}$ is the position vector of the center of mass of the Universe, and ${\vec{\omega}}$ is the angular velocity vector of the Universe \hyperlink{p2a1}{(\hspace{+0.120em}see Appendix I\hspace{+0.120em})}

\par \bigskip \noindent A reference frame S is non-rotating if the angular velocity ${\vec{\omega}}$ of the Universe relative to S is equal to zero, and the reference frame S is also inertial if the acceleration ${\vec{\mathit{A}}}$ of the center of mass of the Universe relative to S is equal to zero.

\vspace{+1.20em}

\par {\centering\subsection*{The New Dynamics}}\addcontentsline{toc}{subsection}{2. The New Dynamics}

\par \bigskip\smallskip \noindent $[\,1\,]$ A force is always caused by the interaction between two or more particles.

\par \bigskip \noindent $[\,2\,]$ The total force ${\mathbf{T}}_i$ acting on a particle $i$ is always zero $[ \, {\mathbf{T}}_i \,=\, 0 \, ]$

\par \bigskip \noindent $[\,3\,]$ In this paper, we assume that all non-kinetic forces always obey Newton's third law in its weak form and in its strong form.

\newpage

\par \bigskip {\centering\subsection*{The Kinetic Force}}\addcontentsline{toc}{subsection}{3. The Kinetic Force}

\par \bigskip\smallskip \noindent The kinetic force ${\mathbf{K}}_{\hspace{+0.060em}ij}$ exerted on a particle $i$ of mass $m_i$ by another particle $j$ of mass $m_j$, caused by the interaction between particle $i$ and particle $j$, is given by:

\par \bigskip ${\mathbf{K}}_{\hspace{+0.060em}ij} ~=~ - \; m_i \, m_j \, {\mathrm{M}}^{\scriptscriptstyle -1} \, (\hspace{+0.045em}{\mathbf{a}}_{\hspace{+0.045em}i} - {\mathbf{a}}_{j})$

\par \bigskip \noindent where ${\mathbf{a}}_{\hspace{+0.045em}i}$ is the universal acceleration of particle $i$, ${\mathbf{a}}_{j}$ is the universal acceleration of \hbox {particle $j$}, and ${\mathrm{M}}$ is the mass of the Universe.

\par \bigskip \noindent From the above equation it follows that the net kinetic force ${\mathbf{K}}_{i}$ (\hspace{+0.237em}$ = \sum_{j}^{\scriptscriptstyle{\mathit{All}}} \, {\mathbf{K}}_{\hspace{+0.060em}ij}$\hspace{+0.237em}) acting on a particle $i$ of mass $m_i$ is given by:

\par \bigskip ${\mathbf{K}}_{i} ~=~ - \; m_i \, (\hspace{+0.045em}{\mathbf{a}}_{\hspace{+0.045em}i} - {\mathbf{A}}_{cm})$

\par \bigskip \noindent where ${\mathbf{a}}_{\hspace{+0.045em}i}$ is the universal acceleration of particle $i$ and ${\mathbf{A}}_{cm}$ is the universal acceleration of the center of mass of the Universe.

\par \bigskip \noindent Since the universal acceleration of the center of mass of the Universe ${\mathbf{A}}_{cm}$ is always zero, then the net kinetic force ${\mathbf{K}}_{i}$ acting on a particle $i$ of mass $m_i$ is certainly given by:

\par \bigskip ${\mathbf{K}}_{i} ~=~ - \; m_i \, {\mathbf{a}}_{\hspace{+0.045em}i}$

\par \bigskip \noindent where ${\mathbf{a}}_{\hspace{+0.045em}i}$ is the universal acceleration of particle $i$.

\par \bigskip \noindent The kinetic force ${\mathbf{K}}_{\hspace{+0.060em}ij}$ is considered in the new dynamics, mainly in the $[\,2\,]$ principle, as a new universal force of interaction.

\par \bigskip \noindent Finally, the kinetic force ${\mathbf{K}}_{\hspace{+0.060em}ij}$ always obey Newton's third law in its weak form.

\vspace{+1.65em}

\par {\centering\subsection*{The [$\,$2$\,$] Principle}}\addcontentsline{toc}{subsection}{4. The Second Principle}

\par \bigskip\smallskip \noindent The second principle of the new dynamics establishes that the total force ${\mathbf{T}}_i$ acting on a \hbox {particle $i$} is always zero.

\par \bigskip ${\mathbf{T}}_i ~=~ 0$

\par \bigskip \noindent If the total force ${\mathbf{T}}_i$ is divided into the following two parts: the net kinetic force ${\mathbf{K}}_{i}$ and the net non-kinetic force ${\mathbf{F}}_{i}$ (\hspace{+0.180em}$\sum$ of gravitational forces, electrostatic forces, etc.\hspace{+0.180em}) then we \hbox{have\hspace{+0.060em}:}

\par \bigskip ${\mathbf{K}}_{i} + {\mathbf{F}}_{i} ~=~ 0$

\par \bigskip \noindent Now, substituting (\hspace{+0.180em}${\mathbf{K}}_{i} \,=\, - \; m_i \, {\mathbf{a}}_{\hspace{+0.045em}i}$\hspace{+0.180em}) and rearranging, we finally obtain:

\par \bigskip ${\mathbf{F}}_{i} ~=~ m_i \, {\mathbf{a}}_{\hspace{+0.045em}i}$

\par \bigskip \noindent This equation (\hspace{+0.180em}similar to Newton's second law\hspace{+0.180em}) will be used throughout this paper.

\par \bigskip \noindent On the other hand, in this paper a system of particles is isolated when the system is free of external non-kinetic forces.

\newpage

\par \bigskip {\centering\subsection*{The Definitions}}\addcontentsline{toc}{subsection}{5. The Definitions}

\par \bigskip \noindent For a system of N particles, the following definitions are applicable:

\par \bigskip\bigskip \hspace{-2.40em} \begin{tabular}{lll}
Mass & \hspace{+0.00em} & ${\mathrm{M}} ~\doteq~ \sum_i^{\scriptscriptstyle{\mathrm{N}}} \, m_i$ \vspace{+0.99em} \\
\\
Position {\small CM} 1 & \hspace{+0.00em} & ${\vec{\mathit{R}}}_{cm} ~\doteq~ {\mathrm{M}}^{\scriptscriptstyle -1} \, \sum_i^{\scriptscriptstyle{\mathrm{N}}} \, m_i \, {\vec{\mathit{r}}_{i}}$ \vspace{+0.99em} \\
Velocity {\small CM} 1 & \hspace{+0.00em} & ${\vec{\mathit{V}}}_{cm} ~\doteq~ {\mathrm{M}}^{\scriptscriptstyle -1} \, \sum_i^{\scriptscriptstyle{\mathrm{N}}} \, m_i \, {\vec{\mathit{v}}_{i}}$ \vspace{+0.99em} \\
Acceleration {\small CM} 1 & \hspace{+0.00em} & ${\vec{\mathit{A}}}_{cm} ~\doteq~ {\mathrm{M}}^{\scriptscriptstyle -1} \, \sum_i^{\scriptscriptstyle{\mathrm{N}}} \, m_i \, {\vec{\mathit{a}}_{i}}$ \vspace{+0.99em} \\
\\
Position {\small CM} 2 & \hspace{+0.00em} & ${\mathbf{R}}_{cm} ~\doteq~ {\mathrm{M}}^{\scriptscriptstyle -1} \, \sum_i^{\scriptscriptstyle{\mathrm{N}}} \, m_i \, {\mathbf{r}}_{i}$ \vspace{+0.99em} \\
Velocity {\small CM} 2 & \hspace{+0.00em} & ${\mathbf{V}}_{cm} ~\doteq~ {\mathrm{M}}^{\scriptscriptstyle -1} \, \sum_i^{\scriptscriptstyle{\mathrm{N}}} \, m_i \, {\mathbf{v}}_{i}$ \vspace{+0.99em} \\
Acceleration {\small CM} 2 & \hspace{+0.00em} & ${\mathbf{A}}_{cm} ~\doteq~ {\mathrm{M}}^{\scriptscriptstyle -1} \, \sum_i^{\scriptscriptstyle{\mathrm{N}}} \, m_i \, {\mathbf{a}}_{\hspace{+0.045em}i}$ \vspace{+0.99em} \\
\\
Linear Momentum 1 & \hspace{+0.00em} & ${\mathbf{P}}_1 ~\doteq~ \sum_i^{\scriptscriptstyle{\mathrm{N}}} \, m_i \, {\mathbf{v}}_{i}$ \vspace{+0.99em} \\
Angular Momentum 1 & \hspace{+0.00em} & ${\mathbf{L}}_1 ~\doteq~ \sum_i^{\scriptscriptstyle{\mathrm{N}}} \, m_i \, \big [ \, {\mathbf{r}}_{i} \times {\mathbf{v}}_{i} \, \big ]$ \vspace{+0.99em} \\
Angular Momentum 2 & \hspace{+0.00em} & ${\mathbf{L}}_2 ~\doteq~ \sum_i^{\scriptscriptstyle{\mathrm{N}}} \, m_i \, \big [ \, ({\mathbf{r}}_{i} - {\mathbf{R}}_{cm}) \times ({\mathbf{v}}_{i} - {\mathbf{V}}_{cm}) \, \big ]$ \vspace{+0.99em} \\
\\
Work 1 & \hspace{+0.00em} & ${\mathrm{W}}_1 ~\doteq~ \sum_i^{\scriptscriptstyle{\mathrm{N}}} \int_{\scriptscriptstyle 1}^{\scriptscriptstyle 2} \, {\mathbf{F}}_i \cdot d{\mathbf{r}}_{i} \,=\, \Delta \, {\mathrm{K}}_1$ \vspace{+0.99em} \\
Kinetic Energy 1 & \hspace{+0.00em} & $\Delta \, {\mathrm{K}}_1 ~\doteq~ \sum_i^{\scriptscriptstyle{\mathrm{N}}} \Delta \, \med \; m_i \, ({\mathbf{v}}_{i})^2$ \vspace{+0.99em} \\
Potential Energy 1 & \hspace{+0.00em} & $\Delta \, {\mathrm{U}}_1 ~\doteq~ - \, \sum_i^{\scriptscriptstyle{\mathrm{N}}} \int_{\scriptscriptstyle 1}^{\scriptscriptstyle 2} \, {\mathbf{F}}_i \cdot d{\mathbf{r}}_{i}$ \vspace{+0.99em} \\
Mechanical Energy 1 & \hspace{+0.00em} & ${\mathrm{E}}_1 ~\doteq~ {\mathrm{K}}_1 + {\mathrm{U}}_1$ \vspace{+0.99em} \\
Lagrangian 1 & \hspace{+0.00em} & ${\mathrm{L}}_1 ~\doteq~ {\mathrm{K}}_1 - {\mathrm{U}}_1$ \vspace{+0.99em} \\
\\
Work 2 & \hspace{+0.00em} & ${\mathrm{W}}_2 ~\doteq~ \sum_i^{\scriptscriptstyle{\mathrm{N}}} \int_{\scriptscriptstyle 1}^{\scriptscriptstyle 2} \, {\mathbf{F}}_i \cdot d({\mathbf{r}}_{i} - {\mathbf{R}}_{cm}) \,=\, \Delta \, {\mathrm{K}}_2$ \vspace{+0.99em} \\
Kinetic Energy 2 & \hspace{+0.00em} & $\Delta \, {\mathrm{K}}_2 ~\doteq~ \sum_i^{\scriptscriptstyle{\mathrm{N}}} \Delta \, \med \; m_i \, ({\mathbf{v}}_{i} - {\mathbf{V}}_{cm})^2$ \vspace{+0.99em} \\
Potential Energy 2 & \hspace{+0.00em} & $\Delta \, {\mathrm{U}}_2 ~\doteq~ - \, \sum_i^{\scriptscriptstyle{\mathrm{N}}} \int_{\scriptscriptstyle 1}^{\scriptscriptstyle 2} \, {\mathbf{F}}_i \cdot d({\mathbf{r}}_{i} - {\mathbf{R}}_{cm})$ \vspace{+0.99em} \\
Mechanical Energy 2 & \hspace{+0.00em} & ${\mathrm{E}}_2 ~\doteq~ {\mathrm{K}}_2 + {\mathrm{U}}_2$ \vspace{+0.99em} \\
Lagrangian 2 & \hspace{+0.00em} & ${\mathrm{L}}_2 ~\doteq~ {\mathrm{K}}_2 - {\mathrm{U}}_2$
\end{tabular}

\newpage

\par \bigskip\bigskip \hspace{-2.40em} \begin{tabular}{lll}
Work 3 & \hspace{+0.33em} & ${\mathrm{W}}_3 ~\doteq~ \sum_i^{\scriptscriptstyle{\mathrm{N}}} \Delta \, \med \; {\mathbf{F}}_i \cdot {\mathbf{r}}_{i} \,=\, \Delta \, {\mathrm{K}}_3$ \vspace{+0.99em} \\
Kinetic Energy 3 & \hspace{+0.33em} & $\Delta \, {\mathrm{K}}_3 ~\doteq~ \sum_i^{\scriptscriptstyle{\mathrm{N}}} \Delta \, \med \; m_i \: {\mathbf{a}}_{\hspace{+0.045em}i} \cdot {\mathbf{r}}_{i}$ \vspace{+0.99em} \\
Potential Energy 3 & \hspace{+0.33em} & $\Delta \, {\mathrm{U}}_3 ~\doteq~ - \, \sum_i^{\scriptscriptstyle{\mathrm{N}}} \Delta \, \med \; {\mathbf{F}}_i \cdot {\mathbf{r}}_{i}$ \vspace{+0.99em} \\
Mechanical Energy 3 & \hspace{+0.33em} & ${\mathrm{E}}_3 ~\doteq~ {\mathrm{K}}_3 + {\mathrm{U}}_3$ \vspace{+0.99em} \\
\\
Work 4 & \hspace{+0.33em} & ${\mathrm{W}}_4 ~\doteq~ \sum_i^{\scriptscriptstyle{\mathrm{N}}} \Delta \, \med \; {\mathbf{F}}_i \cdot ({\mathbf{r}}_{i} - {\mathbf{R}}_{cm}) \,=\, \Delta \, {\mathrm{K}}_4$ \vspace{+0.99em} \\
Kinetic Energy 4 & \hspace{+0.33em} & $\Delta \, {\mathrm{K}}_4 ~\doteq~ \sum_i^{\scriptscriptstyle{\mathrm{N}}} \Delta \, \med \; m_i \, \big [ \, ({\mathbf{a}}_{\hspace{+0.045em}i} - {\mathbf{A}}_{cm}) \cdot ({\mathbf{r}}_{i} - {\mathbf{R}}_{cm}) \, \big ]$ \vspace{+0.99em} \\
Potential Energy 4 & \hspace{+0.33em} & $\Delta \, {\mathrm{U}}_4 ~\doteq~ - \, \sum_i^{\scriptscriptstyle{\mathrm{N}}} \Delta \, \med \; {\mathbf{F}}_i \cdot ({\mathbf{r}}_{i} - {\mathbf{R}}_{cm})$ \vspace{+0.99em} \\
Mechanical Energy 4 & \hspace{+0.33em} & ${\mathrm{E}}_4 ~\doteq~ {\mathrm{K}}_4 + {\mathrm{U}}_4$ \vspace{+0.99em} \\
\\
Work 5 & \hspace{+0.33em} & ${\mathrm{W}}_5 ~\doteq~ \sum_i^{\scriptscriptstyle{\mathrm{N}}} \, \big [ \int_{\scriptscriptstyle 1}^{\scriptscriptstyle 2} \, {\mathbf{F}}_i \cdot d({\vec{\mathit{r}}_{i}} - {\vec{\mathit{R}}}) + \Delta \, \med \; {\mathbf{F}}_i \cdot ({\vec{\mathit{r}}_{i}} - {\vec{\mathit{R}}}) \, \big ] \,=\, \Delta \, {\mathrm{K}}_5$ \vspace{+0.99em} \\
Kinetic Energy 5 & \hspace{+0.33em} & $\Delta \, {\mathrm{K}}_5 ~\doteq~ \sum_i^{\scriptscriptstyle{\mathrm{N}}} \Delta \, \med \; m_i \, \big [ \, ({\vec{\mathit{v}}_{i}} - {\vec{\mathit{V}}})^2 + ({\vec{\mathit{a}}_{i}} - {\vec{\mathit{A}}}) \cdot ({\vec{\mathit{r}}_{i}} - {\vec{\mathit{R}}}) \, \big ]$ \vspace{+0.99em} \\
Potential Energy 5 & \hspace{+0.33em} & $\Delta \, {\mathrm{U}}_5 ~\doteq~ - \, \sum_i^{\scriptscriptstyle{\mathrm{N}}} \, \big [ \int_{\scriptscriptstyle 1}^{\scriptscriptstyle 2} \, {\mathbf{F}}_i \cdot d({\vec{\mathit{r}}_{i}} - {\vec{\mathit{R}}}) + \Delta \, \med \; {\mathbf{F}}_i \cdot ({\vec{\mathit{r}}_{i}} - {\vec{\mathit{R}}}) \, \big ]$ \vspace{+0.99em} \\
Mechanical Energy 5 & \hspace{+0.33em} & ${\mathrm{E}}_5 ~\doteq~ {\mathrm{K}}_5 + {\mathrm{U}}_5$ \vspace{+0.99em} \\
\\
Work 6 & \hspace{+0.33em} & ${\mathrm{W}}_6 ~\doteq~ \sum_i^{\scriptscriptstyle{\mathrm{N}}} \, \big [ \int_{\scriptscriptstyle 1}^{\scriptscriptstyle 2} \, {\mathbf{F}}_i \cdot d({\vec{\mathit{r}}_{i}} - {\vec{\mathit{R}}}_{cm}) + \Delta \, \med \; {\mathbf{F}}_i \cdot ({\vec{\mathit{r}}_{i}} - {\vec{\mathit{R}}}_{cm}) \, \big ] \,=\, \Delta \, {\mathrm{K}}_6$ \vspace{+0.99em} \\
Kinetic Energy 6 & \hspace{+0.33em} & $\Delta \, {\mathrm{K}}_6 ~\doteq~ \sum_i^{\scriptscriptstyle{\mathrm{N}}} \Delta \, \med \; m_i \, \big [ \, ({\vec{\mathit{v}}_{i}} - {\vec{\mathit{V}}}_{cm})^2 + ({\vec{\mathit{a}}_{i}} - {\vec{\mathit{A}}}_{cm}) \cdot ({\vec{\mathit{r}}_{i}} - {\vec{\mathit{R}}}_{cm}) \, \big ]$ \vspace{+0.99em} \\
Potential Energy 6 & \hspace{+0.33em} & $\Delta \, {\mathrm{U}}_6 ~\doteq~ - \, \sum_i^{\scriptscriptstyle{\mathrm{N}}} \, \big [ \int_{\scriptscriptstyle 1}^{\scriptscriptstyle 2} \, {\mathbf{F}}_i \cdot d({\vec{\mathit{r}}_{i}} - {\vec{\mathit{R}}}_{cm}) + \Delta \, \med \; {\mathbf{F}}_i \cdot ({\vec{\mathit{r}}_{i}} - {\vec{\mathit{R}}}_{cm}) \, \big ]$ \vspace{+0.99em} \\
Mechanical Energy 6 & \hspace{+0.33em} & ${\mathrm{E}}_6 ~\doteq~ {\mathrm{K}}_6 + {\mathrm{U}}_6$
\end{tabular}

\vspace{+0.60em}

\par {\centering\subsection*{The Relations}}\addcontentsline{toc}{subsection}{6. The Relations}

\par \bigskip\smallskip \noindent From the above definitions, the following relations can be obtained \hyperlink{p2a2}{(\hspace{+0.120em}see Appendix II\hspace{+0.120em})}

\vspace{+1.80em}

\par \noindent ${\mathrm{K}}_1 ~=~ {\mathrm{K}}_2 + \med \; {\mathrm{M}} \: {\mathbf{V}}_{cm}^{\hspace{+0.045em}2}$
\vspace{+0.99em}
\par \noindent ${\mathrm{K}}_3 ~=~ {\mathrm{K}}_4 + \med \; {\mathrm{M}} \: {\mathbf{A}}_{cm} \cdot {\mathbf{R}}_{cm}$
\vspace{+0.99em}
\par \noindent ${\mathrm{K}}_5 ~=~ {\mathrm{K}}_6 + \med \; {\mathrm{M}} \: \big [ \, ({\vec{\mathit{V}}}_{cm} - \hspace{-0.120em}{\vec{\mathit{V}}})^2 + ({\vec{\mathit{A}}}_{cm} - {\vec{\mathit{A}}}) \cdot ({\vec{\mathit{R}}}_{cm} - {\vec{\mathit{R}}}) \, \big ]$
\vspace{+0.99em}
\par \noindent ${\mathrm{K}}_5 ~=~ {\mathrm{K}}_1 + {\mathrm{K}}_3$ $\hspace{+0.540em} \& \hspace{+0.540em}$ ${\mathrm{U}}_5 ~=~ {\mathrm{U}}_1 \hspace{+0.027em}+\hspace{+0.027em} {\mathrm{U}}_3$ $\hspace{+0.630em} \& \hspace{+0.630em}$ ${\mathrm{E}}_5 ~=~ {\mathrm{E}}_1 + {\mathrm{E}}_3$
\vspace{+0.99em}
\par \noindent ${\mathrm{K}}_6 ~=~ {\mathrm{K}}_2 + {\mathrm{K}}_4$ $\hspace{+0.540em} \& \hspace{+0.540em}$ ${\mathrm{U}}_6 ~=~ {\mathrm{U}}_2 \hspace{+0.027em}+\hspace{+0.027em} {\mathrm{U}}_4$ $\hspace{+0.630em} \& \hspace{+0.630em}$ ${\mathrm{E}}_6 ~=~ {\mathrm{E}}_2 + {\mathrm{E}}_4$

\newpage

\par {\centering\subsection*{The Principles}}\addcontentsline{toc}{subsection}{7. The Principles}

\par \bigskip\smallskip \noindent The linear momentum $[ \, {\mathbf{P}}_1 \, ]$ of an isolated system of N particles remains constant if the internal non-kinetic forces obey Newton's third law in its weak form.

\par \bigskip\medskip ${\mathbf{P}}_1 ~=~ {\mathrm{constant}} \hspace{+2.88em} \big [ \; d({\mathbf{P}}_1)/dt ~=~ \sum_i^{\scriptscriptstyle{\mathrm{N}}} \, m_i \, {\mathbf{a}}_{\hspace{+0.045em}i} ~=~ \sum_i^{\scriptscriptstyle{\mathrm{N}}} \, {\mathbf{F}}_i ~=~ 0 \; \big ]$

\par \bigskip\medskip \noindent The angular momentum $[ \, {\mathbf{L}}_1 \, ]$ of an isolated system of N particles remains constant if the internal non-kinetic forces obey Newton's third law in its strong form.

\par \bigskip\medskip ${\mathbf{L}}_1 ~=~ {\mathrm{constant}} \hspace{+2.97em} \big [ \; d({\mathbf{L}}_1)/dt ~=~ \sum_i^{\scriptscriptstyle{\mathrm{N}}} \, m_i \, \big [ \, {\mathbf{r}}_i \times {\mathbf{a}}_{\hspace{+0.045em}i} \, \big ]~=~ \sum_i^{\scriptscriptstyle{\mathrm{N}}} \, {\mathbf{r}}_i \times {\mathbf{F}}_i ~=~ 0 \; \big ]$

\par \bigskip\medskip \noindent The angular momentum $[ \, {\mathbf{L}}_2 \, ]$ of an isolated system of N particles remains constant if the internal non-kinetic forces obey Newton's third law in its strong form.

\par \bigskip\medskip ${\mathbf{L}}_2 ~=~ {\mathrm{constant}} \hspace{+2.97em} \big [ \; d({\mathbf{L}}_2)/dt ~=~ \sum_i^{\scriptscriptstyle{\mathrm{N}}} \, m_i \, \big [ \, ({\mathbf{r}}_i - {\mathbf{R}}_{cm}) \times ({\mathbf{a}}_{\hspace{+0.045em}i} - {\mathbf{A}}_{cm}) \, \big ] ~=~$

\par \bigskip $\hspace{+10.44em} \sum_i^{\scriptscriptstyle{\mathrm{N}}} \, m_i \, \big [ \, ({\mathbf{r}}_i - {\mathbf{R}}_{cm}) \times {\mathbf{a}}_{\hspace{+0.045em}i} \, \big ] ~=~ \sum_i^{\scriptscriptstyle{\mathrm{N}}} \, ({\mathbf{r}}_i - {\mathbf{R}}_{cm}) \times {\mathbf{F}}_i ~=~ 0 \; \big ]$

\par \bigskip\medskip \noindent The mechanical energy $[ \, {\mathrm{E}}_1 \, ]$ and the mechanical energy $[ \, {\mathrm{E}}_2 \, ]$ of a system of N particles remain constant if the system is only subject to kinetic forces and to conservative non-kinetic forces.

\par \bigskip\medskip ${\mathrm{E}}_1 ~=~ {\mathrm{constant}} \hspace{+3.00em} \big [ \; \Delta \; {\mathrm{E}}_1 ~=~ \Delta \; {\mathrm{K}}_1 + \Delta \; {\mathrm{U}}_1 ~=~ 0 \; \big ]$

\par \bigskip ${\mathrm{E}}_2 ~=~ {\mathrm{constant}} \hspace{+3.00em} \big [ \; \Delta \; {\mathrm{E}}_2 ~=~ \Delta \; {\mathrm{K}}_2 + \Delta \; {\mathrm{U}}_2 ~=~ 0 \; \big ]$

\par \bigskip\medskip \noindent The mechanical energy $[ \, {\mathrm{E}}_3 \, ]$ and the mechanical energy $[ \, {\mathrm{E}}_4 \, ]$ of a system of N particles are always zero (\hspace{+0.180em}and therefore they always remain constant\hspace{+0.180em})

\par \bigskip\medskip ${\mathrm{E}}_3 ~=~ {\mathrm{constant}} \hspace{+3.00em} \big [ \; {\mathrm{E}}_3 ~=~ \sum_i^{\scriptscriptstyle{\mathrm{N}}} \, \med \; \big [ \, m_i \: {\mathbf{a}}_{\hspace{+0.045em}i} \cdot {\mathbf{r}}_{i} - {\mathbf{F}}_i \cdot {\mathbf{r}}_{i} \, \big ] ~=~ 0 \; \big ]$

\par \bigskip ${\mathrm{E}}_4 ~=~ {\mathrm{constant}} \hspace{+3.00em} \big [ \; {\mathrm{E}}_4 ~=~ \sum_i^{\scriptscriptstyle{\mathrm{N}}} \, \med \; \big [ \, m_i \: {\mathbf{a}}_{\hspace{+0.045em}i} \cdot ({\mathbf{r}}_{i} - {\mathbf{R}}_{cm}) - {\mathbf{F}}_i \cdot ({\mathbf{r}}_{i} - {\mathbf{R}}_{cm}) \, \big ] ~=~ 0 \; \big ]$

\par \bigskip $\hspace{+10.44em} \sum_i^{\scriptscriptstyle{\mathrm{N}}} \, \med \; m_i \, \big [ \, ({\mathbf{a}}_{\hspace{+0.045em}i} - {\mathbf{A}}_{cm}) \cdot ({\mathbf{r}}_{i} - {\mathbf{R}}_{cm}) \, \big ] \,=\, \sum_i^{\scriptscriptstyle{\mathrm{N}}} \, \med \; m_i \: {\mathbf{a}}_{\hspace{+0.045em}i} \cdot ({\mathbf{r}}_{i} - {\mathbf{R}}_{cm})$

\par \bigskip\medskip \noindent The mechanical energy $[ \, {\mathrm{E}}_5 \, ]$ and the mechanical energy $[ \, {\mathrm{E}}_6 \, ]$ of a system of N particles remain constant if the system is only subject to kinetic forces and to conservative non-kinetic forces.

\par \bigskip\medskip ${\mathrm{E}}_5 ~=~ {\mathrm{constant}} \hspace{+3.00em} \big [ \; \Delta \; {\mathrm{E}}_5 ~=~ \Delta \; {\mathrm{K}}_5 + \Delta \; {\mathrm{U}}_5 ~=~ 0 \; \big ]$

\par \bigskip ${\mathrm{E}}_6 ~=~ {\mathrm{constant}} \hspace{+3.00em} \big [ \; \Delta \; {\mathrm{E}}_6 ~=~ \Delta \; {\mathrm{K}}_6 + \Delta \; {\mathrm{U}}_6 ~=~ 0 \; \big ]$

\newpage

\par \bigskip {\centering\subsection*{Observations}}\addcontentsline{toc}{subsection}{8. Observations}

\par \bigskip\smallskip \noindent All equations of this paper can be applied in any inertial reference frame and also in any non-inertial reference frame.

\par \bigskip\smallskip \noindent Additionally, inertial reference frames and non-inertial reference frames must not introduce fictitious forces into ${\mathbf{F}}_i$.

\par \bigskip\smallskip \noindent In this paper, the magnitudes $[ \, {\mathit{m}},\spb {\mathbf{r}},\spb {\mathbf{v}},\spb {\mathbf{a}},\spb {\mathrm{M}},\spb {\mathbf{R}},\spb {\mathbf{V}}\hspace{-0.120em},\spb {\mathbf{A}},\spb {\mathbf{T}},\spb {\mathbf{K}},\spb {\mathbf{F}},\spb {\mathbf{P}}_1,\spb {\mathbf{L}}_1,\spb {\mathbf{L}}_2,\spb {\mathrm{W}}_1,\spb {\mathrm{K}}_1,\spb {\mathrm{U}}_1,\spb {\mathrm{E}}_1,\spb {\mathrm{L}}_1$ ${\mathrm{W}}_2,\spc {\mathrm{K}}_2,\spc {\mathrm{U}}_2,\spc {\mathrm{E}}_2,\spc {\mathrm{L}}_2,\spc {\mathrm{W}}_3,\spc {\mathrm{K}}_3,\spc {\mathrm{U}}_3,\spc {\mathrm{E}}_3,\spc {\mathrm{W}}_4,\spc {\mathrm{K}}_4,\spc {\mathrm{U}}_4,\spc {\mathrm{E}}_4,\spc {\mathrm{W}}_5,\spc {\mathrm{K}}_5,\spc {\mathrm{U}}_5,\spc {\mathrm{E}}_5,\spc {\mathrm{W}}_6,\spc {\mathrm{K}}_6,\spc {\mathrm{U}}_6$ and ${\mathrm{E}}_6 \, ]$ are invariant under transformations between inertial and non-inertial reference frames.

\par \bigskip\smallskip \noindent The mechanical energy ${\mathrm{E}}_3$ of a system of particles is always zero $[ \, {\mathrm{E}}_3 = {\mathrm{K}}_3 + {\mathrm{U}}_3 = 0 \, ]$

\par \bigskip\smallskip \noindent Therefore, the mechanical energy ${\mathrm{E}}_5$ of a system of particles is always equal to the mechanical energy ${\mathrm{E}}_1$ of the system of particles $[ \, {\mathrm{E}}_5 = {\mathrm{E}}_1 \, ]$

\par \bigskip\smallskip \noindent The mechanical energy ${\mathrm{E}}_4$ of a system of particles is always zero $[ \, {\mathrm{E}}_4 = {\mathrm{K}}_4 + {\mathrm{U}}_4 = 0 \, ]$

\par \bigskip\smallskip \noindent Therefore, the mechanical energy ${\mathrm{E}}_6$ of a system of particles is always equal to the mechanical energy ${\mathrm{E}}_2$ of the system of particles $[ \, {\mathrm{E}}_6 = {\mathrm{E}}_2 \, ]$

\par \bigskip\smallskip \noindent If the potential energy ${\mathrm{U}}_1$ of a system of particles is a homogeneous function of \hbox {degree ${\mathit{k}}$} \hbox {then the} potential energy ${\mathrm{U}}_3$ and the potential energy ${\mathrm{U}}_5$ of the system of particles are \hbox {given by}: $[ \, {\mathrm{U}}_3 = (\frac{{\mathit{k}}}{2}) \, {\mathrm{U}}_1 \, ]$ and $[ \, {\mathrm{U}}_5 = ({\scriptstyle 1 +} \frac{{\mathit{k}}}{2}) \, {\mathrm{U}}_1 \, ]$

\par \bigskip\smallskip \noindent If the potential energy ${\mathrm{U}}_2$ of a system of particles is a homogeneous function of \hbox {degree ${\mathit{k}}$} \hbox {then the} potential energy ${\mathrm{U}}_4$ and the potential energy ${\mathrm{U}}_6$ of the system of particles are \hbox {given by}: $[ \, {\mathrm{U}}_4 = (\frac{{\mathit{k}}}{2}) \, {\mathrm{U}}_2 \, ]$ and $[ \, {\mathrm{U}}_6 = ({\scriptstyle 1 +} \frac{{\mathit{k}}}{2}) \, {\mathrm{U}}_2 \, ]$

\par \bigskip\smallskip \noindent If the potential energy ${\mathrm{U}}_1$ of a system of particles is a homogeneous function of \hbox {degree ${\mathit{k}}$} and if the kinetic energy ${\mathrm{K}}_5$ of the system of particles is equal to zero, then we obtain: $[ \, {\mathrm{K}}_1 = - \, {\mathrm{K}}_3 = {\mathrm{U}}_3 = (\frac{{\mathit{k}}}{2}) \, {\mathrm{U}}_1 = (\frac{{\mathit{k}}}{2 + {\mathit{k}}}) \, {\mathrm{E}}_1 \, ]$

\par \bigskip\smallskip \noindent If the potential energy ${\mathrm{U}}_2$ of a system of particles is a homogeneous function of \hbox {degree ${\mathit{k}}$} and if the kinetic energy ${\mathrm{K}}_6$ of the system of particles is equal to zero, then we obtain: $[ \, {\mathrm{K}}_2 = - \, {\mathrm{K}}_4 = {\mathrm{U}}_4 = (\frac{{\mathit{k}}}{2}) \, {\mathrm{U}}_2 = (\frac{{\mathit{k}}}{2 + {\mathit{k}}}) \, {\mathrm{E}}_2 \, ]$

\par \bigskip\smallskip \noindent If the potential energy ${\mathrm{U}}_1$ of a system of particles is a homogeneous function of \hbox {degree ${\mathit{k}}$} \hbox {and if} the average kinetic energy $\langle {\mathrm{K}}_5 \rangle$ of the system of particles is equal to zero, then we obtain: $[ \, \langle {\mathrm{K}}_1 \rangle = - \, \langle {\mathrm{K}}_3 \rangle = \langle {\mathrm{U}}_3 \rangle = (\frac{{\mathit{k}}}{2}) \, \langle {\mathrm{U}}_1 \rangle = (\frac{{\mathit{k}}}{2 + {\mathit{k}}}) \, \langle {\mathrm{E}}_1 \rangle \, ]$

\par \bigskip\smallskip \noindent If the potential energy ${\mathrm{U}}_2$ of a system of particles is a homogeneous function of \hbox {degree ${\mathit{k}}$} \hbox {and if} the average kinetic energy $\langle {\mathrm{K}}_6 \rangle$ of the system of particles is equal to zero, then we obtain: $[ \, \langle {\mathrm{K}}_2 \rangle = - \, \langle {\mathrm{K}}_4 \rangle = \langle {\mathrm{U}}_4 \rangle = (\frac{{\mathit{k}}}{2}) \, \langle {\mathrm{U}}_2 \rangle = (\frac{{\mathit{k}}}{2 + {\mathit{k}}}) \, \langle {\mathrm{E}}_2 \rangle \, ]$

\newpage

\par \bigskip\smallskip \noindent The average kinetic energy $\langle {\mathrm{K}}_5 \rangle$ and the average kinetic energy $\langle {\mathrm{K}}_6 \rangle$ of a system of particles with bounded motion ( in $\langle {\mathrm{K}}_5 \rangle$ relative to ${\vec{\mathit{R}}}$ \hspace{+0.090em}and\hspace{+0.090em} in $\langle {\mathrm{K}}_6 \rangle$ relative to ${\vec{\mathit{R}}}_{cm}$ ) are always zero.

\par \bigskip\smallskip \noindent The kinetic energy ${\mathrm{K}}_5$ and the kinetic energy ${\mathrm{K}}_6$ of a system of N particles can also \hbox {be expressed} as follows : $[ \; {\mathrm{K}}_5 = \sum_i^{\scriptscriptstyle{\mathrm{N}}} \, \med \; m_i \, ( \, {\dot{\mathit{r}}}_{i} \, {\dot{\mathit{r}}}_{i} + {\ddot{\mathit{r}}}_{i} \, {\mathit{r}}_{i} \, ) \; ]$ where ${\mathit{r}}_{i} \doteq | \, {\vec{\mathit{r}}}_{i} - {\vec{\mathit{R}}} \, |$ and \hbox {$[ \; {\mathrm{K}}_6 = \sum_{j{\scriptscriptstyle >}i}^{\scriptscriptstyle{\mathrm{N}}} \, \med \; m_i \, m_j \, {\mathrm{M}}^{\scriptscriptstyle -1} ( \, {\dot{\mathit{r}}}_{\hspace{+0.060em}ij} \, {\dot{\mathit{r}}}_{\hspace{+0.060em}ij} + {\ddot{\mathit{r}}}_{\hspace{+0.060em}ij} \, {\mathit{r}}_{\hspace{+0.060em}ij} \, ) \; ]$} where ${\mathit{r}}_{\hspace{+0.060em}ij} \,\doteq\, | \: {\vec{\mathit{r}}}_{i} - {\vec{\mathit{r}}}_{j} \: |$

\par \bigskip\smallskip \noindent The kinetic energy ${\mathrm{K}}_5$ and the kinetic energy ${\mathrm{K}}_6$ of a system of N particles can also \hbox {be expressed} as follows : $[ \; {\mathrm{K}}_5 = \sum_i^{\scriptscriptstyle{\mathrm{N}}} \, \med \; m_i \, ( \, {\ddot{\tau}}_{\hspace{+0.045em}i} \, ) \; ]$ where ${\tau}_{i} \doteq \med \; ({\vec{\mathit{r}}}_{i} - {\vec{\mathit{R}}}) \cdot ({\vec{\mathit{r}}}_{i} - {\vec{\mathit{R}}})$ and \hbox {$[ \; {\mathrm{K}}_6 = \sum_{j{\scriptscriptstyle >}i}^{\scriptscriptstyle{\mathrm{N}}} \, \med \; m_i \, m_j \, {\mathrm{M}}^{\scriptscriptstyle -1} ( \, {\ddot{\tau}}_{\hspace{+0.060em}ij} \, ) \; ]$} where ${\tau}_{\hspace{+0.060em}ij} \doteq \med \; ({\vec{\mathit{r}}}_{i} - {\vec{\mathit{r}}}_{j}) \cdot ({\vec{\mathit{r}}}_{i} - {\vec{\mathit{r}}}_{j})$

\par \bigskip\smallskip \noindent The kinetic energy ${\mathrm{K}}_6$ is the only kinetic energy that can be expressed without the necessity of introducing any magnitude that is related to the Universe $[ \:\: $such as\hspace{+0.060em}: ${\mathbf{r}},\: {\mathbf{v}},\: {\mathbf{a}},\: {\vec{\omega}},\: {\vec{\mathit{R}}}$, etc.$ \:\: ]$

\par \bigskip\smallskip \noindent In an isolated system of particles, the potential energy ${\mathrm{U}}_2$ is equal to the potential energy ${\mathrm{U}}_1$ if the internal non-kinetic forces obey Newton's third law in its weak form $[ \, {\mathrm{U}}_2 = {\mathrm{U}}_1 \, ]$

\par \bigskip\smallskip \noindent In an isolated system of particles, the potential energy ${\mathrm{U}}_4$ is equal to the potential energy ${\mathrm{U}}_3$ if the internal non-kinetic forces obey Newton's third law in its weak form $[ \, {\mathrm{U}}_4 = {\mathrm{U}}_3 \, ]$

\par \bigskip\smallskip \noindent In an isolated system of particles, the potential energy ${\mathrm{U}}_6$ is equal to the potential energy ${\mathrm{U}}_5$ if the internal non-kinetic forces obey Newton's third law in its weak form $[ \, {\mathrm{U}}_6 = {\mathrm{U}}_5 \, ]$

\par \bigskip\smallskip \noindent A reference frame S is non-rotating if the angular velocity ${\vec{\omega}}$ of the Universe relative to S is equal to zero, and the reference frame S is also inertial if the acceleration ${\vec{\mathit{A}}}$ of the center of mass of the Universe relative to S is equal to zero.

\par \bigskip\smallskip \noindent If the origin of a non-rotating reference frame S $[ \, {\vec{\omega}} = 0 \, ]$ always coincides with the center of mass of the Universe $[ \, {\vec{\mathit{R}}} = {\vec{\mathit{V}}} = {\vec{\mathit{A}}} = 0 \, ]$ then relative to S: $[ \, {\mathbf{r}}_i = {\vec{\mathit{r}}}_i$, ${\mathbf{v}}_i = {\vec{\mathit{v}}}_i$ and ${\mathbf{a}}_{\hspace{+0.045em}i} = {\vec{\mathit{a}}}_i \, ]$ Therefore, it is easy to see that always: $[ \, {\mathbf{v}}_i = d(\hspace{+0.090em}{\mathbf{r}}_i\hspace{+0.090em})/dt$ $\;$and$\;$ ${\mathbf{a}}_{\hspace{+0.045em}i} = d^2(\hspace{+0.090em}{\mathbf{r}}_i\hspace{+0.090em})/dt^2 \, ]$

\par \bigskip\smallskip \noindent If kinetic forces are excluded, then this paper does not contradict Newton's first and second laws since they are valid in all inertial reference frames. The equation $[ \: {\mathbf{F}}_i \,=\, m_i \, {\mathbf{a}}_{\hspace{+0.045em}i} \: ]$ is a simple reformulation of Newton's second law.

\par \bigskip\smallskip \noindent In this paper, the equation $[ \: {\mathbf{F}}_i \,=\, m_i \, {\mathbf{a}}_{\hspace{+0.045em}i} \: ]$ would be false in all reference frames (\hspace{+0.180em}inertial \hbox {or non-inertial\hspace{+0.180em})} if a new non-kinetic force were always disobeyed Newton's third law in its strong form or in its weak form.

\vspace{-0.90em}

\par \bigskip {\centering\subsection*{Bibliography}}\addcontentsline{toc}{subsection}{9. Bibliography}

\par \bigskip\smallskip \noindent \textbf{A. Einstein}, Relativity: The Special and General Theory.

\par \bigskip\smallskip \noindent \textbf{A. Torassa}, A Reformulation of Classical Mechanics.

\par \bigskip\smallskip \noindent \textbf{E. Mach}, The Science of Mechanics.

\newpage

\par \bigskip {\centering\subsection*{Appendix I}}

\par \medskip {\centering\subsubsection*{The Universe}}\addcontentsline{toc}{subsection}{Appendix I : The Universe}\hypertarget{p2a1}{}

\par \bigskip \noindent The Universe is a system that contains all particles, that is always free of external forces, and that all internal non-kinetic forces always obey Newton's third law in its weak form and in its strong form.

\par \bigskip \noindent The position ${\vec{\mathit{R}}}$, the velocity ${\vec{\mathit{V}}}$ and the acceleration ${\vec{\mathit{A}}}$ of the center of mass of the Universe relative to a reference frame S (and the angular velocity ${\vec{\omega}}$ and the angular acceleration ${\vec{\alpha}}$ \hbox {of the Universe} relative to the reference frame S) are given by:

\par \bigskip\smallskip \hspace{-2.40em} \begin{tabular}{l}
${\mathrm{M}} ~\doteq~ \sum_i^{\scriptscriptstyle{\mathit{All}}} \, m_i$ \vspace{+1.20em} \\
${\vec{\mathit{R}}} ~\doteq~ {\mathrm{M}}^{\scriptscriptstyle -1} \, \sum_i^{\scriptscriptstyle{\mathit{All}}} \, m_i \, {\vec{\mathit{r}}_{i}}$ \vspace{+1.20em} \\
${\vec{\mathit{V}}} ~\doteq~ {\mathrm{M}}^{\scriptscriptstyle -1} \, \sum_i^{\scriptscriptstyle{\mathit{All}}} \, m_i \, {\vec{\mathit{v}}_{i}}$ \vspace{+1.20em} \\
${\vec{\mathit{A}}} ~\doteq~ {\mathrm{M}}^{\scriptscriptstyle -1} \, \sum_i^{\scriptscriptstyle{\mathit{All}}} \, m_i \, {\vec{\mathit{a}}_{i}}$ \vspace{+1.20em} \\
${\vec{\omega}} ~\doteq~ {\mathit{I}}^{\scriptscriptstyle -1}{\vphantom{\sum_1^2}}^{\hspace{-1.500em}\leftrightarrow}\hspace{+0.600em} \cdot {\vec{\mathit{L}}}$ \vspace{+1.20em} \\
${\vec{\alpha}} ~\doteq~ d({\vec{\omega}})/dt$ \vspace{+1.20em} \\
${\mathit{I}}{\vphantom{\sum_1^2}}^{\hspace{-0.555em}\leftrightarrow}\hspace{-0.210em} ~\doteq~ \sum_i^{\scriptscriptstyle{\mathit{All}}} \, m_i \, [ \, |\hspace{+0.090em}{\vec{\mathit{r}}_{i}} - {\vec{\mathit{R}}}\,|^2 \hspace{+0.309em} {\mathrm{1}}{\vphantom{\sum_1^2}}^{\hspace{-0.639em}\leftrightarrow}\hspace{-0.129em} - ({\vec{\mathit{r}}_{i}} - {\vec{\mathit{R}}}) \otimes ({\vec{\mathit{r}}_{i}} - {\vec{\mathit{R}}}) \, ]$ \vspace{+1.20em} \\
${\vec{\mathit{L}}} ~\doteq~ \sum_i^{\scriptscriptstyle{\mathit{All}}} \, m_i \, ({\vec{\mathit{r}}_{i}} - {\vec{\mathit{R}}}) \times ({\vec{\mathit{v}}_{i}} - \hspace{-0.120em}{\vec{\mathit{V}}})$
\end{tabular}

\par \bigskip \noindent where ${\mathrm{M}}$ is the mass of the Universe, ${\mathit{I}}{\vphantom{\sum_1^2}}^{\hspace{-0.555em}\leftrightarrow}\hspace{-0.300em}$ is the inertia tensor of the Universe (relative \hbox {to ${\vec{\mathit{R}}}$)} and ${\vec{\mathit{L}}}$ is the angular momentum of the Universe relative to the reference frame S.

\vspace{+1.50em}

\par {\centering\subsubsection*{The Transformations}}\addcontentsline{toc}{subsection}{Appendix I : The Transformations}

\par \bigskip\medskip \hspace{-1.80em} $({\vec{\mathit{r}}}_i - {\vec{\mathit{R}}}) ~\doteq~ {\mathbf{r}}_i ~=~ {\mathbf{r}}_i\hspace{-0.300em}'$

\par \bigskip \hspace{-1.80em} $({\vec{\mathit{r}}}_i\hspace{-0.150em}' - {\vec{\mathit{R}}}\hspace{+0.015em}') ~\doteq~ {\mathbf{r}}_i\hspace{-0.300em}' ~=~ {\mathbf{r}}_i$

\par \bigskip \hspace{-1.80em} $({\vec{\mathit{v}}}_i - \hspace{-0.120em}{\vec{\mathit{V}}}) - {\vec{\omega}} \times ({\vec{\mathit{r}}}_i - {\vec{\mathit{R}}}) ~\doteq~ {\mathbf{v}}_i ~=~ {\mathbf{v}}_i\hspace{-0.300em}'$

\par \bigskip \hspace{-1.80em} $({\vec{\mathit{v}}}_i\hspace{-0.150em}' - \hspace{-0.120em}{\vec{\mathit{V}}}\hspace{-0.045em}') - {\vec{\omega}}\hspace{+0.060em}' \times ({\vec{\mathit{r}}}_i\hspace{-0.150em}' - {\vec{\mathit{R}}}\hspace{+0.015em}') ~\doteq~ {\mathbf{v}}_i\hspace{-0.300em}' ~=~ {\mathbf{v}}_i$

\par \bigskip \hspace{-1.80em} $({\vec{\mathit{a}}}_i - {\vec{\mathit{A}}}) - 2 \; {\vec{\omega}} \times ({\vec{\mathit{v}}}_i - \hspace{-0.120em}{\vec{\mathit{V}}}) + {\vec{\omega}} \times [ \, {\vec{\omega}} \times ({\vec{\mathit{r}}}_i - {\vec{\mathit{R}}}) \, ] - {\vec{\alpha}} \times ({\vec{\mathit{r}}}_i - {\vec{\mathit{R}}}) ~\doteq~ {\mathbf{a}}_{\hspace{+0.045em}i} ~=~ {\mathbf{a}}_{\hspace{+0.045em}i}\hspace{-0.360em}'$

\par \bigskip \hspace{-1.80em} $({\vec{\mathit{a}}}_i\hspace{-0.150em}' - {\vec{\mathit{A}}}\hspace{-0.045em}') - 2 \; {\vec{\omega}}\hspace{+0.060em}' \times ({\vec{\mathit{v}}}_i\hspace{-0.150em}' - \hspace{-0.120em}{\vec{\mathit{V}}}\hspace{-0.045em}') + {\vec{\omega}}\hspace{+0.060em}' \times [ \, {\vec{\omega}}\hspace{+0.060em}' \times ({\vec{\mathit{r}}}_i\hspace{-0.150em}' - {\vec{\mathit{R}}}\hspace{+0.015em}') \, ] - {\vec{\alpha}}\hspace{+0.060em}' \times ({\vec{\mathit{r}}}_i\hspace{-0.150em}' - {\vec{\mathit{R}}}\hspace{+0.015em}') ~\doteq~ {\mathbf{a}}_{\hspace{+0.045em}i}\hspace{-0.360em}' ~=~ {\mathbf{a}}_{\hspace{+0.045em}i}$

\newpage

\par \bigskip {\centering\subsection*{Appendix II}}

\par \medskip {\centering\subsubsection*{The Relations}}\addcontentsline{toc}{subsection}{Appendix II : The Relations}\hypertarget{p2a2}{}

\par \bigskip \noindent In a system of particles, these relations can be obtained ( The magnitudes ${\mathbf{R}}_{cm}$, ${\mathbf{V}}_{cm}$, ${\mathbf{A}}_{cm}$, ${\vec{\mathit{R}}}_{cm}$, ${\vec{\mathit{V}}}_{cm}$ and ${\vec{\mathit{A}}}_{cm}$ can be replaced by the magnitudes ${\mathbf{R}}$, ${\mathbf{V}}$, ${\mathbf{A}}$, ${\vec{\mathit{R}}}$, ${\vec{\mathit{V}}}$ and ${\vec{\mathit{A}}}$, or by the magnitudes ${\mathbf{r}}_j$, ${\mathbf{v}}_j$, ${\mathbf{a}}_j$, ${\vec{\mathit{r}}}_j$, ${\vec{\mathit{v}}}_j$ and ${\vec{\mathit{a}}}_j$,{\vphantom{\LARGE t}} respectively. On the other hand, $\: {\mathbf{R}} \:=\: {\mathbf{V}} \:=\: {\mathbf{A}} \:=\: 0 \;$)

\par \bigskip\medskip \noindent ${\mathbf{r}}_i \,\doteq\, ({\vec{\mathit{r}}}_i - {\vec{\mathit{R}}})$

\par \bigskip\smallskip \noindent ${\mathbf{R}}_{cm} \,\doteq\, ({\vec{\mathit{R}}}_{cm} - {\vec{\mathit{R}}})$

\par \bigskip\smallskip \noindent $\longrightarrow \hspace{+0.90em} ({\mathbf{r}}_i - {\mathbf{R}}_{cm}) \,=\, ({\vec{\mathit{r}}}_i - {\vec{\mathit{R}}}_{cm})$

\par \bigskip\smallskip \noindent ${\mathbf{v}}_i \,\doteq\, ({\vec{\mathit{v}}}_i - \hspace{-0.120em}{\vec{\mathit{V}}}) - {\vec{\omega}} \times ({\vec{\mathit{r}}}_i - {\vec{\mathit{R}}})$

\par \bigskip\smallskip \noindent ${\mathbf{V}}_{cm} \,\doteq\, ({\vec{\mathit{V}}}_{cm} - \hspace{-0.120em}{\vec{\mathit{V}}}) - {\vec{\omega}} \times ({\vec{\mathit{R}}}_{cm} - {\vec{\mathit{R}}})$

\par \bigskip\smallskip \noindent $\longrightarrow \hspace{+0.90em} ({\mathbf{v}}_i - {\mathbf{V}}_{cm}) \,=\, ({\vec{\mathit{v}}}_i - \hspace{-0.120em}{\vec{\mathit{V}}}_{cm}) - {\vec{\omega}} \times ({\vec{\mathit{r}}}_i - {\vec{\mathit{R}}}_{cm})$

\par \bigskip\smallskip \noindent $({\mathbf{v}}_i - {\mathbf{V}}_{cm}) \cdot ({\mathbf{v}}_i - {\mathbf{V}}_{cm}) \,=\, \big [ \, ({\vec{\mathit{v}}}_i - \hspace{-0.120em}{\vec{\mathit{V}}}_{cm}) - {\vec{\omega}} \times ({\vec{\mathit{r}}}_i - {\vec{\mathit{R}}}_{cm}) \, \big ] \cdot \big [ \, ({\vec{\mathit{v}}}_i - \hspace{-0.120em}{\vec{\mathit{V}}}_{cm}) - {\vec{\omega}} \times ({\vec{\mathit{r}}}_i - {\vec{\mathit{R}}}_{cm}) \, \big ] \,=$

\par \bigskip\smallskip \noindent $({\vec{\mathit{v}}}_i - \hspace{-0.120em}{\vec{\mathit{V}}}_{cm}) \cdot ({\vec{\mathit{v}}}_i - \hspace{-0.120em}{\vec{\mathit{V}}}_{cm}) - 2 \, ({\vec{\mathit{v}}}_i - \hspace{-0.120em}{\vec{\mathit{V}}}_{cm}) \cdot \big [ \, {\vec{\omega}} \times ({\vec{\mathit{r}}}_i - {\vec{\mathit{R}}}_{cm}) \, \big ] + \big [ \, {\vec{\omega}} \times ({\vec{\mathit{r}}}_i - {\vec{\mathit{R}}}_{cm}) \, \big ] \cdot \big [ \, {\vec{\omega}} \times ({\vec{\mathit{r}}}_i - {\vec{\mathit{R}}}_{cm}) \, \big ] \,=$

\par \bigskip\smallskip \noindent $({\vec{\mathit{v}}}_i - \hspace{-0.120em}{\vec{\mathit{V}}}_{cm}) \cdot ({\vec{\mathit{v}}}_i - \hspace{-0.120em}{\vec{\mathit{V}}}_{cm}) + 2 \, ({\vec{\mathit{r}}}_i - {\vec{\mathit{R}}}_{cm}) \cdot \big [ \, {\vec{\omega}} \times ({\vec{\mathit{v}}}_i - \hspace{-0.120em}{\vec{\mathit{V}}}_{cm}) \, \big ] + \big [ \, {\vec{\omega}} \times ({\vec{\mathit{r}}}_i - {\vec{\mathit{R}}}_{cm}) \, \big ] \cdot \big [ \, {\vec{\omega}} \times ({\vec{\mathit{r}}}_i - {\vec{\mathit{R}}}_{cm}) \, \big ] \,=$

\par \bigskip\smallskip \noindent $({\vec{\mathit{v}}}_i - \hspace{-0.120em}{\vec{\mathit{V}}}_{cm}) \cdot ({\vec{\mathit{v}}}_i - \hspace{-0.120em}{\vec{\mathit{V}}}_{cm}) + \big [ \, 2 \; {\vec{\omega}} \times ({\vec{\mathit{v}}}_i - \hspace{-0.120em}{\vec{\mathit{V}}}_{cm}) \, \big ] \cdot ({\vec{\mathit{r}}}_i - {\vec{\mathit{R}}}_{cm}) + \big [ \, {\vec{\omega}} \times ({\vec{\mathit{r}}}_i - {\vec{\mathit{R}}}_{cm}) \, \big ] \cdot \big [ \, {\vec{\omega}} \times ({\vec{\mathit{r}}}_i - {\vec{\mathit{R}}}_{cm}) \, \big ] \,=$

\par \bigskip\smallskip \noindent $({\vec{\mathit{v}}}_i - \hspace{-0.120em}{\vec{\mathit{V}}}_{cm})^2 + \big [ \, 2 \; {\vec{\omega}} \times ({\vec{\mathit{v}}}_i - \hspace{-0.120em}{\vec{\mathit{V}}}_{cm}) \, \big ] \cdot ({\vec{\mathit{r}}}_i - {\vec{\mathit{R}}}_{cm}) + \big [ \, {\vec{\omega}} \times ({\vec{\mathit{r}}}_i - {\vec{\mathit{R}}}_{cm}) \, \big ]^2$

\par \bigskip\smallskip \noindent $({\mathbf{a}}_{\hspace{+0.045em}i} \,-\, {\mathbf{A}}_{cm}) \cdot ({\mathbf{r}}_i \,-\, {\mathbf{R}}_{cm}) \;=\; \big \{ \, ({\vec{\mathit{a}}}_i \,-\, {\vec{\mathit{A}}}_{cm}) \;-\; 2 \; {\vec{\omega}} \times ({\vec{\mathit{v}}}_i \,-\, \hspace{-0.120em}{\vec{\mathit{V}}}_{cm}) \;+\; {\vec{\omega}} \times [ \, {\vec{\omega}} \times ({\vec{\mathit{r}}}_i \,-\, {\vec{\mathit{R}}}_{cm}) \, ] ~\,-$

\par \bigskip\smallskip \noindent ${\vec{\alpha}} \times ({\vec{\mathit{r}}}_i - {\vec{\mathit{R}}}_{cm}) \, \big \} \cdot ({\vec{\mathit{r}}}_i - {\vec{\mathit{R}}}_{cm}) \,=\, ({\vec{\mathit{a}}}_i - {\vec{\mathit{A}}}_{cm}) \cdot ({\vec{\mathit{r}}}_i - {\vec{\mathit{R}}}_{cm}) - \big [ \, 2 \; {\vec{\omega}} \times ({\vec{\mathit{v}}}_i - \hspace{-0.120em}{\vec{\mathit{V}}}_{cm}) \, \big ] \cdot ({\vec{\mathit{r}}}_i - {\vec{\mathit{R}}}_{cm}) ~\,+$

\par \bigskip\smallskip \noindent $\big \{ \, {\vec{\omega}} \times [ \, {\vec{\omega}} \times ({\vec{\mathit{r}}}_i - {\vec{\mathit{R}}}_{cm}) \, ] \, \big \} \cdot ({\vec{\mathit{r}}}_i - {\vec{\mathit{R}}}_{cm}) - \big [ \, {\vec{\alpha}} \times ({\vec{\mathit{r}}}_i - {\vec{\mathit{R}}}_{cm}) \, \big ] \cdot ({\vec{\mathit{r}}}_i - {\vec{\mathit{R}}}_{cm}) \,=\, ({\vec{\mathit{a}}}_i - {\vec{\mathit{A}}}_{cm}) \cdot ({\vec{\mathit{r}}}_i - {\vec{\mathit{R}}}_{cm}) ~\,-$

\par \bigskip\smallskip \noindent $\big [ \, 2 \; {\vec{\omega}} \times ({\vec{\mathit{v}}}_i - \hspace{-0.120em}{\vec{\mathit{V}}}_{cm}) \, \big ] \cdot ({\vec{\mathit{r}}}_i - {\vec{\mathit{R}}}_{cm}) + \big \{ \, \big [ \, {\vec{\omega}} \cdot ({\vec{\mathit{r}}}_i - {\vec{\mathit{R}}}_{cm}) \, \big ] \: {\vec{\omega}} - ( \, {\vec{\omega}} \cdot {\vec{\omega}} \, ) \: ({\vec{\mathit{r}}}_i - {\vec{\mathit{R}}}_{cm}) \, \big \} \cdot ({\vec{\mathit{r}}}_i - {\vec{\mathit{R}}}_{cm}) \,=$

\par \bigskip\smallskip \noindent $({\vec{\mathit{a}}}_i - {\vec{\mathit{A}}}_{cm}) \cdot ({\vec{\mathit{r}}}_i - {\vec{\mathit{R}}}_{cm}) - \big [ \, 2 \; {\vec{\omega}} \times ({\vec{\mathit{v}}}_i - \hspace{-0.120em}{\vec{\mathit{V}}}_{cm}) \, \big ] \cdot ({\vec{\mathit{r}}}_i - {\vec{\mathit{R}}}_{cm}) + \big [ \, {\vec{\omega}} \hspace{+0.120em}\cdot\hspace{+0.090em} ({\vec{\mathit{r}}}_i - {\vec{\mathit{R}}}_{cm}) \, \big ]^2 -\, ( \, {\vec{\omega}} \, )^2 \: ({\vec{\mathit{r}}}_i - {\vec{\mathit{R}}}_{cm})^2$

\par \bigskip\smallskip \noindent $\longrightarrow \hspace{+0.90em} ({\mathbf{v}}_i - {\mathbf{V}}_{cm})^2 + ({\mathbf{a}}_{\hspace{+0.045em}i} - {\mathbf{A}}_{cm}) \cdot ({\mathbf{r}}_i - {\mathbf{R}}_{cm}) \,=\, ({\vec{\mathit{v}}}_i - \hspace{-0.120em}{\vec{\mathit{V}}}_{cm})^2 + ({\vec{\mathit{a}}}_i - {\vec{\mathit{A}}}_{cm}) \cdot ({\vec{\mathit{r}}}_i - {\vec{\mathit{R}}}_{cm})$

\newpage

\par \bigskip {\centering\subsection*{Appendix III}}

\par \medskip {\centering\subsubsection*{The Magnitudes}}\addcontentsline{toc}{subsection}{Appendix III : The Magnitudes}

\par \bigskip \noindent The magnitudes ${\mathbf{L}}_2$, ${\mathrm{W}}_2$, ${\mathrm{K}}_2$, ${\mathrm{U}}_2$, ${\mathrm{W}}_4$, ${\mathrm{K}}_4$, ${\mathrm{U}}_4$, ${\mathrm{W}}_6$, ${\mathrm{K}}_6$ and ${\mathrm{U}}_6$ of a system of N particles can also be expressed as follows:

\par \bigskip\bigskip \noindent ${\mathbf{L}}_2 ~=~ \sum_{j{\scriptscriptstyle >}i}^{\scriptscriptstyle{\mathrm{N}}} \, m_i \, m_j \, {\mathrm{M}}^{\scriptscriptstyle -1} \big [ \, ({\mathbf{r}}_{i} - {\mathbf{r}}_{j}) \times ({\mathbf{v}}_{i} - {\mathbf{v}}_{j}) \, \big ]$

\par \bigskip\bigskip \noindent ${\mathrm{W}}_2 ~=~ \sum_{j{\scriptscriptstyle >}i}^{\scriptscriptstyle{\mathrm{N}}} \, m_i \, m_j \, {\mathrm{M}}^{\scriptscriptstyle -1} \big [ \int_{\scriptscriptstyle 1}^{\scriptscriptstyle 2} \, ({\mathbf{F}}_i / m_i - {\mathbf{F}}_j / m_j) \cdot d({\mathbf{r}}_{i} - {\mathbf{r}}_{j}) \, \big ]$

\par \bigskip\smallskip \noindent $\Delta \, {\mathrm{K}}_2 ~=~ \sum_{j{\scriptscriptstyle >}i}^{\scriptscriptstyle{\mathrm{N}}} \, \Delta \, \med \; m_i \, m_j \, {\mathrm{M}}^{\scriptscriptstyle -1} \, ({\mathbf{v}}_{i} - {\mathbf{v}}_{j})^2 ~=~ {\mathrm{W}}_2$

\par \bigskip\smallskip \noindent $\Delta \, {\mathrm{U}}_2 ~=~ - \, \sum_{j{\scriptscriptstyle >}i}^{\scriptscriptstyle{\mathrm{N}}} \, m_i \, m_j \, {\mathrm{M}}^{\scriptscriptstyle -1} \big [ \int_{\scriptscriptstyle 1}^{\scriptscriptstyle 2} \, ({\mathbf{F}}_i / m_i - {\mathbf{F}}_j / m_j) \cdot d({\mathbf{r}}_{i} - {\mathbf{r}}_{j}) \, \big ]$

\par \bigskip\bigskip \noindent ${\mathrm{W}}_4 ~=~ \sum_{j{\scriptscriptstyle >}i}^{\scriptscriptstyle{\mathrm{N}}} \, \Delta \, \med \; m_i \, m_j \, {\mathrm{M}}^{\scriptscriptstyle -1} \big [ \, ({\mathbf{F}}_i / m_i - {\mathbf{F}}_j / m_j) \cdot ({\mathbf{r}}_{i} - {\mathbf{r}}_{j}) \, \big ]$

\par \bigskip\smallskip \noindent $\Delta \, {\mathrm{K}}_4 ~=~ \sum_{j{\scriptscriptstyle >}i}^{\scriptscriptstyle{\mathrm{N}}} \, \Delta \, \med \; m_i \, m_j \, {\mathrm{M}}^{\scriptscriptstyle -1} \big [ \, ({\mathbf{a}}_{\hspace{+0.045em}i} - {\mathbf{a}}_{j}) \cdot ({\mathbf{r}}_{i} - {\mathbf{r}}_{j}) \, \big ] ~=~ {\mathrm{W}}_4$

\par \bigskip\smallskip \noindent $\Delta \, {\mathrm{U}}_4 ~=~ - \, \sum_{j{\scriptscriptstyle >}i}^{\scriptscriptstyle{\mathrm{N}}} \, \Delta \, \med \; m_i \, m_j \, {\mathrm{M}}^{\scriptscriptstyle -1} \big [ \, ({\mathbf{F}}_i / m_i - {\mathbf{F}}_j / m_j) \cdot ({\mathbf{r}}_{i} - {\mathbf{r}}_{j}) \, \big ]$

\par \bigskip\bigskip \noindent ${\mathrm{W}}_6 ~=~ \sum_{j{\scriptscriptstyle >}i}^{\scriptscriptstyle{\mathrm{N}}} \, m_i \, m_j \, {\mathrm{M}}^{\scriptscriptstyle -1} \big [ \int_{\scriptscriptstyle 1}^{\scriptscriptstyle 2} \, ({\mathbf{F}}_i / m_i - {\mathbf{F}}_j / m_j) \cdot d({\vec{\mathit{r}}_{i}} - {\vec{\mathit{r}}_{j}}) + \Delta \, \med \; ({\mathbf{F}}_i / m_i - {\mathbf{F}}_j / m_j) \cdot ({\vec{\mathit{r}}_{i}} - {\vec{\mathit{r}}_{j}}) \, \big ]$

\par \bigskip\smallskip \noindent $\Delta \, {\mathrm{K}}_6 ~=~ \sum_{j{\scriptscriptstyle >}i}^{\scriptscriptstyle{\mathrm{N}}} \, \Delta \, \med \; m_i \, m_j \, {\mathrm{M}}^{\scriptscriptstyle -1} \big [ \, ({\vec{\mathit{v}}_{i}} - {\vec{\mathit{v}}_{j}})^2 + ({\vec{\mathit{a}}_{i}} - {\vec{\mathit{a}}_{j}}) \cdot ({\vec{\mathit{r}}_{i}} - {\vec{\mathit{r}}_{j}}) \, \big ] ~=~ {\mathrm{W}}_6$

\par \bigskip\smallskip \noindent $\Delta \, {\mathrm{U}}_6 ~=~ - \, \sum_{j{\scriptscriptstyle >}i}^{\scriptscriptstyle{\mathrm{N}}} \, m_i \, m_j \, {\mathrm{M}}^{\scriptscriptstyle -1} \big [ \int_{\scriptscriptstyle 1}^{\scriptscriptstyle 2} \, ({\mathbf{F}}_i / m_i - {\mathbf{F}}_j / m_j) \cdot d({\vec{\mathit{r}}_{i}} - {\vec{\mathit{r}}_{j}}) + \Delta \, \med \; ({\mathbf{F}}_i / m_i - {\mathbf{F}}_j / m_j) \cdot ({\vec{\mathit{r}}_{i}} - {\vec{\mathit{r}}_{j}}) \, \big ]$

\par \bigskip\bigskip \noindent The magnitudes ${\mathrm{W}}_{(1\;{\mathrm{to}}\;6)}$ and ${\mathrm{U}}_{(1\;{\mathrm{to}}\;6)}$ of an isolated system of N particles, whose internal non-kinetic forces obey Newton's third law in its weak form, can be reduced to:

\par \bigskip\bigskip \noindent ${\mathrm{W}}_1 ~=~ {\mathrm{W}}_2 ~=~ \sum_i^{\scriptscriptstyle{\mathrm{N}}} \int_{\scriptscriptstyle 1}^{\scriptscriptstyle 2} \, {\mathbf{F}}_i \cdot d{\vec{\mathit{r}}_{i}}$

\par \bigskip\smallskip \noindent $\Delta \, {\mathrm{U}}_1 ~=~ \Delta \, {\mathrm{U}}_2 ~=~ - \, \sum_i^{\scriptscriptstyle{\mathrm{N}}} \int_{\scriptscriptstyle 1}^{\scriptscriptstyle 2} \, {\mathbf{F}}_i \cdot d{\vec{\mathit{r}}_{i}}$

\par \bigskip\bigskip \noindent ${\mathrm{W}}_3 ~=~ {\mathrm{W}}_4 ~=~ \sum_i^{\scriptscriptstyle{\mathrm{N}}} \Delta \, \med \; {\mathbf{F}}_i \cdot {\vec{\mathit{r}}_{i}}$

\par \bigskip\smallskip \noindent $\Delta \, {\mathrm{U}}_3 ~=~ \Delta \, {\mathrm{U}}_4 ~=~ - \, \sum_i^{\scriptscriptstyle{\mathrm{N}}} \Delta \, \med \; {\mathbf{F}}_i \cdot {\vec{\mathit{r}}_{i}}$

\par \bigskip\bigskip \noindent ${\mathrm{W}}_5 ~=~ {\mathrm{W}}_6 ~=~ \sum_i^{\scriptscriptstyle{\mathrm{N}}} \big [ \int_{\scriptscriptstyle 1}^{\scriptscriptstyle 2} \, {\mathbf{F}}_i \cdot d{\vec{\mathit{r}}_{i}} + \Delta \, \med \; {\mathbf{F}}_i \cdot {\vec{\mathit{r}}_{i}} \, \big ]$

\par \bigskip\smallskip \noindent $\Delta \, {\mathrm{U}}_5 ~=~ \Delta \, {\mathrm{U}}_6 ~=~ - \, \sum_i^{\scriptscriptstyle{\mathrm{N}}} \big [ \int_{\scriptscriptstyle 1}^{\scriptscriptstyle 2} \, {\mathbf{F}}_i \cdot d{\vec{\mathit{r}}_{i}} + \Delta \, \med \; {\mathbf{F}}_i \cdot {\vec{\mathit{r}}_{i}} \, \big ]$

\end{document}

