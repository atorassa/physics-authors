
\documentclass[10pt,fleqn]{article}
%\documentclass[a4paper,10pt]{article}
%\documentclass[letterpaper,10pt]{article}

\usepackage[dvips]{geometry}
\geometry{papersize={131.1mm,210.6mm}}
\geometry{totalwidth=110.1mm,totalheight=174.6mm}

\usepackage{graphicx}

\usepackage[english]{babel}
\usepackage{times}
\usepackage{amsfonts}
\usepackage{amsmath,bm}

\usepackage{hyperref}
\hypersetup{colorlinks=true,linkcolor=black}
\hypersetup{bookmarksnumbered=true,pdfstartview=FitH,pdfpagemode=UseNone}
\hypersetup{pdftitle={A Relational Formulation of Special Relativity}}
\hypersetup{pdfauthor={A. Blato}}

\setlength{\arraycolsep}{1.74pt}

\newcommand{\yya}{27}
\newcommand{\xxa}{33}
\newcommand{\xxc}{45}

\begin{document}

\begin{center}

{\fontsize{10.92}{10.92}\selectfont \sc A Relational Formulation of Special Relativity}

\bigskip \medskip

{\fontsize{9.60}{9.60}\selectfont A. Blato}

\bigskip \medskip

\small

Creative Commons Attribution 3.0 License

\smallskip

(2016) Buenos Aires

\medskip

Argentina

\smallskip

\bigskip \medskip

\parbox{97.80mm}{This article presents a relational formulation of special relativity whose \hbox {kinematic} and dynamic quantities are invariant under transformations between inertial reference frames. In addition, a new universal force is presented.}

\end{center}

\normalsize

\vspace{-1.20em}

\par \bigskip {\centering\subsubsection*{Introduction}}

\bigskip \smallskip

\noindent From an auxiliary massive particle (called auxiliary-point) can be obtained kinematic quantities (such as relational time, relational position, etc.) that are invariant under transformations between inertial reference frames.
\par \bigskip \smallskip
\noindent An auxiliary-point is an arbitrary massive particle free of external forces (or that the net force acting on it is zero)
\par \bigskip \smallskip
\noindent The relational time $( \, \bar{t} \, )$, the relational position $( \, \bar{\mathbf{r}} \, )$, the relational velocity $( \, \bar{\mathbf{v}} \, )$ and the relational acceleration $( \, \bar{\mathbf{a}} \, )$ of a (massive or non-massive) particle \hbox {relative} to an inertial reference frame S are given by:
\par \vspace{+0.21em}
\begin{eqnarray*}
\bar{t} ~\doteq~ \hspace{+0.24em} \gamma \left ( t - \frac{\mathbf{r} \cdot \boldsymbol{\psi}}{c^2} \right )
\end{eqnarray*}
\vspace{-0.45em}
\begin{eqnarray*}
\bar{\mathbf{r}} ~\doteq~ \hspace{-0.09em} \left [ \; \mathbf{r} + \frac{\gamma^2}{\gamma + 1} \frac{( \mathbf{r} \cdot \boldsymbol{\psi} ) \, \boldsymbol{\psi}}{c^2} - \gamma \, \boldsymbol{\psi} \, t \; \right ]
\end{eqnarray*}
\vspace{-0.30em}
\begin{eqnarray*}
\bar{\mathbf{v}} \hspace{-0.12em} ~\doteq~ \hspace{-0.12em} \left [ \; \mathbf{v} + \frac{\gamma^2}{\gamma + 1} \frac{( \mathbf{v} \cdot \boldsymbol{\psi} ) \, \boldsymbol{\psi}}{c^2} - \hspace{+0.114em} \gamma \, \boldsymbol{\psi} \hspace{+0.114em} \; \right ] \frac{1}{\gamma \, ( 1 - \frac{\mathbf{v} \cdot \boldsymbol{\psi}}{c^2} )}
\end{eqnarray*}
\vspace{-0.30em}
\begin{eqnarray*}
\bar{\mathbf{a}} ~\doteq~ \hspace{-0.18em} \left [ \; \mathbf{a} - \frac{\gamma}{\gamma + 1} \frac{( \mathbf{a} \cdot \boldsymbol{\psi} ) \, \boldsymbol{\psi}}{c^2} + \frac{( \mathbf{a} \times \mathbf{v} ) \times \boldsymbol{\psi}}{c^2} \; \right ] \frac{1}{\gamma^2 \, ( 1 - \frac{\mathbf{v} \cdot \boldsymbol{\psi}}{c^2} )^3}
\end{eqnarray*}
\par \vspace{+1.20em}
\noindent where $( \, t, \, \mathbf{r}, \, \mathbf{v}, \, \mathbf{a} \, )$ are the time, the position, the velocity and the acceleration of the particle relative to the inertial reference frame S, $( \, \boldsymbol{\psi} \, )$ is the velocity of the auxiliary-point relative to the inertial reference frame S and $( \, c \, )$ is the speed of light in vacuum. $( \, \boldsymbol{\psi} \, )$ is a constant. {\small $\gamma = ({1 - \boldsymbol{\psi} \cdot \boldsymbol{\psi}/c^2})^{-1/2}$}

\newpage

\noindent The relational frequency $( \, \bar{\nu} \, )$ of a non-massive particle relative to an inertial reference frame S is given by:
\par \vspace{-0.60em}
\begin{eqnarray*}
\bar{\nu} ~\doteq~ \nu \;\, \dfrac{\left ( 1 - \dfrac{\mathbf{c} \cdot \boldsymbol{\psi}}{c^2} \right )}{\sqrt{1 - \dfrac{\boldsymbol{\psi} \cdot \boldsymbol{\psi}}{c^2}}}
\end{eqnarray*}
\par \vspace{+0.45em}
\noindent where $( \, \nu \, )$ is the frequency of the non-massive particle relative to the inertial reference frame S, $( \, \mathbf{c} \, )$ is the velocity of the non-massive particle relative to the inertial reference frame S, $( \, \boldsymbol{\psi} \, )$ is the velocity of the auxiliary-point relative to the inertial reference frame S and $( \, c \, )$ is the speed of light in vacuum.
\par \vspace{+0.75em}
\noindent The relational mass $( \, m \, )$ of a massive particle is given by:
\par \vspace{-0.60em}
\begin{eqnarray*}
m ~\doteq~ \dfrac{m_o}{\sqrt{1 - \dfrac{\bar{\mathbf{v}} \cdot \bar{\mathbf{v}}}{c^2}}}
\end{eqnarray*}
\par \vspace{+0.45em}
\noindent where $( \, m_o \, )$ is the rest mass of the massive particle, $( \, \bar{\mathbf{v}} \, )$ is the relational velocity of the massive particle and $( \, c \, )$ is the speed of light in vacuum.
\par \vspace{+0.75em}
\noindent The relational mass $( \, m \, )$ of a non-massive particle is given by:
\par \vspace{-0.60em}
\begin{eqnarray*}
m ~\doteq~ \dfrac{h \, \bar{\nu}}{c^2}
\end{eqnarray*}
\par \vspace{+0.45em}
\noindent where $( \, h \, )$ is the Planck constant, $( \, \bar{\nu} \, )$ is the relational frequency of the non-massive particle and $( \, c \, )$ is the speed of light in vacuum.

\vspace{-0.60em}

\par \bigskip {\centering\subsubsection*{Observations}}

\bigskip \smallskip

\noindent \S \ In arbitrary inertial reference frames \hbox {{\small ( $\bar{t}_{\alpha} \ne \tau_{\alpha}$ or $\bar{\mathbf{r}}_{\alpha} \ne 0$ ) ( $\alpha$ = auxiliary-point )}} a constant must be add in the definition of relational time such that the relational time and the proper time of the auxiliary-point are the same \hbox {{\small ( $\bar{t}_{\alpha} = \tau_{\alpha}$ )}} and another constant must be add in the definition of relational position such that the relational position of the auxiliary-point is zero \hbox {{\small ( $\bar{\mathbf{r}}_{\alpha} = 0$ )}}
\par \bigskip \smallskip
\noindent \S \ In the particular case of an isolated system of ( massive or non-massive ) \hbox {particles, inertial} observers should preferably use an auxiliary-point such that the linear momentum of the isolated system of particles is zero \hbox {{\small ( $\sum_z m_z \hspace{+0.09em} \bar{\mathbf{v}}_z = 0$ )}}

\newpage

\par \bigskip {\centering\subsubsection*{The Relational Dynamics}}

\bigskip \smallskip

\noindent If we consider a ( massive or non-massive ) particle with relational mass $m$ then the linear momentum $\mathbf{P}$ of the particle, the angular momentum $\mathbf{L}$ of the particle, the net force $\mathbf{F}$ acting on the particle, the work $\mathrm{W}$ done by the net force acting on the particle, and the kinetic energy $\mathrm{K}$ of the particle, for an inertial reference frame, are given by:
\par \vspace{-0.30em}
\begin{eqnarray*}
\mathbf{P} ~\doteq~ m \, \bar{\mathbf{v}}
\end{eqnarray*}
\vspace{-0.60em}
\begin{eqnarray*}
\mathbf{L} ~\doteq~ \mathbf{P} \times \bar{\mathbf{r}} ~=~ m \, \bar{\mathbf{v}} \times \bar{\mathbf{r}}
\end{eqnarray*}
\vspace{-0.30em}
\begin{eqnarray*}
\mathbf{F} ~=~ \dfrac{d\hspace{+0.045em}\mathbf{P}}{d\bar{t}} ~=~ m \, \bar{\mathbf{a}} \, + \dfrac{dm}{d\bar{t}} \, \bar{\mathbf{v}}
\end{eqnarray*}
\vspace{-0.30em}
\begin{eqnarray*}
\mathrm{W} ~\doteq~ \hspace{-0.36em} \int_{\scriptscriptstyle 1}^{\hspace{+0.09em}{\scriptscriptstyle 2}} \mathbf{F} \cdot d\bar{\mathbf{r}} ~=~ \hspace{-0.36em} \int_{\scriptscriptstyle 1}^{\hspace{+0.09em}{\scriptscriptstyle 2}} \dfrac{d\hspace{+0.045em}\mathbf{P}}{d\bar{t}} \cdot d\bar{\mathbf{r}} ~=~ \Delta \, \mathrm{K}
\end{eqnarray*}
\vspace{-0.30em}
\begin{eqnarray*}
\mathrm{K} ~\doteq~ m \, c^2
\end{eqnarray*}
\par \vspace{+0.90em}
\noindent where $( \, \bar{t}, \, \bar{\mathbf{r}}, \, \bar{\mathbf{v}}, \, \bar{\mathbf{a}} \, )$ are the relational time, the relational position, the relational velocity and the relational acceleration of the particle relative to the inertial reference frame and $( \, c \, )$ is the speed of light in vacuum. The kinetic energy $( \, \mathrm{K}_o \, )$ of a massive particle at rest is $( \, m_o \, c^2 \, )$
\par \vspace{+0.90em}
\noindent Forces and fields must be expressed only with relational quantities ( the Lorentz force must be expressed with the relational velocity $\bar{\mathbf{v}}$, the electric field must be expressed with the relational position $\bar{\mathbf{r}}$, etc. )

\vspace{+0.60em}

\par \bigskip {\centering\subsubsection*{Conclusions}}

\bigskip \smallskip

\noindent \S \ In this article, the quantities $( \, \bar{t}, \bar{\mathbf{r}}, \bar{\mathbf{v}}, \bar{\mathbf{a}}, \bar{\nu}, m, \mathbf{P}, \mathbf{L}, \mathbf{F}, \mathrm{W}, \mathrm{K} \, )$ are invariant under transformations between inertial reference frames.
\par \bigskip \smallskip
\noindent \S \ However, this article considers (\hspace{+0.21em}{\small 1}\hspace{+0.21em}) that it would also be possible to obtain kinematic and dynamic quantities $( \, \bar{t}, \bar{\mathbf{r}}, \bar{\mathbf{v}}, \bar{\mathbf{a}}, \bar{\nu}, m, \mathbf{P}, \mathbf{L}, \mathbf{F}, \mathrm{W}, \mathrm{K} \, )$ that would be invariant under transformations between inertial and non-inertial reference frames and (\hspace{+0.21em}{\small 2}\hspace{+0.21em}) that the dynamic quantities $( \, m, \mathbf{P}, \mathbf{L}, \mathbf{F}, \mathrm{W}, \mathrm{K} \, )$ would also be given by the equations of this article.

\newpage

\par \bigskip {\centering\subsubsection*{The Kinetic Force}}

\bigskip \smallskip

\noindent In an isolated system of ( massive or non-massive ) particles, the kinetic force $\mathbf{K}_{\hspace{+0.012em}ij}$ exerted on a particle $i$ with relational mass $m_i$ by another particle $j$ with relational mass $m_j$ is given by:
\par \vspace{-0.45em}
\begin{eqnarray*}
\mathbf{K}_{\hspace{+0.012em}ij} ~=\, - \; \dfrac{d}{d\hspace{+0.06em}\bar{t}_i} \, \Bigg [ \; \dfrac{m_i \, m_j}{\mathrm{M}} \, ( \, \bar{\mathbf{v}}_i - \bar{\mathbf{v}}_j \, ) \; \Bigg ]
\end{eqnarray*}
\par \vspace{+0.60em}
\noindent where $\bar{t}_i$ is the relational time of the particle $i$, $\bar{\mathbf{v}}_i$ is the relational velocity of the particle $i$, $\bar{\mathbf{v}}_j$ is the relational velocity of the particle $j$ and $\mathrm{M}$ {\small ( $ = \sum_z m_z$ )} is the relational mass of the isolated system of particles.
\par \vspace{+0.60em}
\noindent From the above equation it follows that the net kinetic force $\mathbf{K}_i$ {\small ( $ = \sum_z \, \mathbf{K}_{\hspace{+0.012em}iz}$ )} acting on the particle $i$ is given by:
\par \vspace{-0.60em}
\begin{eqnarray*}
\mathbf{K}_i ~=\, - \; \dfrac{d}{d\hspace{+0.06em}\bar{t}_i} \, \Big [ \; m_i \, \bar{\mathbf{v}}_i \; \Big ]
\end{eqnarray*}
\par \vspace{+0.60em}
\noindent where $\bar{t}_i$ is the relational time of the particle $i$, $m_i$ is the relational mass of the particle $i$ and $\bar{\mathbf{v}}_i$ is the relational velocity of the particle $i$.
\par \vspace{+0.60em}
\noindent Now, substituting ( $\mathbf{F}_i = d\hspace{+0.012em}( m_i \, \bar{\mathbf{v}}_i ) / d\hspace{+0.06em}\bar{t}_i$ ) and rearranging, we obtain:
\par \vspace{-0.81em}
\begin{eqnarray*}
\mathbf{T}_i ~\doteq~ \mathbf{K}_i + \mathbf{F}_i ~=~ 0
\end{eqnarray*}
\par \vspace{+0.60em}
\noindent Therefore, in an isolated system of ( massive or non-massive ) particles, the total force $\mathbf{T}_i$ acting on a particle $i$ is always zero.
\par \vspace{+0.60em}
\noindent Inertial observers must use an auxiliary-point such that the linear momentum of the isolated system of particles is zero \hbox {{\small ( $\sum_z m_z \, \bar{\mathbf{v}}_z = 0$ )}}

\vspace{+0.60em}

\par \bigskip {\centering\subsubsection*{Bibliography}}

\bigskip \smallskip

\par \noindent \textbf{A. Einstein}, Relativity: The Special and General Theory.
\bigskip \smallskip
\par \noindent \textbf{E. Mach}, The Science of Mechanics.
\bigskip \smallskip
\par \noindent \textbf{W. Pauli}, Theory of Relativity.
\bigskip \smallskip
\par \noindent \textbf{A. French}, Special Relativity.

\newpage

\par \bigskip {\centering\subsubsection*{Appendix I}}

\smallskip

\par \bigskip {\centering\subsubsection*{Generalized Lorentz Transformations}}

\bigskip \smallskip

\noindent The time $( \, {t}' \, )$, the position $( \, {\mathbf{r}}' \, )$, the velocity $( \, {\mathbf{v}}' \, )$ and the acceleration $( \, {\mathbf{a}}' \, )$ of a (massive or non-massive) particle relative to an inertial reference frame S' are given by:
\par \vspace{+0.21em}
\begin{eqnarray*}
{t}' ~=~ \hspace{+0.24em} \gamma \left ( t - \frac{\mathbf{r} \cdot \mathbf{V}}{c^2} \right )
\end{eqnarray*}
\vspace{-0.45em}
\begin{eqnarray*}
{\mathbf{r}}' ~=~ \hspace{-0.09em} \left [ \; \mathbf{r} + \frac{\gamma^2}{\gamma + 1} \frac{( \mathbf{r} \cdot \mathbf{V} ) \, \mathbf{V}}{c^2} - \gamma \, \mathbf{V} \, t \; \right ]
\end{eqnarray*}
\vspace{-0.30em}
\begin{eqnarray*}
{\mathbf{v}}' \hspace{-0.12em} ~=~ \hspace{-0.12em} \left [ \; \mathbf{v} + \frac{\gamma^2}{\gamma + 1} \frac{( \mathbf{v} \cdot \mathbf{V} ) \, \mathbf{V}}{c^2} - \hspace{+0.114em} \gamma \, \mathbf{V} \hspace{+0.114em} \; \right ] \frac{1}{\gamma \, ( 1 - \frac{\mathbf{v} \cdot \mathbf{V}}{c^2} )}
\end{eqnarray*}
\vspace{-0.30em}
\begin{eqnarray*}
{\mathbf{a}}' ~=~ \hspace{-0.18em} \left [ \; \mathbf{a} - \frac{\gamma}{\gamma + 1} \frac{( \mathbf{a} \cdot \mathbf{V} ) \, \mathbf{V}}{c^2} + \frac{( \mathbf{a} \times \mathbf{v} ) \times \mathbf{V}}{c^2} \; \right ] \frac{1}{\gamma^2 \, ( 1 - \frac{\mathbf{v} \cdot \mathbf{V}}{c^2} )^3}
\end{eqnarray*}
\par \vspace{+1.20em}
\noindent where $( \, t, \, \mathbf{r}, \, \mathbf{v}, \, \mathbf{a} \, )$ are the time, the position, the velocity and the acceleration of the particle relative to an inertial reference frame S, $( \, ${\small $\mathbf{V}$}$ \, )$ is the velocity of the inertial reference frame S' relative to the inertial reference frame S and $( \, c \, )$ is the speed of light in vacuum. $( \, ${\small $\mathbf{V}$}$ \, )$ is a constant. {\small $\gamma = ({1 - \mathbf{V} \hspace{-0.114em}\cdot\hspace{-0.114em} \mathbf{V}/c^2})^{-1/2}$}

\vspace{+0.60em}

\par \bigskip {\centering\subsubsection*{Transformation of Frequency}}

\bigskip \smallskip

\noindent The frequency $( \, \nu\hspace{+0.03em}' \, )$ of a non-massive particle relative to an inertial reference frame S' is given by:
\par \vspace{-0.60em}
\begin{eqnarray*}
\nu\hspace{+0.03em}' ~=~ \: \nu \;\, \dfrac{\left ( 1 - \dfrac{\mathbf{c} \cdot \mathbf{V}}{c^2} \right )}{\sqrt{1 - \dfrac{\mathbf{V} \cdot \mathbf{V}}{c^2}}}
\end{eqnarray*}
\par \vspace{+0.45em}
\noindent where $( \, \nu \, )$ is the frequency of the non-massive particle relative to an inertial reference frame S, $( \: \mathbf{c} \: )$ is the velocity of the non-massive particle relative to \hbox {the inertial reference} frame S, $( \, ${\small $\mathbf{V}$}$ \, )$ is the velocity of the inertial reference frame S' relative to the inertial reference frame S and $( \, c \, )$ is the speed of light in vacuum.

\newpage

\par \bigskip {\centering\subsubsection*{Appendix II}}

\smallskip

\par \bigskip {\centering\subsubsection*{System of Equations}}

\bigskip \bigskip

\begin{center}
\begin{tabular}{ccccc}
& & {\framebox(\xxa,\yya){[1]}} \\
& & {\makebox(\xxa,\yya){$\downarrow$ $d\bar{t}$ $\downarrow$}} \\
{\framebox(\xxa,\yya){[4]}} & {\makebox(\xxc,\yya){$\leftarrow \times \: \bar{\mathbf{r}} \leftarrow$}} & {\framebox(\xxa,\yya){[2]}} \\
{\makebox(\xxa,\yya){$\downarrow$ $d\bar{t}$ $\downarrow$}} & & {\makebox(\xxa,\yya){$\downarrow$ $d\bar{t}$ $\downarrow$}} \\
{\framebox(\xxa,\yya){[5]}} & {\makebox(\xxc,\yya){$\leftarrow \times \: \bar{\mathbf{r}} \leftarrow$}} & {\framebox(\xxa,\yya){[3]}} & {\makebox(\xxc,\yya){$\rightarrow \hspace{-0.001em} \int \hspace{+0.03em} d\bar{\mathbf{r}} \hspace{+0.001em} \rightarrow$}} & {\framebox(\xxa,\yya){[6]}}
\end{tabular}
\end{center}
\par \vspace{+0.90em}
\begin{eqnarray*}
\hspace{+1.26em} [\,1\,] \;\;\;\;\;\; \dfrac{1}{\mu} \; \bigg [ \,\, \int \mathbf{P} \; d\bar{t} \; - \int \hspace{-0.30em} \int \mathbf{F} \; d\bar{t} \, d\bar{t} \,\, \bigg ] =~ 0
\end{eqnarray*}
\par \vspace{+0.15em}
\begin{eqnarray*}
\hspace{+1.26em} [\,2\,] \;\;\;\;\;\; \dfrac{1}{\mu} \; \bigg [ \,\, \mathbf{P} \; - \int \mathbf{F} \; d\bar{t} \,\, \bigg ] =~ 0
\end{eqnarray*}
\par \vspace{+0.15em}
\begin{eqnarray*}
\hspace{+1.26em} [\,3\,] \;\;\;\;\;\; \dfrac{1}{\mu} \; \bigg [ \,\, \dfrac{d\hspace{+0.045em}\mathbf{P}}{d\bar{t}} \, - \; \mathbf{F} \,\, \bigg ] =~ 0
\end{eqnarray*}
\par \vspace{+0.15em}
\begin{eqnarray*}
\hspace{+1.26em} [\,4\,] \;\;\;\;\;\; \dfrac{1}{\mu} \; \bigg [ \,\, \mathbf{P} \; - \int \mathbf{F} \; d\bar{t} \,\, \bigg ] \times \bar{\mathbf{r}} ~=~ 0
\end{eqnarray*}
\par \vspace{+0.15em}
\begin{eqnarray*}
\hspace{+1.26em} [\,5\,] \;\;\;\;\;\; \dfrac{1}{\mu} \; \bigg [ \,\, \dfrac{d\hspace{+0.045em}\mathbf{P}}{d\bar{t}} \, - \; \mathbf{F} \,\, \bigg ] \times \bar{\mathbf{r}} ~=~ 0
\end{eqnarray*}
\par \vspace{+0.15em}
\begin{eqnarray*}
\hspace{+1.26em} [\,6\,] \;\;\;\;\;\; \dfrac{1}{\mu} \; \bigg [ \,\, \int \dfrac{d\hspace{+0.045em}\mathbf{P}}{d\bar{t}} \cdot d\bar{\mathbf{r}} \; - \int \mathbf{F} \cdot d\bar{\mathbf{r}} \,\, \bigg ] =~ 0
\end{eqnarray*}
\par \vspace{+0.33em}
\begin{eqnarray*}
\hspace{+1.26em} [\,\mu\,] \; \textrm{is an arbitrary ( universal ) constant with dimension of mass.}
\end{eqnarray*}

\newpage

\par \bigskip {\centering\subsubsection*{Appendix III}}

\smallskip

\par \bigskip {\centering\subsubsection*{System of Equations}}

\bigskip \bigskip

\begin{center}
\begin{tabular}{ccccc}
& & {\framebox(\xxa,\yya){[1]}} \\
& & {\makebox(\xxa,\yya){$\downarrow$ $d\bar{t}$ $\downarrow$}} \\
{\framebox(\xxa,\yya){[4]}} & {\makebox(\xxc,\yya){$\leftarrow \times \: \bar{\mathbf{r}} \leftarrow$}} & {\framebox(\xxa,\yya){[2]}} \\
{\makebox(\xxa,\yya){$\downarrow$ $d\bar{t}$ $\downarrow$}} & & {\makebox(\xxa,\yya){$\downarrow$ $d\bar{t}$ $\downarrow$}} \\
{\framebox(\xxa,\yya){[5]}} & {\makebox(\xxc,\yya){$\leftarrow \times \: \bar{\mathbf{r}} \leftarrow$}} & {\framebox(\xxa,\yya){[3]}} & {\makebox(\xxc,\yya){$\rightarrow \hspace{-0.001em} \int \hspace{+0.03em} d\bar{\mathbf{r}} \hspace{+0.001em} \rightarrow$}} & {\framebox(\xxa,\yya){[6]}}
\end{tabular}
\end{center}
\par \vspace{+0.90em}
\begin{eqnarray*}
\hspace{+1.26em} [\,1\,] \;\;\;\;\;\; \dfrac{1}{\mu} \; \bigg [ \,\, \int m \, \bar{\mathbf{v}} \; d\bar{t} \; - \int \hspace{-0.30em} \int \mathbf{F} \; d\bar{t} \, d\bar{t} \,\, \bigg ] =~ 0
\end{eqnarray*}
\par \vspace{+0.15em}
\begin{eqnarray*}
\hspace{+1.26em} [\,2\,] \;\;\;\;\;\; \dfrac{1}{\mu} \; \bigg [ \,\, m \, \bar{\mathbf{v}} \, - \int \mathbf{F} \; d\bar{t} \,\, \bigg ] =~ 0
\end{eqnarray*}
\par \vspace{+0.15em}
\begin{eqnarray*}
\hspace{+1.26em} [\,3\,] \;\;\;\;\;\; \dfrac{1}{\mu} \; \bigg [ \,\, m \, \bar{\mathbf{a}} \, + \dfrac{dm}{d\bar{t}} \, \bar{\mathbf{v}} \, - \; \mathbf{F} \,\, \bigg ] =~ 0
\end{eqnarray*}
\par \vspace{+0.15em}
\begin{eqnarray*}
\hspace{+1.26em} [\,4\,] \;\;\;\;\;\; \dfrac{1}{\mu} \; \bigg [ \,\, m \, \bar{\mathbf{v}} \, - \int \mathbf{F} \; d\bar{t} \,\, \bigg ] \times \bar{\mathbf{r}} ~=~ 0
\end{eqnarray*}
\par \vspace{+0.15em}
\begin{eqnarray*}
\hspace{+1.26em} [\,5\,] \;\;\;\;\;\; \dfrac{1}{\mu} \; \bigg [ \,\, m \, \bar{\mathbf{a}} \, + \dfrac{dm}{d\bar{t}} \, \bar{\mathbf{v}} \, - \; \mathbf{F} \,\, \bigg ] \times \bar{\mathbf{r}} ~=~ 0
\end{eqnarray*}
\par \vspace{+0.15em}
\begin{eqnarray*}
\hspace{+1.26em} [\,6\,] \;\;\;\;\;\; \dfrac{1}{\mu} \; \bigg [ \,\, m \, c^2 \, - \int \mathbf{F} \cdot d\bar{\mathbf{r}} \,\, \bigg ] =~ 0
\end{eqnarray*}
\par \vspace{+0.33em}
\begin{eqnarray*}
\hspace{+1.26em} [\,\mu\,] \; \textrm{is an arbitrary ( universal ) constant with dimension of mass.}
\end{eqnarray*}

\end{document}

