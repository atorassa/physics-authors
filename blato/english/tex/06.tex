
\documentclass[10pt,fleqn]{article}
%\documentclass[a4paper,10pt]{article}
%\documentclass[letterpaper,10pt]{article}

\usepackage[dvips]{geometry}
\geometry{papersize={129.0mm,162.0mm}}
\geometry{totalwidth=108.0mm,totalheight=126.0mm}

\usepackage{graphicx}

\usepackage[english]{babel}
\usepackage{times}
\usepackage{amsfonts}
\usepackage{amsmath,bm}

\usepackage{hyperref}
\hypersetup{colorlinks=true,linkcolor=black}
\hypersetup{bookmarksnumbered=true,pdfstartview=FitH,pdfpagemode=UseNone}
\hypersetup{pdftitle={A New Dynamics in Special Relativity}}
\hypersetup{pdfauthor={A. Blato}}

\setlength{\arraycolsep}{1.74pt}

\newcommand{\med}{\raise.5ex\hbox{$\scriptstyle 1$}\kern-.15em/\kern-.12em\lower.45ex\hbox{$\scriptstyle 2$}\;}

\begin{document}

\begin{center}

{\fontsize{10.98}{10.98}\selectfont \sc A New Dynamics in Special Relativity}

\bigskip \medskip

{A. Blato}

\bigskip \medskip

\small

Creative Commons Attribution 3.0 License

\smallskip

(2016) Buenos Aires

\medskip

Argentina

\smallskip

\bigskip \medskip

\parbox{72.00mm}{In special relativity, this article presents a new dynamics which can be applied in any inertial reference frame.}

\end{center}

\normalsize

\vspace{-1.20em}

\par \bigskip {\centering\subsubsection*{Introduction}}

\bigskip \smallskip

\noindent In special relativity, the relativistic position $( \, {\boldsymbol{\varphi}} \, )$, the relativistic velocity $( \, \dot{\boldsymbol{\varphi}} \, )$ and the relativistic acceleration $( \, \ddot{\boldsymbol{\varphi}} \, )$ of a particle are given by:
\par \vspace{-0.30em}
\begin{eqnarray*}
{\boldsymbol{\varphi}} ~\doteq~ \mathbf{r}
\end{eqnarray*}
\par \vspace{+0.30em}
\begin{eqnarray*}
\dot{\boldsymbol{\varphi}} ~\doteq~ \frac{d{\boldsymbol{\varphi}}}{d{\tau}} ~=~ \frac{\mathbf{v}}{\sqrt{1 - \frac{v^2}{c^2}}}
\end{eqnarray*}
\par \vspace{-0.30em}
\begin{eqnarray*}
\ddot{\boldsymbol{\varphi}} ~\doteq~ \frac{d{\dot{\boldsymbol{\varphi}}}}{d{\tau}} ~=~ \frac{1}{\sqrt{1 - \frac{v^2}{c^2}}} \left [ \; \frac{\mathbf{a}}{\sqrt{1 - \frac{v^2}{c^2}}} + \frac{(\mathbf{a} \cdot \mathbf{v}) \: \mathbf{v}}{c^2\hspace{-0.18em}\left(1 - \frac{v^2}{c^2}\right)^{\hspace{-0.09em}{3/2}}} \; \right ]
\end{eqnarray*}
\par \vspace{+1.20em}
\noindent where $( \, \mathbf{r}, \, \mathbf{v}, \, \mathbf{a} \, )$ are the position, the velocity and the acceleration of the particle. $( \, {\tau} \, )$ is the proper time of the particle. {\small $d{\tau} = \sqrt{1 - v^2/c^2} \; d{t}$}

\newpage

\par \bigskip {\centering\subsubsection*{The Poincarian Dynamics}}

\bigskip \smallskip

\noindent In special relativity, if we consider a particle with rest mass $m_o$ then the linear momentum $\mathbf{P}$ of the particle, the net Poincarian force $\mathbf{\widehat{F}}$ acting on the particle, the work $\mathrm{W}$ done by the net Poincarian force acting on the particle, and the kinetic energy $\mathrm{K}$ of the particle, for an inertial reference frame, are given by:
\par \vspace{-0.30em}
\begin{eqnarray*}
\mathbf{P} ~\doteq~ m_o \, \dot{\boldsymbol{\varphi}}
\end{eqnarray*}
\vspace{-0.30em}
\begin{eqnarray*}
\mathbf{\widehat{F}} ~=~ \frac{d\mathbf{P}}{d{\tau}} ~=~ m_o \, \ddot{\boldsymbol{\varphi}}
\end{eqnarray*}
\vspace{-0.30em}
\begin{eqnarray*}
\mathrm{W} ~\doteq~ \int_1^{\hspace{+0.06em}2} \mathbf{\widehat{F}} \cdot d{{\boldsymbol{\varphi}}} ~=~ \Delta \, \mathrm{K}
\end{eqnarray*}
\vspace{-0.30em}
\begin{eqnarray*}
\mathrm{K} ~\doteq~ \med \, m_o \, ( \dot{\boldsymbol{\varphi}} \cdot \dot{\boldsymbol{\varphi}} )
\end{eqnarray*}
\par \vspace{+0.90em}
\noindent where $( \, {\boldsymbol{\varphi}}, \, \dot{\boldsymbol{\varphi}}, \, \ddot{\boldsymbol{\varphi}} \, )$ are the relativistic position, the relativistic velocity and the relativistic acceleration of the particle relative to the inertial reference frame.
\par \vspace{+1.50em}
\noindent {\small $\mathbf{\widehat{F}} = \gamma \: \mathbf{F}$} ( where {\small $\gamma$} is the Lorentz factor and {\small $\mathbf{F}$} is the net Einsteinian force acting on the particle )
\par \vspace{+1.50em}
\noindent The relativistic acceleration $\ddot{\boldsymbol{\varphi}}$ of a particle is always in the direction of the net Poincarian force $\mathbf{\widehat{F}}$ acting on the particle.

\end{document}

