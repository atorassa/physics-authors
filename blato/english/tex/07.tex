
\documentclass[10pt,fleqn]{article}
%\documentclass[a4paper,10pt]{article}
%\documentclass[letterpaper,10pt]{article}

\usepackage[dvips]{geometry}
\geometry{papersize={128.4mm,198.6mm}}
\geometry{totalwidth=107.4mm,totalheight=162.6mm}

\usepackage{graphicx}

\usepackage[english]{babel}
\usepackage{times}
\usepackage{amsfonts}
\usepackage{amsmath,bm}

\usepackage{hyperref}
\hypersetup{colorlinks=true,linkcolor=black}
\hypersetup{bookmarksnumbered=true,pdfstartview=FitH,pdfpagemode=UseNone}
\hypersetup{pdftitle={Special Relativity with Massive and Non-massive Particles}}
\hypersetup{pdfauthor={A. Blato}}

\setlength{\arraycolsep}{1.74pt}

\newcommand{\yya}{27}
\newcommand{\xxa}{33}
\newcommand{\xxc}{45}

\begin{document}

\begin{center}

{\fontsize{9.60}{9.60}\selectfont \sc Special Relativity with Massive and Non-massive Particles}

\bigskip \medskip

{\fontsize{9.60}{9.60}\selectfont A. Blato}

\bigskip \medskip

\small

Creative Commons Attribution 3.0 License

\smallskip

(2016) Buenos Aires

\medskip

Argentina

\smallskip

\bigskip \medskip

\parbox{78.99mm}{In special relativity, this article presents a relativistic dynamics of massive and non-massive particles which can be applied in any inertial reference frame.}

\end{center}

\normalsize

\vspace{-1.20em}

\par \bigskip {\centering\subsubsection*{Introduction}}

\bigskip \smallskip

\noindent In special relativity, the total position $( \, \bar{\mathbf{r}} \, )$ of a (massive or non-massive) particle is always zero.
\par \vspace{-0.60em}
\begin{eqnarray*}
\bar{\mathbf{r}} ~=~ 0
\end{eqnarray*}
\par \vspace{+0.60em}
\noindent The total position $( \, \bar{\mathbf{r}} \, )$ of a (massive or non-massive) particle is defined by the kinetic position $( \, \hat{\mathbf{r}} \, )$ and the dynamic position $( \, \check{\mathbf{r}} \, )$ as follows:
\par \vspace{-0.60em}
\begin{eqnarray*}
\hat{\mathbf{r}} - \check{\mathbf{r}} ~=~ 0
\end{eqnarray*}
\par \vspace{+0.60em}
\noindent The kinetic position $( \, \hat{\mathbf{r}} \, )$ of a (massive or non-massive) particle is given by:
\par \vspace{-0.60em}
\begin{eqnarray*}
\hat{\mathbf{r}} ~\doteq~ \dfrac{1}{\mu} \, \int m \, \mathbf{v} \, dt
\end{eqnarray*}
\par \vspace{+0.45em}
\noindent where $( \, \mu \, )$ is an arbitrary ( universal ) constant, $( \, m \, )$ is the relativistic mass of the particle, $( \, \mathbf{v} \, )$ is the velocity of the particle and $( \, t \, )$ is time.
\par \vspace{+0.60em}
\noindent The dynamic position $( \, \check{\mathbf{r}} \, )$ of a (massive or non-massive) particle is given by:
\par \vspace{-0.60em}
\begin{eqnarray*}
\check{\mathbf{r}} ~\doteq~ \dfrac{1}{\mu} \, \int \hspace{-0.30em} \int \mathbf{F} \; dt \, dt
\end{eqnarray*}
\par \vspace{+0.45em}
\noindent where $( \, \mu \, )$ is the arbitrary ( universal ) constant, $( \, \mathbf{F} \, )$ is the net force acting on the particle and $( \, t \, )$ is time.

\newpage

\noindent The relativistic mass $( \, m \, )$ of a massive particle is given by:
\par \vspace{-0.60em}
\begin{eqnarray*}
m ~\doteq~ \dfrac{m_o}{\sqrt{1 - \frac{v^2}{c^2}}}
\end{eqnarray*}
\par \vspace{+0.45em}
\noindent where $( \, m_o \, )$ is the rest mass of the massive particle, $( \, v \, )$ is the speed of the massive particle and $( \, c \, )$ is the speed of light in vacuum.
\par \vspace{+0.60em}
\noindent The relativistic mass $( \, m \, )$ of a non-massive particle is given by:
\par \vspace{-0.60em}
\begin{eqnarray*}
m ~\doteq~ \dfrac{h \, \nu}{c^2}
\end{eqnarray*}
\par \vspace{+0.45em}
\noindent where $( \, h \, )$ is the Planck constant, $( \, \nu \, )$ is the frequency of the non-massive particle and $( \, c \, )$ is the speed of light in vacuum.
\par \vspace{+0.75em}
\noindent Now, the total position $( \, \bar{\mathbf{r}} \, )$ of a (massive or non-massive) particle can also be expressed as follows:
\par \vspace{-0.60em}
\begin{eqnarray*}
\dfrac{1}{\mu} \; \bigg [ \,\, \int m \, \mathbf{v} \, dt \; - \int \hspace{-0.30em} \int \mathbf{F} \; dt \, dt \,\, \bigg ] =~ 0
\end{eqnarray*}
\par \vspace{+0.60em}
\noindent Differentiating the above equation with respect to time, yields:
\par \vspace{-0.60em}
\begin{eqnarray*}
\dfrac{1}{\mu} \; \bigg [ \,\, m \, \mathbf{v} \; - \int \mathbf{F} \; dt \,\, \bigg ] =~ 0
\end{eqnarray*}
\par \vspace{+0.60em}
\noindent Differentiating again with respect to time, we have:
\par \vspace{-0.60em}
\begin{eqnarray*}
\dfrac{1}{\mu} \; \bigg [ \,\, m \, \mathbf{a} \, + \dfrac{dm}{dt} \, \mathbf{v} \; - \; \mathbf{F} \,\, \bigg ] =~ 0
\end{eqnarray*}
\par \vspace{+0.60em}
\noindent Multiplying by $( \, \mu \, )$ and rearranging, we finally obtain:
\par \vspace{-0.45em}
\begin{eqnarray*}
\mathbf{F} ~=~ m \, \mathbf{a} \, + \dfrac{dm}{dt} \, \mathbf{v}
\end{eqnarray*}
\par \vspace{+0.60em}
\noindent This equation ( similar to Newton's second law for $v \ll c$ ) will be used in the next section of this article.

\newpage

\par \bigskip {\centering\subsubsection*{The Relativistic Dynamics}}

\bigskip \smallskip

\noindent If we consider a (massive or non-massive) particle with relativistic mass $m$ then the linear momentum $\mathbf{P}$ of the particle, the angular momentum $\mathbf{L}$ of the particle, the net force $\mathbf{F}$ acting on the particle, the work $\mathrm{W}$ done by the net force acting on the particle, and the kinetic energy $\mathrm{K}$ of the particle, for an inertial reference frame, are given by:
\par \vspace{-0.30em}
\begin{eqnarray*}
\mathbf{P} ~\doteq~ m \, \mathbf{v}
\end{eqnarray*}
\vspace{-0.60em}
\begin{eqnarray*}
\mathbf{L} ~\doteq~ \mathbf{P} \times \mathbf{r}
\end{eqnarray*}
\vspace{-0.30em}
\begin{eqnarray*}
\mathbf{F} ~=~ \dfrac{d\hspace{+0.045em}\mathbf{P}}{dt} ~=~ m \, \mathbf{a} \, + \dfrac{dm}{dt} \, \mathbf{v}
\end{eqnarray*}
\vspace{-0.30em}
\begin{eqnarray*}
\mathrm{W} ~\doteq~ \hspace{-0.36em} \int_{\scriptscriptstyle 1}^{\hspace{+0.09em}{\scriptscriptstyle 2}} \mathbf{F} \cdot d\mathbf{r} ~=~ \hspace{-0.36em} \int_{\scriptscriptstyle 1}^{\hspace{+0.09em}{\scriptscriptstyle 2}} \dfrac{d\hspace{+0.045em}\mathbf{P}}{dt} \cdot d\mathbf{r} ~=~ \Delta \, \mathrm{K}
\end{eqnarray*}
\vspace{-0.30em}
\begin{eqnarray*}
\mathrm{K} ~\doteq~ m \, c^2
\end{eqnarray*}
\par \vspace{+0.90em}
\noindent where $( \, \mathbf{r}, \, \mathbf{v}, \, \mathbf{a} \, )$ are the position, the velocity and the acceleration of the particle relative to the inertial reference frame and $( \, c \, )$ is the speed of light in vacuum. The kinetic energy $( \, \mathrm{K}_o \, )$ of a massive particle at rest is $( \, m_o \, c^2 \, )$

\vspace{+0.60em}

\par \bigskip {\centering\subsubsection*{Bibliography}}

\bigskip \smallskip

\par \noindent \textbf{A. Einstein}, Relativity: The Special and General Theory.
\bigskip
\par \noindent \textbf{E. Mach}, The Science of Mechanics.
\bigskip
\par \noindent \textbf{W. Pauli}, Theory of Relativity.
\bigskip
\par \noindent \textbf{A. French}, Special Relativity.

\newpage

\par \bigskip {\centering\subsubsection*{Appendix}}

\smallskip

\par \bigskip {\centering\subsubsection*{System of Equations}}

\bigskip \bigskip

\begin{center}
\begin{tabular}{ccccc}
& & {\framebox(\xxa,\yya){[1]}} \\
& & {\makebox(\xxa,\yya){$\downarrow$ $dt$ $\downarrow$}} \\
{\framebox(\xxa,\yya){[4]}} & {\makebox(\xxc,\yya){$\leftarrow \times \: \mathbf{r} \leftarrow$}} & {\framebox(\xxa,\yya){[2]}} \\
{\makebox(\xxa,\yya){$\downarrow$ $dt$ $\downarrow$}} & & {\makebox(\xxa,\yya){$\downarrow$ $dt$ $\downarrow$}} \\
{\framebox(\xxa,\yya){[5]}} & {\makebox(\xxc,\yya){$\leftarrow \times \: \mathbf{r} \leftarrow$}} & {\framebox(\xxa,\yya){[3]}} & {\makebox(\xxc,\yya){$\rightarrow \hspace{-0.001em} \int \hspace{+0.03em} d\mathbf{r} \hspace{+0.001em} \rightarrow$}} & {\framebox(\xxa,\yya){[6]}}
\end{tabular}
\end{center}
\par \vspace{+0.90em}
\begin{eqnarray*}
\hspace{+1.26em} [\,1\,] \;\;\;\;\;\; \dfrac{1}{\mu} \; \bigg [ \,\, \int \mathbf{P} \; dt \; - \int \hspace{-0.30em} \int \mathbf{F} \; dt \, dt \,\, \bigg ] =~ 0
\end{eqnarray*}
\par \vspace{+0.15em}
\begin{eqnarray*}
\hspace{+1.26em} [\,2\,] \;\;\;\;\;\; \dfrac{1}{\mu} \; \bigg [ \,\, \mathbf{P} \; - \int \mathbf{F} \; dt \,\, \bigg ] =~ 0
\end{eqnarray*}
\par \vspace{+0.15em}
\begin{eqnarray*}
\hspace{+1.26em} [\,3\,] \;\;\;\;\;\; \dfrac{1}{\mu} \; \bigg [ \,\, \dfrac{d\hspace{+0.045em}\mathbf{P}}{dt} \, - \; \mathbf{F} \,\, \bigg ] =~ 0
\end{eqnarray*}
\par \vspace{+0.15em}
\begin{eqnarray*}
\hspace{+1.26em} [\,4\,] \;\;\;\;\;\; \dfrac{1}{\mu} \; \bigg [ \,\, \mathbf{P} \; - \int \mathbf{F} \; dt \,\, \bigg ] \times \mathbf{r} ~=~ 0
\end{eqnarray*}
\par \vspace{+0.15em}
\begin{eqnarray*}
\hspace{+1.26em} [\,5\,] \;\;\;\;\;\; \dfrac{1}{\mu} \; \bigg [ \,\, \dfrac{d\hspace{+0.045em}\mathbf{P}}{dt} \, - \; \mathbf{F} \,\, \bigg ] \times \mathbf{r} ~=~ 0
\end{eqnarray*}
\par \vspace{+0.15em}
\begin{eqnarray*}
\hspace{+1.26em} [\,6\,] \;\;\;\;\;\; \dfrac{1}{\mu} \; \bigg [ \,\, \int \dfrac{d\hspace{+0.045em}\mathbf{P}}{dt} \cdot d\mathbf{r} \; - \int \mathbf{F} \cdot d\mathbf{r} \,\, \bigg ] =~ 0
\end{eqnarray*}

\end{document}

