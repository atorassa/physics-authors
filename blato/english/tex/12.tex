
\documentclass[10pt,fleqn]{article}
%\documentclass[a4paper,10pt]{article}
%\documentclass[letterpaper,10pt]{article}

\usepackage[dvips]{geometry}
\geometry{papersize={131.1mm,216.0mm}}
\geometry{totalwidth=110.1mm,totalheight=180.0mm}

\usepackage{graphicx}

\usepackage[english]{babel}
\usepackage{times}
\usepackage{amsfonts}
\usepackage{amsmath,bm}

\usepackage{hyperref}
\hypersetup{colorlinks=true,linkcolor=black}
\hypersetup{bookmarksnumbered=true,pdfstartview=FitH,pdfpagemode=UseNone}
\hypersetup{pdftitle={Non-Inertial Frames in Special Relativity}}
\hypersetup{pdfauthor={A. Blato}}

\setlength{\arraycolsep}{1.74pt}

\newcommand{\yya}{27}
\newcommand{\xxa}{33}
\newcommand{\xxc}{45}

\begin{document}

\begin{center}

{\fontsize{10.50}{10.50}\selectfont \sc Non-Inertial Frames in Special Relativity}

\bigskip \medskip

{\fontsize{9.60}{9.60}\selectfont A. Blato}

\bigskip \medskip

\small

Creative Commons Attribution 3.0 License

\smallskip

(2018) Buenos Aires

\medskip

Argentina

\smallskip

\bigskip \medskip

\parbox{86.40mm}{This article presents a new formulation of special relativity which is invariant under transformations between inertial and non-inertial (\,non-rotating\,) frames. Additionally, a simple solution to the twin paradox is presented and a new universal force is proposed.}

\end{center}

\normalsize

\vspace{-1.50em}

\par \bigskip {\centering\subsubsection*{Introduction}}

\bigskip \smallskip

\noindent The intrinsic mass $( \, m \, )$ and the frequency factor $( \, f \, )$ of a massive particle \hbox {are given by}:
\par \vspace{-0.60em}
\begin{eqnarray*}
m ~\doteq~ m_o
\end{eqnarray*}
\vspace{-0.90em}
\begin{eqnarray*}
f ~\doteq~ \Big ( 1 - \dfrac{\mathbf{v} \cdot \mathbf{v}}{c^2} \hspace{+0.15em} \Big )^{\hspace{-0.24em}-\hspace{+0.03em}1/2}
\end{eqnarray*}
\par \vspace{+0.60em}
\noindent where $( \, m_o \, )$ is the rest mass of the massive particle, $( \, \mathbf{v} \, )$ is the relational velocity of the massive particle and $( \, c \, )$ is the speed of light in vacuum.
\par \vspace{+0.60em}
\noindent The intrinsic mass $( \, m \, )$ and the frequency factor $( \, f \, )$ of a non-massive particle \hbox {are given by}:
\par \vspace{-0.60em}
\begin{eqnarray*}
m ~\doteq~ \dfrac{h \, \kappa}{c^2}
\end{eqnarray*}
\vspace{-0.60em}
\begin{eqnarray*}
f ~\doteq~ \dfrac{\nu}{\kappa}
\end{eqnarray*}
\par \vspace{+0.60em}
\noindent where $( \hspace{+0.33em} h \hspace{+0.33em} )$ is the Planck constant, \hspace{+0.06em}$( \hspace{+0.30em} \nu \hspace{+0.30em} )$ is the relational frequency of the \hbox {non-massive} particle, $( \, \kappa \, )$ is a positive universal constant with dimension of frequency and $( \, c \, )$ is the speed of light in vacuum.
\par \vspace{+0.60em}
\noindent In this article, a massive particle is a particle with non-zero rest mass and a non-massive particle is a particle with zero rest mass.

\newpage

\par \bigskip {\centering\subsubsection*{The Invariant Kinematics}}

\bigskip \smallskip

\noindent The special position $( \, \bar{\mathbf{r}} \, )$, the special velocity $( \, \bar{\mathbf{v}} \, )$ and the special acceleration $( \, \bar{\mathbf{a}} \, )$ of a ( massive or non-massive ) particle are given by:
\par \vspace{-0.30em}
\begin{eqnarray*}
\bar{\mathbf{r}} ~\doteq~ \int f \, \mathbf{v} \; d\hspace{+0.012em}t
\end{eqnarray*}
\vspace{-0.45em}
\begin{eqnarray*}
\bar{\mathbf{v}} ~\doteq~ \dfrac{d\hspace{+0.036em}\bar{\mathbf{r}}}{d\hspace{+0.012em}t} ~=~ f \, \mathbf{v}
\end{eqnarray*}
\vspace{-0.30em}
\begin{eqnarray*}
\bar{\mathbf{a}} ~\doteq~ \dfrac{d\hspace{+0.021em}\bar{\mathbf{v}}}{d\hspace{+0.012em}t} ~=~ f \, \dfrac{d\hspace{+0.021em}\mathbf{v}}{d\hspace{+0.012em}t} + \dfrac{d\hspace{-0.12em}f}{d\hspace{+0.012em}t} \, \mathbf{v}
\end{eqnarray*}
\par \vspace{+1.20em}
\noindent where $( \, f \, )$ is the frequency factor of the particle, $( \, \mathbf{v} \, )$ is the relational velocity of the particle and $( \, t \, )$ is the relational time of the particle.

\vspace{+0.60em}

\par \bigskip {\centering\subsubsection*{The Invariant Dynamics}}

\bigskip \smallskip

\noindent If we consider a ( massive or non-massive ) particle with intrinsic mass $( \, m \, )$ then the linear momentum $( \, \mathbf{P} \, )$ of the particle, the angular momentum $( \, \mathbf{L} \, )$ of the particle, the net force $( \, \mathbf{F} \, )$ acting on the particle, the work $( \, \mathrm{W} \, )$ done by the net force acting on the particle, and the kinetic energy $( \, \mathrm{K} \, )$ of the particle are given by:
\par \vspace{-0.30em}
\begin{eqnarray*}
\mathbf{P} ~\doteq~ m \, \bar{\mathbf{v}} ~=~ m \, f \, \mathbf{v}
\end{eqnarray*}
\vspace{-0.30em}
\begin{eqnarray*}
\mathbf{L} ~\doteq~ \mathbf{P} \hspace{+0.24em}\dot{\times}\hspace{+0.30em} \mathbf{r} ~=~ m \, \bar{\mathbf{v}} \hspace{+0.24em}\dot{\times}\hspace{+0.30em} \mathbf{r} ~=~ m \, f \, \mathbf{v} \hspace{+0.24em}\dot{\times}\hspace{+0.30em} \mathbf{r}
\end{eqnarray*}
\vspace{-0.30em}
\begin{eqnarray*}
\mathbf{F} ~=~ \dfrac{d\hspace{+0.045em}\mathbf{P}}{d\hspace{+0.012em}t} ~=~ m \, \bar{\mathbf{a}} ~=~ m \, \bigg [ \, f \, \dfrac{d\hspace{+0.021em}\mathbf{v}}{d\hspace{+0.012em}t} + \dfrac{d\hspace{-0.12em}f}{d\hspace{+0.012em}t} \, \mathbf{v} \, \bigg ]
\end{eqnarray*}
\vspace{-0.15em}
\begin{eqnarray*}
\mathrm{W} ~\doteq~ \hspace{-0.36em} \int_{\scriptscriptstyle 1}^{\hspace{+0.09em}{\scriptscriptstyle 2}} \mathbf{F} \cdot d\hspace{+0.036em}\mathbf{r} ~=~ \hspace{-0.36em} \int_{\scriptscriptstyle 1}^{\hspace{+0.09em}{\scriptscriptstyle 2}} \dfrac{d\hspace{+0.045em}\mathbf{P}}{d\hspace{+0.012em}t} \cdot d\hspace{+0.036em}\mathbf{r} ~=~ \Delta \, \mathrm{K}
\end{eqnarray*}
\vspace{-0.30em}
\begin{eqnarray*}
\mathrm{K} ~\doteq~ m \, f \, c^2
\end{eqnarray*}
\par \vspace{+0.90em}
\noindent where $( \: f, \: \mathbf{r}, \: \mathbf{v}, \: t, \: \bar{\mathbf{v}}, \: \bar{\mathbf{a}} \: )$ are the frequency factor, the relational position, the relational velocity, the relational time, the special velocity and the special acceleration of the particle and $( \, c \, )$ is the speed of light in vacuum. The kinetic energy $( \, \mathrm{K}_o \, )$ of a massive particle at relational rest is $( \, m_o \, c^2 \, )$

\newpage

\par \bigskip {\centering\subsubsection*{Relational Quantities}}

\bigskip \smallskip

\noindent From an auxiliary massive particle ( called auxiliary-point ) some kinematic quantities ( called relational quantities ) can be obtained. These are invariant under transformations between inertial and non-inertial (\,non-rotating\,) frames.
\par \medskip \smallskip
\noindent An auxiliary-point is an arbitrary massive particle that is free of forces ( that is, the net force acting on it is zero )
\par \medskip \smallskip
\noindent The relational time $( \, t \, )$, the relational position $( \, \mathbf{r} \, )$, the relational velocity $( \, \mathbf{v} \, )$ and the relational acceleration $( \, \mathbf{a} \, )$ of a (\,massive or non-massive\,) particle \hbox {relative} to an inertial or non-inertial (\,non-rotating\,) frame S are given by:
\par \vspace{-0.18em}
\begin{eqnarray*}
t ~\doteq~ \int_{\hspace{-0.045em}\mathtt{O}}^{\mathtt{t}} \gamma \; \mathtt{dt} \, - \, \gamma \; \frac{\vec{r} \cdot \vec{\varphi}}{c^2}
\end{eqnarray*}
\vspace{-0.15em}
\begin{eqnarray*}
\mathbf{r} ~\doteq~ \vec{r} \: + \, \frac{\gamma^2}{\gamma + 1} \, \frac{( \vec{r} \cdot \vec{\varphi} \hspace{+0.09em}) \: \vec{\varphi}}{c^2} \, - \int_{\hspace{-0.045em}\mathtt{O}}^{\mathtt{t}} \gamma \; \vec{\varphi} \; \mathtt{dt}
\end{eqnarray*}
\vspace{-0.15em}
\begin{eqnarray*}
\mathbf{v} ~\doteq~ \dfrac{d\mathbf{r}}{dt} \hspace{+1.20em} , \hspace{+1.11em} \mathbf{a} ~\doteq~ \dfrac{d\mathbf{v}}{dt}
\end{eqnarray*}
\par \vspace{+1.20em}
\noindent where $(\hspace{+0.12em}\mathtt{t}, \, \vec{r}\hspace{+0.21em})$ are the time and the position of the particle relative to the \hbox{frame S}, $( \, \vec{\varphi} \, )$ is the velocity of the auxiliary-point relative to the frame S, $( \, c \, )$ is the speed of light in vacuum, and {\small $\, \gamma ~\doteq~ (\,{1 - \vec{\varphi} \cdot \vec{\varphi}\hspace{+0.09em}/\hspace{+0.09em}c^2}\,)^{-1/2}$}
\par \medskip \smallskip
\noindent The velocity of the auxiliary-point $( \, \vec{\varphi} \, )$ is a constant in inertial frames and the factor $( \, \gamma \, )$ is a constant in non-inertial (\,uniform circular motion\,) frames.
\par \medskip \smallskip
\noindent The relational velocity of light (\,non-massive particle\,) in vacuum is $( \, \mathbf{c} \, )$ and $( \, \mathbf{c} \cdot \mathbf{c} \, )$ is a constant in inertial and non-inertial (\,non-rotating\,) frames.
\par \medskip \smallskip
\noindent The relational frequency $( \, \nu \, )$ of a non-massive particle relative to an inertial or non-inertial (\,non-rotating\,) frame S is given by:
\par \vspace{-0.18em}
\begin{eqnarray*}
\nu ~\doteq~ \mathtt{v} \;\, \gamma \: \left ( 1 - \dfrac{\vec{c} \cdot \vec{\varphi}}{c^2} \: \right )
\end{eqnarray*}
\par \vspace{+0.90em}
\noindent where $( \, \mathtt{v} \, )$ is the frequency of the non-massive particle relative to the frame S, $( \, \vec{c} \, )$ is the velocity of the non-massive particle relative to the frame S, $( \, \vec{\varphi} \, )$ is the velocity of the auxiliary-point relative to the frame S, $( \, c \, )$ is the speed of light in vacuum, and {\small $\, \gamma ~\doteq~ (\,{1 - \vec{\varphi} \cdot \vec{\varphi}\hspace{+0.09em}/\hspace{+0.09em}c^2}\,)^{-1/2}$}

\newpage

\noindent \S \ In arbitrary frames \hbox {{\small ( $t_{\alpha} \ne \tau_{\alpha}$ or \hspace{+0.06em}$\mathbf{r}_{\alpha} \ne 0$ ) ( $\alpha$ = auxiliary-point )}} a constant must be add in the definition of relational time such that the relational time and the proper time of the auxiliary-point are the same \hbox {{\small ( $t_{\alpha} = \tau_{\alpha}$ )}} and another \hbox {constant} must be add in the definition of relational position such that the relational \hbox {position} of the auxiliary-point is zero \hbox {{\small ( $\mathbf{r}_{\alpha} = 0$ )}}
\par \medskip \smallskip
\noindent \S \ In the particular case of an isolated system of ( massive or non-massive ) particles, all observers should preferably use an auxiliary-point such that the linear momentum of the isolated system of particles is zero \hbox {{\small ( $\sum_z m_z \hspace{+0.09em} \bar{\mathbf{v}}_z = 0$ )}}
\par \medskip \smallskip
\noindent \S \ It is important to emphasize that any auxiliary-point must be a free massive particle ( that is, the net force acting on it must be zero )

\vspace{-0.6em}

\par \bigskip {\centering\subsubsection*{Relational Metric}}

\bigskip \smallskip

\noindent It is known that in inertial frames the local geometry is Euclidean and that in non-inertial frames the local geometry is in general non-Euclidean.
\par \medskip \smallskip
\noindent According to this article, in an inertial or non-inertial (\,non-rotating\,) frame S the local line element must be obtained from the relational line element.
\par \medskip \smallskip
\noindent Therefore, in the frame S the relational line element (\,in rectilinear coordinates\,) and the local line element are given by:
\par \vspace{+0.06em}
\begin{eqnarray*}
\hspace{-2.50em} ds^{\hspace{+0.03em}2} ~=~ c^2 dt^{\hspace{+0.03em}2} - d\mathbf{r}^{\hspace{+0.03em}2}
\end{eqnarray*}
\vspace{-0.45em}
\begin{eqnarray*}
ds^{\hspace{+0.03em}2} ~=~ \Big [ \; \Big ( \, 1 + \frac{\vec{\mathrm w} \cdot \vec{r}}{c^2} \: \Big )^{\hspace{-0.18em}2} \hspace{-0.21em}- \Big ( \, \frac{\vec{\phi} \times \vec{r}}{c} \: \Big )^{\hspace{-0.18em}2} \,\, \Big ] \, c^2 \, \mathtt{dt}^2 \,-\, 2 \, \Big ( \, \vec{\phi} \times \vec{r} \: \Big ) \, d\vec{r} \,\, \mathtt{dt} \,-\, d\vec{r}^{\hspace{+0.24em}2}
\end{eqnarray*}
\vspace{+0.06em}
\begin{eqnarray*}
\hspace{+0.03em} \vec{\mathrm w} ~\doteq~ - \; \gamma^1 \: \Big ( \, \vec{\alpha} \: + \frac{\gamma^2}{\gamma + 1} \, \frac{(\hspace{+0.09em} \vec{\alpha} \cdot \vec{\varphi} \hspace{+0.09em}) \: \vec{\varphi}}{c^2} \: \Big ) \hspace{+1.08em} , \hspace{+1.08em} \vec{\phi} ~\doteq~ - \; \gamma^0 \: \Big ( \, \frac{\gamma^2}{\gamma + 1} \, \frac{(\hspace{+0.09em} \vec{\alpha} \times \vec{\varphi} \hspace{+0.09em})}{c^2} \: \Big )
\end{eqnarray*}
\par \vspace{+1.20em}
\noindent where $( \, t, \, \mathbf{r} \, )$ are relational time and relational position relative to the frame S, $( \, \mathtt{t}, \, \vec{r} \, )$ are time and position relative to the frame S, $( \, \vec{\varphi}, \, \vec{\alpha} \, )$ are the velocity and the acceleration of the auxiliary-point relative to the frame S, $( \, c \, )$ is the speed of light in vacuum, and {\small $\, \gamma ~\doteq~ (\,{1 - \vec{\varphi} \cdot \vec{\varphi}\hspace{+0.09em}/\hspace{+0.09em}c^2}\,)^{-1/2}$}
\par \vspace{+0.90em}
\noindent The frame S is inertial when $( \: \vec{\alpha} ~=~ 0 \: )$ the frame S is non-inertial (\,rectilinear accelerated motion\,) when $( \: \vec{\alpha} ~\ne~ 0 \: )$ {\small \&} $( \: \vec{\alpha} \times \vec{\varphi} ~=~ 0 \: )$ and the frame S is non-inertial (\,uniform circular motion\,) when $( \: \vec{\alpha} ~\ne~ 0 \: )$ {\small \&} $( \: \vec{\alpha} \hspace{+0.12em}\cdot\hspace{+0.12em} \vec{\varphi} ~=~ 0 \: )$

\newpage

\par \bigskip {\centering\subsubsection*{General Observations}}

\bigskip \smallskip

\noindent \S \ Forces and fields must be expressed with relational quantities ( the Lorentz force must be expressed with the relational velocity {\small $\mathbf{v}$}, the electric field must be expressed with the relational position {\small $\mathbf{r}$}, etc. )
\par \medskip \smallskip
\noindent \S \ The operator {\small $( \, \dot{\times} \, )$} must be replaced by the operator {\small $( \, \times \, )$} or the operator {\small $( \, \wedge \, )$} as follows: {\small $( \, \mathbf{a} \hspace{+0.24em}\dot{\times}\hspace{+0.30em} \mathbf{b} = \mathbf{b} \times \mathbf{a} \, )$} or {\small $( \, \mathbf{a} \hspace{+0.24em}\dot{\times}\hspace{+0.30em} \mathbf{b} = \mathbf{b} \wedge \mathbf{a} \, )$}
\par \medskip \smallskip
\noindent \S \ Inertial and non-inertial observers must not introduce fictitious forces into ${\mathbf{F}}$.
\par \medskip \smallskip
\noindent \S \ According to this article and special relativity, intrinsic mass is not additive.
\par \medskip \smallskip
\noindent \S \ The intrinsic mass quantity {\small $( \: m \: )$} is invariant under transformations between inertial and non-inertial (\,all\,) frames.
\par \medskip \smallskip
\noindent \S \ The relational quantities {\small $( \: \nu, t, \mathbf{r}, \mathbf{v}, \mathbf{a} \: )$} are invariant under transformations between inertial and non-inertial (\,non-rotating\,) frames.
\par \medskip \smallskip
\noindent \S \ Therefore, the kinematic and dynamic quantities {\small $( \: f, \bar{\mathbf{r}}, \bar{\mathbf{v}}, \bar{\mathbf{a}}, \mathbf{P}, \mathbf{L}, \mathbf{F}, \mathrm{W}, \mathrm{K} \: )$} are invariant under transformations between inertial and non-inertial (\,non-rotating\,) frames.
\par \medskip \smallskip
\noindent \S \ However, it is natural to consider the following generalization:
\par \medskip \smallskip
\noindent $\bullet$ It would also be possible to obtain relational quantities {\small $( \: \nu, t, \mathbf{r}, \mathbf{v}, \mathbf{a} \: )$} that would be invariant under transformations between inertial and non\hspace{+0.03em}-\hspace{+0.03em}inertial \hbox{(\,all\,) frames}.
\par \medskip \smallskip
\noindent $\bullet$ The kinematic and dynamic quantities {\small $( \: f, \bar{\mathbf{r}}, \bar{\mathbf{v}}, \bar{\mathbf{a}}, \mathbf{P}, \mathbf{L}, \mathbf{F}, \mathrm{W}, \mathrm{K} \: )$} would also be given by the equations of this article.
\par \medskip \smallskip
\noindent $\bullet$ Therefore, the kinematic and dynamic quantities {\small $( \: f, \bar{\mathbf{r}}, \bar{\mathbf{v}}, \bar{\mathbf{a}}, \mathbf{P}, \mathbf{L}, \mathbf{F}, \mathrm{W}, \mathrm{K} \: )$} would be invariant under transformations between inertial and non\hspace{+0.03em}-\hspace{+0.03em}inertial \hbox{(\,all\,) frames}.

\vspace{+0.45em}

\par \bigskip {\centering\subsubsection*{Bibliography}}

\bigskip \medskip

\par \noindent [1] \, R. A. Nelson, J. Math. Phys. {\bf 28}, 2379 (1987).
\medskip \smallskip
\par \noindent [2] \, R. A. Nelson, J. Math. Phys. {\bf 35}, 6224 (1994).
\medskip \smallskip
\par \noindent [3] \, H. Nikoli\'c,\hspace{+0.12em} Phys. Rev. A\hspace{+0.12em} {\bf 61},\hspace{+0.09em} 032109\hspace{+0.09em} (2000).
\medskip \smallskip
\par \noindent [4] \, V. V. Voytik, Gravit. Cosmol. {\bf 19}, 193 (2013).
\medskip \smallskip
\par \noindent [5] \, C. M{\o}ller, The Theory of Relativity (1952).

\newpage

\par \bigskip {\centering\subsubsection*{The Twin Paradox}}

\bigskip \smallskip

\noindent If a clock A is at rest at the origin O of an inertial or non-inertial (\,uniform circular motion\,) frame S and another clock B is at rest at the origin O' of a non-inertial (\,uniform circular motion\,) frame S' then the relational time $t_{A}$ of clock A and the relational time $t_{B}$ of clock B are given by:
\par \vspace{-0.66em}
\begin{eqnarray*}
t_{A} ~=~ \int_0^{\mathtt{t}_A} \gamma_{(\,\vec\varphi\;)} \; \mathtt{dt}_{A} \, - \, \gamma_{(\,\vec\varphi\;)} \; \frac{\vec{r}_{A} \cdot \vec{\varphi}}{c^2}
\end{eqnarray*}
\begin{eqnarray*}
t_{B} ~=~ \int_0^{\mathtt{t}_B} \gamma_{(\,\vec\varphi\,')} \; \mathtt{dt}_{B} \, - \, \gamma_{(\,\vec\varphi\,')} \; \frac{\vec{r}_{B} \cdot \vec{\varphi}{\,'}}{c^2}
\end{eqnarray*}
\par \vspace{+0.81em}
\noindent The position of the origin O relative to the frame S is always zero {\small ( $\vec{r}_{A} = 0$ )} and since $\,\gamma_{(\,\vec\varphi\;)}\,$ is a constant in the frame S, we obtain:
\par \vspace{-0.66em}
\begin{eqnarray*}
t_{A} ~=~ \int_0^{\mathtt{t}_A} \gamma_{(\,\vec\varphi\;)} \; \mathtt{dt}_{A}
\end{eqnarray*}
\begin{eqnarray*}
t_{A} ~=~ \gamma_{(\,\vec\varphi\;)} \; \mathtt{t}_{A}
\end{eqnarray*}
\par \vspace{+0.81em}
\noindent The position of the origin O' relative to the frame S' is always zero {\small ( $\vec{r}_{B} = 0$ )} and since $\,\gamma_{(\,\vec\varphi\,')}\,$ is a constant in the frame S', we obtain:
\par \vspace{-0.66em}
\begin{eqnarray*}
t_{B} ~=~ \int_0^{\mathtt{t}_B} \gamma_{(\,\vec\varphi\,')} \; \mathtt{dt}_{B}
\end{eqnarray*}
\begin{eqnarray*}
t_{B} ~=~ \gamma_{(\,\vec\varphi\,')} \; \mathtt{t}_{B}
\end{eqnarray*}
\par \vspace{+0.81em}
\noindent The clocks A and B spatially coincide at the relational time \hbox {{\small ( $t_0 = t_{0A} = t_{0B}$ )}} and at the relational time {\small ( $t = t_{A} = t_{B}$ )} Since {\small ( $t_{A} = t_{B}$ )} then we have:
\par \vspace{-0.66em}
\begin{eqnarray*}
\gamma_{(\,\vec\varphi\;)} \; \mathtt{t}_{A} ~=~ \gamma_{(\,\vec\varphi\,')} \; \mathtt{t}_{B}
\end{eqnarray*}
\par \vspace{+0.81em}
\noindent Therefore, if {\small $\,\gamma_{(\,\vec\varphi\;)} > \gamma_{(\,\vec\varphi\,')}\,$} then {\small (\:$\mathtt{t}_{A} < \mathtt{t}_{B}$\:)} if {\small $\,\gamma_{(\,\vec\varphi\;)} = \gamma_{(\,\vec\varphi\,')}\,$} then \hbox {\small {(\:$\mathtt{t}_{A} = \mathtt{t}_{B}$\:)}} and if {\small $\,\gamma_{(\,\vec\varphi\;)} < \gamma_{(\,\vec\varphi\,')}\,$} then {\small (\:$\mathtt{t}_{A} > \mathtt{t}_{B}$\:)}
\par \vspace{+0.90em}
\noindent Where {\small ( $\vec\varphi$ )} is the velocity of the auxiliary-point relative to the frame S and \hbox {{\small ( $\vec\varphi\,'$ )}} is the velocity of the auxiliary-point relative to the frame S'.

\newpage

\par \bigskip {\centering\subsubsection*{The Kinetic Force}}

\bigskip \smallskip

\noindent The kinetic force \hbox {$\mathbf{K}^{a}_{\hspace{+0.012em}ij}$ exerted} on a particle $i$ with intrinsic mass $m_i$ by another particle $j$ with intrinsic mass $m_j$ \hbox {is given by}:
\par \vspace{-0.54em}
\begin{eqnarray*}
\mathbf{K}^{a}_{\hspace{+0.012em}ij} ~=\, - \; \Bigg [ \; \dfrac{m_i \, m_j}{\mathbb{M}} \, ( \, \bar{\mathbf{a}}_{\hspace{+0.045em}i} - \bar{\mathbf{a}}_{j} \, ) \; \Bigg ]
\end{eqnarray*}
\par \vspace{+0.60em}
\noindent where $\bar{\mathbf{a}}_{\hspace{+0.045em}i}$ is the special acceleration of particle $i$, $\bar{\mathbf{a}}_{j}$ is the special acceleration of particle $j$ and $\mathbb{M}$ {\small ( $ = \sum_z m_z$ )} is the sum of the intrinsic masses of all the particles of the Universe.
\par \vspace{+0.60em}
\noindent The kinetic force $\mathbf{K}^{u}_{\hspace{+0.030em}i}$ exerted on a particle $i$ with intrinsic mass $m_i$ by the Universe is given by:
\par \vspace{-0.45em}
\begin{eqnarray*}
\mathbf{K}^{u}_{\hspace{+0.030em}i} ~=\, - \; m_i \; \dfrac{\sum_z m_z \, \bar{\mathbf{a}}_{\hspace{+0.045em}z}}{\sum_z m_z}
\end{eqnarray*}
\par \vspace{+0.60em}
\noindent where $m_z$ and $\bar{\mathbf{a}}_{\hspace{+0.045em}z}$ are the intrinsic mass and the special acceleration of the \textit{z}-th particle of the Universe.
\par \vspace{+0.60em}
\noindent From the above equations it follows that the net kinetic force $\mathbf{K}_i$ {\small ( $ = \sum_j \, \mathbf{K}^{a}_{\hspace{+0.012em}ij}$ $+ \; \mathbf{K}^{u}_{\hspace{+0.030em}i}$ )} acting on a particle $i$ with intrinsic mass $m_i$ is given by:
\par \vspace{-0.60em}
\begin{eqnarray*}
\mathbf{K}_i ~=\, - \; m_i \, \bar{\mathbf{a}}_{\hspace{+0.045em}i}
\end{eqnarray*}
\par \vspace{+0.60em}
\noindent where $\bar{\mathbf{a}}_{\hspace{+0.045em}i}$ is the special acceleration of particle $i$.
\par \vspace{+0.60em}
\noindent Now, substituting ( $\mathbf{F}_i = m_i \, \bar{\mathbf{a}}_{\hspace{+0.045em}i}$ ) and rearranging, we obtain:
\par \vspace{-0.75em}
\begin{eqnarray*}
\mathbf{K}_i + \mathbf{F}_i ~=~ 0
\end{eqnarray*}
\par \vspace{+0.60em}
\noindent If we define $\mathbf{T}_i$ ( $\doteq\, \mathbf{K}_i + \mathbf{F}_i$ ) as the total force acting on the particle $i$ then:
\par \vspace{-0.75em}
\begin{eqnarray*}
\mathbf{T}_i ~=~ 0
\end{eqnarray*}
\par \vspace{+0.60em}
\noindent Therefore, the total force $\mathbf{T}_i$ acting on any particle $i$ is always zero.
\par \vspace{+0.75em}
\noindent On the other hand, if an observer uses an auxiliary-point such that the linear momentum of the Universe ( that is, an isolated system of particles ) is zero {\hbox {\small ( $\sum_z m_z \hspace{+0.09em} \bar{\mathbf{v}}_z = 0$ )}} then for this observer the kinetic force $\mathbf{K}^{u}_{\hspace{+0.030em}i}$ exerted on any particle $i$ by the Universe is also zero, since {\hbox {\small ( $\sum_z m_z \hspace{+0.09em} \bar{\mathbf{a}}_z = 0$ )}}

\newpage

\par \bigskip {\centering\subsubsection*{Appendix I}}

\smallskip

\par \bigskip {\centering\subsubsection*{System of Equations I}}

\bigskip \bigskip

\begin{center}
\begin{tabular}{ccccc}
& & {\framebox(\xxa,\yya){[1]}} \\
& & {\makebox(\xxa,\yya){$\downarrow$ $d\hspace{+0.012em}t$ $\downarrow$}} \\
{\framebox(\xxa,\yya){[4]}} & {\makebox(\xxc,\yya){$\leftarrow \hspace{+0.24em}\dot{\times}\hspace{+0.30em} \mathbf{r} \leftarrow$}} & {\framebox(\xxa,\yya){[2]}} \\
{\makebox(\xxa,\yya){$\downarrow$ $d\hspace{+0.012em}t$ $\downarrow$}} & & {\makebox(\xxa,\yya){$\downarrow$ $d\hspace{+0.012em}t$ $\downarrow$}} \\
{\framebox(\xxa,\yya){[5]}} & {\makebox(\xxc,\yya){$\leftarrow \hspace{+0.24em}\dot{\times}\hspace{+0.30em} \mathbf{r} \leftarrow$}} & {\framebox(\xxa,\yya){[3]}} & {\makebox(\xxc,\yya){$\rightarrow \hspace{-0.001em} \int \hspace{+0.03em} d\hspace{+0.036em}\mathbf{r} \hspace{+0.001em} \rightarrow$}} & {\framebox(\xxa,\yya){[6]}}
\end{tabular}
\end{center}
\par \vspace{+0.90em}
\begin{eqnarray*}
\hspace{+1.17em} [\,1\,] \;\;\;\;\;\; \dfrac{1}{\mu} \; \bigg [ \,\, \int \mathbf{P} \; d\hspace{+0.012em}t \; - \int \hspace{-0.30em} \int \mathbf{F} \; d\hspace{+0.012em}t \, d\hspace{+0.012em}t \,\, \bigg ] =~ 0
\end{eqnarray*}
\par \vspace{+0.15em}
\begin{eqnarray*}
\hspace{+1.17em} [\,2\,] \;\;\;\;\;\; \dfrac{1}{\mu} \; \bigg [ \,\, \mathbf{P} \; - \int \mathbf{F} \; d\hspace{+0.012em}t \,\, \bigg ] =~ 0
\end{eqnarray*}
\par \vspace{+0.15em}
\begin{eqnarray*}
\hspace{+1.17em} [\,3\,] \;\;\;\;\;\; \dfrac{1}{\mu} \; \bigg [ \,\, \dfrac{d\hspace{+0.045em}\mathbf{P}}{d\hspace{+0.012em}t} \, - \; \mathbf{F} \,\, \bigg ] =~ 0
\end{eqnarray*}
\par \vspace{+0.15em}
\begin{eqnarray*}
\hspace{+1.17em} [\,4\,] \;\;\;\;\;\; \dfrac{1}{\mu} \; \bigg [ \,\, \mathbf{P} \; - \int \mathbf{F} \; d\hspace{+0.012em}t \,\, \bigg ] \hspace{+0.24em}\dot{\times}\hspace{+0.30em} \mathbf{r} ~=~ 0
\end{eqnarray*}
\par \vspace{+0.15em}
\begin{eqnarray*}
\hspace{+1.17em} [\,5\,] \;\;\;\;\;\; \dfrac{1}{\mu} \; \bigg [ \,\, \dfrac{d\hspace{+0.045em}\mathbf{P}}{d\hspace{+0.012em}t} \, - \; \mathbf{F} \,\, \bigg ] \hspace{+0.24em}\dot{\times}\hspace{+0.30em} \mathbf{r} ~=~ 0
\end{eqnarray*}
\par \vspace{+0.15em}
\begin{eqnarray*}
\hspace{+1.17em} [\,6\,] \;\;\;\;\;\; \dfrac{1}{\mu} \; \bigg [ \,\, \int \dfrac{d\hspace{+0.045em}\mathbf{P}}{d\hspace{+0.012em}t} \cdot d\hspace{+0.036em}\mathbf{r} \; - \int \mathbf{F} \cdot d\hspace{+0.036em}\mathbf{r} \,\, \bigg ] =~ 0
\end{eqnarray*}
\par \vspace{+0.33em}
\begin{eqnarray*}
\hspace{+1.17em} [\,\mu\,] \; \textrm{is an arbitrary constant with dimension of mass {\small $( \, \mathrm{M} \, )$}}
\end{eqnarray*}

\newpage

\par \bigskip {\centering\subsubsection*{Appendix II}}

\smallskip

\par \bigskip {\centering\subsubsection*{System of Equations II}}

\bigskip \bigskip

\begin{center}
\begin{tabular}{ccccc}
& & {\framebox(\xxa,\yya){[1]}} \\
& & {\makebox(\xxa,\yya){$\downarrow$ $d\hspace{+0.012em}t$ $\downarrow$}} \\
{\framebox(\xxa,\yya){[4]}} & {\makebox(\xxc,\yya){$\leftarrow \hspace{+0.24em}\dot{\times}\hspace{+0.30em} \mathbf{r} \leftarrow$}} & {\framebox(\xxa,\yya){[2]}} \\
{\makebox(\xxa,\yya){$\downarrow$ $d\hspace{+0.012em}t$ $\downarrow$}} & & {\makebox(\xxa,\yya){$\downarrow$ $d\hspace{+0.012em}t$ $\downarrow$}} \\
{\framebox(\xxa,\yya){[5]}} & {\makebox(\xxc,\yya){$\leftarrow \hspace{+0.24em}\dot{\times}\hspace{+0.30em} \mathbf{r} \leftarrow$}} & {\framebox(\xxa,\yya){[3]}} & {\makebox(\xxc,\yya){$\rightarrow \hspace{-0.001em} \int \hspace{+0.03em} d\hspace{+0.036em}\mathbf{r} \hspace{+0.001em} \rightarrow$}} & {\framebox(\xxa,\yya){[6]}}
\end{tabular}
\end{center}
\par \vspace{+0.90em}
\begin{eqnarray*}
\hspace{+1.17em} [\,1\,] \;\;\;\;\;\; \dfrac{1}{\mu} \; \bigg [ \,\, m \, \bar{\mathbf{r}} \, - \int \hspace{-0.30em} \int \mathbf{F} \; d\hspace{+0.012em}t \, d\hspace{+0.012em}t \,\, \bigg ] =~ 0
\end{eqnarray*}
\par \vspace{+0.15em}
\begin{eqnarray*}
\hspace{+1.17em} [\,2\,] \;\;\;\;\;\; \dfrac{1}{\mu} \; \bigg [ \,\, m \, \bar{\mathbf{v}} \, - \int \mathbf{F} \; d\hspace{+0.012em}t \,\, \bigg ] =~ 0
\end{eqnarray*}
\par \vspace{+0.15em}
\begin{eqnarray*}
\hspace{+1.17em} [\,3\,] \;\;\;\;\;\; \dfrac{1}{\mu} \; \bigg [ \,\, m \, \bar{\mathbf{a}} \: - \; \mathbf{F} \,\, \bigg ] =~ 0
\end{eqnarray*}
\par \vspace{+0.15em}
\begin{eqnarray*}
\hspace{+1.17em} [\,4\,] \;\;\;\;\;\; \dfrac{1}{\mu} \; \bigg [ \,\, m \, \bar{\mathbf{v}} \, - \int \mathbf{F} \; d\hspace{+0.012em}t \,\, \bigg ] \hspace{+0.24em}\dot{\times}\hspace{+0.30em} \mathbf{r} ~=~ 0
\end{eqnarray*}
\par \vspace{+0.15em}
\begin{eqnarray*}
\hspace{+1.17em} [\,5\,] \;\;\;\;\;\; \dfrac{1}{\mu} \; \bigg [ \,\, m \, \bar{\mathbf{a}} \: - \; \mathbf{F} \,\, \bigg ] \hspace{+0.24em}\dot{\times}\hspace{+0.30em} \mathbf{r} ~=~ 0
\end{eqnarray*}
\par \vspace{+0.15em}
\begin{eqnarray*}
\hspace{+1.17em} [\,6\,] \;\;\;\;\;\; \dfrac{1}{\mu} \; \bigg [ \,\, m \, f \, c^2 \, - \int \mathbf{F} \cdot d\hspace{+0.036em}\mathbf{r} \,\, \bigg ] =~ 0
\end{eqnarray*}
\par \vspace{+0.33em}
\begin{eqnarray*}
\hspace{+1.17em} [\,\mu\,] \; \textrm{is an arbitrary constant with dimension of mass {\small $( \, \mathrm{M} \, )$}}
\end{eqnarray*}

\end{document}

