
\documentclass[10pt,fleqn]{article}
%\documentclass[a4paper,10pt]{article}
%\documentclass[letterpaper,10pt]{article}

\usepackage[dvips]{geometry}
\geometry{papersize={131.1mm,195.0mm}}
\geometry{totalwidth=110.1mm,totalheight=159.0mm}

\usepackage{graphicx}

\usepackage[english]{babel}
\usepackage{times}
\usepackage{amsfonts}
\usepackage{amsmath,bm}

\usepackage{hyperref}
\hypersetup{colorlinks=true,linkcolor=black}
\hypersetup{bookmarksnumbered=true,pdfstartview=FitH,pdfpagemode=UseNone}
\hypersetup{pdftitle={Special Relativity with Absolute Space and Time}}
\hypersetup{pdfauthor={A. Blato}}

\setlength{\arraycolsep}{1.74pt}

\begin{document}

\begin{center}

{\sc Special Relativity with Absolute Space and Time}

\bigskip \medskip

{A. Blato}

\bigskip \medskip

\small

Creative Commons Attribution 3.0 License

\smallskip

(2016) Buenos Aires

\medskip

Argentina

\smallskip

\bigskip \medskip

\parbox{78.60mm}{In special relativity, this article presents kinematic quantities that are invariant under Lorentz transformations.}

\end{center}

\normalsize

\vspace{-1.20em}

\par \bigskip {\centering\subsubsection*{Introduction}}

\bigskip \smallskip

\noindent From an auxiliary point object (called free-point) can be obtained kinematic quantities (such as absolute position, absolute time, etc.) that are invariant under Lorentz transformations.
\par \bigskip \smallskip
\noindent The free-point is a point object (massive particle) that must always be free of internal and external forces (or the net force acting on it must always be zero)
\par \bigskip \smallskip
\noindent The absolute position $( \, \breve{x}_i, \, \breve{y}_i, \, \breve{z}_i \, )$ and the absolute time $( \, \breve{t}_i \, )$ of a particle $i$ relative to an inertial reference frame S are given by:
\par \vspace{+0.30em}
\begin{eqnarray*}
\breve{x}_i ~\doteq~ \dfrac{x_i - V_x \, t_i}{\sqrt{1 - \dfrac{V_x^2}{c^2}}} \;\;\;\;\;\; , \;\;\;\;\;\; \breve{y}_i ~\doteq~ y_i \;\;\;\;\;\; , \;\;\;\;\;\; \breve{z}_i ~\doteq~ z_i
\end{eqnarray*}
\vspace{+0.60em}
\begin{eqnarray*}
\breve{t}_i ~\doteq~ \dfrac{t_i - \dfrac{V_x \, x_i}{c^2}}{\sqrt{1 - \dfrac{V_x^2}{c^2}}}
\end{eqnarray*}
\par \vspace{+1.50em}
\noindent where $( \, x_i, \, y_i, \, z_i, \, t_i \, )$ are the position and the time of the particle $i$ relative to the inertial reference frame S, $( \, V_x \, )$ is the velocity ( on the $x$ axis ) of the free-point relative to the inertial reference frame S and $( \, c \, )$ is the speed of light in vacuum.

\newpage

\par \bigskip {\centering\subsubsection*{Observations}}

\bigskip \smallskip

\noindent In this article, the kinematic quantities $( \, \breve{x}, \, \breve{y}, \, \breve{z}, \, \breve{t} \, )$ are always invariant under Lorentz transformations.
\par \bigskip \smallskip
\noindent From these quantities, it would be possible to obtain the absolute position $\breve{\mathbf{r}}$, the absolute velocity $\breve{\mathbf{v}}$ and the absolute acceleration $\breve{\mathbf{a}}$ of a particle (with rest mass $m_o$) relative to an inertial reference frame S.
\par \bigskip \smallskip
\noindent The linear momentum $\mathbf{P}$, the force $\mathbf{F}$, the work $\mathrm{W}$ and the kinetic energy $\mathrm{K}$, for the inertial reference frame S, would be given by:
\par \vspace{+0.30em}
\begin{eqnarray*}
\mathbf{P} ~\doteq~ \dfrac{m_o \, \breve{\mathbf{v}}}{\sqrt{1 - \dfrac{\breve{v}^2}{c^2}}}
\end{eqnarray*}
\vspace{+0.60em}
\begin{eqnarray*}
\mathbf{F} ~=~ \dfrac{d\mathbf{P}}{d\breve{t}}
\end{eqnarray*}
\vspace{+0.30em}
\begin{eqnarray*}
\mathrm{W} ~\doteq~ \int_1^{\hspace{+0.06em}2} \mathbf{F} \cdot d\breve{\mathbf{r}} ~=~ \Delta \, \mathrm{K}
\end{eqnarray*}
\vspace{+0.30em}
\begin{eqnarray*}
\mathrm{K} ~\doteq~ m_o \, c^2 \left ( \: \dfrac{1}{\sqrt{1 - \dfrac{\breve{v}^2}{c^2}}} - 1 \: \right )
\end{eqnarray*}
\par \vspace{+1.50em}
\noindent According to this article, the quantities $( \, \breve{\mathbf{r}}, \, \breve{\mathbf{v}}, \, \breve{\mathbf{a}}, \, \mathbf{P}, \, \mathbf{F}, \, \mathrm{W}, \, \mathrm{K} \, )$ would also be invariant under Lorentz transformations.
\par \bigskip \smallskip
\noindent However, this article considers, on one hand, that it would also be possible to obtain kinematic and dynamic quantities $( \, \breve{\mathbf{r}}, \, \breve{\mathbf{v}}, \, \breve{\mathbf{a}}, \, \mathbf{P}, \, \mathbf{F}, \, \mathrm{W}, \, \mathrm{K} \, )$ that would be invariant under transformations between inertial and non-inertial reference frames and, on the other hand, that the dynamic quantities $( \, \mathbf{P}, \, \mathbf{F}, \, \mathrm{W}, \, \mathrm{K} \, )$ would also be given by the above equations.

\end{document}

