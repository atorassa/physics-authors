
\documentclass[10pt,fleqn]{article}
%\documentclass[a4paper,10pt]{article}
%\documentclass[letterpaper,10pt]{article}

\usepackage[dvips]{geometry}
\geometry{papersize={131.1mm,210.6mm}}
\geometry{totalwidth=110.1mm,totalheight=174.6mm}

\usepackage{graphicx}

\usepackage[english]{babel}
\usepackage{times}
\usepackage{amsfonts}
\usepackage{amsmath,bm}

\usepackage{hyperref}
\hypersetup{colorlinks=true,linkcolor=black}
\hypersetup{bookmarksnumbered=true,pdfstartview=FitH,pdfpagemode=UseNone}
\hypersetup{pdftitle={A New Formulation of Special Relativity}}
\hypersetup{pdfauthor={A. Blato}}

\setlength{\arraycolsep}{1.74pt}

\begin{document}

\begin{center}

{\fontsize{10.98}{10.98}\selectfont \sc A New Formulation of Special Relativity}

\bigskip \medskip

{A. Blato}

\bigskip \medskip

\small

Creative Commons Attribution 3.0 License

\smallskip

(2016) Buenos Aires

\medskip

Argentina

\smallskip

\bigskip \medskip

\parbox{78.00mm}{This article presents a new formulation of special relativity whose kinematic and dynamic quantities are invariant under generalized Lorentz transformations.}

\end{center}

\normalsize

\vspace{-1.20em}

\par \bigskip {\centering\subsubsection*{Introduction}}

\bigskip \smallskip

\noindent From an auxiliary point object (called free-point) can be obtained kinematic quantities (such as absolute time, absolute position, etc.) that are invariant under generalized Lorentz transformations.
\par \bigskip \smallskip
\noindent The free-point is a point object (massive particle) that must always be free of internal and external forces (or the net force acting on it must always be zero)
\par \bigskip \smallskip
\noindent The absolute time $( \, \breve{t} \, )$, the absolute position $( \, \breve{\mathbf{r}} \, )$, the absolute velocity $( \, \breve{\mathbf{v}} \, )$ and the absolute acceleration $( \, \breve{\mathbf{a}} \, )$ of a particle relative to an inertial reference frame S are given by:
\par \vspace{+0.21em}
\begin{eqnarray*}
\breve{t} ~\doteq~ \hspace{+0.24em} \gamma \left ( t - \frac{\mathbf{r} \cdot \boldsymbol{\psi}}{c^2} \right )
\end{eqnarray*}
\vspace{-0.45em}
\begin{eqnarray*}
\breve{\mathbf{r}} ~\doteq~ \hspace{-0.09em} \left [ \; \mathbf{r} + \frac{\gamma^2}{\gamma + 1} \frac{( \mathbf{r} \cdot \boldsymbol{\psi} ) \, \boldsymbol{\psi}}{c^2} - \gamma \, \boldsymbol{\psi} \, t \; \right ]
\end{eqnarray*}
\vspace{-0.30em}
\begin{eqnarray*}
\breve{\mathbf{v}} \hspace{-0.12em} ~\doteq~ \hspace{-0.12em} \left [ \; \mathbf{v} + \frac{\gamma^2}{\gamma + 1} \frac{( \mathbf{v} \cdot \boldsymbol{\psi} ) \, \boldsymbol{\psi}}{c^2} - \hspace{+0.114em} \gamma \, \boldsymbol{\psi} \hspace{+0.114em} \; \right ] \frac{1}{\gamma \, ( 1 - \frac{\mathbf{v} \cdot \boldsymbol{\psi}}{c^2} )}
\end{eqnarray*}
\vspace{-0.30em}
\begin{eqnarray*}
\breve{\mathbf{a}} ~\doteq~ \hspace{-0.18em} \left [ \; \mathbf{a} - \frac{\gamma}{\gamma + 1} \frac{( \mathbf{a} \cdot \boldsymbol{\psi} ) \, \boldsymbol{\psi}}{c^2} + \frac{( \mathbf{a} \times \mathbf{v} ) \times \boldsymbol{\psi}}{c^2} \; \right ] \frac{1}{\gamma^2 \, ( 1 - \frac{\mathbf{v} \cdot \boldsymbol{\psi}}{c^2} )^3}
\end{eqnarray*}
\par \vspace{+1.20em}
\noindent where $( \, t, \, \mathbf{r}, \, \mathbf{v}, \, \mathbf{a} \, )$ are the time, the position, the velocity and the acceleration of the particle relative to the inertial reference frame S, $( \, \boldsymbol{\psi} \, )$ is the velocity of the free-point relative to the inertial reference frame S and $( \, c \, )$ is the speed of light in vacuum. $( \, \boldsymbol{\psi} \, )$ is a constant. {\small $\gamma = ({1 - \boldsymbol{\psi} \cdot \boldsymbol{\psi}/c^2})^{-1/2}$}

\newpage

\par \bigskip {\centering\subsubsection*{The Dynamics}}

\bigskip \smallskip

\noindent If we consider a particle with rest mass $m_o$ then the linear momentum $\mathbf{P}$ of the particle, the net force $\mathbf{F}$ acting on the particle, the work $\mathrm{W}$ done by the net force acting on the particle, and the kinetic energy $\mathrm{K}$ of the particle, for an inertial reference frame, are given by:
\par \vspace{-0.30em}
\begin{eqnarray*}
\mathbf{P} ~\doteq~ \frac{m_o \, \breve{\mathbf{v}}}{\sqrt{1 - \frac{\breve{v}^2}{c^2}}}
\end{eqnarray*}
\vspace{-0.30em}
\begin{eqnarray*}
\mathbf{F} ~=~ \frac{d\mathbf{P}}{d\breve{t}} ~=~ \frac{m_o \, \breve{\mathbf{a}}}{\sqrt{1 - \frac{\breve{v}^2}{c^2}}} + \frac{m_o \, \breve{\mathbf{v}}}{\left(1 - \frac{\breve{v}^2}{c^2}\right)^{\hspace{-0.03em}{3/2}}} \frac{(\breve{\mathbf{v}} \cdot \breve{\mathbf{a}})}{c^2}
\end{eqnarray*}
\vspace{-0.30em}
\begin{eqnarray*}
\mathrm{W} ~\doteq~ \int_1^{\hspace{+0.06em}2} \mathbf{F} \cdot d\breve{\mathbf{r}} ~=~ \Delta \, \mathrm{K}
\end{eqnarray*}
\vspace{-0.30em}
\begin{eqnarray*}
\mathrm{K} ~\doteq~ m_o \, c^2 \left ( \: \frac{1}{\sqrt{1 - \frac{\breve{v}^2}{c^2}}} - 1 \: \right )
\end{eqnarray*}
\par \vspace{+1.20em}
\noindent Forces and fields must be expressed only with absolute quantities. For example, the Lorentz force must be expressed with the absolute velocity $\breve{\mathbf{v}}$, the electric field must be expressed with the absolute position $\breve{\mathbf{r}}$, etc.

\vspace{+0.60em}

\par \bigskip {\centering\subsubsection*{Observations}}

\bigskip \smallskip

\noindent \S \ In this article, the quantities $( \, \breve{t}, \, \breve{\mathbf{r}}, \, \breve{\mathbf{v}}, \, \breve{\mathbf{a}}, \, \mathbf{P}, \, \mathbf{F}, \, \mathrm{W}, \, \mathrm{K} \, )$ are invariant under generalized Lorentz transformations.
\par \bigskip \smallskip
\noindent \S \ In arbitrary inertial reference frames ( $\breve{t}_{\mathrm{fp}} \ne \tau_{\mathrm{\hspace{+0.06em}fp}}$ or $\breve{\mathbf{r}}_{\mathrm{fp}} \ne 0$ ) \hbox {( ${\mathrm{fp}}$ = free-point )} a constant must be included in the definition of absolute time such that the \hbox {absolute} time and the proper time of the free-point are the same \hbox {( $\breve{t}_{\mathrm{fp}} = \tau_{\mathrm{\hspace{+0.06em}fp}}$ )} and another constant must also be included in the definition of absolute position such that the absolute position of the free-point is zero \hbox {( $\breve{\mathbf{r}}_{\mathrm{fp}} = 0$ )}
\par \bigskip \smallskip
\noindent \S \ Finally, this article considers, on one hand, that it would also be possible to obtain kinematic and dynamic quantities $( \, \breve{t}, \, \breve{\mathbf{r}}, \, \breve{\mathbf{v}}, \, \breve{\mathbf{a}}, \, \mathbf{P}, \, \mathbf{F}, \, \mathrm{W}, \, \mathrm{K} \, )$ that would be invariant under transformations between inertial and non-inertial reference frames and, on the other hand, that the dynamic quantities $( \, \mathbf{P}, \, \mathbf{F}, \, \mathrm{W}, \, \mathrm{K} \, )$ would also be given by the above equations.

\newpage

\par \bigskip {\centering\subsubsection*{Appendix}}

\smallskip

\par \bigskip {\centering\subsubsection*{Generalized Lorentz Transformations}}

\bigskip \smallskip

\noindent The time $( \, {t}' \, )$, the position $( \, {\mathbf{r}}' \, )$, the velocity $( \, {\mathbf{v}}' \, )$ and the acceleration $( \, {\mathbf{a}}' \, )$ of a particle relative to an inertial reference frame S' are given by:
\par \vspace{+0.21em}
\begin{eqnarray*}
{t}' ~=~ \hspace{+0.24em} \gamma \left ( t - \frac{\mathbf{r} \cdot \mathbf{V}}{c^2} \right )
\end{eqnarray*}
\vspace{-0.45em}
\begin{eqnarray*}
{\mathbf{r}}' ~=~ \hspace{-0.09em} \left [ \; \mathbf{r} + \frac{\gamma^2}{\gamma + 1} \frac{( \mathbf{r} \cdot \mathbf{V} ) \, \mathbf{V}}{c^2} - \gamma \, \mathbf{V} \, t \; \right ]
\end{eqnarray*}
\vspace{-0.30em}
\begin{eqnarray*}
{\mathbf{v}}' \hspace{-0.12em} ~=~ \hspace{-0.12em} \left [ \; \mathbf{v} + \frac{\gamma^2}{\gamma + 1} \frac{( \mathbf{v} \cdot \mathbf{V} ) \, \mathbf{V}}{c^2} - \hspace{+0.114em} \gamma \, \mathbf{V} \hspace{+0.114em} \; \right ] \frac{1}{\gamma \, ( 1 - \frac{\mathbf{v} \cdot \mathbf{V}}{c^2} )}
\end{eqnarray*}
\vspace{-0.30em}
\begin{eqnarray*}
{\mathbf{a}}' ~=~ \hspace{-0.18em} \left [ \; \mathbf{a} - \frac{\gamma}{\gamma + 1} \frac{( \mathbf{a} \cdot \mathbf{V} ) \, \mathbf{V}}{c^2} + \frac{( \mathbf{a} \times \mathbf{v} ) \times \mathbf{V}}{c^2} \; \right ] \frac{1}{\gamma^2 \, ( 1 - \frac{\mathbf{v} \cdot \mathbf{V}}{c^2} )^3}
\end{eqnarray*}
\par \vspace{+1.20em}
\noindent where $( \, t, \, \mathbf{r}, \, \mathbf{v}, \, \mathbf{a} \, )$ are the time, the position, the velocity and the acceleration of the particle relative to an inertial reference frame S, $( \, ${\small $\mathbf{V}$}$ \, )$ is the velocity of the inertial reference frame S' relative to the inertial reference frame S and $( \, c \, )$ is the speed of light in vacuum. $( \, ${\small $\mathbf{V}$}$ \, )$ is a constant. {\small $\gamma = ({1 - \mathbf{V} \hspace{-0.114em}\cdot\hspace{-0.114em} \mathbf{V}/c^2})^{-1/2}$}

\vspace{+0.60em}

\par \bigskip {\centering\subsubsection*{Bibliography}}

\bigskip \smallskip

\par \noindent \textbf{A. Einstein}, Relativity: The Special and General Theory.
\bigskip
\par \noindent \textbf{E. Mach}, The Science of Mechanics.
\bigskip
\par \noindent \textbf{R. Resnick and D. Halliday}, Physics.
\bigskip
\par \noindent \textbf{J. Kane and M. Sternheim}, Physics.
\bigskip
\par \noindent \textbf{B. Russell}, ABC of Relativity.
\bigskip
\par \noindent \textbf{A. French}, Special Relativity.

\end{document}

