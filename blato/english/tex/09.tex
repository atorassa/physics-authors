
\documentclass[10pt,fleqn]{article}
%\documentclass[a4paper,10pt]{article}
%\documentclass[letterpaper,10pt]{article}

\usepackage[dvips]{geometry}
\geometry{papersize={132.3mm,210.0mm}}
\geometry{totalwidth=111.3mm,totalheight=174.0mm}

\usepackage{graphicx}

\usepackage[english]{babel}
\usepackage{times}
\usepackage{chngpage}
\usepackage{amsfonts}
\usepackage{amsmath,bm}

\usepackage{hyperref}
\hypersetup{colorlinks=true,linkcolor=black}
\hypersetup{bookmarksnumbered=true,pdfstartview=FitH,pdfpagemode=UseNone}
\hypersetup{pdftitle={An Alternative Formulation of Special Relativity}}
\hypersetup{pdfauthor={A. Blato}}

\setlength{\arraycolsep}{1.74pt}

\newcommand{\yya}{27}
\newcommand{\xxa}{33}
\newcommand{\xxc}{45}

\begin{document}

\begin{center}

{\fontsize{9.96}{9.96}\selectfont \sc An Alternative Formulation of Special Relativity}

\bigskip \medskip

{\fontsize{9.60}{9.60}\selectfont A. Blato}

\bigskip \medskip

\small

Creative Commons Attribution 3.0 License

\smallskip

(2016) Buenos Aires

\medskip

Argentina

\smallskip

\bigskip \medskip

\parbox{83.40mm}{This article presents an alternative formulation of special relativity which can be applied in any inertial reference frame. In addition, a new universal force is proposed.}

\end{center}

\normalsize

\vspace{-1.50em}

\par \bigskip {\centering\subsubsection*{Introduction}}

\bigskip \smallskip

\noindent The intrinsic mass $( \, m \, )$ and the frequency factor $( \, f \, )$ of a massive particle \hbox {are given by}:
\par \vspace{-0.60em}
\begin{eqnarray*}
m ~\doteq~ m_o
\end{eqnarray*}
\vspace{-0.90em}
\begin{eqnarray*}
f ~\doteq~ \Big ( 1 - \dfrac{\mathbf{v} \cdot \mathbf{v}}{c^2} \hspace{+0.15em} \Big )^{\hspace{-0.24em}-\hspace{+0.03em}1/2}
\end{eqnarray*}
\par \vspace{+0.60em}
\noindent where $( \, m_o \, )$ is the rest mass of the massive particle, $( \, \mathbf{v} \, )$ is the velocity of the massive particle and $( \, c \, )$ is the speed of light in vacuum.
\par \vspace{+0.60em}
\noindent The intrinsic mass $( \, m \, )$ and the frequency factor $( \, f \, )$ of a non-massive particle \hbox {are given by}:
\par \vspace{-0.60em}
\begin{eqnarray*}
m ~\doteq~ \dfrac{h \, \kappa}{c^2}
\end{eqnarray*}
\vspace{-0.60em}
\begin{eqnarray*}
f ~\doteq~ \dfrac{\nu}{\kappa}
\end{eqnarray*}
\par \vspace{+0.60em}
\noindent where $( \, h \, )$ is the Planck constant, $( \, \nu \, )$ is the frequency of the non-massive particle, $( \, \kappa \, )$ is a positive universal constant with dimension of frequency and $( \, c \, )$ is the speed of light in vacuum.
\par \vspace{+0.60em}
\noindent In this article, a massive particle is a particle with non-zero rest mass and a non-massive particle is a particle with zero rest mass.

\newpage

\par \bigskip {\centering\subsubsection*{The Alternative Kinematics}}

\bigskip \smallskip

\noindent The special position $( \, \bar{\mathbf{r}} \, )$, the special velocity $( \, \bar{\mathbf{v}} \, )$ and the special acceleration $( \, \bar{\mathbf{a}} \, )$ of a ( massive or non-massive ) particle are given by:
\par \vspace{-0.30em}
\begin{eqnarray*}
\bar{\mathbf{r}} ~\doteq~ \int f \, \mathbf{v} \; d\hspace{+0.012em}t
\end{eqnarray*}
\vspace{-0.60em}
\begin{eqnarray*}
\bar{\mathbf{v}} ~\doteq~ \dfrac{d\hspace{+0.036em}\bar{\mathbf{r}}}{d\hspace{+0.012em}t} ~=~ f \, \mathbf{v}
\end{eqnarray*}
\vspace{-0.45em}
\begin{eqnarray*}
\bar{\mathbf{a}} ~\doteq~ \dfrac{d\hspace{+0.021em}\bar{\mathbf{v}}}{d\hspace{+0.012em}t} ~=~ f \, \dfrac{d\hspace{+0.021em}\mathbf{v}}{d\hspace{+0.012em}t} + \dfrac{d\hspace{-0.12em}f}{d\hspace{+0.012em}t} \, \mathbf{v}
\end{eqnarray*}
\par \vspace{+1.20em}
\noindent where $( \, f \, )$ and $( \, \mathbf{v} \, )$ are the frequency factor and the velocity of the particle.

\vspace{+0.60em}

\par \bigskip {\centering\subsubsection*{The Alternative Dynamics}}

\bigskip \smallskip

\noindent If we consider a ( massive or non-massive ) particle with intrinsic mass $( \, m \, )$ then the linear momentum $( \, \mathbf{P} \, )$ of the particle, the angular momentum $( \, \mathbf{L} \, )$ of the particle, the net force $( \, \mathbf{F} \, )$ acting on the particle, the work $( \, \mathrm{W} \, )$ done by the net force acting on the particle, and the kinetic energy $( \, \mathrm{K} \, )$ of the particle \hbox {are given by}:
\par \vspace{-0.30em}
\begin{eqnarray*}
\mathbf{P} ~\doteq~ m \, \bar{\mathbf{v}} ~=~ m \, f \, \mathbf{v}
\end{eqnarray*}
\vspace{-0.30em}
\begin{eqnarray*}
\mathbf{L} ~\doteq~ \mathbf{P} \hspace{+0.24em}\dot{\times}\hspace{+0.30em} \mathbf{r} ~=~ m \, \bar{\mathbf{v}} \hspace{+0.24em}\dot{\times}\hspace{+0.30em} \mathbf{r} ~=~ m \, f \, \mathbf{v} \hspace{+0.24em}\dot{\times}\hspace{+0.30em} \mathbf{r}
\end{eqnarray*}
\vspace{-0.30em}
\begin{eqnarray*}
\mathbf{F} ~=~ \dfrac{d\hspace{+0.045em}\mathbf{P}}{d\hspace{+0.012em}t} ~=~ m \, \bar{\mathbf{a}} ~=~ m \, \bigg [ \, f \, \dfrac{d\hspace{+0.021em}\mathbf{v}}{d\hspace{+0.012em}t} + \dfrac{d\hspace{-0.12em}f}{d\hspace{+0.012em}t} \, \mathbf{v} \, \bigg ]
\end{eqnarray*}
\vspace{-0.15em}
\begin{eqnarray*}
\mathrm{W} ~\doteq~ \hspace{-0.36em} \int_{\scriptscriptstyle 1}^{\hspace{+0.09em}{\scriptscriptstyle 2}} \mathbf{F} \cdot d\hspace{+0.036em}\mathbf{r} ~=~ \hspace{-0.36em} \int_{\scriptscriptstyle 1}^{\hspace{+0.09em}{\scriptscriptstyle 2}} \dfrac{d\hspace{+0.045em}\mathbf{P}}{d\hspace{+0.012em}t} \cdot d\hspace{+0.036em}\mathbf{r} ~=~ \Delta \, \mathrm{K}
\end{eqnarray*}
\vspace{-0.30em}
\begin{eqnarray*}
\mathrm{K} ~\doteq~ m \, f \, c^2
\end{eqnarray*}
\par \vspace{+0.90em}
\noindent where $( \, f, \, \mathbf{r}, \, \mathbf{v}, \, \bar{\mathbf{v}}, \, \bar{\mathbf{a}} \, )$ are the frequency factor, the position, the velocity, the special velocity and the special acceleration of the particle and $( \, c \, )$ is the speed of light in vacuum. The kinetic energy $( \, \mathrm{K}_o \, )$ of a massive particle at rest is $( \, m_o \, c^2 \, )$ \S \ On the other hand, {\small $( \, \mathbf{a} \hspace{+0.24em}\dot{\times}\hspace{+0.30em} \mathbf{b} = \mathbf{b} \times \mathbf{a} \, )$} or {\small $( \, \mathbf{a} \hspace{+0.24em}\dot{\times}\hspace{+0.30em} \mathbf{b} = \mathbf{b} \wedge \mathbf{a} \, )$}

\newpage

\begin{adjustwidth}{+0.36mm}{+0.36mm}

\par \bigskip {\centering\subsubsection*{The Kinetic Force}}

\bigskip \smallskip

\noindent The kinetic force \hbox {$\mathbf{K}^{a}_{\hspace{+0.012em}ij}$ exerted} on a particle $i$ with intrinsic mass $m_i$ by another particle $j$ with intrinsic mass $m_j$ \hbox {is given by}:
\par \vspace{-0.30em}
\begin{eqnarray*}
\mathbf{K}^{a}_{\hspace{+0.012em}ij} ~=\, - \; \Bigg [ \; \dfrac{m_i \, m_j}{\mathbb{M}} \, ( \, \bar{\mathbf{a}}_{\hspace{+0.045em}i} - \bar{\mathbf{a}}_{j} \, ) \; \Bigg ]
\end{eqnarray*}
\par \vspace{+0.90em}
\noindent where $\bar{\mathbf{a}}_{\hspace{+0.045em}i}$ is the special acceleration of particle $i$, $\bar{\mathbf{a}}_{j}$ is the special acceleration of particle $j$ and $\mathbb{M}$ {\small ( $ = \sum_z m_z$ )} is the sum of the intrinsic masses of all the particles of the Universe.
\par \vspace{+0.60em}
\noindent The kinetic force $\mathbf{K}^{u}_{\hspace{+0.030em}i}$ exerted on a particle $i$ with intrinsic mass $m_i$ by the Universe is given by:
\par \vspace{-0.30em}
\begin{eqnarray*}
\mathbf{K}^{u}_{\hspace{+0.030em}i} ~=\, - \; m_i \; \dfrac{\sum_z m_z \, \bar{\mathbf{a}}_{\hspace{+0.045em}z}}{\sum_z m_z}
\end{eqnarray*}
\par \vspace{+0.90em}
\noindent where $m_z$ and $\bar{\mathbf{a}}_{\hspace{+0.045em}z}$ are the intrinsic mass and the special acceleration of the \textit{z}-th particle of the Universe.
\par \vspace{+0.60em}
\noindent From the above equations it follows that the net kinetic force $\mathbf{K}_i$ {\small ( $ = \sum_j \, \mathbf{K}^{a}_{\hspace{+0.012em}ij}$ $+ \; \mathbf{K}^{u}_{\hspace{+0.030em}i}$ )} acting on a particle $i$ with intrinsic mass $m_i$ is given by:
\par \vspace{-0.30em}
\begin{eqnarray*}
\mathbf{K}_i ~=\, - \; m_i \, \bar{\mathbf{a}}_{\hspace{+0.045em}i}
\end{eqnarray*}
\par \vspace{+0.90em}
\noindent where $\bar{\mathbf{a}}_{\hspace{+0.045em}i}$ is the special acceleration of particle $i$.
\par \vspace{+0.60em}
\noindent Now, substituting ( $\mathbf{F}_i = m_i \, \bar{\mathbf{a}}_{\hspace{+0.045em}i}$ ) and rearranging, we obtain:
\par \vspace{-0.30em}
\begin{eqnarray*}
\mathbf{T}_i ~\doteq~ \mathbf{K}_i + \mathbf{F}_i ~=~ 0
\end{eqnarray*}
\par \vspace{+0.90em}
\noindent Therefore, the total force $\mathbf{T}_i$ acting on a particle $i$ is always zero.

\vspace{-0.60em}

\par \bigskip {\centering\subsubsection*{Bibliography}}

\bigskip \smallskip

\par \noindent \textbf{A. Einstein}, Relativity: The Special and General Theory.
\bigskip \smallskip
\par \noindent \textbf{E. Mach}, The Science of Mechanics.
\bigskip \smallskip
\par \noindent \textbf{W. Pauli}, Theory of Relativity.

\end{adjustwidth}

\newpage

\par \bigskip {\centering\subsubsection*{Appendix I}}

\smallskip

\par \bigskip {\centering\subsubsection*{System of Equations I}}

\bigskip \bigskip

\begin{center}
\begin{tabular}{ccccc}
& & {\framebox(\xxa,\yya){[1]}} \\
& & {\makebox(\xxa,\yya){$\downarrow$ $d\hspace{+0.012em}t$ $\downarrow$}} \\
{\framebox(\xxa,\yya){[4]}} & {\makebox(\xxc,\yya){$\leftarrow \hspace{+0.24em}\dot{\times}\hspace{+0.30em} \mathbf{r} \leftarrow$}} & {\framebox(\xxa,\yya){[2]}} \\
{\makebox(\xxa,\yya){$\downarrow$ $d\hspace{+0.012em}t$ $\downarrow$}} & & {\makebox(\xxa,\yya){$\downarrow$ $d\hspace{+0.012em}t$ $\downarrow$}} \\
{\framebox(\xxa,\yya){[5]}} & {\makebox(\xxc,\yya){$\leftarrow \hspace{+0.24em}\dot{\times}\hspace{+0.30em} \mathbf{r} \leftarrow$}} & {\framebox(\xxa,\yya){[3]}} & {\makebox(\xxc,\yya){$\rightarrow \hspace{-0.001em} \int \hspace{+0.03em} d\hspace{+0.036em}\mathbf{r} \hspace{+0.001em} \rightarrow$}} & {\framebox(\xxa,\yya){[6]}}
\end{tabular}
\end{center}
\par \vspace{+0.90em}
\begin{eqnarray*}
\hspace{+1.35em} [\,1\,] \;\;\;\;\;\; \dfrac{1}{\mu} \; \bigg [ \,\, \int \mathbf{P} \; d\hspace{+0.012em}t \; - \int \hspace{-0.30em} \int \mathbf{F} \; d\hspace{+0.012em}t \, d\hspace{+0.012em}t \,\, \bigg ] =~ 0
\end{eqnarray*}
\par \vspace{+0.15em}
\begin{eqnarray*}
\hspace{+1.35em} [\,2\,] \;\;\;\;\;\; \dfrac{1}{\mu} \; \bigg [ \,\, \mathbf{P} \; - \int \mathbf{F} \; d\hspace{+0.012em}t \,\, \bigg ] =~ 0
\end{eqnarray*}
\par \vspace{+0.15em}
\begin{eqnarray*}
\hspace{+1.35em} [\,3\,] \;\;\;\;\;\; \dfrac{1}{\mu} \; \bigg [ \,\, \dfrac{d\hspace{+0.045em}\mathbf{P}}{d\hspace{+0.012em}t} \, - \; \mathbf{F} \,\, \bigg ] =~ 0
\end{eqnarray*}
\par \vspace{+0.15em}
\begin{eqnarray*}
\hspace{+1.35em} [\,4\,] \;\;\;\;\;\; \dfrac{1}{\mu} \; \bigg [ \,\, \mathbf{P} \; - \int \mathbf{F} \; d\hspace{+0.012em}t \,\, \bigg ] \hspace{+0.24em}\dot{\times}\hspace{+0.30em} \mathbf{r} ~=~ 0
\end{eqnarray*}
\par \vspace{+0.15em}
\begin{eqnarray*}
\hspace{+1.35em} [\,5\,] \;\;\;\;\;\; \dfrac{1}{\mu} \; \bigg [ \,\, \dfrac{d\hspace{+0.045em}\mathbf{P}}{d\hspace{+0.012em}t} \, - \; \mathbf{F} \,\, \bigg ] \hspace{+0.24em}\dot{\times}\hspace{+0.30em} \mathbf{r} ~=~ 0
\end{eqnarray*}
\par \vspace{+0.15em}
\begin{eqnarray*}
\hspace{+1.35em} [\,6\,] \;\;\;\;\;\; \dfrac{1}{\mu} \; \bigg [ \,\, \int \dfrac{d\hspace{+0.045em}\mathbf{P}}{d\hspace{+0.012em}t} \cdot d\hspace{+0.036em}\mathbf{r} \; - \int \mathbf{F} \cdot d\hspace{+0.036em}\mathbf{r} \,\, \bigg ] =~ 0
\end{eqnarray*}
\par \vspace{+0.33em}
\begin{eqnarray*}
\hspace{+1.35em} [\,\mu\,] \; \textrm{is an arbitrary constant with dimension of mass {\small $( \, \mathrm{M} \, )$}}
\end{eqnarray*}

\newpage

\par \bigskip {\centering\subsubsection*{Appendix II}}

\smallskip

\par \bigskip {\centering\subsubsection*{System of Equations II}}

\bigskip \bigskip

\begin{center}
\begin{tabular}{ccccc}
& & {\framebox(\xxa,\yya){[1]}} \\
& & {\makebox(\xxa,\yya){$\downarrow$ $d\hspace{+0.012em}t$ $\downarrow$}} \\
{\framebox(\xxa,\yya){[4]}} & {\makebox(\xxc,\yya){$\leftarrow \hspace{+0.24em}\dot{\times}\hspace{+0.30em} \mathbf{r} \leftarrow$}} & {\framebox(\xxa,\yya){[2]}} \\
{\makebox(\xxa,\yya){$\downarrow$ $d\hspace{+0.012em}t$ $\downarrow$}} & & {\makebox(\xxa,\yya){$\downarrow$ $d\hspace{+0.012em}t$ $\downarrow$}} \\
{\framebox(\xxa,\yya){[5]}} & {\makebox(\xxc,\yya){$\leftarrow \hspace{+0.24em}\dot{\times}\hspace{+0.30em} \mathbf{r} \leftarrow$}} & {\framebox(\xxa,\yya){[3]}} & {\makebox(\xxc,\yya){$\rightarrow \hspace{-0.001em} \int \hspace{+0.03em} d\hspace{+0.036em}\mathbf{r} \hspace{+0.001em} \rightarrow$}} & {\framebox(\xxa,\yya){[6]}}
\end{tabular}
\end{center}
\par \vspace{+0.90em}
\begin{eqnarray*}
\hspace{+1.35em} [\,1\,] \;\;\;\;\;\; \dfrac{1}{\mu} \; \bigg [ \,\, m \, \bar{\mathbf{r}} \, - \int \hspace{-0.30em} \int \mathbf{F} \; d\hspace{+0.012em}t \, d\hspace{+0.012em}t \,\, \bigg ] =~ 0
\end{eqnarray*}
\par \vspace{+0.15em}
\begin{eqnarray*}
\hspace{+1.35em} [\,2\,] \;\;\;\;\;\; \dfrac{1}{\mu} \; \bigg [ \,\, m \, \bar{\mathbf{v}} \, - \int \mathbf{F} \; d\hspace{+0.012em}t \,\, \bigg ] =~ 0
\end{eqnarray*}
\par \vspace{+0.15em}
\begin{eqnarray*}
\hspace{+1.35em} [\,3\,] \;\;\;\;\;\; \dfrac{1}{\mu} \; \bigg [ \,\, m \, \bar{\mathbf{a}} \: - \; \mathbf{F} \,\, \bigg ] =~ 0
\end{eqnarray*}
\par \vspace{+0.15em}
\begin{eqnarray*}
\hspace{+1.35em} [\,4\,] \;\;\;\;\;\; \dfrac{1}{\mu} \; \bigg [ \,\, m \, \bar{\mathbf{v}} \, - \int \mathbf{F} \; d\hspace{+0.012em}t \,\, \bigg ] \hspace{+0.24em}\dot{\times}\hspace{+0.30em} \mathbf{r} ~=~ 0
\end{eqnarray*}
\par \vspace{+0.15em}
\begin{eqnarray*}
\hspace{+1.35em} [\,5\,] \;\;\;\;\;\; \dfrac{1}{\mu} \; \bigg [ \,\, m \, \bar{\mathbf{a}} \: - \; \mathbf{F} \,\, \bigg ] \hspace{+0.24em}\dot{\times}\hspace{+0.30em} \mathbf{r} ~=~ 0
\end{eqnarray*}
\par \vspace{+0.15em}
\begin{eqnarray*}
\hspace{+1.35em} [\,6\,] \;\;\;\;\;\; \dfrac{1}{\mu} \; \bigg [ \,\, m \, f \, c^2 \, - \int \mathbf{F} \cdot d\hspace{+0.036em}\mathbf{r} \,\, \bigg ] =~ 0
\end{eqnarray*}
\par \vspace{+0.33em}
\begin{eqnarray*}
\hspace{+1.35em} [\,\mu\,] \; \textrm{is an arbitrary constant with dimension of mass {\small $( \, \mathrm{M} \, )$}}
\end{eqnarray*}

\end{document}

