
\documentclass[10pt]{article}
%\documentclass[a4paper,10pt]{article}
%\documentclass[letterpaper,10pt]{article}

\usepackage[dvips]{geometry}
\geometry{papersize={134.1mm,177.0mm}}
\geometry{totalwidth=113.1mm,totalheight=141.0mm}

\usepackage{graphicx}

\usepackage[english]{babel}
\usepackage[latin1]{inputenc}
\usepackage[T1]{fontenc}
\usepackage{times}
\usepackage{amsfonts}
\usepackage{amsmath,bm}

\usepackage{hyperref}
\hypersetup{colorlinks=true,linkcolor=black,urlcolor=black}
\hypersetup{bookmarksnumbered=true,pdfstartview=FitH,pdfpagemode=UseNone}
\hypersetup{pdftitle={The Principle of Least Action}}
\hypersetup{pdfauthor={A. Blato}}

\setlength{\arraycolsep}{1.74pt}

\begin{document}

\begin{center}

{\LARGE The Principle of Least Action}

\bigskip \medskip

{\large A. Blato}

\bigskip \medskip

\small

Creative Commons Attribution 3.0 License

\smallskip

(2015) Buenos Aires

\medskip

Argentina

\smallskip

\bigskip \medskip

\parbox{87.60mm}{In classical mechanics, this article obtains the principle of least action for a single particle in a didactic and simple way.}

\end{center}

\normalsize

\vspace{-1.20em}

\par \bigskip {\centering\subsection*{Introduction}}

\bigskip \medskip

\noindent Let us consider the following tautological equation for a single particle:
\par \bigskip \smallskip
\noindent $\dfrac{d(\mathbf{v} \cdot \delta \mathbf{r})}{dt} = \delta \dfrac{1}{2} \mathbf{v} \cdot \mathbf{v} + \mathbf{a} \cdot \delta \mathbf{r}$
\par \bigskip \smallskip
\noindent Now, integrating with respect to time from $t_1$ to $t_2$, yields:
\par \bigskip \smallskip
\noindent $\int_{t_1}^{t_2} \left[ \dfrac{d(\mathbf{v} \cdot \delta \mathbf{r})}{dt} \right] \hspace{+0.12em} dt = \int_{t_1}^{t_2} \left[ \delta \dfrac{1}{2} \mathbf{v} \cdot \mathbf{v} + \mathbf{a} \cdot \delta \mathbf{r} \right] \hspace{+0.12em} dt$
\par \bigskip \smallskip
\noindent The left side of the equation is zero, therefore, we obtain:
\par \bigskip \smallskip
\noindent $0 = \int_{t_1}^{t_2} \left[ \delta \dfrac{1}{2} \mathbf{v} \cdot \mathbf{v} + \mathbf{a} \cdot \delta \mathbf{r} \right] \hspace{+0.12em} dt$
\par \bigskip \smallskip
\noindent In classical mechanics, this last tautological equation is the mathematical basis of the principle of least action for a single particle.

\newpage

\noindent Now, multiplying by $m$ (mass of the particle) the following tautological equation is obtained:
\par \bigskip \smallskip
\noindent $0 = \int_{t_1}^{t_2} \left[ \delta \dfrac{1}{2} \hspace{+0.12em} m \hspace{+0.12em} (\mathbf{v} \cdot \mathbf{v}) + m \hspace{+0.12em} \mathbf{a} \cdot \delta \mathbf{r} \right] \hspace{+0.12em} dt$
\par \bigskip \smallskip
\noindent Substituting $\mathbf{a}=\mathbf{F}/m$ (Newton's second law) the following empirical equation is obtained:
\par \bigskip \smallskip
\noindent $0 = \int_{t_1}^{t_2} \left[ \delta \dfrac{1}{2} \hspace{+0.12em} m \hspace{+0.12em} (\mathbf{v} \cdot \mathbf{v}) + \mathbf{F} \cdot \delta \mathbf{r} \right] \hspace{+0.12em} dt$
\par \bigskip \smallskip
\noindent If the particle is only subject to conservative forces then $\delta V = - \mathbf{F} \cdot \delta \mathbf{r}$ and since $T = \dfrac{1}{2} m \hspace{+0.12em} (\mathbf{v} \cdot \mathbf{v})$ yields:
\par \bigskip \smallskip
\noindent $0 = \int_{t_1}^{t_2} \left[ \hspace{+0.12em} \delta T - \delta V \hspace{+0.12em} \right] \hspace{+0.12em} dt$
\par \bigskip \smallskip
\noindent That is:
\par \bigskip \smallskip
\noindent $0 = \delta \int_{t_1}^{t_2} \left[ \hspace{+0.12em} T - V \hspace{+0.12em} \right] \hspace{+0.12em} dt$
\par \bigskip \smallskip
\noindent Or else:
\par \bigskip \smallskip
\noindent $\delta \int_{t_1}^{t_2} \left[ \hspace{+0.12em} T - V \hspace{+0.12em} \right] \hspace{+0.12em} dt = 0$
\par \bigskip \smallskip
\noindent Finally we obtain:
\par \bigskip \smallskip
\noindent $\delta \int_{t_1}^{t_2} L \hspace{+0.24em} dt = 0$
\par \bigskip \smallskip
\noindent Since $L= T - V$.

\newpage

\par \bigskip {\centering\subsection*{Annex}}

\bigskip \medskip

\noindent $\dfrac{d}{dt}(\bf{v} \cdot \delta \bf{r}) = \ldots$
\par \bigskip
\par \bigskip
\noindent $\dfrac{d}{dt}(m \, \bf{v} \cdot \delta \bf{r}) = \ldots$
\par \bigskip
\noindent $\sum_i \dfrac{d}{dt}(m_i \, {\bf{v}}_i \cdot \delta {\bf{r}}_i) = \ldots$
\par \bigskip
\noindent $\sum_{i,j} \dfrac{d}{dt}(m_i \, {\bf{v}}_i \cdot \dfrac{\partial {\bf{r}}_i}{\partial q_j} \; \delta q_j) = \ldots$
\par \bigskip
\par \bigskip
\noindent $\sum_{i,j} \dfrac{d}{dt}(m_i \, {\bf{v}}_i \cdot \dfrac{\partial {\bf{r}}_i}{\partial q_j} \; \delta q_j) = \sum_{i,j} m_i \, {\bf{v}}_i \cdot \dfrac{d}{dt}(\dfrac{\partial {\bf{r}}_i}{\partial q_j} \; \delta q_j) + \sum_{i,j} m_i \, {\bf{a}}_i \cdot \dfrac{\partial {\bf{r}}_i}{\partial q_j} \; \delta q_j$
\par \bigskip
\noindent $\sum_{i,j} \dfrac{d}{dt}(m_i \, {\bf{v}}_i \cdot \dfrac{\partial {\bf{r}}_i}{\partial q_j} \; \delta q_j) - \sum_{i,j} m_i \, {\bf{v}}_i \cdot \dfrac{d}{dt}(\dfrac{\partial {\bf{r}}_i}{\partial q_j} \; \delta q_j) = \sum_{i,j} m_i \, {\bf{a}}_i \cdot \dfrac{\partial {\bf{r}}_i}{\partial q_j} \; \delta q_j$
\par \bigskip
\noindent $\sum_{i,j} \big [ \dfrac{d}{dt}(m_i \, {\bf{v}}_i \cdot \dfrac{\partial {\bf{r}}_i}{\partial q_j}) - m_i \, {\bf{v}}_i \cdot \dfrac{d}{dt}(\dfrac{\partial {\bf{r}}_i}{\partial q_j}) \big ] \, \delta q_j = \sum_{i,j} m_i \, {\bf{a}}_i \cdot \dfrac{\partial {\bf{r}}_i}{\partial q_j} \; \delta q_j$
\par \bigskip
\par \bigskip
\noindent $\sum_{i,j} \big [ \dfrac{d}{dt}(m_i \, {\bf{v}}_i \cdot \dfrac{\partial {\bf{r}}_i}{\partial q_j}) - m_i \, {\bf{v}}_i \cdot \dfrac{d}{dt}(\dfrac{\partial {\bf{r}}_i}{\partial q_j}) \big ] \, \delta q_j = \sum_{i,j} {\bf{F}}_i \cdot \dfrac{\partial {\bf{r}}_i}{\partial q_j} \; \delta q_j$

\bigskip \smallskip

\par \bigskip {\centering\subsection*{Bibliography}}

\bigskip \medskip

\noindent \href{http://en.wikipedia.org/wiki/Principle\_of\_least\_action}{http://en.wikipedia.org/wiki/Principle\_of\_least\_action}
\par \bigskip \smallskip
\noindent \href{http://en.wikipedia.org/wiki/Virtual\_displacement}{http://en.wikipedia.org/wiki/Virtual\_displacement}
\par \bigskip \smallskip
\noindent \href{http://vixra.org/abs/1112.0052}{http://vixra.org/abs/{\small 1112.0052}}

\end{document}

