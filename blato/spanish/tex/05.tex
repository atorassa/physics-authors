
\documentclass[10pt,fleqn]{article}
%\documentclass[a4paper,10pt]{article}
%\documentclass[letterpaper,10pt]{article}

\usepackage[dvips]{geometry}
\geometry{papersize={131.1mm,202.5mm}}
\geometry{totalwidth=110.1mm,totalheight=166.5mm}

\usepackage{graphicx}

\usepackage[spanish]{babel}
\usepackage[latin1]{inputenc}
\usepackage[T1]{fontenc}
\usepackage{times}
\usepackage{amsfonts}
\usepackage{amsmath,bm}
\spanishdecimal{.}

\frenchspacing

\usepackage{hyperref}
\hypersetup{colorlinks=true,linkcolor=black}
\hypersetup{bookmarksnumbered=true,pdfstartview=FitH,pdfpagemode=UseNone}
\hypersetup{pdftitle={Relatividad Especial \& Segunda Ley de Newton}}
\hypersetup{pdfauthor={A. Blato}}

\setlength{\arraycolsep}{1.74pt}

\newcommand{\med}{\raise.5ex\hbox{$\scriptstyle 1$}\kern-.15em/\kern-.12em\lower.45ex\hbox{$\scriptstyle 2$}\;}

\begin{document}

\begin{center}

{\fontsize{10.77}{10.77}\selectfont \sc Relatividad Especial \hspace{+0.06em}\&\hspace{+0.06em} Segunda Ley de Newton}

\bigskip \medskip

{A. Blato}

\bigskip \medskip

\small

Licencia Creative Commons Atribuci�n 3.0

\smallskip

(2016) Buenos Aires

\medskip

Argentina

\smallskip

\bigskip \medskip

\parbox{87.60mm}{En relatividad especial, este art�culo demuestra que la segunda ley de Newton puede ser aplicada en cualquier sistema de referencia inercial.}

\end{center}

\normalsize

\vspace{-1.20em}

\par \bigskip {\centering\subsubsection*{Introducci�n}}

\bigskip \smallskip

\noindent En relatividad especial, el momento lineal $\mathbf{P}$ de una part�cula con masa en reposo $m_o$ est� dado por:
\par \vspace{-0.30em}
\begin{eqnarray*}
\mathbf{P} ~\doteq~ \frac{m_o \, \mathbf{v}}{\sqrt{1 - \frac{v^2}{c^2}}}
\end{eqnarray*}
\par \vspace{+0.60em}
\noindent La relaci�n entre la fuerza einsteniana neta $\mathbf{F}$ que act�a sobre la part�cula y el momento lineal $\mathbf{P}$ de la part�cula, est� dada por:
\par \vspace{-0.30em}
\begin{eqnarray*}
\mathbf{F} ~=~ \frac{d\mathbf{P}}{d{t}} ~=~ m_o \left [ \; \frac{\mathbf{a}}{\sqrt{1 - \frac{v^2}{c^2}}} + \frac{(\mathbf{a} \cdot \mathbf{v}) \: \mathbf{v}}{c^2\hspace{-0.18em}\left(1 - \frac{v^2}{c^2}\right)^{\hspace{-0.09em}{3/2}}} \; \right ]
\end{eqnarray*}
\par \vspace{+0.60em}
\noindent Ahora, puesto que $\mathbf{a} = \mathbf{1} \cdot \mathbf{a}$ ( tensor unitario ) y $( \mathbf{a} \cdot \mathbf{v} ) \: \mathbf{v} = ( \mathbf{v} \otimes \mathbf{v} ) \cdot \mathbf{a}$ \hbox {( producto} tensorial o di�dico ) entonces reemplazando se tiene:
\par \vspace{-0.30em}
\begin{eqnarray*}
\mathbf{F} ~=~ m_o \left [ \; \frac{\mathbf{1} \cdot \mathbf{a}}{\sqrt{1 - \frac{v^2}{c^2}}} + \frac{( \mathbf{v} \otimes \mathbf{v} ) \cdot \mathbf{a}}{c^2\hspace{-0.18em}\left(1 - \frac{v^2}{c^2}\right)^{\hspace{-0.09em}{3/2}}} \; \right ]
\end{eqnarray*}
\par \vspace{+0.60em}
\noindent que es:
\par \vspace{-0.30em}
\begin{eqnarray*}
\mathbf{F} ~=~ m_o \left [ \; \frac{\mathbf{1}}{\sqrt{1 - \frac{v^2}{c^2}}} + \frac{\hspace{-0.33em}( \mathbf{v} \otimes \mathbf{v} )}{c^2\hspace{-0.18em}\left(1 - \frac{v^2}{c^2}\right)^{\hspace{-0.09em}{3/2}}} \; \right ] \cdot \hspace{+0.06em}\mathbf{a}
\end{eqnarray*}

\newpage

\noindent Pasando el tensor entre corchetes de la ecuaci�n anterior ( denominado en este art�culo como el tensor de Newton ) al primer miembro, se obtiene:
\par \vspace{-0.30em}
\begin{eqnarray*}
\left [ \; \frac{\mathbf{1}}{\sqrt{1 - \frac{v^2}{c^2}}} + \frac{\hspace{-0.33em}( \mathbf{v} \otimes \mathbf{v} )}{c^2\hspace{-0.18em}\left(1 - \frac{v^2}{c^2}\right)^{\hspace{-0.09em}{3/2}}} \; \right ]^{\hspace{-0.15em}-1} \hspace{-0.24em}\cdot \hspace{+0.12em}\mathbf{F} ~=~ m_o \, \mathbf{a}
\end{eqnarray*}
\par \vspace{+0.90em}
\noindent Si identificamos el primer miembro de la ecuaci�n anterior como la fuerza newtoniana neta $\mathbf{\overline{F}}$ que act�a sobre la part�cula, entonces finalmente se tiene:
\par \vspace{-0.60em}
\begin{eqnarray*}
\mathbf{\overline{F}} ~=~ m_o \, \mathbf{a}
\end{eqnarray*}
\par \vspace{+0.60em}
\noindent Por lo tanto, la fuerza newtoniana neta $\mathbf{\overline{F}}$ que act�a sobre una part�cula siempre tiene igual direcci�n y sentido que la aceleraci�n $\mathbf{a}$ de la part�cula.

\vspace{-0.60em}

\par \bigskip {\centering\subsubsection*{Din�mica Newtoniana}}

\bigskip \smallskip

\noindent En relatividad especial, sea una part�cula con masa en reposo $m_o$ entonces el momento lineal $\mathbf{P}$ de la part�cula, la fuerza newtoniana neta $\mathbf{\overline{F}}$ que act�a sobre la part�cula, el trabajo $\mathrm{W}$ realizado por la fuerza newtoniana neta que act�a sobre la part�cula y la energ�a cin�tica $\mathrm{K}$ de la part�cula, para un sistema de referencia inercial, est�n dados por:
\par \vspace{-0.30em}
\begin{eqnarray*}
\mathbf{P} ~\doteq~ m_o \, \mathbf{v}
\end{eqnarray*}
\vspace{-0.30em}
\begin{eqnarray*}
\mathbf{\overline{F}} ~=~ \frac{d\mathbf{P}}{d{t}} ~=~ m_o \, \mathbf{a}
\end{eqnarray*}
\vspace{-0.30em}
\begin{eqnarray*}
\mathrm{W} ~\doteq~ \int_1^{\hspace{+0.06em}2} \mathbf{\overline{F}} \cdot d{\mathbf{r}} ~=~ \Delta \, \mathrm{K}
\end{eqnarray*}
\vspace{-0.30em}
\begin{eqnarray*}
\mathrm{K} ~\doteq~ \med \, m_o \, ( \mathbf{v} \cdot \mathbf{v} )
\end{eqnarray*}
\par \vspace{+0.90em}
\noindent donde $( \, \mathbf{r}, \, \mathbf{v}, \, \mathbf{a} \, )$ son la posici�n, la velocidad y la aceleraci�n de la part�cula respecto al sistema de referencia inercial. {\small $\mathbf{\overline{F}} ~=~ \mathbf{N}^{-1} \cdot \mathbf{F}$}, donde {\small $\mathbf{N}$} es el tensor de Newton y {\small $\mathbf{F}$} es la fuerza einsteniana neta que act�a sobre la part�cula.

\end{document}

