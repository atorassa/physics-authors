
\documentclass[10pt,fleqn]{article}
%\documentclass[a4paper,10pt]{article}
%\documentclass[letterpaper,10pt]{article}

\usepackage[dvips]{geometry}
\geometry{papersize={134.1mm,210.6mm}}
\geometry{totalwidth=113.1mm,totalheight=174.6mm}

\usepackage{graphicx}

\usepackage[spanish]{babel}
\usepackage[latin1]{inputenc}
\usepackage[T1]{fontenc}
\usepackage{times}
\usepackage{amsfonts}
\usepackage{amsmath,bm}
\spanishdecimal{.}

\frenchspacing

\usepackage{hyperref}
\hypersetup{colorlinks=true,linkcolor=black}
\hypersetup{bookmarksnumbered=true,pdfstartview=FitH,pdfpagemode=UseNone}
\hypersetup{pdftitle={Transformaciones Vectoriales de Lorentz}}
\hypersetup{pdfauthor={A. Blato}}

\setlength{\arraycolsep}{1.74pt}

\begin{document}

\begin{center}

{\fontsize{10.20}{10.20}\selectfont \sc Transformaciones Vectoriales de Lorentz}

\bigskip \medskip

{\fontsize{9.60}{9.60}\selectfont A. Blato}

\bigskip \medskip

\small

Licencia Creative Commons Atribuci�n 3.0

\smallskip

(2016) Buenos Aires

\medskip

Argentina

\smallskip

\bigskip \medskip

\parbox{99.00mm}{Este art�culo presenta las transformaciones vectoriales de Lorentz de tiempo, espacio, velocidad y aceleraci�n.}

\end{center}

\normalsize

\vspace{-1.20em}

\par \bigskip {\centering\subsubsection*{Introducci�n}}

\bigskip \smallskip

\noindent Sean dos sistemas de referencia inerciales S y S' cuyos or�genes coinciden en el tiempo cero ( para ambos sistemas ) entonces el tiempo $( \, {t}' \, )$, la posici�n $( \, {\mathbf{r}}' \, )$ la velocidad $( \, {\mathbf{v}}' \, )$ y la aceleraci�n $( \, {\mathbf{a}}' \, )$ de una part�cula (masiva o no masiva) respecto al sistema de referencia inercial S' est�n dados por:
\par \vspace{+0.21em}
\begin{eqnarray*}
{t}' ~=~ \hspace{+0.24em} \gamma \left ( t - \frac{\mathbf{r} \cdot \mathbf{V}}{c^2} \right )
\end{eqnarray*}
\vspace{+0.15em}
\begin{eqnarray*}
{\mathbf{r}}' ~=~ \hspace{-0.09em} \left [ \; \mathbf{r} + \frac{\gamma^2}{\gamma + 1} \frac{( \mathbf{r} \cdot \mathbf{V} ) \, \mathbf{V}}{c^2} - \gamma \, \mathbf{V} \, t \; \right ]
\end{eqnarray*}
\vspace{+0.30em}
\begin{eqnarray*}
{\mathbf{v}}' \hspace{-0.12em} ~=~ \hspace{-0.12em} \left [ \; \mathbf{v} + \frac{\gamma^2}{\gamma + 1} \frac{( \mathbf{v} \cdot \mathbf{V} ) \, \mathbf{V}}{c^2} - \hspace{+0.114em} \gamma \, \mathbf{V} \hspace{+0.114em} \; \right ] \frac{1}{\gamma \, ( 1 - \frac{\mathbf{v} \cdot \mathbf{V}}{c^2} )}
\end{eqnarray*}
\vspace{+0.30em}
\begin{eqnarray*}
{\mathbf{a}}' ~=~ \hspace{-0.18em} \left [ \; \mathbf{a} - \frac{\gamma}{\gamma + 1} \frac{( \mathbf{a} \cdot \mathbf{V} ) \, \mathbf{V}}{c^2} + \frac{( \mathbf{a} \times \mathbf{v} ) \times \mathbf{V}}{c^2} \; \right ] \frac{1}{\gamma^2 \, ( 1 - \frac{\mathbf{v} \cdot \mathbf{V}}{c^2} )^3}
\end{eqnarray*}
\par \vspace{+1.80em}
\noindent donde $( \, t, \, \mathbf{r}, \, \mathbf{v}, \, \mathbf{a} \, )$ son el tiempo, la posici�n, la velocidad y la aceleraci�n de la part�cula respecto al sistema de referencia inercial S, $( \, ${\small $\mathbf{V}$}$ \, )$ es la velocidad del \hbox {sistema} de referencia inercial S' respecto al sistema de referencia inercial S y $( \, c \, )$ es la velocidad de la luz en el vac�o. \hspace{-0.09em}$( \, ${\small $\mathbf{V}$}$ \, )$ es una constante. \hbox {{\small $\gamma = ({1 - \mathbf{V} \hspace{-0.114em}\cdot\hspace{-0.114em} \mathbf{V}/c^2})^{-1/2}$}}

\newpage

\noindent $\bullet$
$
\hspace{+1.20em} {\mathbf{v}}' ~=~ \dfrac{{d\mathbf{r}}'}{{dt}'} ~=~ \dfrac{{d\mathbf{r}}'}{{dt}'} \: \dfrac{dt}{dt} ~=~ \dfrac{{d\mathbf{r}}'}{dt} \: \dfrac{dt}{{dt}'} ~=~ \left ( \dfrac{{d\mathbf{r}}'}{dt} \right ) \dfrac{1}{\left ( \dfrac{{dt}'}{dt} \right )}
$

\par \vspace{+2.07em}

\noindent $\bullet$
$
\hspace{+1.20em} {\mathbf{a}}' ~=~ \dfrac{{d\mathbf{v}}'}{{dt}'} ~=~ \dfrac{{d\mathbf{v}}'}{{dt}'} \: \dfrac{dt}{dt} ~=~ \dfrac{{d\mathbf{v}}'}{dt} \: \dfrac{dt}{{dt}'} ~=~ \left ( \dfrac{{d\mathbf{v}}'}{dt} \right ) \dfrac{1}{\left ( \dfrac{{dt}'}{dt} \right )}
$

\par \vspace{+2.07em}

\noindent $\bullet$
$
\hspace{+1.20em} {dt}' ~=~ \hspace{+0.24em} \gamma \left ( dt - \frac{d\mathbf{r} \cdot \mathbf{V}}{c^2} \right )
$

\par \vspace{+2.07em}

\noindent $\bullet$
$
\hspace{+1.20em} \left ( \dfrac{{dt}'}{dt} \right ) ~=~ \hspace{+0.24em} \gamma \left ( 1 - \frac{\mathbf{v} \cdot \mathbf{V}}{c^2} \right )
$

\par \vspace{+2.07em}

\noindent $\bullet$
$
\hspace{+1.20em} {d\mathbf{r}}' ~=~ \hspace{-0.09em} \left [ \; d\mathbf{r} + \frac{\gamma^2}{\gamma + 1} \frac{( d\mathbf{r} \cdot \mathbf{V} ) \, \mathbf{V}}{c^2} - \gamma \, \mathbf{V} \, dt \; \right ]
$

\par \vspace{+2.07em}

\noindent $\bullet$
$
\hspace{+1.20em} \left ( \dfrac{{d\mathbf{r}}'}{dt} \right ) ~=~ \hspace{-0.09em} \left [ \; \mathbf{v} + \frac{\gamma^2}{\gamma + 1} \frac{( \mathbf{v} \cdot \mathbf{V} ) \, \mathbf{V}}{c^2} - \gamma \, \mathbf{V} \; \right ]
$

\par \vspace{+2.07em}

\noindent $\bullet$
$
\hspace{+1.20em} {d\mathbf{v}}' ~=~ \Big [ \; d\mathbf{m} \cdot n - \mathbf{m} \cdot dn \; \Big ] \: \dfrac{1}{n^2}
$

\par \vspace{+2.07em}

\noindent $\bullet$
$
\hspace{+1.20em} \mathbf{m} ~=~ \hspace{-0.18em} \left [ \; \mathbf{v} + \frac{\gamma^2}{\gamma + 1} \frac{( \mathbf{v} \cdot \mathbf{V} ) \, \mathbf{V}}{c^2} - \hspace{+0.114em} \gamma \, \mathbf{V} \hspace{+0.114em} \; \right ]
$

\par \vspace{+2.07em}

\noindent $\bullet$
$
\hspace{+1.20em} d\mathbf{m} ~=~ \hspace{-0.18em} \left [ \; d\mathbf{v} + \frac{\gamma^2}{\gamma + 1} \frac{( d\mathbf{v} \cdot \mathbf{V} ) \, \mathbf{V}}{c^2} \; \right ]
$

\par \vspace{+2.07em}

\noindent $\bullet$
$
\hspace{+1.20em} n ~=~ \hspace{-0.18em} \left [ \; \gamma \, ( 1 - \frac{\mathbf{v} \cdot \mathbf{V}}{c^2} ) \; \right ]
$

\par \vspace{+2.07em}

\noindent $\bullet$
$
\hspace{+1.20em} dn ~=~ \hspace{-0.18em} \left [ \; - \, \gamma \, \frac{d\mathbf{v} \cdot \mathbf{V}}{c^2} \; \right ]
$

\par \vspace{+2.07em}

\noindent $\bullet$
$
\hspace{+1.20em} \left ( \dfrac{{d\mathbf{v}}'}{dt} \right ) ~=~ \hspace{-0.18em} \left [ \; \mathbf{a} - \frac{\gamma}{\gamma + 1} \frac{( \mathbf{a} \cdot \mathbf{V} ) \, \mathbf{V}}{c^2} + \frac{( \mathbf{a} \times \mathbf{v} ) \times \mathbf{V}}{c^2} \; \right ] \frac{1}{\gamma \, ( 1 - \frac{\mathbf{v} \cdot \mathbf{V}}{c^2} )^2}
$

\newpage

\par \bigskip {\centering\subsubsection*{Bibliograf�a}}

\bigskip \smallskip

\par \noindent https://it.wikipedia.org/wiki/Trasformazione\_di\_Lorentz
\bigskip \medskip
\par \noindent https://en.wikipedia.org/wiki/Lorentz\_transformation
\bigskip \medskip
\par \noindent https://arxiv.org/abs/physics/0507099
\bigskip \medskip
\par \noindent https://arxiv.org/abs/physics/0702191
\bigskip \medskip
\par \noindent https://archive.org/details/blato\_links
\bigskip \medskip
\par \noindent https://archive.org/details/@a\_blato

\par \vspace{+1.20em}

\par \bigskip {\centering\subsubsection*{Ap�ndice I}}

\bigskip \medskip

\noindent En la bibliograf�a anterior, si la velocidad del sistema de referencia inercial S' respecto al sistema de referencia inercial S no es igual a cero $( \, ${\small $\mathbf{V} \ne 0$}$ \, )$ entonces:

\par \vspace{+1.80em}

$\dfrac{\gamma - 1}{\mathbf{V}^2} ~=~ \dfrac{\gamma^2}{\gamma + 1} \, \dfrac{1}{c^2}$ \hspace{+2.40em} $( \: ${\small $\mathbf{V}^2 = \mathbf{V} \hspace{-0.114em}\cdot\hspace{-0.114em} \mathbf{V}$}$ \: )$

\par \vspace{+2.10em}

\noindent donde $( \, c \, )$ es la velocidad de la luz en el vac�o. \hbox {{\small $\gamma = ({1 - \mathbf{V} \hspace{-0.114em}\cdot\hspace{-0.114em} \mathbf{V}/c^2})^{-1/2}$}}

\par \vspace{+1.20em}

\par \bigskip {\centering\subsubsection*{Ap�ndice II}}

\bigskip \medskip

\noindent Derivada del cociente:
$
\hspace{+1.20em} \mathbf{a} ~=~ \dfrac{\mathbf{m}}{n} \hspace{+0.75em} \rightarrow \hspace{+0.75em} d\mathbf{a} ~=~ \Big [ \; d\mathbf{m} \cdot n - \mathbf{m} \cdot dn \; \Big ] \: \dfrac{1}{n^2}
$

\par \vspace{+2.40em}

\noindent Regla de la expulsi�n:
$
\hspace{+1.20em} \mathbf{a} \times ( \, \mathbf{b} \times \mathbf{c} \, ) ~=~ \mathbf{b} \, ( \, \mathbf{a} \cdot \mathbf{c} \, ) - \mathbf{c} \, ( \, \mathbf{a} \cdot \mathbf{b} \, )
$

\par \vspace{+2.61em}

\noindent Anticonmutatividad:
$
\hspace{+1.20em} ( \, \mathbf{a} \times \mathbf{b} \, ) ~=~ - ( \, \mathbf{b} \times \mathbf{a} \, )
$

\end{document}

