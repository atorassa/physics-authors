
\documentclass[10pt]{article}
%\documentclass[a4paper,10pt]{article}
%\documentclass[letterpaper,10pt]{article}

\usepackage[dvips]{geometry}
\geometry{papersize={134.1mm,183.0mm}}
\geometry{totalwidth=113.1mm,totalheight=147.0mm}

\usepackage{graphicx}

\usepackage[spanish]{babel}
\usepackage[latin1]{inputenc}
\usepackage[T1]{fontenc}
\usepackage{times}
\usepackage{amsfonts}
\usepackage{amsmath,bm}
\spanishdecimal{.}

\frenchspacing

\usepackage{hyperref}
\hypersetup{colorlinks=true,linkcolor=black}
\hypersetup{bookmarksnumbered=true,pdfstartview=FitH,pdfpagemode=UseNone}
\hypersetup{pdftitle={Sobre el Principio de M�nima Acci�n}}
\hypersetup{pdfauthor={A. Blato}}

\setlength{\arraycolsep}{1.74pt}

\begin{document}

\begin{center}

{\LARGE Sobre el Principio de M�nima Acci�n}

\bigskip \medskip

{\large A. Blato}

\bigskip \medskip

\small

Licencia Creative Commons Atribuci�n 3.0

\smallskip

(2015) Buenos Aires

\medskip

Argentina

\smallskip

\bigskip \medskip

\parbox{90.60mm}{En mec�nica cl�sica, este art�culo obtiene el principio de m�nima acci�n de una manera did�ctica y sencilla para una sola part�cula.}

\end{center}

\normalsize

\vspace{-1.20em}

\par \bigskip {\centering\subsection*{Introducci�n}}

\bigskip \medskip

\noindent Consideremos la siguiente ecuaci�n tautol�gica (ecuaci�n que no se puede refutar emp�ricamente) para una sola part�cula:
\par \bigskip \smallskip
\noindent $\dfrac{d(\mathbf{v} \cdot \delta \mathbf{r})}{dt} = \delta \dfrac{1}{2} \mathbf{v} \cdot \mathbf{v} + \mathbf{a} \cdot \delta \mathbf{r}$
\par \bigskip \smallskip
\noindent Ahora integrando con respecto al tiempo desde $t_1$ a $t_2$, resulta:
\par \bigskip \smallskip
\noindent $\int_{t_1}^{t_2} \left[ \dfrac{d(\mathbf{v} \cdot \delta \mathbf{r})}{dt} \right] \hspace{+0.12em} dt = \int_{t_1}^{t_2} \left[ \delta \dfrac{1}{2} \mathbf{v} \cdot \mathbf{v} + \mathbf{a} \cdot \delta \mathbf{r} \right] \hspace{+0.12em} dt$
\par \bigskip \smallskip
\noindent El lado izquierdo de la ecuaci�n es igual a cero, por lo tanto, se obtiene:
\par \bigskip \smallskip
\noindent $0 = \int_{t_1}^{t_2} \left[ \delta \dfrac{1}{2} \mathbf{v} \cdot \mathbf{v} + \mathbf{a} \cdot \delta \mathbf{r} \right] \hspace{+0.12em} dt$
\par \bigskip \smallskip
\noindent Esta �ltima ecuaci�n a�n es tautol�gica y ser�a la ecuaci�n, la estructura o la base matem�tica del principio de m�nima acci�n en mec�nica cl�sica para una sola part�cula.

\newpage

\noindent Ahora si multiplicamos por la masa m de la part�cula que es un invariante nos da la siguiente ecuaci�n que tambi�n es tautol�gica:
\par \bigskip \smallskip
\noindent $0 = \int_{t_1}^{t_2} \left[ \delta \dfrac{1}{2} \hspace{+0.12em} m \hspace{+0.12em} (\mathbf{v} \cdot \mathbf{v}) + m \hspace{+0.12em} \mathbf{a} \cdot \delta \mathbf{r} \right] \hspace{+0.12em} dt$
\par \bigskip \smallskip
\noindent Pero como en todo sistema inercial $\mathbf{a}=\mathbf{F}/m$ (as� como en todo sistema no inercial considerando a las fuerzas ficticias) entonces reemplazando nos da la siguiente ecuaci�n que ya es emp�rica:
\par \bigskip \smallskip
\noindent $0 = \int_{t_1}^{t_2} \left[ \delta \dfrac{1}{2} \hspace{+0.12em} m \hspace{+0.12em} (\mathbf{v} \cdot \mathbf{v}) + \mathbf{F} \cdot \delta \mathbf{r} \right] \hspace{+0.12em} dt$
\par \bigskip \smallskip
\noindent Si sobre la part�cula s�lo act�an fuerzas conservativas entonces $\delta V = - \mathbf{F} \cdot \delta \mathbf{r}$ y como $T = \dfrac{1}{2} m \hspace{+0.12em} (\mathbf{v} \cdot \mathbf{v})$ reemplazando queda:
\par \bigskip \smallskip
\noindent $0 = \int_{t_1}^{t_2} \left[ \hspace{+0.12em} \delta T - \delta V \hspace{+0.12em} \right] \hspace{+0.12em} dt$
\par \bigskip \smallskip
\noindent Esto es:
\par \bigskip \smallskip
\noindent $0 = \delta \int_{t_1}^{t_2} \left[ \hspace{+0.12em} T - V \hspace{+0.12em} \right] \hspace{+0.12em} dt$
\par \bigskip \smallskip
\noindent O bien:
\par \bigskip \smallskip
\noindent $\delta \int_{t_1}^{t_2} \left[ \hspace{+0.12em} T - V \hspace{+0.12em} \right] \hspace{+0.12em} dt = 0$
\par \bigskip \smallskip
\noindent Finalmente se obtiene:
\par \bigskip \smallskip
\noindent $\delta \int_{t_1}^{t_2} L \hspace{+0.24em} dt = 0$
\par \bigskip \smallskip
\noindent Puesto que $L= T - V$.

\newpage

\par \bigskip {\centering\subsection*{Anexo}}

\bigskip \medskip

\noindent $\dfrac{d}{dt}(\bf{v} \cdot \delta \bf{r}) = \ldots$
\par \bigskip
\par \bigskip
\noindent $\dfrac{d}{dt}(m \, \bf{v} \cdot \delta \bf{r}) = \ldots$
\par \bigskip
\noindent $\sum_i \dfrac{d}{dt}(m_i \, {\bf{v}}_i \cdot \delta {\bf{r}}_i) = \ldots$
\par \bigskip
\noindent $\sum_{i,j} \dfrac{d}{dt}(m_i \, {\bf{v}}_i \cdot \dfrac{\partial {\bf{r}}_i}{\partial q_j} \; \delta q_j) = \ldots$
\par \bigskip
\par \bigskip
\noindent $\sum_{i,j} \dfrac{d}{dt}(m_i \, {\bf{v}}_i \cdot \dfrac{\partial {\bf{r}}_i}{\partial q_j} \; \delta q_j) = \sum_{i,j} m_i \, {\bf{v}}_i \cdot \dfrac{d}{dt}(\dfrac{\partial {\bf{r}}_i}{\partial q_j} \; \delta q_j) + \sum_{i,j} m_i \, {\bf{a}}_i \cdot \dfrac{\partial {\bf{r}}_i}{\partial q_j} \; \delta q_j$
\par \bigskip
\noindent $\sum_{i,j} \dfrac{d}{dt}(m_i \, {\bf{v}}_i \cdot \dfrac{\partial {\bf{r}}_i}{\partial q_j} \; \delta q_j) - \sum_{i,j} m_i \, {\bf{v}}_i \cdot \dfrac{d}{dt}(\dfrac{\partial {\bf{r}}_i}{\partial q_j} \; \delta q_j) = \sum_{i,j} m_i \, {\bf{a}}_i \cdot \dfrac{\partial {\bf{r}}_i}{\partial q_j} \; \delta q_j$
\par \bigskip
\noindent $\sum_{i,j} \big [ \dfrac{d}{dt}(m_i \, {\bf{v}}_i \cdot \dfrac{\partial {\bf{r}}_i}{\partial q_j}) - m_i \, {\bf{v}}_i \cdot \dfrac{d}{dt}(\dfrac{\partial {\bf{r}}_i}{\partial q_j}) \big ] \, \delta q_j = \sum_{i,j} m_i \, {\bf{a}}_i \cdot \dfrac{\partial {\bf{r}}_i}{\partial q_j} \; \delta q_j$
\par \bigskip
\par \bigskip
\noindent $\sum_{i,j} \big [ \dfrac{d}{dt}(m_i \, {\bf{v}}_i \cdot \dfrac{\partial {\bf{r}}_i}{\partial q_j}) - m_i \, {\bf{v}}_i \cdot \dfrac{d}{dt}(\dfrac{\partial {\bf{r}}_i}{\partial q_j}) \big ] \, \delta q_j = \sum_{i,j} {\bf{F}}_i \cdot \dfrac{\partial {\bf{r}}_i}{\partial q_j} \; \delta q_j$

\bigskip \smallskip

\par \bigskip {\centering\subsection*{Bibliograf�a}}

\bigskip \medskip

{\fontsize{8.83}{8.83}\selectfont

\noindent http://forum.lawebdefisica.com/threads/28510-Estructura-del-principio-de-m�nima-acci�n
\par \bigskip \smallskip
\noindent http://es.wikipedia.org/wiki/Principio\_de\_m�nima\_acci�n
\par \bigskip \smallskip
\noindent http://es.wikipedia.org/wiki/Desplazamiento\_virtual

}

\end{document}

