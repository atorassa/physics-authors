
\documentclass[10pt,fleqn]{article}
%\documentclass[a4paper,10pt]{article}
%\documentclass[letterpaper,10pt]{article}

\usepackage[dvips]{geometry}
\geometry{papersize={131.1mm,162.0mm}}
\geometry{totalwidth=110.1mm,totalheight=126.0mm}

\usepackage{graphicx}

\usepackage[spanish]{babel}
\usepackage[latin1]{inputenc}
\usepackage[T1]{fontenc}
\usepackage{times}
\usepackage{amsfonts}
\usepackage{amsmath,bm}
\spanishdecimal{.}

\frenchspacing

\usepackage{hyperref}
\hypersetup{colorlinks=true,linkcolor=black}
\hypersetup{bookmarksnumbered=true,pdfstartview=FitH,pdfpagemode=UseNone}
\hypersetup{pdftitle={Una Nueva Din�mica en Relatividad Especial}}
\hypersetup{pdfauthor={A. Blato}}

\setlength{\arraycolsep}{1.74pt}

\newcommand{\med}{\raise.5ex\hbox{$\scriptstyle 1$}\kern-.15em/\kern-.12em\lower.45ex\hbox{$\scriptstyle 2$}\;}

\begin{document}

\begin{center}

{\fontsize{10.77}{10.77}\selectfont \sc Una Nueva Din�mica en Relatividad Especial}

\bigskip \medskip

{A. Blato}

\bigskip \medskip

\small

Licencia Creative Commons Atribuci�n 3.0

\smallskip

(2016) Buenos Aires

\medskip

Argentina

\smallskip

\bigskip \medskip

\parbox{83.10mm}{En relatividad especial, este art�culo presenta una nueva din�mica que puede ser aplicada en cualquier sistema de referencia inercial.}

\end{center}

\normalsize

\vspace{-1.20em}

\par \bigskip {\centering\subsubsection*{Introducci�n}}

\bigskip \smallskip

\noindent En relatividad especial, la posici�n relativista $( \, {\boldsymbol{\varphi}} \, )$, la velocidad relativista $( \, \dot{\boldsymbol{\varphi}} \, )$ y la aceleraci�n relativista $( \, \ddot{\boldsymbol{\varphi}} \, )$ de una part�cula est�n dadas por:
\par \vspace{-0.30em}
\begin{eqnarray*}
{\boldsymbol{\varphi}} ~\doteq~ \mathbf{r}
\end{eqnarray*}
\par \vspace{+0.30em}
\begin{eqnarray*}
\dot{\boldsymbol{\varphi}} ~\doteq~ \frac{d{\boldsymbol{\varphi}}}{d{\tau}} ~=~ \frac{\mathbf{v}}{\sqrt{1 - \frac{v^2}{c^2}}}
\end{eqnarray*}
\par \vspace{-0.30em}
\begin{eqnarray*}
\ddot{\boldsymbol{\varphi}} ~\doteq~ \frac{d{\dot{\boldsymbol{\varphi}}}}{d{\tau}} ~=~ \frac{1}{\sqrt{1 - \frac{v^2}{c^2}}} \left [ \; \frac{\mathbf{a}}{\sqrt{1 - \frac{v^2}{c^2}}} + \frac{(\mathbf{a} \cdot \mathbf{v}) \: \mathbf{v}}{c^2\hspace{-0.18em}\left(1 - \frac{v^2}{c^2}\right)^{\hspace{-0.09em}{3/2}}} \; \right ]
\end{eqnarray*}
\par \vspace{+1.20em}
\noindent donde $( \, \mathbf{r}, \, \mathbf{v}, \, \mathbf{a} \, )$ son la posici�n, la velocidad y la aceleraci�n de la part�cula. $( \, {\tau} \, )$ es el tiempo propio de la part�cula. {\small $d{\tau} = \sqrt{1 - v^2/c^2} \; d{t}$}

\newpage

\par \bigskip {\centering\subsubsection*{Din�mica Poincariana}}

\bigskip \smallskip

\noindent En relatividad especial, sea una part�cula con masa en reposo $m_o$ entonces el momento lineal $\mathbf{P}$ de la part�cula, la fuerza poincariana neta $\mathbf{\widehat{F}}$ que act�a sobre la part�cula, el trabajo $\mathrm{W}$ realizado por la fuerza poincariana neta que act�a sobre la part�cula y la energ�a cin�tica $\mathrm{K}$ de la part�cula, para un sistema de referencia inercial, est�n dados por:
\par \vspace{-0.30em}
\begin{eqnarray*}
\mathbf{P} ~\doteq~ m_o \, \dot{\boldsymbol{\varphi}}
\end{eqnarray*}
\vspace{-0.30em}
\begin{eqnarray*}
\mathbf{\widehat{F}} ~=~ \frac{d\mathbf{P}}{d{\tau}} ~=~ m_o \, \ddot{\boldsymbol{\varphi}}
\end{eqnarray*}
\vspace{-0.30em}
\begin{eqnarray*}
\mathrm{W} ~\doteq~ \int_1^{\hspace{+0.06em}2} \mathbf{\widehat{F}} \cdot d{{\boldsymbol{\varphi}}} ~=~ \Delta \, \mathrm{K}
\end{eqnarray*}
\vspace{-0.30em}
\begin{eqnarray*}
\mathrm{K} ~\doteq~ \med \, m_o \, ( \dot{\boldsymbol{\varphi}} \cdot \dot{\boldsymbol{\varphi}} )
\end{eqnarray*}
\par \vspace{+0.90em}
\noindent $( \, {\boldsymbol{\varphi}}, \, \dot{\boldsymbol{\varphi}}, \, \ddot{\boldsymbol{\varphi}} \, )$ son la posici�n relativista, la velocidad relativista y la aceleraci�n relativista de la part�cula respecto al sistema de referencia inercial.
\par \vspace{+1.50em}
\noindent {\small $\mathbf{\widehat{F}} = \gamma \: \mathbf{F}$} ( donde {\small $\gamma$} es el factor de Lorentz y {\small $\mathbf{F}$} es la fuerza einsteniana neta que act�a sobre la part�cula )
\par \vspace{+1.50em}
\noindent La fuerza poincariana neta $\mathbf{\widehat{F}}$ que act�a sobre una part�cula siempre tiene igual direcci�n y sentido que la aceleraci�n relativista $\ddot{\boldsymbol{\varphi}}$ de la part�cula.

\end{document}

