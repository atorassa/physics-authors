
\documentclass[10pt,fleqn]{article}
%\documentclass[a4paper,10pt]{article}
%\documentclass[letterpaper,10pt]{article}

\usepackage[dvips]{geometry}
\geometry{papersize={134.1mm,195.0mm}}
\geometry{totalwidth=113.1mm,totalheight=159.0mm}

\usepackage{graphicx}

\usepackage[spanish]{babel}
\usepackage[latin1]{inputenc}
\usepackage[T1]{fontenc}
\usepackage{times}
\usepackage{amsfonts}
\usepackage{amsmath,bm}
\spanishdecimal{.}

\frenchspacing

\usepackage{hyperref}
\hypersetup{colorlinks=true,linkcolor=black}
\hypersetup{bookmarksnumbered=true,pdfstartview=FitH,pdfpagemode=UseNone}
\hypersetup{pdftitle={Relatividad Especial con Espacio y Tiempo Absolutos}}
\hypersetup{pdfauthor={A. Blato}}

\setlength{\arraycolsep}{1.74pt}

\begin{document}

\begin{center}

{\sc Relatividad Especial con Espacio y Tiempo Absolutos}

\bigskip \medskip

{A. Blato}

\bigskip \medskip

\small

Licencia Creative Commons Atribuci�n 3.0

\smallskip

(2016) Buenos Aires

\medskip

Argentina

\smallskip

\bigskip \medskip

\parbox{90.60mm}{En relatividad especial, este art�culo presenta magnitudes cinem�ticas que son invariantes bajo las transformaciones de Lorentz.}

\end{center}

\normalsize

\vspace{-1.20em}

\par \bigskip {\centering\subsubsection*{Introducci�n}}

\bigskip \smallskip

\noindent A partir de un cuerpo puntual auxiliar (denominado free-point) es posible obtener magnitudes cinem�ticas (denominadas absolutas) que son invariantes bajo las transformaciones de Lorentz.
\par \bigskip \smallskip
\noindent El free-point es un cuerpo puntual (part�cula masiva) que est� siempre libre de fuerzas externas e internas (o que su fuerza resultante est� siempre en equilibrio)
\par \bigskip \smallskip
\noindent La posici�n absoluta $( \, \breve{x}_i, \, \breve{y}_i, \, \breve{z}_i \, )$ y el tiempo absoluto $( \, \breve{t}_i \, )$ de una part�cula $i$ respecto a un sistema de referencia inercial S, est�n dados por:
\par \vspace{+0.30em}
\begin{eqnarray*}
\breve{x}_i ~\doteq~ \dfrac{x_i - V_x \, t_i}{\sqrt{1 - \dfrac{V_x^2}{c^2}}} \;\;\;\;\;\; , \;\;\;\;\;\; \breve{y}_i ~\doteq~ y_i \;\;\;\;\;\; , \;\;\;\;\;\; \breve{z}_i ~\doteq~ z_i
\end{eqnarray*}
\vspace{+0.60em}
\begin{eqnarray*}
\breve{t}_i ~\doteq~ \dfrac{t_i - \dfrac{V_x \, x_i}{c^2}}{\sqrt{1 - \dfrac{V_x^2}{c^2}}}
\end{eqnarray*}
\par \vspace{+1.50em}
\noindent donde $( \, x_i, \, y_i, \, z_i, \, t_i \, )$ son la posici�n y el tiempo de la part�cula $i$ respecto al sistema de referencia inercial S, $( \, V_x \, )$ es la velocidad ( sobre el eje $x$ ) del free-point respecto al sistema de referencia inercial S y $( \, c \, )$ es la velocidad de la luz en el vac�o.

\newpage

\par \bigskip {\centering\subsubsection*{Observaciones}}

\bigskip \smallskip

\noindent En este art�culo, las magnitudes cinem�ticas $( \, \breve{x}, \, \breve{y}, \, \breve{z}, \, \breve{t} \, )$ son invariantes bajo las transformaciones de Lorentz.
\par \bigskip \smallskip
\noindent A partir de las magnitudes anteriores ser�a posible obtener la posici�n absoluta $\breve{\mathbf{r}}$, la velocidad absoluta $\breve{\mathbf{v}}$ y la aceleraci�n absoluta $\breve{\mathbf{a}}$ de una part�cula con masa en reposo $m_o$ respecto a un sistema de referencia inercial S.
\par \bigskip \smallskip
\noindent Luego, el momento lineal $\mathbf{P}$, la fuerza $\mathbf{F}$, el trabajo $\mathrm{W}$ y la energ�a cin�tica $\mathrm{K}$, para el sistema de referencia inercial S, estar�an dados por:
\par \vspace{+0.30em}
\begin{eqnarray*}
\mathbf{P} ~\doteq~ \dfrac{m_o \, \breve{\mathbf{v}}}{\sqrt{1 - \dfrac{\breve{v}^2}{c^2}}}
\end{eqnarray*}
\vspace{+0.60em}
\begin{eqnarray*}
\mathbf{F} ~=~ \dfrac{d\mathbf{P}}{d\breve{t}}
\end{eqnarray*}
\vspace{+0.30em}
\begin{eqnarray*}
\mathrm{W} ~\doteq~ \int_1^{\hspace{+0.06em}2} \mathbf{F} \cdot d\breve{\mathbf{r}} ~=~ \Delta \, \mathrm{K}
\end{eqnarray*}
\vspace{+0.30em}
\begin{eqnarray*}
\mathrm{K} ~\doteq~ m_o \, c^2 \left ( \: \dfrac{1}{\sqrt{1 - \dfrac{\breve{v}^2}{c^2}}} - 1 \: \right )
\end{eqnarray*}
\par \vspace{+1.50em}
\noindent Seg�n este art�culo, las magnitudes $( \, \breve{\mathbf{r}}, \, \breve{\mathbf{v}}, \, \breve{\mathbf{a}}, \, \mathbf{P}, \, \mathbf{F}, \, \mathrm{W}, \, \mathrm{K} \, )$ ser�an invariantes tambi�n bajo las transformaciones de Lorentz.
\par \bigskip \smallskip
\noindent Sin embargo, este art�culo considera, por un lado, que ser�a tambi�n posible \hbox {obtener} magnitudes cinem�ticas y din�micas $( \, \breve{\mathbf{r}}, \, \breve{\mathbf{v}}, \, \breve{\mathbf{a}}, \, \mathbf{P}, \, \mathbf{F}, \, \mathrm{W}, \, \mathrm{K} \, )$ que ser�an invariantes bajo transformaciones entre sistemas de referencia inerciales y no inerciales y, por otro lado, que las magnitudes din�micas $( \, \mathbf{P}, \, \mathbf{F}, \, \mathrm{W}, \, \mathrm{K} \, )$ \hbox {estar�an} dadas tambi�n por las ecuaciones de arriba.

\end{document}

